\chapter{Auxiliary results}
This appendix contains results which are used throughout the dissertation. We separate the appendix into sections depending on which chapter the result is first used.

%\section{Entropy}
\section{Gibbs measures}
We use the following result in \thref{lem:pp-3-3} and also \thref{prop:walters-cor-4-2-1}. First, we need the following definitions.

\begin{definition}
	Let $f : X \to \reals$ be a function of a convex set $X$. We say that $f$ is \key{strictly convex} if, for all $x, y \in X$ and for all $\alpha \in [0, 1]$, we have
	\[
		f(\alpha x + (1 - \alpha)y) \leq \alpha f(x) + (1 - \alpha)f(y),
	\]
	with equality if and only if $x = y$ or $\alpha = 0$ or $1$.
	
	In general, $f$ is strictly convex if, for any set $\{x_j \in X \mid j \in \{1, \dots, k\}\}$ and for any $\{\alpha_j \in [0, 1] \mid j \in \{1, \dots, k\},\ \sum_{j = 1}^k{\alpha_j} = 1\}$, we have
	\[
		f\left(\sum_{j = 1}^k{\alpha_j x_j}\right) \leq \sum_{j = 1}^k{\alpha_j} f(x_j),
	\]
	with equality if and only if $x_1 = x_2 = \dots = x_k$ whenever $\alpha_j \neq 0$ for all $j = 1, \dots, k$.
	
	The function $f$ is \key{strictly concave} if the $-f$ is strictly convex.
\end{definition}

\begin{theorem} \label{thm:walters-4-2-xlogx-convex}
	Let $f : [0, +\infty) \to \reals$ be the function defined by
	\[
		f(x) =
		\begin{cases}
			0, & \text{if } x = 0; \\
			x\log{x}, & \text{if } x \neq 0.
		\end{cases}
	\]
	Then $f$ is \emph{strictly convex}.
	\begin{proof}
		For $x > 0$ we have $f'(x) = 1 + \log{x}$ and $f''(x) = 1 / x$. Suppose that $y > x$ and let $\alpha \in (0, 1)$. In particular, this means that $x < \alpha x + (1 - \alpha)y < y$. Now $f$ is clearly continuous, and hence we may apply the Mean Value Theorem. So there exists $z \in (\alpha x + (1 - \alpha)y, y)$ such that
		\[
			f'(z) = \frac{f(y) - f(\alpha x + (1 - \alpha)y)}{y - \alpha x - (1 - \alpha)y},
		\]
		and so
		\[
			f(y) - f(\alpha x + (1 - \alpha)y) = f'(z)\alpha(y - x).
		\]
		
		Similarly, there exists $w \in (x, \alpha x + (1 - \alpha)y)$ such that
		\[
			f(\alpha x + (1 - \alpha)y) - f(x) = f'(w)(1 - \alpha)(y - x).
		\]
		
		Since $x > 0$, we have $f''(x) = 1 / x > 0$ and so $f'(z) > f'(w)$. Hence
		\begin{align*}
			&(1 - \alpha)(f(y) - f(\alpha x + (1 - \alpha)y)) = f'(z)\alpha(1 - \alpha)(y - x) \\
			& \quad > f'(w)\alpha(1 - \alpha)(y - x) = \alpha(f(\alpha x + (1 - \alpha)y) - f(x)).
		\end{align*}
		Cancelling terms, this gives
		\[
			f(\alpha x + (1 - \alpha)y) < \alpha f(x) + (1 - \alpha)f(y)
		\]
		for $y > x > 0$. We may apply the same argument for $x, y \geq 0$ and $x \neq y$.
	\end{proof}
\end{theorem}

%\section{Gibbs measures}
The follow result is used in the proof of \thref{lem:pp-prop-3-4}. %First, we need to a definition.

%\begin{definition}
%	Let $X$ be a convex set and suppose $f : X \to \reals$ (or $\complex$). We say that the function $f$ is \key{concave} on $X$ if for all $x, y \in X$ we have
%	\[
%		f(\alpha x + (1 - \alpha)y) \geq \alpha f(x) + (1 - \alpha)f(y),
%	\]
%	for all $\alpha \in [0, 1]$. The function $f$ is \key{strictly concave} if the inequality is strict.~\cite[p11]{cambini-martein:generalized}
%\end{definition}

\begin{lemma} \label{lem:pp-3-3}
	Suppose that $(p_1, \dots, p_k), (q_1, \dots, q_k)$ are probability vectors with $p_j > 0$ for all $j = 1, \dots, k$. Then
	\[
		-\sum_{j = 1}^k{q_j \log{q_j}} + \sum_{j = 1}^k{q_j \log{p_j}} \leq 0,
	\]
	with equality if and only if $p_j = q_j$ for all $j = 1, \dots, k$.
	\begin{proof}
		We have
		\begin{align}
			-\sum_{j = 1}^k{q_j \log{q_j}} + \sum_{j = 1}^k{q_j \log{p_j}} &= \sum_{j = 1}^k{q_j \log\left(\frac{p_j}{q_j}\right)} \nonumber \\
				&= \sum_{j = 1}^k{-p_j \frac{q_j}{p_j} \log\left(\frac{q_j}{p_j}\right)} \nonumber \\
				&= \sum_{j = 1}^k{p_j \phi\left(\frac{q_j}{p_j}\right)}, \label{fml:pp-3-3-sums}
		\end{align}
		where $\phi(x) = -x \log{x}$, with the convention that $\phi(0) = 0$. By \thref{thm:walters-4-2-xlogx-convex}, we have that $x \log{x}$ is a strictly convex function, and so $\phi$ is strictly concave. Continuing from \eqref{fml:pp-3-3-sums}, we have
		\[
			-\sum_{j = 1}^k{q_j \log{q_j}} + \sum_{j = 1}^k{q_j \log{p_j}} \leq \phi\left(\sum_{j = 1}^k{p_j \frac{q_j}{p_j}}\right) = \phi(1) = 0,
		\]
		with equality if and only if all the $(q_j / p_j)$ terms are equal.
	\end{proof}
\end{lemma}

\section{Concentration bounds}

The following result is used in the proof of \thref{thm:cm-4-1}.

\begin{proposition}\label{prop:logs-thm-4-1}
	Let $\ell, k \in \naturals$ be such that $\ell + k \leq n$. Then
	\[
		\left|\left(\frac{\ell}{n}\right)\log\left(\frac{\ell}{n}\right) - \left(\frac{\ell + k}{n}\right)\log\left(\frac{\ell + k}{n}\right)\right| \leq \left(\frac{k}{n}\right)\log{n}.
	\]
	
	\begin{proof}
		We have
		\begin{align*}
			&\left(\frac{\ell}{n}\right)\log\left(\frac{\ell}{n}\right) - \left(\frac{\ell + k}{n}\right)\log\left(\frac{\ell + k}{n}\right) \\
				&\quad = \left(\frac{k}{n}\right)\log{n} - \left[\left(\frac{\ell + k}{n}\right)\log(\ell + k) - \left(\frac{\ell}{n}\right)\log{\ell}\right].
		\end{align*}
		Since $k \geq 1$, we have
		\begin{align*}
			0 &\leq \left(\frac{\ell + k}{n}\right)\log(\ell + k) - \left(\frac{\ell}{n}\right)\log{\ell} \\
				&\leq \left(\frac{\ell + k}{n}\right)\log(\ell + k) + \left(\frac{k - \ell}{n}\right)\log{\ell} \\
				&\leq \left(\frac{\ell + k}{n}\right)\log{n} + \left(\frac{k - \ell}{n}\right)\log{n} \\
				&\leq 2\left(\frac{k}{n}\right)\log{n}.
		\end{align*}
		Hence
		\[
			-\left(\frac{k}{n}\right)\log{n} \leq \left(\frac{\ell}{n}\right)\log\left(\frac{\ell}{n}\right) - \left(\frac{\ell + k}{n}\right)\log\left(\frac{\ell + k}{n}\right) \leq \left(\frac{k}{n}\right)\log{n},
		\]
		in other words,
		\[
			\left|\left(\frac{\ell}{n}\right)\log\left(\frac{\ell}{n}\right) - \left(\frac{\ell + k}{n}\right)\log\left(\frac{\ell + k}{n}\right)\right| \leq \left(\frac{k}{n}\right)\log{n}.
		\]
	\end{proof}
\end{proposition}

The following result allows us to write the $k$-block conditional empirical entropy $\hat{h}_{k(n)}(x_0^{n - 1})$ in a more useful form. It is used in the proof of \thref{thm:cm-4-2}.

\begin{lemma}\label{lem:cm-4-1}
	Let $\phi \in F_\theta$. We have
	\begin{equation}
		\hat{h}_{k(n)}(x_0^{n - 1}) = \frac{1}{n}\sum_{j = 0}^{n - 1}(-\phi \circ \sigma^j(x)) + \hat{\Delta}_{k(n)}(x_0^{n - 1}) + O(\theta^{k(n)}),
	\end{equation}
	where
	\[
		|E(\hat{\Delta}_{k(n)})| \leq \frac{M|A|^{k(n)}}{n},
	\]
	for some $M > 0$.
	\begin{proof}
		We follow the proof given in \cite[p10-11]{chazottes-maldonado:cbfee}.
		
		We have
		\begin{align}
			\hat{h}_k(x_0^{n - 1}) &= h_k(\E_k(\seedot; x_0^{n - 1})) \nonumber \\
				&= -\sum_{a_0^{k - 1} \in A^k}{\E_k(a_0^{k - 1}; x_0^{n - 1}) \log{\frac{\E_k(a_0^{k - 1}; x_0^{n - 1})}{\E_{k - 1}(a_0^{k - 2}; x_0^{n - 1})}}} \nonumber \\
				&= \hat{\Delta}_k(x_0^{n - 1}) -\sum_{a_0^{k - 1} \in A^k}\E_k(a_0^{k - 1}; x_0^{n - 1}) \log{\frac{\mu_\phi[a_0^{k - 1}]}{\mu_\phi[a_1^{k - 1}]}}, \label{fml:cm-11}
		\end{align}
		where
		\begin{align*}
			\hat{\Delta}_k(x_0^{n - 1}) &=-\sum_{a_0^{k - 1} \in A^k}{\E_k(a_0^{k - 1}; x_0^{n - 1}) \log{\frac{\E_k(a_0^{k - 1}; x_0^{n - 1})}{\E_{k - 1}(a_0^{k - 2}; x_0^{n - 1})}}} \\
				& \qquad + \sum_{a_0^{k - 1} \in A^k}\E_k(a_0^{k - 1}; x_0^{n - 1}) \log{\frac{\mu_\phi[a_0^{k - 1}]}{\mu_\phi[a_1^{k - 1}]}} \\
				&=-\sum_{a_0^{k - 1} \in A^k}{\E_k(a_0^{k - 1}; x_0^{n - 1}) \log{\frac{\E_k(a_0^{k - 1}; x_0^{n - 1})}{\mu_\phi[a_0^{k - 1}]}}} \\
				& \qquad + \sum_{a_0^{k - 1} \in A^k}\E_k(a_0^{k - 1}; x_0^{n - 1}) \log{\frac{\E_{k - 1}(a_0^{k - 2}; x_0^{n - 1})}{\mu_\phi[a_1^{k - 1}]}}.
		\end{align*}
		Since $\E_k(\seedot; x_0^{n - 1})$ is locally $\sigma$-invariant, we have
		\[
			\sum_{a_0 \in A}{\E_k(a_0^{k - 1}; x_0^{n - 1})} = \E_{k - 1}(a_1^{k - 1}; x_0^{n - 1}).
		\]
		It is also clear that
		\[
			\sum_{a_{k - 1} \in A}{\mathcal{E}_k(a_0^{k - 1}; x_0^{n - 1})} = \mathcal{E}_{k - 1}(a_0^{k - 2}; x_0^{n - 1}).
		\]
		This means that
		\begin{align*}
			\sum_{a_0^{k - 1} \in A^k}&\E_k(a_0^{k - 1}; x_0^{n - 1}) \log{\frac{\E_{k - 1}(a_0^{k - 2}; x_0^{n - 1})}{\mu_\phi[a_1^{k - 1}]}} \\
				&= \sum_{a_0^{k - 1} \in A^k}\E_k(a_0^{k - 1}; x_0^{n - 1}) \log{\frac{\sum_{a_{k} \in A}{\mathcal{E}_k(a_0^{k - 1}; x_0^{n - 1})}}{\mu_\phi[a_1^{k - 1}]}} \\
				&= \sum_{a_0^{k - 1} \in A^k}\E_k(a_0^{k - 1}; x_0^{n - 1}) \log{\frac{\mathcal{E}_k(a_0^{k - 1}; x_0^{n - 1})}{\mu_\phi[a_1^{k - 1}]}} \\
				&= \sum_{a_1^{k - 1} \in A^{k - 1}}\sum_{a_0 \in A}\E_k(a_0^{k - 1}; x_0^{n - 1}) \log{\frac{\mathcal{E}_k(a_0^{k - 1}; x_0^{n - 1})}{\mu_\phi[a_1^{k - 1}]}} \\
				&= \sum_{a_1^{k - 1} \in A^{k - 1}}\E_{k - 1}(a_1^{k - 1}; x_0^{n - 1}) \log{\frac{\mathcal{E}_{k - 1}(a_1^{k - 1}; x_0^{n - 1})}{\mu_\phi[a_1^{k - 1}]}} \\
				&= H_{k - 1}(\E_{k - 1}(\seedot; x_0^{n - 1})).
		\end{align*}
		Hence $\hat{\Delta}_k(x_0^{n - 1}) = -H_k(\E_k(\seedot; x_0^{n - 1})) + H_{k - 1}(\E_{k - 1}(\seedot; x_0^{n - 1}))$.
		
		There is a formula due to \cite[Formulae (4.15), (4.16)]{gabrielli-galves-guiol:fluctuations} which gives the bound
		\[
			|E_{\mu_\phi}(\hat{\Delta}_{k(n)})| \leq \frac{M|A|^k}{n}
		\]
		for all $n \geq 1$, where $M > 0$ is a strictly positive constant.
		
		We now deal with the summation in Formula \eqref{fml:cm-11}. For $y \in \Sigma$ we put
		\[
			\phi_k(y) = \log\frac{\mu_\phi[y_0^{k - 1}]}{\mu_\phi[y_1^{k - 1}]}.
		\]
		By \thref{prop:pp-3-2}, for all $y \in \Sigma$ we have
		\[
			\|\phi_k(y) - \phi(y)\|_\infty = \left\|\log\frac{\mu_\phi[y_0^{k - 1}]}{\mu_\phi[y_1^{k - 1}]} - \phi(y)\right\|_\infty \leq |\phi|_\theta \theta^k.
		\]
		So we may reasonably replace any instance of $\phi_k(y)$ with $\phi(y) + O(\theta^k)$. By the way $\E_k(\seedot; x_0^{n - 1})$ is defined, we have
		\begin{align*}
			-\sum_{a_0^{k - 1} \in A^k}\E_k(a_0^{k - 1}; x_0^{n - 1}) \log{\frac{\mu_\phi[a_0^{k - 1}]}{\mu_\phi[a_1^{k - 1}]}} &= -\sum_{a_0^{k - 1} \in A^k}\E_k(a_0^{k - 1}; x_0^{n - 1}) \phi_k(a) \\
				&= \frac{1}{n}\sum_{j = 0}^{n - 1}{(-\phi \circ \sigma^j(x))} + O(\theta^k).
		\end{align*}
		The result follows by substituting this back into Formula \eqref{fml:cm-11}.
	\end{proof}
\end{lemma}
