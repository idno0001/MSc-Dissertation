\chapter{Entropy}
\begin{mdframed}[linewidth=2,leftmargin=108,rightmargin=108,skipbelow=30]
	\textbf{Throughout this chapter $(X, \B, \mu)$ will denote a probability space.}
\end{mdframed}

\section{Isomorphisms of measure-preserving transformations}
One of the main problems in ergodic theory is to classify measure-preserving transformations. To this end, we want to decide the conditions required for two measure-preserving transformations to be `the same' -- up to sets of measure zero.

\subsection{Isomorphism and conjugacy of measure spaces}
\emph{This subsection predominantly follows \cite[Chapter 2]{walters:intro-to-ergodic-theory}.}

We begin by defining when two probability spaces are isomorphic or conjugate.

\begin{definition}
	Two probability spaces $(X_1, \B_1, \mu_1), (X_2, \B_2, \mu_2)$ are \key{isomorphic} if there exists $M_1 \in \B_1$, $M_2 \in \B_2$ such that $\mu_1(M_1) = 1 = \mu_2(M_2)$ and if there exists an invertible measure-preserving transformation $\phi: M_1 \to M_2$.
\end{definition}

Let $A, C \subset \B$. We define an equivalence relation on $\B$: we have $A \sim C$ if and only if $\mu(A \symdiff C) = 0$. In other words, $A$ and $C$ belong to the same equivalence class if they are equal almost everywhere. It can be easily checked that $\sim$ is indeed an equivalence relation.

Let $\tilde{\B}$ denote the collection of all equivalence classes in $\B$. Since $\B$ is a $\sigma$-algebra, it is clear that $\tilde{B}$ is also a $\sigma$-algebra. We can define a measure $\tilde{\mu} : \tilde{\B} \to \reals^+$ by $\tilde{\mu}(\tilde{B}) = \mu(B)$, where $B$ belongs to the equivalence class $\tilde{B}$.

\begin{definition}
	A \key{measure algebra} is a Boolean $\sigma$-algebra equipped with a measure.
\end{definition}

In view of this definition, we see that $(\tilde{\B}, \tilde{\mu})$ is a \key{measure algebra}.

\begin{definition}
	Let $(X_1, \B_1, \mu_1), (X_2, \B_2, \mu_2)$ be probability spaces with corresponding measure algebras $(\tilde{\B}_1, \tilde{\mu}_1), (\tilde{\B}_2, \tilde{\mu}_2)$, respectively.
	
	We say $(\tilde{\B}_1, \tilde{\mu}_1)$ and $(\tilde{\B}_2, \tilde{\mu}_2)$ are \key{isomorphic} if there exists a bijection $\phi : \tilde{\B}_2 \to \tilde{\B}_1$ which preserves complmentation and countable unions and intersections such that $\tilde{\mu}_1(\phi \tilde{B}) = \tilde{\mu}_2(\tilde{B})$ for all $\tilde{B} \in \tilde{\B}_2$.
	
	The probability spaces $(X_1, \B_1, \mu_1)$ and $(X_2, \B_2, \mu_2)$ are \key{conjugate} if their corresponding measure algebras are isomorphic.
\end{definition}

\begin{proposition}
	If two probability spaces are isomorphic then they are also conjugate.
	\begin{proof}
		Suppose $(X_1, \B_1, \mu_1), (X_2, \B_2, \mu_2)$ are isomorphic probability spaces with corresponding measure algebras $(\tilde{\B}_1, \tilde{\mu}_1), (\tilde{\B}_2, \tilde{\mu}_2)$. By definition, this means there exists $M_1 \in \B_1$, $M_2 \in \B_2$ such that $\mu_1(M_1) = 1 = \mu_2(M_2)$ and there exists an invertible measure-preserving transformation $\phi: M_1 \to M_2$.
		
		Now we can define the map
		\[
			\psi : \tilde{\B}_2 \to \tilde{\B}_1 : \tilde{B} \mapsto (\phi^{-1}(M_2 \cap B))^\sim.
		\]
		This is clearly a bijection and, since $\phi$ is measure-preserving and $M_2 = X_2$ almost everywhere, we have
		\[
			\tilde{\mu}_1(\psi\tilde{B}) = \tilde{\mu}_1(\phi^{-1}(M_2 \cap B))^\sim = \tilde{\mu}_2(M_2 \cap B)^\sim = \tilde{\mu}_2(\tilde{B}),
		\]
		for all $\tilde{B} \in \tilde{\B}_2$. Therefore the measure algebras are isomorphic and hence the corresponding measure spaces are isomorphic.
	\end{proof}
\end{proposition}

The converse statement is not necessarily true. Indeed, suppose we have the probability space $(X_1, \B_1, \mu_1)$ consisting of exactly one point, and another probability space $(X_2, \B_2, \mu_2)$ consisting of exactly two points, with $\B_2 = \{\emptyset, X_2\}$. It is easy to see that the measure algebras are isomorphic and hence the measure spaces are conjugate.

We need to choose $M_1 \in \B_1$, $M_2 \in \B_2$ such that $\mu_1(M_1) = 1 = \mu_2(M_2)$; the only possibility is $M_1 = X_1$ and $M_2 = X_2$. However there does not exist bijection between these two sets, so the probability spaces are \emph{isomorphic}.

\subsection{A motivational example}
We describe a scenario when two measure-preserving transformations could be considered `the same'. We follow the example in \cite[p58]{walters:intro-to-ergodic-theory}.

\section{Conditional expectation}
\begin{definition}
	Conditional expectation operator $E(\:\cdot \mid \C) : L^1(X, \B, \mu) \to L^1(X, \C, \mu)$ is defined...
\end{definition}

The operator $E(f \mid \C)$ is uniquely determined by the requirement that:
\begin{enumerate}
	\item $E(f \mid \C)$ is $\C$-measurable, and
	\item for all $C \in \C$ we have
	\[
	\int_C{f\ d\mu} = \int_C{E(f \mid \C)\ d\mu}.
	\]
\end{enumerate}

We can think of $E(f \mid \C)$ as the best approximation of $f$ in the smaller space $\C$ of measurable functions.~\cite[Lecture 21]{ergodic-lectures}

...

\emph{The remainder of this chapter predominantly follows \cite[Chapter 4]{walters:intro-to-ergodic-theory}.}

\section{Conditional entropy}
\begin{theorem}\label{thm:walters-4.3}
	Let $(X, \B, \mu)$ be a probability space and let $\A, \B, \D$ be finite sub-$\sigma$-algebras of $\B$. Suppose $T : X \to X$ is a measure-preserving transformation. Then
	\begin{enumerate}
		\item $H_\mu(\A \join \C \mid \D) = H_\mu(\A \mid \D) + H_\mu(\C \mid \A \join \D)$,
		\item $H_\mu(\A \join \C) = H_\mu(\A) + H_\mu(\C \mid \A)$,
		\item if $\A \subset \C$  then $H_\mu(\A \mid \D) \leq H_\mu(\C \mid \D)$,
		\item if $\A \subset \C$  then $H_\mu(\A) \leq H_\mu(\C)$,
		\item if $\C \subset \D$ then $H_\mu(\A \mid \C) \geq H_\mu(\A \mid \D)$,
		\item $H_\mu(\A) \geq H_\mu(\A \mid \D)$,
		\item $H_\mu(\A \join \C \mid \D) \leq H_\mu(\A \mid \D) + H_\mu(\C \mid \D)$,
		\item $H_\mu(\A \join \C) \leq H_\mu(\A) + H_\mu(\C)$,
		\item $H_\mu(T^{-1}\A \mid T^{-1}\C) = H_\mu(\A \mid \C)$,
		\item $H_\mu(T^{-1}\A) = H_\mu(\A)$.
	\end{enumerate}
	\begin{proof}
		Coming soon!
	\end{proof}
\end{theorem}

...

\begin{theorem}\label{thm:walters-4.12}
	Let $\A, \C$ be finite sub-algebras of $\B$ and let $T$ be a measure-preserving transformation of a probability space $(X, \B, \mu)$ Then
	\begin{enumerate}
		\item $h_\mu(T, \A) \leq H_\mu(\A)$,
		\item $h_\mu(T, \A \join \C) \leq h_\mu(T, \A) + h_\mu(T, \C)$,
		\item if $\A \subset \C$ then $h_\mu(T, \A) \leq h_\mu(T, \C)$,
		\item $h_\mu(T, \A) \leq h_\mu(T, \C) + H_\mu(\A \mid \C)$,
		\item $h_\mu(T, T^{-1}{\A}) = h_\mu(T, \A)$,
		\item if $k \geq 1$ then $h_\mu(T, \A) = h_\mu\left(T, \bigjoin\limits_{i = 0}^{k - 1}{T^{-i}{\A}}\right)$,
		\item if $T$ is invertible and $k \geq 1$ then $h_\mu(T, \A) = h_\mu\left(T, \bigjoin\limits_{i = -k}^k{T^{-i}{\A}}\right)$.
	\end{enumerate}
	\begin{proof}
		Coming soon!
	\end{proof}
\end{theorem}
