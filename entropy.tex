\chapter{Entropy}
\begin{mdframed}[linewidth=2,leftmargin=108,rightmargin=108,skipbelow=30]
	\textbf{Throughout this chapter $(X, \B, \mu)$ will denote a probability space.}
\end{mdframed}

\section{Isomorphisms of measure-preserving transformations}\label{sec:isos-of-mpts}
One of the main problems in ergodic theory is to classify measure-preserving transformations. To this end, we want to decide the conditions required for two measure-preserving transformations to be `the same' -- up to sets of measure zero.

\emph{This section predominantly follows material in \cite[Chapter 2]{walters:intro-to-ergodic-theory}.}

\subsection{Isomorphism and conjugacy of measure spaces}

We begin by defining when two probability spaces are isomorphic or conjugate.

\begin{definition}
	Two probability spaces $(X_1, \B_1, \mu_1), (X_2, \B_2, \mu_2)$ are \key{isomorphic} if there exists $M_1 \in \B_1$, $M_2 \in \B_2$ such that $\mu_1(M_1) = 1 = \mu_2(M_2)$ and if there exists an invertible measure-preserving transformation $\phi: M_1 \to M_2$.
\end{definition}

Let $A, C \subset \B$. We define an equivalence relation on $\B$: we have $A \sim C$ if and only if $\mu(A \symdiff C) = 0$. In other words, $A$ and $C$ belong to the same equivalence class if they are equal almost everywhere. It can be easily checked that $\sim$ is indeed an equivalence relation.

Let $\tilde{\B}$ denote the collection of all equivalence classes in $\B$. Since $\B$ is a $\sigma$-algebra, it is clear that $\tilde{\B}$ is also a $\sigma$-algebra. We can define a measure $\tilde{\mu} : \tilde{\B} \to \reals^+$ by $\tilde{\mu}(\tilde{B}) = \mu(B)$, where $B$ belongs to the equivalence class $\tilde{B}$.

\begin{definition}
	A \key{measure algebra} is a Boolean $\sigma$-algebra equipped with a measure.
\end{definition}

In view of this definition, we see that $(\tilde{\B}, \tilde{\mu})$ is a \key{measure algebra}.

\begin{definition}
	Let $(X_1, \B_1, \mu_1), (X_2, \B_2, \mu_2)$ be probability spaces with corresponding measure algebras $(\tilde{\B}_1, \tilde{\mu}_1), (\tilde{\B}_2, \tilde{\mu}_2)$, respectively.
	
	We say $(\tilde{\B}_1, \tilde{\mu}_1)$ and $(\tilde{\B}_2, \tilde{\mu}_2)$ are \key{isomorphic} if there exists a bijection $\phi : \tilde{\B}_2 \to \tilde{\B}_1$ which preserves complementation and countable unions and intersections such that $\tilde{\mu}_1(\phi \tilde{B}) = \tilde{\mu}_2(\tilde{B})$ for all $\tilde{B} \in \tilde{\B}_2$.
	
	The probability spaces $(X_1, \B_1, \mu_1)$ and $(X_2, \B_2, \mu_2)$ are \key{conjugate} if their corresponding measure algebras are isomorphic.
\end{definition}

\begin{proposition}
	If two probability spaces are isomorphic then they are also conjugate.
	\begin{proof}
		Suppose $(X_1, \B_1, \mu_1), (X_2, \B_2, \mu_2)$ are isomorphic probability spaces with corresponding measure algebras $(\tilde{\B}_1, \tilde{\mu}_1), (\tilde{\B}_2, \tilde{\mu}_2)$. By definition, this means there exists $M_1 \in \B_1$, $M_2 \in \B_2$ such that $\mu_1(M_1) = 1 = \mu_2(M_2)$ and there exists an invertible measure-preserving transformation $\phi: M_1 \to M_2$.
		
		Now we can define the map
		\[
			\psi : \tilde{\B}_2 \to \tilde{\B}_1 : \tilde{B} \mapsto (\phi^{-1}(M_2 \cap B))^\sim.
		\]
		This is clearly a bijection and, since $\phi$ is measure-preserving and $M_2 = X_2$ almost everywhere, we have
		\[
			\tilde{\mu}_1(\psi\tilde{B}) = \tilde{\mu}_1(\phi^{-1}(M_2 \cap B))^\sim = \tilde{\mu}_2(M_2 \cap B)^\sim = \tilde{\mu}_2(\tilde{B}),
		\]
		for all $\tilde{B} \in \tilde{\B}_2$. Therefore the measure algebras are isomorphic and hence the corresponding measure spaces are isomorphic.
	\end{proof}
\end{proposition}

The converse statement is not necessarily true. Indeed, suppose we have the probability space $(X_1, \B_1, \mu_1)$ consisting of exactly one point, and another probability space $(X_2, \B_2, \mu_2)$ consisting of exactly two points, with $\B_2 = \{\emptyset, X_2\}$. It is easy to see that the measure algebras are isomorphic and hence the measure spaces are conjugate.

We need to choose $M_1 \in \B_1$, $M_2 \in \B_2$ such that $\mu_1(M_1) = 1 = \mu_2(M_2)$; the only possibility is $M_1 = X_1$ and $M_2 = X_2$. However there does not exist bijection between these two sets, so the probability spaces are \emph{isomorphic}.

\subsection{A motivational example}
We describe a scenario when two measure-preserving transformations could be considered `the same'. We follow the example in \cite[p58]{walters:intro-to-ergodic-theory}.

We first introduce a new probability space.

\begin{comment}
Let $Y = \{0, 1\}$ and let $(p_0, p_1)$ be a probability vector with no zero entries. Then $(Y, 2^Y, \nu)$ is a measure space, with measure $\nu$ defined by $\nu(y) = p_y$ for $y \in Y$. Now let $X = \{(x_j)_{j = 0}^\infty \mid x_j \in Y\}$, the space of infinite sequences with entries in $Y = \{0, 1\}$.
\end{comment}
\subsubsection{Bernoulli shifts}
Let $Y = \{0, 1, \dots, k\}$ be a set of $k$ symbols and let $p = (p_0, p_1, \dots, p_k)$ be a probability vector with no zero entries. Let $X = \{(x_j)_{j = 0}^\infty \mid x_j \in Y \text{ for all } j \geq 0\}$ be the space of infinite sequences with entries in $Y$. We may define a measure $\nu$ on cylinders of length $n$ by
\[
	\nu[x_0, x_1, \dots, x_{n - 1}] = p_{x_0} p_{x_1} \dots p_{x_{n - 1}}.
\]
Such measures are known as \key{Bernoulli measures}. Let $\sigma : X \to X$ be the one-sided, left shift map on $X$.

\begin{proposition}
	The measure $\nu$ is $\sigma$-invariant.
	\begin{proof}
		We have
		\begin{align*}
			\nu(\sigma^{-1}[x_1, \dots, x_n]) &= \nu\left(\bigsqcup_{j = 0}^{k - 1}{[j, x_1, \dots, x_n]}\right) \\
				&= \sum_{j = 0}^{k - 1}{\nu[j, x_1, \dots, x_n]} \\
				&= \sum_{j = 0}^{k - 1}{p_j p_{x_1} \dots p_{x_n}} \\
				&= p_{x_1} \dots p_{x_n} \\
				&= \nu[x_1, \dots, x_n].
		\end{align*}
		(We have used the fact that $\sum_{j = 0}^{k - 1}{p_j} = 1$ on the penultimate line.)
	\end{proof}
\end{proposition}

The shift map $\sigma$ is called the one-sided $(p_0, p_1, \dots, p_{k - 1})$-shift.

We are now ready to present two measure-preserving transformations which we argue are `the same'.

\subsubsection{The \texorpdfstring{($\mathbf{\frac{1}{2}, \frac{1}{2}}$)}{(1/2, 1/2)}-shift and the doubling map}
Let $T : ([0, 1), \B, \mu) \to ([0, 1), \B, \mu) : x \mapsto 2x \bmod 1$ be the doubling map, where $\B$ is the Borel $\sigma$-algebra on $[0, 1)$ and $\mu$ is Lebesgue measure.

Let $\sigma : (X, \C, \nu) \to (X, \C, \nu)$ be the $(\frac{1}{2}, \frac{1}{2})$-shift, where $X = \{(x_j)_{j = 0}^\infty \mid x_j \in \{0, 1\} \text{ for all } j \geq 0\}$, $\C$ is the $\sigma$-algebra generated by all cylinders in $X$, and $\nu$ is the Bernoulli measure as described above.

Define the map $\phi : X \to [0, 1)$ by
\[
	\phi(x_0, x_1, \dots) = \sum_{j = 0}^\infty{\frac{x_j}{2^{j + 1}}} = \frac{x_0}{2^1} + \frac{x_1}{2^2} + \frac{x_2}{2^3} + \dots.
\]
It is easy to see that $\phi$ maps the binary expansion of a number to the actual number itself.

Let $E = \{(x_j)_{j = 0}^\infty \in X \mid (x_j)_{j = N}^\infty \text{ is constant for some } N \geq 0\}$ be the set of sequences in $X$ whose coordinates are eventually constant. Now, if the binary expansion of a number is \emph{not} eventually constant, then this binary expansion is unique. Therefore $\phi$ is \emph{injective} on $X \setminus E$. It is also clear that $\phi$ is \emph{surjective}, since every number in $[0, 1)$ has at least one binary expansion. It is also easy to see that $\phi \circ \sigma = T \circ \phi$.

We now show that $\phi$ is measure-preserving. Let $[\frac{a}{2^s}, \frac{a + 1}{2^s}] \subset [0, 1)$ be a dyadic interval, where $s \in \naturals$. We can write
\[
	\frac{a}{2^s} = \sum_{j = 0}^{s - 1}{\frac{a_j}{2^j}} \quad \text{and} \quad \frac{a + 1}{2^s} = \sum_{j = 0}^\infty{\frac{a_j}{2^j}},
\]
where $a_j \in \{0, 1\}$ for $j = 0, 1, \dots, s - 2$ and $a_k = 1$ for $k \geq s - 1$. In other words, the binary expansion of all numbers in the interval $[\frac{a}{2^s}, \frac{a + 1}{2^s}]$ agree in the first $s$ positions. Thus, \begin{align*}
	\nu\left(\phi^{-1}\left[\frac{a}{2^s}, \frac{a + 1}{2^s}\right]\right) &= \nu[a_0, a_1, \dots, a_{s - 1}] \\
		&= \frac{1}{2^s} \\
		&= \mu\left[\frac{a}{2^s}, \frac{a + 1}{2^s}\right].
\end{align*}
Hence $\phi$ is measure-preserving on dyadic intervals, which generate the Borel $\sigma$-algebra $\B$ on $[0, 1)$. We may therefore apply the Kolmogorov Extension Theorem and it follows that $\phi$ is \emph{measure-preserving} on all Borel sets $B \in \B$.

Let $D = \{\frac{a}{2^s} \in [0, 1) \mid s \in \naturals,\ 0 \leq a < 2^s\}$ be the set of all dyadic rationals in $[0, 1)$. Clearly, $T^{-1}D = D$ and this means that $T^{-1}([0, 1) \setminus D) = [0, 1) \setminus D$. It is also clear that $\sigma^{-1}E = E$ and so $\sigma^{-1}(X \setminus E) = X \setminus E$. So by the above observations, we see that $\phi: X \setminus E \to [0, 1) \setminus D$ is a bijection. It is also clear that $\phi \circ \sigma(x) = T \circ \phi(x)$ for all $x \in X \setminus E$.

Finally, we have $D \subset \rationals$ which gives $\mu(D) = 0$, and we also note that there are countably many sequences in $E$, thus $\nu(E) = 0$. Therefore $\phi$ is an invertible measure-preserving transformation between $X$ and $[0, 1)$ (modulo sets of measure zero), that is, the measure-preserving transformations are \emph{isomorphic}. Therefore it makes sense to say that these measure-preserving transformations are `the same'.

\subsection{\texorpdfstring{\sloppy Isomorphism and conjugacy of measure-preserving transformations}{Isomorphism and conjugacy of measure-preserving transformations}}
We now formalise the ideas illustrated in the above example.

\begin{definition}
	\sloppy Let $(X_1, \B_1, \mu_1, T_1)$, $(X_2, \B_2, \mu_2, T_2)$ be measure-preserving transformations of probability spaces. We say that $T_1$ is \key{isomorphic} to $T_2$ if there exists $M_1 \in \B_1$, $M_2 \in \B_2$ such that $\mu_1(M_1) = 1 = \mu_2(M_2)$ with
	\begin{enumerate}
		\item $T_1{M_1} \subset M_1$ and $T_2{M_2} \subset M_2$, and \label{mpt-iso-i}
		\item there exists an invertible measure-preserving transformation $\phi : M_1 \to M_2$ such that $\phi \circ T_1(x) = T_2 \circ \phi(x)$ for all $x \in M_1$. \label{mpt-iso-ii}
	\end{enumerate}
	If this is the case, we write $T_1 \simeq T_2$.
\end{definition}

Now suppose $T_1 \simeq T_2$ with $M_1$, $M_2$ and $\phi : M_1 \to M_2$ as in the above definition. Then for $n \geq 1$ we clearly have $T_1^n{M_1} \subset M_1$ and $T_2^n{M_2} \subset M_2$, satisfying condition \ref{mpt-iso-i}. This in turn gives that $\phi \circ T_1^n(x) = T_2^n \circ \phi(x)$ for all $x \in M_1$, satisfying condition \ref{mpt-iso-ii}. In other words, if $T_1 \simeq T_2$, then $T_1^n \simeq T_2^n$ for all $n \geq 1$.

We also have the notion of conjugacy of measure-preserving transformations.

\begin{definition}
	Let $(X_1, \B_1, \mu_1, T_1)$, $(X_2, \B_2, \mu_2, T_2)$ be measure-preserving transformations of probability spaces. We say that $T_1$ is \key{conjugate} to $T_2$ if there exists an isomorphism $\Phi : (\tilde{\B}_2, \tilde{\mu}_2) \to (\tilde{\B}_1, \tilde{\mu}_1)$ of measure algebras such that $\Phi \circ \tilde{T}_2^{-1} = \tilde{T}_1^{-1} \circ \Phi$.
\end{definition}

It can be easily checked that isomorphism and conjugacy are equivalence relations on the set of all measure-preserving transformations.

As with probability spaces, isomorphic measure-preserving transformations are also conjugate:

\begin{theorem}\label{thm:walters-2-5}
	Let $(X_1, \B_1, \mu_1, T_1)$, $(X_2, \B_2, \mu_2, T_2)$ be measure-preserving transformations of probability spaces and suppose that $T_1 \simeq T_2$. Then $T_1$ is conjugate to $T_2$.
	
	\begin{proof}
		Suppose $T_1 \simeq T_2$, so a measure-preserving transformation $\phi : M_1 \to M_2$ such that $\phi \circ T_1(x) = T_2 \circ \phi(x)$ for all $x \in M_1$, where $M_1, M_2$ are as in the definition.
		
		Define $\Phi : (\tilde{\B}_2, \tilde{\mu}_2) \to (\tilde{\B}_1, \tilde{\mu}_1)$ by $\Phi(\tilde{B}) \mapsto (\phi^{-1}(B \cap M_2))^\sim$ for $B \in B_2$. Recall that $\tilde{B}$ is an equivalence class, so it is easy to see that $\Phi$ is an isomorphism. We also have
		\[
			\tilde{T}_1^{-1} \circ \Phi(\tilde{B}) = \tilde{T}_1^{-1} \circ (\phi^{-1}(B \cap M_2))^\sim = \phi^{-1} \circ \tilde{T}_2^{-1} (B \cap M_2)^\sim = \Phi \circ \tilde{T}_2^{-1}(B)
		\]
		for all $B \in \B_2$. Hence $T_1$ is conjugate to $T_2$.
	\end{proof}
\end{theorem}

The converse of this theorem is not necessarily true. However, we will find it useful to know the conditions for which the converse holds. We need the following definition from \cite[Definition A.21]{einsiedler-ward:ergodic-nt}.

\begin{definition}
	Let $Y$ be a set of countably or finitely many points, where each $y \in Y$ has positive measure $p_y > 0$. Let $s = 1 - \sum_{y \in Y}{p_y}$ and let $\mathcal{L}[0, s]$ denote the $\sigma$-algebra of Lebesgue measurable sets on the closed interval $[0, s]$. Let $\lambda_{[0, s]}$ denote Lebesgue measure on $[0, s]$.
	
	If the probability space $(X, \B, \mu)$ is isomorphic to the probability space
	\[
		\left([0, s] \sqcup Y,\ \mathcal{L}[0, s],\ \lambda_{[0, s]} + \sum_{y \in Y}{p_y \delta_y} \right),
	\]
	where $\delta_y$ is Dirac measure at $y$, then $(X, \B, \mu)$ is a \key{Lebesgue space}.
\end{definition}

We will also use the following result, which is proved in \cite[Theorem 12]{royden:real-analysis}.

\begin{lemma} \label{lem:walters-thm-2-2}
	For $j = 1, 2$, let $(X_j, \B(X_j), \mu_j)$ be complete separable metric spaces endowed with Borel $\sigma$-algebra $\B(X_j)$ and probability measure $\mu_j$. Suppose that $\Phi: \tilde{\B}(X_2) \to \tilde{\B}(X_1)$ is an isomorphism of measure algebras. Then there exists $M_1 \in \B(X_1)$, $M_2 \in \B(X_2)$ such that $\mu_1(M_1) = 1 = \mu_2(M_2)$, and an invertible measure-preserving transformation $\phi: M_1 \to M_2$ such that $\Phi(\tilde{B}) = (\phi^{-1}(B \cap M_2))^\sim$ for all $B \in \B(X_2)$.
	
	If $\psi$ is any other isomorphism $(X_1, \B(X_1), \mu_1)$ to $(X_2, \B(X_2), \mu_2)$ which induces $\Phi$, then $\mu_1\{x \in X_1 \mid \phi(x) \neq \psi(x)\} = 0$.
\end{lemma}

The following result gives the conditions for which the converse of Theorem \ref{thm:walters-2-5} is true.

\begin{theorem} \label{thm:walters-2-6}
	Suppose that either $(X_1, \B_1, \mu_1)$, $(X_2, \B_2, \mu_2)$ are Lebesgue spaces, or that $X_1, X_2$ are each complete separable metric spaces with corresponding Borel $\sigma$-algebras $\B_1, \B_2$. Suppose that $T_1 : X_1 \to X_1$, $T_2 : X_2 \to X_2$ be measure-preserving transformations and that $T_1$ is conjugate to $T_2$. Then $T_1 \simeq T_2$.
	\begin{proof}
		Suppose $\Phi : (\tilde{B}_2, \tilde{\mu}_2) \to (\tilde{B}_1, \tilde{\mu}_1)$ is an isomorphism of measure algebras such that $\Phi \circ \tilde{T}_2^{-1} = \tilde{T}_1^{-1} \circ \Phi$. By Lemma \ref{lem:walters-thm-2-2} there exists sets $X'_1 \in \B_1$, $X'_2 \in \B_2$ such that $\mu_1(X'_1) = 1 = \mu(X'_2)$, and there exists an invertible measure-preserving transformation $\phi : X'_1 \to X'_2$ such that $\Phi(\tilde{B}) = (\phi^{-1}(B \cap X'_2))^\sim$ for all $B \in \B_2$. Then we have $\tilde{\phi}^{-1} \circ \tilde{T}_2^{-1} = \tilde{T}_1^{-1} \circ \tilde{\phi}^{-1}$, i.e. $T_2 \circ \phi = \phi \circ T_1$ almost everywhere.
		
		Now put
		\[
			A_1 = \{x \in X_1 \mid T_2 \circ \phi(x) = \phi \circ T_1(x)\} \quad \text{and} \quad M_1 = \bigcap_{n = 0}^\infty{T_1^{-n}{A_1}}.
		\]
		Then $\mu_1(M_1) = 1$, $T_1^{-1}{M_1} \supset M_1$ which means that $M_1 \supset T_1 M_1$. We then define $M_2 = \phi M_1$ so that $T_2 M_2 \subset M_2$. Hence $T_1 \simeq T_2$.
	\end{proof}
\end{theorem}

As we mentioned briefly at the beginning of Section \ref{sec:isos-of-mpts}, we want to be able to decide when two measure-preserving transformations are `the same'. In view of the above discussion, `the same' can be replaced with `conjugate' or `isomorphic'. \key{Entropy} is one of the main conjugacy and isomorphism invariants studied in ergodic theory, and the remainder of this chapter will describe how the entropy of a measure-preserving transformation is defined.

\emph{The rest of this chapter predominantly follows \cite[Chapter 4]{walters:intro-to-ergodic-theory}.}

\section{Entropy of partitions and sub-\texorpdfstring{$\sigma$}{sigma}-algebras}
\hl{Things to define:}
	\begin{enumerate}
		\item join
		\item notation for partitions, sub-sigma-algebra
	\end{enumerate}

\dots

Here is Theorem 4.2 from Walters. We use this in the final chapter and also in a proof in the appendix.

\begin{theorem} \label{thm:walters-4-2-xlogx-convex}
	Let $f : [0, +\infty) \to \reals$ be the function defined by
	\[
		f(x) =
		\begin{cases}
			0, & \text{if } x = 0; \\
			x\log{x}, & \text{if } x \neq 0.
		\end{cases}
	\]
	Then $f$ is \emph{strictly convex}, i.e. for all $x, y \in [0, +\infty)$ we have
	\[
		f(\alpha x + (1 - \alpha)y) \leq \alpha f(x) + (1 - \alpha)f(y),
	\]
	for all $\alpha \in [0, 1]$, with equality if and only if $x = y$ or $\alpha = 0$ or $1$. We can generalise the notion of convexity as
	\[
		f\left(\sum_{j = 1}^k{\alpha_j x_j}\right) \leq \sum_{j = 1}^k{\alpha_j} f(x_j),
	\]
	where $x_j \in [0, +\infty)$, $\alpha_j \in (0, 1)$ such that $\sum_{j = 1}^k{\alpha_j} = 1$, and with equality if and only if the $x_j$ values are all equal whenever $\alpha_j \neq 0$ for all $j = 1, \dots, k$.
	\begin{proof}
		\hl{TODO. (notes p98)}
	\end{proof}
\end{theorem}

\begin{corollary} \label{cor:walters-4-2-1}
	\hl{TODO.}
\end{corollary}

\hl{In particular, $\log{|A|}$ is an upper bound for entropy by partitions.}

\dots

\section{Conditional expectation}
\begin{definition}
	Conditional expectation operator $E(\seedot \mid \C) : L^1(X, \B, \mu) \to L^1(X, \C, \mu)$ is defined...
\end{definition}

The operator $E(f \mid \C)$ is uniquely determined by the requirement that:
\begin{enumerate}
	\item $E(f \mid \C)$ is $\C$-measurable, and
	\item for all $C \in \C$ we have
	\[
	\int_C{f\ d\mu} = \int_C{E(f \mid \C)\ d\mu}.
	\]
\end{enumerate}

We can think of $E(f \mid \C)$ as the best approximation of $f$ in the smaller space $\C$ of measurable functions.~\cite[Lecture 21]{ergodic-lectures}

...

\emph{The remainder of this chapter predominantly follows \cite[Chapter 4]{walters:intro-to-ergodic-theory}.}

\section{Conditional entropy}
\begin{theorem}\label{thm:walters-4.3}
	Let $(X, \B, \mu)$ be a probability space and let $\A, \B, \D$ be finite sub-$\sigma$-algebras of $\B$. Suppose $T : X \to X$ is a measure-preserving transformation. Then
	\begin{enumerate}
		\item $H_\mu(\A \join \C \mid \D) = H_\mu(\A \mid \D) + H_\mu(\C \mid \A \join \D)$,
		\item $H_\mu(\A \join \C) = H_\mu(\A) + H_\mu(\C \mid \A)$,
		\item if $\A \subset \C$  then $H_\mu(\A \mid \D) \leq H_\mu(\C \mid \D)$,
		\item if $\A \subset \C$  then $H_\mu(\A) \leq H_\mu(\C)$,
		\item if $\C \subset \D$ then $H_\mu(\A \mid \C) \geq H_\mu(\A \mid \D)$,
		\item $H_\mu(\A) \geq H_\mu(\A \mid \D)$,
		\item $H_\mu(\A \join \C \mid \D) \leq H_\mu(\A \mid \D) + H_\mu(\C \mid \D)$,
		\item $H_\mu(\A \join \C) \leq H_\mu(\A) + H_\mu(\C)$,
		\item $H_\mu(T^{-1}\A \mid T^{-1}\C) = H_\mu(\A \mid \C)$,
		\item $H_\mu(T^{-1}\A) = H_\mu(\A)$.
	\end{enumerate}
	\begin{proof}
		Coming soon!
	\end{proof}
\end{theorem}

...

\begin{theorem}\label{thm:walters-4.12}
	Let $\A, \C$ be finite sub-algebras of $\B$ and let $T$ be a measure-preserving transformation of a probability space $(X, \B, \mu)$ Then
	\begin{enumerate}
		\item $h_\mu(T, \A) \leq H_\mu(\A)$,
		\item $h_\mu(T, \A \join \C) \leq h_\mu(T, \A) + h_\mu(T, \C)$,
		\item if $\A \subset \C$ then $h_\mu(T, \A) \leq h_\mu(T, \C)$,
		\item $h_\mu(T, \A) \leq h_\mu(T, \C) + H_\mu(\A \mid \C)$,
		\item $h_\mu(T, T^{-1}{\A}) = h_\mu(T, \A)$,
		\item if $k \geq 1$ then $h_\mu(T, \A) = h_\mu\left(T, \bigjoin\limits_{i = 0}^{k - 1}{T^{-i}{\A}}\right)$,
		\item if $T$ is invertible and $k \geq 1$ then $h_\mu(T, \A) = h_\mu\left(T, \bigjoin\limits_{i = -k}^k{T^{-i}{\A}}\right)$.
	\end{enumerate}
	\begin{proof}
		Coming soon!
	\end{proof}
\end{theorem}
