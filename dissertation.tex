\documentclass[12pt,MSc]{muthesis}
\usepackage{amssymb,amsmath}
\usepackage[T1]{fontenc}
\usepackage{fouriernc}
%\usepackage{eucal}
\usepackage{verbatim}
\usepackage{enumitem}
\usepackage[stretch=10]{microtype}
\usepackage{todonotes}
\usepackage{hyperref}
\usepackage[thmmarks,thref,amsmath,hyperref]{ntheorem}

% Font candidates:
% \usepackage{lmodern}
%
% The following fonts are nice but \nu looks too much like v.
% \usepackage{txfonts} (\mathcal looks nice apart from \C which looks like C.)
% \usepackage{fouriernc}
% \usepackage[bitstream-charter]{mathdesign}
%
% Note - can set standard text font to (e.g.) fouriernc by:
% \renewcommand*{\rmdefault}{fnc}

% Inkscape graphics. Compiles new diagram if svg file changes.
\newcommand{\executeiffilenewer}[3]{%
	\ifnum\pdfstrcmp{\pdffilemoddate{#1}}%
	{\pdffilemoddate{#2}}>0%
	{\immediate\write18{#3}}\fi%
}
%\newcommand{\includesvg}[1]{%
%	\executeiffilenewer{fig/#1.svg}{fig/#1.pdf}%
%	{inkscape -z -D --file=fig/#1.svg %
%		--export-pdf=fig/#1.pdf --export-latex}%
%		\input{fig/#1.pdf_tex}%
%}

% Theorem environments.
%\newcounter{ctr}[section]
%\def\thectr{\thesection.\arabic{ctr}}
\newcounter{ctr}[chapter]
\def\thectr{\thechapter.\arabic{ctr}}
\theorembodyfont{}
\theoremseparator{.}
\theoremstyle{plain}
\theoremsymbol{\ensuremath{\blacksquare}}
\newtheorem{lemma}[ctr]{Lemma}
\newtheorem{proposition}[ctr]{Proposition}
\newtheorem{theorem}[ctr]{Theorem}
\newtheorem{corollary}[ctr]{Corollary}
\theoremstyle{nonumberplain}
\newtheorem{definition}{Definition}
\newtheorem{example}{Example}
\newtheorem{notation}{Notation}
\newtheorem{remark}{Remark}
\theoremheaderfont{\it}\theorembodyfont{\upshape}
\newtheorem{claim}{Claim}
\theoremheaderfont{\sc}\theorembodyfont{\upshape}
\theoremsymbol{\ensuremath{\square}}
\newtheorem{proof}{Proof}

% Redefine lists to use roman numerals.
\setenumerate[1]{label=(\roman{*})}

% Key words.
\newcommand{\key}[1]{\emph{#1}}

% Redefine the spacing used in the cases environment. (Default: 1.2)
\makeatletter
\renewcommand*\env@cases[1][1.0]{%
	\let\@ifnextchar\new@ifnextchar
	\left\lbrace
	\def\arraystretch{#1}%
	\array{@{}l@{\quad}l@{}}%
}
\makeatother

% Frequently used symbols.
\newcommand{\A}{\mathcal{A}}
\newcommand{\B}{\mathcal{B}}
\newcommand{\complex}{\mathbb{C}}
\newcommand{\C}{\mathcal{C}}
\newcommand{\D}{\mathcal{D}}
\newcommand{\E}{\mathcal{E}}
\newcommand{\naturals}{\mathbb{N}}
\newcommand{\rationals}{\mathbb{Q}}
\newcommand{\reals}{\mathbb{R}}
\newcommand{\integers}{\mathbb{Z}}
\newcommand{\join}{\vee}
\newcommand{\bigjoin}{\bigvee}
\DeclareMathOperator{\card}{card}
\DeclareMathOperator{\id}{id}
\DeclareMathOperator{\Lip}{Lip}
\DeclareMathOperator{\var}{var}
\DeclareMathOperator{\Var}{Var}
\newcommand{\symdiff}{\bigtriangleup}

% Avoids inconsistent use of notation.
\renewcommand{\subset}{\subseteq}
\renewcommand{\supset}{\supseteq}
\renewcommand{\tilde}[1]{\widetilde{#1}}

% Special `almost everywhere' notation.
%	(Possible alternatives: \circeq, \accentset{\circ}{=} [requires package accents])
\newcommand{\eqae}{\ \mathring{=}\ }	% `equal almost everywhere'
\newcommand{\subsetae}{\ \mathring{\subset}\ }
\newcommand{\supsetae}{\ \mathring{\supset}\ }

% \relmiddle is like \middle, but gives the same spacing as \mid.
\newcommand{\relmiddle}[1]{\mathrel{}\middle#1\mathrel{}}
\newcommand{\midmid}{\relmiddle|}
\newcommand{\seedot}{\mathrel{}\cdot\mathrel{}}
\newcommand{\midcolon}{\,:\,}

\begin{document}
\title{Concentration Bounds for Entropy Estimation of One-Dimensional Gibbs Measures}
\author{Anthony Chiu}
% Faculty of Life Sciences people should comment the next line out
\school{Mathematics}
\faculty{Engineering and Physical Sciences}
\immediate\write18{texcount.pl -1 -inc -sum dissertation.tex > thesis.count}
\def\wordcount{\documentclass[12pt,MSc]{muthesis}
\usepackage{amssymb,amsmath}
\usepackage[T1]{fontenc}
\usepackage{fouriernc}
%\usepackage{eucal}
\usepackage{verbatim}
\usepackage{enumitem}
%\usepackage{accents}
\usepackage{todonotes}
\usepackage{hyperref}
\usepackage[thmmarks,thref,amsmath,hyperref]{ntheorem}

% Font candidates:
% \usepackage{lmodern}
%
% The following fonts are nice but \nu looks too much like v.
% \usepackage{txfonts} (\mathcal looks nice apart from \C which looks like C.)
% \usepackage{fouriernc}
% \usepackage[bitstream-charter]{mathdesign}
%
% Note - can set standard text font to (e.g.) fouriernc by:
% \renewcommand*{\rmdefault}{fnc}

% Inkscape graphics. Compiles new diagram if svg file changes.
\newcommand{\executeiffilenewer}[3]{%
	\ifnum\pdfstrcmp{\pdffilemoddate{#1}}%
	{\pdffilemoddate{#2}}>0%
	{\immediate\write18{#3}}\fi%
}
%\newcommand{\includesvg}[1]{%
%	\executeiffilenewer{fig/#1.svg}{fig/#1.pdf}%
%	{inkscape -z -D --file=fig/#1.svg %
%		--export-pdf=fig/#1.pdf --export-latex}%
%		\input{fig/#1.pdf_tex}%
%}

% Theorem environments.
%\newcounter{ctr}[section]
%\def\thectr{\thesection.\arabic{ctr}}
\newcounter{ctr}[chapter]
\def\thectr{\thechapter.\arabic{ctr}}
\theorembodyfont{}
\theoremseparator{.}
\theoremstyle{plain}
\theoremsymbol{\ensuremath{\blacksquare}}
\newtheorem{lemma}[ctr]{Lemma}
\newtheorem{proposition}[ctr]{Proposition}
\newtheorem{theorem}[ctr]{Theorem}
\newtheorem{corollary}[ctr]{Corollary}
\theoremstyle{nonumberplain}
\newtheorem{definition}{Definition}
\newtheorem{example}{Example}
\newtheorem{notation}{Notation}
\newtheorem{remark}{Remark}
\theoremheaderfont{\it}\theorembodyfont{\upshape}
\newtheorem{claim}{Claim}
\theoremheaderfont{\sc}\theorembodyfont{\upshape}
\theoremsymbol{\ensuremath{\square}}
\newtheorem{proof}{Proof}

% Redefine lists to use roman numerals.
\setenumerate[1]{label=(\roman{*})}

% Key words.
\newcommand{\key}[1]{\emph{#1}}

% Redefine the spacing used in the cases environment. (Default: 1.2)
\makeatletter
\renewcommand*\env@cases[1][1.0]{%
	\let\@ifnextchar\new@ifnextchar
	\left\lbrace
	\def\arraystretch{#1}%
	\array{@{}l@{\quad}l@{}}%
}
\makeatother

% Frequently used symbols.
\newcommand{\A}{\mathcal{A}}
\newcommand{\B}{\mathcal{B}}
\newcommand{\complex}{\mathbb{C}}
\newcommand{\C}{\mathcal{C}}
\newcommand{\D}{\mathcal{D}}
\newcommand{\E}{\mathcal{E}}
\newcommand{\naturals}{\mathbb{N}}
\newcommand{\rationals}{\mathbb{Q}}
\newcommand{\reals}{\mathbb{R}}
\newcommand{\integers}{\mathbb{Z}}
\newcommand{\join}{\vee}
\newcommand{\bigjoin}{\bigvee}
\DeclareMathOperator{\card}{card}
\DeclareMathOperator{\Lip}{Lip}
\DeclareMathOperator{\var}{var}
\DeclareMathOperator{\Var}{Var}
\newcommand{\symdiff}{\bigtriangleup}

% Avoids inconsistent use of notation.
\renewcommand{\subset}{\subseteq}
\renewcommand{\supset}{\supseteq}
\renewcommand{\tilde}[1]{\widetilde{#1}}

% Special `almost everywhere' notation.
%	(Possible alternatives: \circeq, \accentset{\circ}{=} [requires package accents])
\newcommand{\eqae}{\ \mathring{=}\ }	% `equal almost everywhere'
\newcommand{\subsetae}{\ \mathring{\subset}\ }
\newcommand{\supsetae}{\ \mathring{\supset}\ }

% \relmiddle is like \middle, but gives the same spacing as \mid.
\newcommand{\relmiddle}[1]{\mathrel{}\middle#1\mathrel{}}
\newcommand{\midmid}{\relmiddle|}
\newcommand{\seedot}{\mathrel{}\cdot\mathrel{}}
\newcommand{\midcolon}{\,:\,}

\begin{document}
\title{Concentration Bounds For Entropy Estimation of One-Dimensional Gibbs Measures}
\author{Anthony Chiu}
% Faculty of Life Sciences people should comment the next line out
\school{Mathematics}
\faculty{Engineering and Physical Sciences}
\immediate\write18{texcount.pl -1 -inc -sum thesis.tex > thesis.count}
\def\wordcount{\documentclass[12pt,MSc]{muthesis}
\usepackage{amssymb,amsmath}
\usepackage[T1]{fontenc}
\usepackage{fouriernc}
%\usepackage{eucal}
\usepackage{verbatim}
\usepackage{enumitem}
%\usepackage{accents}
\usepackage{todonotes}
\usepackage{hyperref}
\usepackage[thmmarks,thref,amsmath,hyperref]{ntheorem}

% Font candidates:
% \usepackage{lmodern}
%
% The following fonts are nice but \nu looks too much like v.
% \usepackage{txfonts} (\mathcal looks nice apart from \C which looks like C.)
% \usepackage{fouriernc}
% \usepackage[bitstream-charter]{mathdesign}
%
% Note - can set standard text font to (e.g.) fouriernc by:
% \renewcommand*{\rmdefault}{fnc}

% Inkscape graphics. Compiles new diagram if svg file changes.
\newcommand{\executeiffilenewer}[3]{%
	\ifnum\pdfstrcmp{\pdffilemoddate{#1}}%
	{\pdffilemoddate{#2}}>0%
	{\immediate\write18{#3}}\fi%
}
%\newcommand{\includesvg}[1]{%
%	\executeiffilenewer{fig/#1.svg}{fig/#1.pdf}%
%	{inkscape -z -D --file=fig/#1.svg %
%		--export-pdf=fig/#1.pdf --export-latex}%
%		\input{fig/#1.pdf_tex}%
%}

% Theorem environments.
%\newcounter{ctr}[section]
%\def\thectr{\thesection.\arabic{ctr}}
\newcounter{ctr}[chapter]
\def\thectr{\thechapter.\arabic{ctr}}
\theorembodyfont{}
\theoremseparator{.}
\theoremstyle{plain}
\theoremsymbol{\ensuremath{\blacksquare}}
\newtheorem{lemma}[ctr]{Lemma}
\newtheorem{proposition}[ctr]{Proposition}
\newtheorem{theorem}[ctr]{Theorem}
\newtheorem{corollary}[ctr]{Corollary}
\theoremstyle{nonumberplain}
\newtheorem{definition}{Definition}
\newtheorem{example}{Example}
\newtheorem{notation}{Notation}
\newtheorem{remark}{Remark}
\theoremheaderfont{\it}\theorembodyfont{\upshape}
\newtheorem{claim}{Claim}
\theoremheaderfont{\sc}\theorembodyfont{\upshape}
\theoremsymbol{\ensuremath{\square}}
\newtheorem{proof}{Proof}

% Redefine lists to use roman numerals.
\setenumerate[1]{label=(\roman{*})}

% Key words.
\newcommand{\key}[1]{\emph{#1}}

% Redefine the spacing used in the cases environment. (Default: 1.2)
\makeatletter
\renewcommand*\env@cases[1][1.0]{%
	\let\@ifnextchar\new@ifnextchar
	\left\lbrace
	\def\arraystretch{#1}%
	\array{@{}l@{\quad}l@{}}%
}
\makeatother

% Frequently used symbols.
\newcommand{\A}{\mathcal{A}}
\newcommand{\B}{\mathcal{B}}
\newcommand{\complex}{\mathbb{C}}
\newcommand{\C}{\mathcal{C}}
\newcommand{\D}{\mathcal{D}}
\newcommand{\E}{\mathcal{E}}
\newcommand{\naturals}{\mathbb{N}}
\newcommand{\rationals}{\mathbb{Q}}
\newcommand{\reals}{\mathbb{R}}
\newcommand{\integers}{\mathbb{Z}}
\newcommand{\join}{\vee}
\newcommand{\bigjoin}{\bigvee}
\DeclareMathOperator{\card}{card}
\DeclareMathOperator{\Lip}{Lip}
\DeclareMathOperator{\var}{var}
\DeclareMathOperator{\Var}{Var}
\newcommand{\symdiff}{\bigtriangleup}

% Avoids inconsistent use of notation.
\renewcommand{\subset}{\subseteq}
\renewcommand{\supset}{\supseteq}
\renewcommand{\tilde}[1]{\widetilde{#1}}

% Special `almost everywhere' notation.
%	(Possible alternatives: \circeq, \accentset{\circ}{=} [requires package accents])
\newcommand{\eqae}{\ \mathring{=}\ }	% `equal almost everywhere'
\newcommand{\subsetae}{\ \mathring{\subset}\ }
\newcommand{\supsetae}{\ \mathring{\supset}\ }

% \relmiddle is like \middle, but gives the same spacing as \mid.
\newcommand{\relmiddle}[1]{\mathrel{}\middle#1\mathrel{}}
\newcommand{\midmid}{\relmiddle|}
\newcommand{\seedot}{\mathrel{}\cdot\mathrel{}}
\newcommand{\midcolon}{\,:\,}

\begin{document}
\title{Concentration Bounds For Entropy Estimation of One-Dimensional Gibbs Measures}
\author{Anthony Chiu}
% Faculty of Life Sciences people should comment the next line out
\school{Mathematics}
\faculty{Engineering and Physical Sciences}
\immediate\write18{texcount.pl -1 -inc -sum thesis.tex > thesis.count}
\def\wordcount{\documentclass[12pt,MSc]{muthesis}
\usepackage{amssymb,amsmath}
\usepackage[T1]{fontenc}
\usepackage{fouriernc}
%\usepackage{eucal}
\usepackage{verbatim}
\usepackage{enumitem}
%\usepackage{accents}
\usepackage{todonotes}
\usepackage{hyperref}
\usepackage[thmmarks,thref,amsmath,hyperref]{ntheorem}

% Font candidates:
% \usepackage{lmodern}
%
% The following fonts are nice but \nu looks too much like v.
% \usepackage{txfonts} (\mathcal looks nice apart from \C which looks like C.)
% \usepackage{fouriernc}
% \usepackage[bitstream-charter]{mathdesign}
%
% Note - can set standard text font to (e.g.) fouriernc by:
% \renewcommand*{\rmdefault}{fnc}

% Inkscape graphics. Compiles new diagram if svg file changes.
\newcommand{\executeiffilenewer}[3]{%
	\ifnum\pdfstrcmp{\pdffilemoddate{#1}}%
	{\pdffilemoddate{#2}}>0%
	{\immediate\write18{#3}}\fi%
}
%\newcommand{\includesvg}[1]{%
%	\executeiffilenewer{fig/#1.svg}{fig/#1.pdf}%
%	{inkscape -z -D --file=fig/#1.svg %
%		--export-pdf=fig/#1.pdf --export-latex}%
%		\input{fig/#1.pdf_tex}%
%}

% Theorem environments.
%\newcounter{ctr}[section]
%\def\thectr{\thesection.\arabic{ctr}}
\newcounter{ctr}[chapter]
\def\thectr{\thechapter.\arabic{ctr}}
\theorembodyfont{}
\theoremseparator{.}
\theoremstyle{plain}
\theoremsymbol{\ensuremath{\blacksquare}}
\newtheorem{lemma}[ctr]{Lemma}
\newtheorem{proposition}[ctr]{Proposition}
\newtheorem{theorem}[ctr]{Theorem}
\newtheorem{corollary}[ctr]{Corollary}
\theoremstyle{nonumberplain}
\newtheorem{definition}{Definition}
\newtheorem{example}{Example}
\newtheorem{notation}{Notation}
\newtheorem{remark}{Remark}
\theoremheaderfont{\it}\theorembodyfont{\upshape}
\newtheorem{claim}{Claim}
\theoremheaderfont{\sc}\theorembodyfont{\upshape}
\theoremsymbol{\ensuremath{\square}}
\newtheorem{proof}{Proof}

% Redefine lists to use roman numerals.
\setenumerate[1]{label=(\roman{*})}

% Key words.
\newcommand{\key}[1]{\emph{#1}}

% Redefine the spacing used in the cases environment. (Default: 1.2)
\makeatletter
\renewcommand*\env@cases[1][1.0]{%
	\let\@ifnextchar\new@ifnextchar
	\left\lbrace
	\def\arraystretch{#1}%
	\array{@{}l@{\quad}l@{}}%
}
\makeatother

% Frequently used symbols.
\newcommand{\A}{\mathcal{A}}
\newcommand{\B}{\mathcal{B}}
\newcommand{\complex}{\mathbb{C}}
\newcommand{\C}{\mathcal{C}}
\newcommand{\D}{\mathcal{D}}
\newcommand{\E}{\mathcal{E}}
\newcommand{\naturals}{\mathbb{N}}
\newcommand{\rationals}{\mathbb{Q}}
\newcommand{\reals}{\mathbb{R}}
\newcommand{\integers}{\mathbb{Z}}
\newcommand{\join}{\vee}
\newcommand{\bigjoin}{\bigvee}
\DeclareMathOperator{\card}{card}
\DeclareMathOperator{\Lip}{Lip}
\DeclareMathOperator{\var}{var}
\DeclareMathOperator{\Var}{Var}
\newcommand{\symdiff}{\bigtriangleup}

% Avoids inconsistent use of notation.
\renewcommand{\subset}{\subseteq}
\renewcommand{\supset}{\supseteq}
\renewcommand{\tilde}[1]{\widetilde{#1}}

% Special `almost everywhere' notation.
%	(Possible alternatives: \circeq, \accentset{\circ}{=} [requires package accents])
\newcommand{\eqae}{\ \mathring{=}\ }	% `equal almost everywhere'
\newcommand{\subsetae}{\ \mathring{\subset}\ }
\newcommand{\supsetae}{\ \mathring{\supset}\ }

% \relmiddle is like \middle, but gives the same spacing as \mid.
\newcommand{\relmiddle}[1]{\mathrel{}\middle#1\mathrel{}}
\newcommand{\midmid}{\relmiddle|}
\newcommand{\seedot}{\mathrel{}\cdot\mathrel{}}
\newcommand{\midcolon}{\,:\,}

\begin{document}
\title{Concentration Bounds For Entropy Estimation of One-Dimensional Gibbs Measures}
\author{Anthony Chiu}
% Faculty of Life Sciences people should comment the next line out
\school{Mathematics}
\faculty{Engineering and Physical Sciences}
\immediate\write18{texcount.pl -1 -inc -sum thesis.tex > thesis.count}
\def\wordcount{\input{thesis.count}}

% Uncomment the line below to suppress the `List of Tables' page (optional)
\tablespagefalse

% Uncomment the line below to suppress the `List of Figures' page (optional)
\figurespagefalse

\beforeabstract

Abstract here.

\afterabstract

% The next part is optional; however it is a good place to thank your
% supervisor and the people responsible for providing computer support ;-)
\prefacesection{Acknowledgements}
I would like to thank...

% The next line is NOT optional and MUST appear
\afterpreface

% Finally, you can start writing about all the new theorems you have proved
% and all the new results that you have discovered

\input{conventions}
\input{introduction}
\input{background}
\input{entropy}
\input{gibbs}
\input{concentration_bounds}

\appendix
\input{appendix}

\bibliographystyle{alpha}
\bibliography{references}

% If you need more than one appendix, then just use another \chapter command
%\chapter{Yet Another Appendix}

\end{document}
}

% Uncomment the line below to suppress the `List of Tables' page (optional)
\tablespagefalse

% Uncomment the line below to suppress the `List of Figures' page (optional)
\figurespagefalse

\beforeabstract

Abstract here.

\afterabstract

% The next part is optional; however it is a good place to thank your
% supervisor and the people responsible for providing computer support ;-)
\prefacesection{Acknowledgements}
I would like to thank...

% The next line is NOT optional and MUST appear
\afterpreface

% Finally, you can start writing about all the new theorems you have proved
% and all the new results that you have discovered

\prefacesection{Conventions}
The following outlines the conventions used throughout this dissertation.

\section*{Notation}
\begin{trivlist}
	\item $\naturals = \{1, 2, 3, \ldots\}$, the set of natural numbers -- positive integers \emph{not} including $0$.
	\item $\naturals_0 = \{0, 1, 2, \ldots\}$, the set of nonnegative integers.
	\item $\reals^+ = \{a \in \reals \mid a \geq 0\}$, the set of nonnegative real numbers including $0$.
\end{trivlist}

\section*{Propositions, theorems, examples, etc.}
Throughout this dissertation a black square $\blacksquare$ marks the end of each definition, remark and results where the proof has been omitted. If we have a claim within a proof, then the proof of the claim is also marked with a black square.

A white square $\square$ is used to mark the end of all other proofs.

\chapter{Introduction}
\section{Overview}
The main aim of this dissertation is to provide a more accessible account of \cite{chazottes-maldonado:cbfee}, which we will do in Chapter \ref{chap:concentration-bounds}. To achieve this, the following chapters are devoted to describing the key concepts required to provide the reader with the relevant background knowledge.

A property of measure-preserving transformations called \key{entropy} makes up a crucial part of this dissertation. We will show that, if two measure-preserving transformations are `the same', then they have the same entropy. (In Chapter \ref{chap:entropy} we will formally define `the same'.)

The main ideas in \cite{chazottes-maldonado:cbfee} focus on methods for estimating entropy. For example, there is a theorem which says that, for almost all $(x, y)$ we have
\[
	\frac{1}{n} \log{W_n(x, y)} \to h(\nu),
\]
as $n \to +\infty$, where $W_n$ is a function we will define later and $h(\nu)$ is the entropy of a measure $\nu$. We will see later that this is called the hitting time entropy estimator.

Although it may seem obvious, it is worth emphasising that entropy estimators give \emph{estimates} for the entropy. For example, consider two sample pairs $(x_1, y_1)$, $(x_2, y_2)$ which achieve convergence for the above function. If we fix $n \geq 1$, it is possible that $\frac{1}{n} \log{W_n(x_1, y_1)}$ gives a value which is close to $h(\nu)$, whereas $\frac{1}{n} \log{W_n(x_2, y_2)}$ gives a value which is far away from $h(\nu)$. In the final part of this dissertation, we will be particularly interested in these fluctuation properties of entropy estimators.

We will work with \key{Gibbs measures}, which is a class of measures on shifts of finite type with a distinguishing property. Therefore Chapter \ref{chap:sft} will provide the relevant background for shifts of finite type, and Chapter \ref{chap:gibbs} will define Gibbs measures and its main properties.

\section{Preliminaries}
Before we begin with the main background material, this section briefly introduces some concepts and definitions which will be used throughout this dissertation.

\begin{definition}
	Let $(X, d_X)$ and $(Y, d_Y)$ be metric spaces. A function $f : X \to Y$ is a \key{Lipschitz function} if there exists a constant $K > 0$ such that
	\[
		d_Y(f(x), f(y)) \leq Kd_X(x, y)
	\]
	for all $x, y \in X$. If this is the case, we say that $f$ is a Lipschitz function with \key{Lipschitz constant} $K$.~\cite[p154]{searcoid:metric-spaces}
\end{definition}

\begin{definition}
	A transformation $T : (X_1, \B_1, \mu_1) \to (X_2, \B_2, \mu_2)$ is \key{measure-preserving} if:
	\begin{enumerate}
		\item $T$ is measurable, i.e. if $B_2 \in \B_2$, then $T^{-1}{B_2} \in \B_1$, and
		\item $\mu_1(T^{-1}{B_2}) = \mu_2(B_2)$ for all $B_2 \in \B_2$.
	\end{enumerate}
	This agrees with our usual definition when $X_1 = X_2$.
\end{definition}

\begin{definition}
	Let $X$ be a compact metric space with Borel $\sigma$-algebra $\B$. We let $M(X)$ denote the set of all probability measures on $(X, \B)$.
	
	Let $T : X \to X$ be a continuous mapping on $X$. We let $M(X, T)$ denote the set of $T$-invariant probability measures on $(X, B)$.
\end{definition}

\begin{definition}
	The \key{symmetric difference} of two sets $A, B$ is defined $(A \setminus B) \cup (B \setminus A)$. We will write this as
	\[
		A \symdiff B := (A \setminus B) \cup (B \setminus A).
	\]
\end{definition}
\chapter{Background}
\section{Preliminaries}
This section briefly introduces some concepts which will be used throughout this dissertation.

\begin{definition}
	Let $(X, d_X)$ and $(Y, d_Y)$ be metric spaces. A function $f : X \to Y$ is a \key{Lipschitz function} if there exists a constant $K > 0$ such that
	\[
		d_Y(f(x), f(y)) \leq Kd_X(x, y)
	\]
	for all $x, y \in X$. We say that $f$ is a Lipschitz function with \key{Lipschitz constant} $K$.~\cite[p154]{searcoid:metric-spaces}
%	
%	More generally, we say that $f$ is \key{H\"older continuous} if there exists some constants $K > 0$ and $\alpha \in (0, 1]$ such that
%	\[
%	d_Y(f(x), f(y)) \leq K(d_X(x, y))^\alpha
%	\]
%	for all $x, y \in X$. In this case, we say that $f$ is H\"older continuous with \key{H\"older exponent} $\alpha$ and \key{H\"older constant} $K$.~\cite[p143]{brin-stuck:dynsys}
\end{definition}

\begin{definition}
	A transformation $T : (X_1, \B_1, \mu_1) \to (X_2, \B_2, \mu_2)$ is \key{measure-preserving} if:
	\begin{enumerate}
		\item $T$ is measurable, i.e. if $B_2 \in \B_2$, then $T^{-1}{B_2} \in \B_1$, and
		\item $\mu_1(T^{-1}{B_2}) = \mu_2(B_2)$ for all $B_2 \in \B_2$.
	\end{enumerate}
	This agrees with our usual definition when $X_1 = X_2$.
\end{definition}

\begin{definition}
	Let $X$ be a compact metric space with Borel $\sigma$-algebra $\B$. We let $M(X)$ denote the set of all probability measures on $(X, \B)$.
	
	Let $T : X \to X$ be a continuous mapping on $X$. We let $M(X, T)$ denote the set of $T$-invariant probability measures on $(X, B)$.
\end{definition}

\begin{definition}
	The \key{symmetric difference} of two sets $A, B$ is defined $(A \setminus B) \cup (B \setminus A)$. We will write this as
	\begin{equation*}
		A \symdiff B := (A \setminus B) \cup (B \setminus A).
	\end{equation*}
\end{definition}

\section{Shifts of finite type}
\emph{A large portion of this section follows \cite[Chapter 1]{parry-pollicott:zeta-fns-periodic-orbits}.}
\subsection{The basics}
Let $A$ be a $k \times k$ matrix with entries in $\{0, 1\}$. A \key{(two-sided) shift of finite type} $\Sigma_A$ is defined by
\[
	\Sigma_A = \{(x_j)_{j = -\infty}^\infty \mid A_{x_j, x_{j + 1}} = 1,\ j \in \integers\}.
\]
Similarly, a \key{(one-sided) shift of finite type} $\Sigma_A^+$ is defined by
\[
	\Sigma_A^+ = \{(x_j)_{j = 0}^\infty \mid A_{x_j, x_{j + 1}} = 1,\ j \in \naturals_0\}.
\]

Let $x = (x_j)_{j = -\infty}^\infty \in \Sigma_A$. We define the \key{(two-sided, left) shift map} $\sigma : \Sigma_A \to \Sigma_A$ by
\[
	(\sigma(x))_j = x_{j + 1}.
\]
which shifts each coordinate of $x$ one position to the left.

Now let $x = (x_j)_{j = 0}^\infty \in \Sigma_A^+$. We similarly define the \key{(one-sided, left) shift map} $\sigma^+ : \Sigma_A^+ \to \Sigma_A^+$ by
\[
	(\sigma^+(x))_j = x_{j + 1}.
\]
As with the one-sided case, this shifts the coordinates of $x$ one position to the left but also deletes the first coordinate $x_0$. It is clear that $\sigma^+$ is not invertible, whereas $\sigma$ is invertible.

To avoid excessive use of subscripts and superscripts, we will often write $\sigma$ for both the one-sided and two-sided shift maps. It should be clear from the context which of these maps $\sigma$ denotes.

If $x = (x_j)_{j = -\infty}^\infty \in \Sigma_A$, then we call $(x_j)_{j = -\infty}^0$ the \key{past}, $x_0$ the \key{present}, and $(x_j)_{j = 0}^\infty$ the \key{future}.

\subsubsection{Irreducibility, aperiodicity and cylinders}
Let $A$ is a $k \times k$ matrix with entries in $\{0, 1\}$, and let $\Sigma_A$ (or $\Sigma_A^+$) be the associated shift of finite type. We may consider $A$ to be the adjacency matrix of a directed graph $G_A$ with $k$ vertices.

We say that $A$ is \key{irreducible} if, for each $i, j \in \{1, \dots, k\}$, there exists $n = n(i, j) > 0$ such that $(A^n)_{i, j} > 0$. Alternatively, $A$ is irreducible if there exists an edge-path between any two vertices in the corresponding graph $G_A$. In this case, we say that the shift of finite type $\Sigma_A$ (or $\Sigma_A^+$) is irreducible.

If there exists $n > 0$ such that $(A^n)_{i, j} > 0$ for all $i, j \in \{1, \dots, k\}$, then we say that $A$ is \key{aperiodic}. That is, $A$ is aperiodic if all edge-paths between any two vertices in $G_A$ can be chosen to be of the same length. As before, this means that $\Sigma_A$ (or $\Sigma_A^+$) is aperiodic.

A \key{cylinder} $C$ on $\Sigma_A$ is defined
\[
	C = [i_{-m}, \dots, i_{-1}, i_0, i_1, \dots, i_n]_{-m, n} = \{(x_j)_{j = -\infty}^\infty \in \Sigma \mid x_j = i_j \text{ for } -m \leq j \leq n\}.
\]
Similarly, a cylinder $C^+$ on $\Sigma_A^+$ is given by
\[
	C^+ = [i_0, i_1, \dots, i_{n - 1}, i_n]_{0, n} = \{(x_j)_{j = 0}^\infty \in \Sigma \mid x_j = i_j \text{ for } 0 \leq j \leq n\}.
\]
In other words, a cylinder is the set of all sequences which agree in the given positions.

\subsection{Function spaces for shifts of finite type}
Let $\Sigma_A$ and $\Sigma_A^+$ be two-sided and one-sided shifts of finite type, respectively, and let $\theta \in (0, 1)$ be fixed.

\subsubsection{Metrics for shifts of finite type}
Let $x = (x_j)_{j = -\infty}^\infty, y = (y_j)_{j = -\infty}^\infty \in \Sigma_A$. We define $n = n(x, y) \geq 0$ to be the largest integer such that $x_j = y_j$ for all $|j| < n$, but $x_n \neq y_n$ or $x_{-n} \neq y_{-n}$. If $x_j = y_j$ for all $j \in \integers$ then we define $n = +\infty$.

We define the map $d_\theta : \Sigma_A \times \Sigma_A \to \reals^+$ by
\[
	d_\theta(x, y) =
	\begin{cases}
		\theta^n, & \text{if } x \neq y; \\
		0 & \text{if } x = y.
	\end{cases}
\]
It can be shown that $d_\theta$ is a \key{metric} on $\Sigma_A$, so sequences in $\Sigma_A$ are `close' if they agree for a large number of leading coordinates.

Similarly, if $x = (x_j)_{j = 0}^\infty, y = (y_j)_{j = 0}^\infty \in \Sigma_A^+$, then $n = n(x, y) \geq 0$ is defined to be the largest integer such that $x_j = y_j$ for all $0 \leq j < n$ but $x_n \neq y_n$. We define the map $d_\theta : \Sigma_A^+ \times \Sigma_A^+ \to \reals^+$ in the same way as the two-sided case, and it can be shown that $d_\theta$ is also a metric.

\subsubsection{The space of Lipschitz functions}
\begin{definition}
	Let $f : \Sigma_A \to \complex$ be a continuous function and let $n \geq 0$. We define the \key{$n$-th variation of $f$} by
	\[
		\var_n(f) = \sup\{|f(x) - f(y)| \mid x, y \in \Sigma_A,\ x_j = y_j \text{ for } |j| < n\}.
	\]
	
	Similarly, if $g : \Sigma_A^+ \to \complex$ is a continuous function, then the \key{$n$-th variation of $g$} is given by
	\[
	\var_n(g) = \sup\{|g(x) - g(y)| \mid x, y \in \Sigma_A^+,\ x_j = y_j \text{ for } 0 \leq j < n\}.
	\]
	That is, $\var_n(g)$ indicates how much $g$ varies on cylinders of length $n$.~\cite[Lecture 8]{magic-ergodic}
\end{definition}

It is easy to see that $\var_n(f) \leq K\theta^n$ for all $n \geq 0$ if and only if $|f(x) - f(y)| \leq Kd_\theta(x, y)$, i.e. $f$ is a Lipschitz function. This allows the following definition to be given in terms of $n$-th variations of continuous functions.

\begin{definition}
	Define
	\[
		F_\theta = F_\theta(\Sigma_A) = \{f \in C(\Sigma_A, \complex) \mid \var_n(f) \leq K\theta^n \text{ for all } n \geq 0, \text{ for some } K > 0\}
	\]
	to be the space of Lipschitz functions with respect to the metric $d_\theta$.
	
	We define $F_\theta^+ = F_\theta^+(\Sigma_A^+)$ in the same way, replacing $\Sigma_A$ with $\Sigma_A^+$.
\end{definition}

\begin{definition}
	Let $f \in F_\theta$ (or $F_\theta^+)$. Define
	\[
		|f|_\theta = \sup_{n \geq 0}\left\{\frac{\var_n(f)}{\theta^n}\right\}
	\]
	to be the least Lipschitz constant of $f$. In other words, $|f|_\theta$ is the smallest $K > 0$ such that $\var_n(f) \leq K\theta^n$ for all $n \geq 0$.
	
	We can then define a \key{norm} on $F_\theta$ (or $F_\theta^+$) by
	\[
		\|f\|_\theta = |f|_\infty+ |f|_\theta,
	\]
	where $|f|_\infty = \sup_{x \in \Sigma}\{|f(x)|\}$.
\end{definition}

\begin{proposition}
	The spaces $(F_\theta, \|\cdot\|_\theta)$ and $(F_\theta^+, \|\cdot\|_\theta)$ are Banach spaces.
\end{proposition}

\begin{definition}
	Let $f, g \in F_\theta$ (or $F_\theta^+$). We say $f$ and $g$ are \key{cohomologous} if there exists a continuous function $h$ such that $f = g + h \circ \sigma - h$. In this case, we write $f \sim g$.
	
	If $f$ is cohomologous to $0$, then we say $f$ is a \key{coboundary}.
\end{definition}

\begin{remark}
	It is clear that $\sim$ is an equivalence relation.
\end{remark}

\begin{proposition} \label{prop:pp-1-2}
	Suppose that $f \in F_\theta$. Then there exists $g, h \in F_{\theta^{1 / 2}}$ such that $f = g + h - h \circ \sigma$, where $g(x) = g(y)$ if $x_j = y_j$ for all $j \geq 0$. In other words, the value of $g(x)$ is determined only by the future coordinates of $x$.
	\begin{proof}
		For each $j = 1, \dots, k$ we choose a sequence $(a_n^{(j)})_{n = -\infty}^0$ from the past such that $a_0^{(j)} = j$. Now we define a function $\phi : \Sigma_A \to \Sigma_A : x \mapsto x'$, where
		\[
			(x')_n
			\begin{cases}
				x_n, & \text{if } n \geq 0; \\
				a_n^{(x_0)}, & \text{if } n \leq 0.
			\end{cases}
		\]
		So $\phi$ fixes the future coordinates of $x$, but replaces the past with $(a_n^{(x_0)})_{n = -\infty}^0$.
		
		Let $h(x) = \sum_{n = 0}^\infty{(f(\sigma^n{x}) - f(\sigma^n \phi{x}))}$. Note that $h$ converges because for all $n \geq 0$,
		\[
			|f(\sigma^n{x}) - f(\sigma^n \phi{x})| \leq \var_n(f) \leq |f|_\theta \theta^n,
		\]
		We have
		\begin{align*}
			h(x) - h(\sigma{x}) &= \sum_{n = 0}^\infty{(f(\sigma^n{x}) - f(\sigma^n \phi{x}))} - \sum_{n = 0}^\infty{(f(\sigma^{n + 1}{x}) - f(\sigma^n \phi \sigma{x}))} \\
				&= f(x) - \left(f(\phi{x}) + \sum_{n = 0}^\infty{(f(\sigma^{n + 1} \phi{x}) - f(\sigma^n \phi \sigma{x}))}\right) \\
				&= f(x) - g(x),
		\end{align*}
		where $g(x) := f(\phi{x}) + \sum_{n = 0}^\infty{(f(\sigma^{n + 1} \phi{x}) - f(\sigma^n \phi \sigma{x}))}$. Since all terms in $g$ contain $\phi$, we see that $g$ depends only on its future coordinates.
		
		It remains to show that $h \in F_{\theta^{1 / 2}}$ and it we will immediately have $g \in F_{\theta^{1 / 2}}$. It is sufficient to show that $\var_{2N}(h) \leq K\theta^N$ for all $N \geq 0$, for some constant $K > 0$, because this gives that
		\[
			\var_{2N + 1}(h) \leq K\theta^N = \frac{K}{\theta^{1 / 2}}(\theta^{1 / 2})^{2N + 1}.
		\]
		
		Let $x, y \in \Sigma_A$ be such that $x_j = y_j$ for $|j| \leq 2N$. Then for all $n = 0, \dots, N$, we have
		\[
			|f(\sigma^n{x}) - f(\sigma^n{y})| \leq |f|_\theta \theta^{2N - n} \quad \text{and} \quad |f(\sigma^n \phi{x}) - f(\sigma^n \phi{y})| \leq |f|_\theta \theta^{2N - n}.
		\]
		By definition, for all $n \geq 0$ we have
		\[
			|f(\sigma^n{x}) - f(\sigma^n \phi{x})| \leq |f|_\theta \theta^n \quad \text{and} \quad |f(\sigma^n {y}) - f(\sigma^n \phi{y})| \leq |f|_\theta \theta^n.
		\]
		Hence
		\begin{align*}
			|h(x) - h(y)| &= \left|\sum_{n = 0}^\infty{(f(\sigma^n{x}) - f(\sigma^n \phi{x}) - f(\sigma^n{y}) + f(\sigma^n \phi{y}))}\right| \\
				&\leq \sum_{n = 0}^N{|f(\sigma^n{x}) - f(\sigma^n{y})| + | f(\sigma^n \phi{x}) -  f(\sigma^n \phi{y})|} \\
				&\quad + \sum_{n = N + 1}^\infty{|f(\sigma^n{x}) - f(\sigma^n \phi{x})| + | f(\sigma^n \phi{y}) -  f(\sigma^n {y})|} \\
				&\leq 2|f|_\theta \sum_{n = 0}^N{\theta^{2N - n}} + 2|f|_\theta \sum_{n + N + 1}^\infty{\theta^n} \\
				&= 2|f|_theta \theta^{2N} \left(\frac{\theta^{-N - 1} - 1}{\theta^{-1} - 1}\right) + |f|_\theta \frac{\theta^{N + 1}}{1 - \theta} \\
				&\leq 4|f|_\theta \frac{\theta^N}{1 - \theta}.
		\end{align*}
		Therefore
		\[
			\var_{2N}(h) \leq \left(\frac{4|f|_\theta}{1 - \theta}\right)\theta^N
		\]
		and so $h \in F_{\theta^{1 / 2}}$.
	\end{proof}
\end{proposition}

The above result allows us to write $f \in F_\theta$ using $g \in F_{\theta^{1 / 2}}^+$. This allows us to apply results to $g$ which hold for $F_\theta^+$ but not necessarily for $F_\theta$. We will encounter such uses for this proposition in due course.

\begin{comment}
If a function $f : \Sigma \to \complex$ is $\alpha$-H\"older continuous, $\alpha \in (0, 1]$, then it is a Lipschitz function with respect to $d_{\theta^\alpha}$.

Suppose we have $0 < \theta < \theta' < 1$. Then
\[
	F_{\theta'}(\Sigma) \supset F_\theta(\Sigma) \quad \text{and} \quad F_{\theta'}^+(\Sigma^+) \supset F_\theta^+(\Sigma^+).
\]
We can therefore define
\[
	F = \bigcup_{0 < \theta < 1}{F_\theta(\Sigma)} \quad \text{and} \quad F^+ = \bigcup_{0 < \theta < 1}{F_\theta^+(\Sigma^+)},
\]
the spaces of all H\"older continuous functions.


We now consider a class of functions lying in $F_\theta^+$ for all $0 < \theta < 1$. For all $m \geq 1$ we define
\[
	F_m^+ = \{f : \Sigma^+ \to \complex \mid f(x) = f(y) \text{ if } x_j = y_j, \text{ for all } 0 \leq j < m\},
\]
the set of all locally constant functions which depend on the first $m$ terms of $x \in \Sigma^+$. It is clear that
\[
	F_1^+ \subset F_2^+ \subset F_3^+ \subset \dots.
\]
and, for $f \in F_m^+$, $\var_m(f) = 0$. Hence
\[
	\bigcup_{m = 1}^\infty{F_m^+} \subset \bigcap_{0 < \theta < 1}{F_\theta^+}.
\]

\begin{proposition}
	Suppose $0 < \theta < \theta' < 1$. Then for all $m \geq 0$ we have
	\begin{equation*}
		|f - f_m|_{\theta'} \leq |f|_\theta \left(\frac{\theta}{\theta'}\right)^m.
	\end{equation*}
\end{proposition}
\end{comment}

\section{The Ruelle operator}
Throughout this section, let $\Sigma = \Sigma_A^+$ be a (one-sided) shift of finite type.

\begin{definition}
	Let $f \in F_\theta^+$. The \key{Ruelle operator} (or \key{transfer operator}) $L_f : F_\theta^+ \to F_\theta^+$ (or, more generally, $L_f : C(\Sigma, \complex) \to C(\Sigma, \complex)$) is defined
	\[
		(L_f{w})(x) = \sum_{y \in \Sigma \midcolon \sigma{y} = x}{e^{f(y)} w(y)} = \sum_{j \midcolon A_{j, x_0} = 1}{e^{f(j, x_0, x_1, \dots)} w(j, x_0, x_1, \dots)},
	\]
	where $x = (x_j)_{j = 0}^\infty \in \Sigma$. This is a bounded linear operator.
	
	The $n$-th iterate of $L_f$ is given by
	\[
		(L_f^n{w})(x) = \sum_{y \in \Sigma \midcolon \sigma^n{y} = x}{e^{f^n(y)} w(y)}.
	\]
	
	If $f$ is also real-valued and $L_f{1} = 1$, then we say that $f$ or $L_f$ is \key{normalised}.
\end{definition}

\begin{proposition}
	Let $f \in F_\theta^+$ with $f = u + iv$, where $u, v \in F_\theta^+$ are real-valued functions. If $L_u$ is normalised, i.e. $L_u{1} = 1$, then for all $n \geq 0$,
	\[
		|L_f^n{w}|_\theta \leq K|w|_\infty + \theta^n |w|_\theta
	\]
	for all $w \in F_\theta^+$, where $K > 0$ is a constant depending only on $f$ and $\theta$.
\end{proposition}

\begin{theorem}[Ruelle's Perron-Frobenius Theorem] \label{thm:rpf}
	Suppose $\Sigma = \Sigma_A^+$ is an aperiodic shift of finite type and let $f \in F_\theta^+$ be a real-valued function. Then
	\begin{enumerate}
		\item There is a simple maximal eigenvalue $\lambda$ of $L_f : C(\Sigma, \reals) \to C(\Sigma, \reals)$ with a corresponding eigenfunction $h \in C(\Sigma_A^+, \reals)$, with $h > 0$. \label{rpf:1}
		\item The remainder of the spectrum of $L_f$ is contained in a disc of radius strictly less than $\lambda$. \label{rpf:2}
		\item There is a unique probability measure $\mu$ such that $L_f^*{\mu} = \lambda\mu$. That is,
		\[
			\int{L_f{v}\ d\mu} = \lambda \int{v\ d\mu},
		\]
		for all $v \in C(\Sigma, \reals)$. Additionally, if $h$ is the eigenfunction as in \ref{rpf:1} and $\int{h\ d\mu} = 1$, then the measure $\nu$ defined by $d\nu = h\ d\mu$ is a $\sigma$-invariant probability measure. \label{rpf:3}
		\item If $h$ is the eigenfunction as in \ref{rpf:1} and $\int{h\ d\mu} = 1$, then for all $v \in C(\Sigma, \reals)$,
		\[
			\frac{1}{\lambda^n}L_f^n{v} \to h \int{v\ d\mu}
		\]
		uniformly. \label{rpf:4}
	\end{enumerate}
\end{theorem}

\chapter{Entropy} \label{chap:entropy}
\section{Overview}
Entropy is an important property used to distinguish measure-preserving transformations from each other and is used extensively in ergodic theory. Chapter \ref{chap:concentration-bounds} looks at methods for estimating entropy and finding inequalities to describe how these `estimators' behave. This chapter focuses on defining entropy and explaining why it is a useful property.

Throughout this chapter $(X, \B, \mu)$ will denote a probability space.

\section{Isomorphisms of measure-preserving transformations}\label{sec:isos-of-mpts}
One of the main problems in ergodic theory is to classify measure-preserving transformations. To this end, we want to decide the conditions required for two measure-preserving transformations to be `the same' -- up to sets of measure zero.

\emph{This section predominantly follows material in \cite[Chapter 2]{walters:intro-to-ergodic-theory}.}

\subsection{Isomorphism and conjugacy of measure spaces}

We begin by defining when two probability spaces are isomorphic or conjugate.

\begin{definition}
	Two probability spaces $(X_1, \B_1, \mu_1), (X_2, \B_2, \mu_2)$ are \key{isomorphic} if there exists $M_1 \in \B_1$, $M_2 \in \B_2$ such that $\mu_1(M_1) = 1 = \mu_2(M_2)$ and if there exists an invertible measure-preserving transformation $\phi: M_1 \to M_2$.
\end{definition}

Let $A, C \subset \B$. We define an equivalence relation on $\B$: we have $A \sim C$ if and only if $\mu(A \symdiff C) = 0$. In other words, $A$ and $C$ belong to the same equivalence class if they are equal almost everywhere. It can be easily checked that $\sim$ is indeed an equivalence relation.

Let $\tilde{\B}$ denote the collection of all equivalence classes in $\B$. Since $\B$ is a $\sigma$-algebra, it is clear that $\tilde{\B}$ is also a $\sigma$-algebra. We can define a measure $\tilde{\mu} : \tilde{\B} \to \reals^+$ by $\tilde{\mu}(\tilde{B}) = \mu(B)$, where $B$ belongs to the equivalence class $\tilde{B}$.

\begin{definition}
	A \key{measure algebra} is a Boolean $\sigma$-algebra equipped with a measure.
\end{definition}

In view of this definition, we see that $(\tilde{\B}, \tilde{\mu})$ is a \key{measure algebra}.

\begin{definition}
	Let $(X_1, \B_1, \mu_1), (X_2, \B_2, \mu_2)$ be probability spaces with corresponding measure algebras $(\tilde{\B}_1, \tilde{\mu}_1), (\tilde{\B}_2, \tilde{\mu}_2)$, respectively.
	
	We say $(\tilde{\B}_1, \tilde{\mu}_1)$ and $(\tilde{\B}_2, \tilde{\mu}_2)$ are \key{isomorphic} if there exists a bijection $\phi : \tilde{\B}_2 \to \tilde{\B}_1$ which preserves complementation and countable unions and intersections such that $\tilde{\mu}_1(\phi \tilde{B}) = \tilde{\mu}_2(\tilde{B})$ for all $\tilde{B} \in \tilde{\B}_2$.
	
	The probability spaces $(X_1, \B_1, \mu_1)$ and $(X_2, \B_2, \mu_2)$ are \key{conjugate} if their corresponding measure algebras are isomorphic.
\end{definition}

\begin{proposition}
	If two probability spaces are isomorphic, then they are also conjugate.
	\begin{proof}
		Suppose $(X_1, \B_1, \mu_1), (X_2, \B_2, \mu_2)$ are isomorphic probability spaces with corresponding measure algebras $(\tilde{\B}_1, \tilde{\mu}_1), (\tilde{\B}_2, \tilde{\mu}_2)$. By definition, this means there exists $M_1 \in \B_1$, $M_2 \in \B_2$ such that $\mu_1(M_1) = 1 = \mu_2(M_2)$ and there exists an invertible measure-preserving transformation $\phi: M_1 \to M_2$.
		
		Now we can define the map
		\[
			\psi : \tilde{\B}_2 \to \tilde{\B}_1 : \tilde{B} \mapsto (\phi^{-1}(M_2 \cap B))^\sim.
		\]
		This is clearly a bijection and, since $\phi$ is measure-preserving and $M_2 = X_2$ almost everywhere, we have
		\[
			\tilde{\mu}_1(\psi\tilde{B}) = \tilde{\mu}_1(\phi^{-1}(M_2 \cap B))^\sim = \tilde{\mu}_2(M_2 \cap B)^\sim = \tilde{\mu}_2(\tilde{B}),
		\]
		for all $\tilde{B} \in \tilde{\B}_2$. Therefore the measure algebras are isomorphic and hence the corresponding measure spaces are conjugate.
	\end{proof}
\end{proposition}

The converse statement is not necessarily true. Indeed, suppose we have the probability space $(X_1, \B_1, \mu_1)$ consisting of exactly one point, and another probability space $(X_2, \B_2, \mu_2)$ consisting of exactly two points, with $\B_2 = \{\emptyset, X_2\}$. It is easy to see that the measure algebras are isomorphic and hence the measure spaces are conjugate.

We need to choose $M_1 \in \B_1$, $M_2 \in \B_2$ such that $\mu_1(M_1) = 1 = \mu_2(M_2)$; the only possibility is $M_1 = X_1$ and $M_2 = X_2$. However there does not exist bijection between these two sets, so the probability spaces are \emph{isomorphic}.

\subsection{A motivational example}
We describe a scenario when two measure-preserving transformations could be considered `the same'. We follow the example in \cite[p58]{walters:intro-to-ergodic-theory}.

We first introduce a new probability space.

\begin{comment}
Let $Y = \{0, 1\}$ and let $(p_0, p_1)$ be a probability vector with no zero entries. Then $(Y, 2^Y, \nu)$ is a measure space, with measure $\nu$ defined by $\nu(y) = p_y$ for $y \in Y$. Now let $X = \{(x_j)_{j = 0}^\infty \mid x_j \in Y\}$, the space of infinite sequences with entries in $Y = \{0, 1\}$.
\end{comment}
\subsubsection{Bernoulli shifts}
Let $Y = \{0, 1, \dots, k - 1\}$ be a set of $k - 1$ symbols and let $p = (p_0, p_1, \dots, p_{k - 1})$ be a probability vector with no zero entries. Let $X = \{(x_j)_{j = 0}^\infty \mid x_j \in Y \text{ for all } j \geq 0\}$ be the space of infinite sequences with entries in $Y$. We may define a measure $\nu$ on cylinders of length $n$ by
\[
	\nu[x_0, x_1, \dots, x_{n - 1}] = p_{x_0} p_{x_1} \dots p_{x_{n - 1}}.
\]
Such measures are known as \key{Bernoulli measures}. Let $\sigma : X \to X$ be the one-sided, left shift map on $X$.

\begin{proposition}
	The measure $\nu$ is $\sigma$-invariant.
	\begin{proof}
		We have
		\begin{align*}
			\nu(\sigma^{-1}[x_1, \dots, x_n]) &= \nu\left(\bigsqcup_{j = 0}^{k - 1}{[j, x_1, \dots, x_n]}\right) \\
				&= \sum_{j = 0}^{k - 1}{\nu[j, x_1, \dots, x_n]} \\
				&= \sum_{j = 0}^{k - 1}{p_j p_{x_1} \dots p_{x_n}} \\
				&= p_{x_1} \dots p_{x_n} \\
				&= \nu[x_1, \dots, x_n].
		\end{align*}
		(We have used the fact that $\sum_{j = 0}^{k - 1}{p_j} = 1$ on the penultimate line.)
	\end{proof}
\end{proposition}

The shift map $\sigma : (X, \nu) \to (X, \nu)$ is called the one-sided $(p_0, p_1, \dots, p_{k - 1})$-shift.

We are now ready to present two measure-preserving transformations which we argue are `the same'.

\subsubsection{The \texorpdfstring{$\mathbf{\left(\frac{1}{2}, \frac{1}{2}\right)}$}{(1/2, 1/2)}-shift and the doubling map}
Let $T : ([0, 1), \B, \mu) \to ([0, 1), \B, \mu) : x \mapsto 2x \bmod 1$ be the doubling map, where $\B$ is the Borel $\sigma$-algebra on $[0, 1)$ and $\mu$ is Lebesgue measure.

Let $\sigma : (X, \C, \nu) \to (X, \C, \nu)$ be the $\left(\frac{1}{2}, \frac{1}{2}\right)$-shift, where
\[
	X := \{(x_j)_{j = 0}^\infty \mid x_j \in \{0, 1\} \text{ for all } j \geq 0\},
\]
$\C$ is the $\sigma$-algebra generated by all cylinders in $X$, and $\nu$ is the Bernoulli measure as described above with $p = \left(\frac{1}{2}, \frac{1}{2}\right)$.

Define the map $\phi : X \to [0, 1)$ by
\[
	\phi(x_0, x_1, \dots) = \sum_{j = 0}^\infty{\frac{x_j}{2^{j + 1}}} = \frac{x_0}{2^1} + \frac{x_1}{2^2} + \frac{x_2}{2^3} + \dots.
\]
It is easy to see that $\phi$ maps the binary expansion of a number to the actual number itself.

Let $E := \{(x_j)_{j = 0}^\infty \in X \mid (x_j)_{j = N}^\infty \text{ is constant for some } N \geq 0\}$ be the set of sequences in $X$ whose coordinates are eventually constant. Now, if the binary expansion of a number is \emph{not} eventually constant, then this binary expansion is unique. Therefore $\phi$ is \emph{injective} on $X \setminus E$. It is also clear that $\phi$ is \emph{surjective}, since every number in $[0, 1)$ has at least one binary expansion. In addition, it is easy to that $\phi \circ \sigma = T \circ \phi$.

We now show that $\phi$ is measure-preserving. A dyadic interval is an interval of the form $\left[\frac{a}{2^s}, \frac{a + 1}{2^s}\right] \subset [0, 1)$, where $s \in \naturals$. We can write
\[
	\frac{a}{2^s} = \sum_{j = 0}^{s - 1}{\frac{a_j}{2^j}} \quad \text{and} \quad \frac{a + 1}{2^s} = \sum_{j = 0}^\infty{\frac{a_j}{2^j}},
\]
where $a_j \in \{0, 1\}$ for $j = 0, 1, \dots, s - 2$ and $a_k = 1$ for $k \geq s - 1$. In other words, the binary expansion of all numbers in the interval $\left[\frac{a}{2^s}, \frac{a + 1}{2^s}\right]$ agree in the first $s$ positions. Thus,
\begin{align*}
	\nu\left(\phi^{-1}\left[\frac{a}{2^s}, \frac{a + 1}{2^s}\right]\right) &= \nu[a_0, a_1, \dots, a_{s - 1}] \\
		&= \frac{1}{2^s} \\
		&= \mu\left[\frac{a}{2^s}, \frac{a + 1}{2^s}\right].
\end{align*}
Hence $\phi$ is measure-preserving on dyadic intervals, which generate the Borel $\sigma$-algebra $\B$ on $[0, 1)$. We may therefore apply the Kolmogorov Extension Theorem and it follows that $\phi$ is \emph{measure-preserving} on all Borel sets $B \in \B$.

Let $D := \left\{\frac{a}{2^s} \in [0, 1) \mid s \in \naturals,\ 0 \leq a < 2^s\right\}$ be the set of all dyadic rationals in $[0, 1)$. Clearly, $T^{-1}D = D$ and this means that $T^{-1}([0, 1) \setminus D) = [0, 1) \setminus D$. It is also clear that $\sigma^{-1}E = E$ and so $\sigma^{-1}(X \setminus E) = X \setminus E$. So by the above observations, we see that $\phi: X \setminus E \to [0, 1) \setminus D$ is a bijection. It is also clear that $\phi \circ \sigma(x) = T \circ \phi(x)$ for all $x \in X \setminus E$.

Finally, we have $D \subset \rationals$ which gives $\mu(D) = 0$, and we also note that there are countably many sequences in $E$, thus $\nu(E) = 0$. Therefore $\phi$ is an invertible measure-preserving transformation between $X$ and $[0, 1)$ (modulo sets of measure zero), that is, the measure-preserving transformations are \emph{isomorphic}. Therefore it makes sense to say that these measure-preserving transformations are `the same'.

\subsection{\texorpdfstring{\sloppy Isomorphism and conjugacy of measure-preserving transformations}{Isomorphism and conjugacy of measure-preserving transformations}}
We now formalise the ideas illustrated in the above example.

\begin{definition}
	\sloppy Let $(X_1, \B_1, \mu_1, T_1)$, $(X_2, \B_2, \mu_2, T_2)$ be measure-preserving transformations of probability spaces. We say that $T_1$ is \key{isomorphic} to $T_2$ if there exists $M_1 \in \B_1$, $M_2 \in \B_2$ such that $\mu_1(M_1) = 1 = \mu_2(M_2)$ with
	\begin{enumerate}
		\item $T_1{M_1} \subset M_1$ and $T_2{M_2} \subset M_2$, and \label{mpt-iso-i}
		\item there exists an invertible measure-preserving transformation $\phi : M_1 \to M_2$ such that $\phi \circ T_1(x) = T_2 \circ \phi(x)$ for all $x \in M_1$. \label{mpt-iso-ii}
	\end{enumerate}
	If this is the case, we write $T_1 \simeq T_2$.
\end{definition}

Now suppose that $T_1 \simeq T_2$ with $M_1$, $M_2$ and $\phi : M_1 \to M_2$ as in the above definition. Then for $n \geq 1$ we clearly have $T_1^n{M_1} \subset M_1$ and $T_2^n{M_2} \subset M_2$, satisfying condition \ref{mpt-iso-i}. This in turn gives that $\phi \circ T_1^n(x) = T_2^n \circ \phi(x)$ for all $x \in M_1$, satisfying condition \ref{mpt-iso-ii}. In other words, if $T_1 \simeq T_2$, then $T_1^n \simeq T_2^n$ for all $n \geq 1$.

We also have the notion of conjugacy of measure-preserving transformations.

\begin{definition}
	Let $(X_1, \B_1, \mu_1, T_1)$, $(X_2, \B_2, \mu_2, T_2)$ be measure-preserving transformations of probability spaces. We say that $T_1$ is \key{conjugate} to $T_2$ if there exists an isomorphism $\Phi : (\tilde{\B}_2, \tilde{\mu}_2) \to (\tilde{\B}_1, \tilde{\mu}_1)$ of measure algebras such that $\Phi \circ \tilde{T}_2^{-1} = \tilde{T}_1^{-1} \circ \Phi$.
\end{definition}

It can be easily checked that isomorphism and conjugacy are equivalence relations on the set of all measure-preserving transformations.

As with probability spaces, isomorphic measure-preserving transformations are also conjugate. We show this in the following result.

\begin{theorem}\label{thm:walters-2-5}
	Let $(X_1, \B_1, \mu_1, T_1)$, $(X_2, \B_2, \mu_2, T_2)$ be measure-preserving transformations of probability spaces and suppose that $T_1 \simeq T_2$. Then $T_1$ is conjugate to $T_2$.
	
	\begin{proof}
		Suppose that $T_1 \simeq T_2$, so there exists a measure-preserving transformation $\phi : M_1 \to M_2$ such that $\phi \circ T_1(x) = T_2 \circ \phi(x)$ for all $x \in M_1$, where $M_1, M_2$ are as in the definition.
		
		Define $\Phi : (\tilde{\B}_2, \tilde{\mu}_2) \to (\tilde{\B}_1, \tilde{\mu}_1)$ by $\Phi(\tilde{B}) \mapsto (\phi^{-1}(B \cap M_2))^\sim$ for $B \in B_2$. Recall that $\tilde{B}$ is an equivalence class, so it is easy to see that $\Phi$ is an isomorphism. We also have
		\[
			\tilde{T}_1^{-1} \circ \Phi(\tilde{B}) = \tilde{T}_1^{-1} \circ (\phi^{-1}(B \cap M_2))^\sim = \phi^{-1} \circ \tilde{T}_2^{-1} (B \cap M_2)^\sim = \Phi \circ \tilde{T}_2^{-1}(B)
		\]
		for all $B \in \B_2$. Hence $T_1$ is conjugate to $T_2$.
	\end{proof}
\end{theorem}

The converse of this theorem is not necessarily true. However, we will find it useful to know the conditions for which the converse holds. We need the following definition from \cite[Definition A.21]{einsiedler-ward:ergodic-nt}.

\begin{definition}
	Let $Y$ be a set of countably or finitely many points, where each $y \in Y$ has positive measure $p_y > 0$ such that $\sum_{y \in Y}{p_y} \leq 1$. Put $s := 1 - \sum_{y \in Y}{p_y}$ and let $\mathcal{L}[0, s]$ denote the $\sigma$-algebra of Lebesgue measurable sets on the closed interval $[0, s]$. Let $\lambda_{[0, s]}$ denote Lebesgue measure on $[0, s]$.
	
	If the probability space $(X, \B, \mu)$ is isomorphic to the probability space
	\[
		\left([0, s] \sqcup Y,\ \mathcal{L}[0, s],\ \lambda_{[0, s]} + \sum_{y \in Y}{p_y \delta_y} \right),
	\]
	where $\delta_y$ is the Dirac measure at $y$, then we say that $(X, \B, \mu)$ is a \key{Lebesgue space}.
\end{definition}

We will also use the following result, which is proved in \cite[Theorem 12]{royden:real-analysis}.

\begin{lemma} \label{lem:walters-thm-2-2}
	For $j = 1, 2$, let $(X_j, \B(X_j), \mu_j)$ be complete separable metric spaces endowed with Borel $\sigma$-algebra $\B(X_j)$ and probability measure $\mu_j$. Suppose that $\Phi: \tilde{\B}(X_2) \to \tilde{\B}(X_1)$ is an isomorphism of measure algebras. Then there exists $M_1 \in \B(X_1)$, $M_2 \in \B(X_2)$ such that $\mu_1(M_1) = 1 = \mu_2(M_2)$, and an invertible measure-preserving transformation $\phi: M_1 \to M_2$ such that $\Phi(\tilde{B}) = (\phi^{-1}(B \cap M_2))^\sim$ for all $B \in \B(X_2)$.
	
	If $\psi$ is any other isomorphism $(X_1, \B(X_1), \mu_1)$ to $(X_2, \B(X_2), \mu_2)$ which induces $\Phi$, then $\mu_1\{x \in X_1 \mid \phi(x) \neq \psi(x)\} = 0$.
\end{lemma}

The following result gives the conditions for which the converse of \thref{thm:walters-2-5} is true.

\begin{theorem} \label{thm:walters-2-6}
	Suppose that either $(X_1, \B_1, \mu_1)$, $(X_2, \B_2, \mu_2)$ are Lebesgue spaces, or that $X_1, X_2$ are each complete separable metric spaces with corresponding Borel $\sigma$-algebras $\B_1, \B_2$. Suppose that $T_1 : X_1 \to X_1$, $T_2 : X_2 \to X_2$ are measure-preserving transformations and that $T_1$ is conjugate to $T_2$. Then $T_1 \simeq T_2$.
	\begin{proof}
		Suppose that $\Phi : (\tilde{B}_2, \tilde{\mu}_2) \to (\tilde{B}_1, \tilde{\mu}_1)$ is an isomorphism of measure algebras such that $\Phi \circ \tilde{T}_2^{-1} = \tilde{T}_1^{-1} \circ \Phi$. By \thref{lem:walters-thm-2-2} there exists sets $X'_1 \in \B_1$, $X'_2 \in \B_2$ such that $\mu_1(X'_1) = 1 = \mu(X'_2)$, and there exists an invertible measure-preserving transformation $\phi : X'_1 \to X'_2$ such that $\Phi(\tilde{B}) = (\phi^{-1}(B \cap X'_2))^\sim$ for all $B \in \B_2$. Then we have $\tilde{\phi}^{-1} \circ \tilde{T}_2^{-1} = \tilde{T}_1^{-1} \circ \tilde{\phi}^{-1}$, i.e. $T_2 \circ \phi = \phi \circ T_1$ almost everywhere.
		
		Now put
		\[
			A_1 := \{x \in X_1 \mid T_2 \circ \phi(x) = \phi \circ T_1(x)\} \quad \text{and} \quad M_1 := \bigcap_{n = 0}^\infty{T_1^{-n}{A_1}}.
		\]
		Then $\mu_1(M_1) = 1$ and $T_1^{-1}{M_1} \supset M_1$ which means that $M_1 \supset T_1 M_1$. We then define $M_2 := \phi M_1$ so that $T_2 M_2 \subset M_2$. Hence $T_1 \simeq T_2$.
	\end{proof}
\end{theorem}

As we mentioned briefly at the beginning of Section \ref{sec:isos-of-mpts}, we want to be able to decide when two measure-preserving transformations are `the same'. In view of the above discussion, `the same' can be replaced with `conjugate' or `isomorphic'. \key{Entropy} is one of the main conjugacy and isomorphism invariants studied in ergodic theory, and the remainder of this chapter will describe how the entropy of a measure-preserving transformation is defined.

The rest of this chapter predominantly follows \cite[Chapter 4]{walters:intro-to-ergodic-theory} unless otherwise stated. In particular, any definitions relating to \emph{information} is derived from \cite[p33-34]{parry-pollicott:zeta-fns-periodic-orbits}

\section{Entropy of partitions and sub-\texorpdfstring{$\sigma$}{sigma}-algebras}
\subsection{Partitions and sub-\texorpdfstring{$\sigma$}{sigma}-algebras}

We begin with a finite partition $\alpha = \{A_1, \dots, A_m\}$ of $(X, \B, \mu)$, i.e. the $A_j$ are pairwise disjoint and $X = \bigsqcup_{j = 1}^m{A_j}$. For clarity, we will denote partitions by the Greek letters, usually $\alpha, \beta$ or $\gamma$. Consider the collection of all elements of $\B$ such that their unions are elements of $\alpha$. Such a collection is a sub-$\sigma$-algebra of $\B$ and we will denote it by $\A(\alpha)$.

On the other hand, consider a finite sub-$\sigma$-algebra $\C = \{C_1, \dots, C_n\}$ of $\B$. We will use script uppercase letters to denote sub-$\sigma$-algebras, usually $\A, \C$ or $\D$. We can form a partition of $X$ by $\{B_1, \dots, B_n\}$, where $B_j = C_j$ or $X \setminus C_j$. We denote this partition by $\alpha(\C)$.

Note that if $\C$ is a sub-$\sigma$-algebra of $\B$ and $\gamma$ is a partition of $X$, then $\A(\alpha(\C)) = \C$ and $\alpha(\A(\gamma)) = \gamma$. This means that there is a one-to-one correspondence between finite partitions of $X$ and finite sub-$\sigma$-algebras of $\B$. Hence, in a lot of cases, we may use ``partition'' and ``sub-$\sigma$-algebra'' interchangeably.

If $T: X \to X$ is a measure-preserving transformation and $n \geq 0$, then $T^{-n}{\alpha}$ denotes the partition $\{T^{-n}{A_1}, \dots, T^{-n}{A_k}\}$.

\begin{remark}
	Let $\alpha = \{A_1, \dots, A_m\}$ be a finite partition of $(X, \B, \mu)$. Throughout this chapter, we may assume without loss of generality that $\mu(A_j) > 0$ for all $j = 1, \dots, m$. Indeed, we may index $\alpha$ so that
	\[
		\mu(A_j)
		\begin{cases}
			> 0, & \text{if } 1 \leq j \leq p; \\
			= 0, & \text{if } p + 1 \leq j \leq m.
		\end{cases}
	\]
	Then we may form a new partition $\alpha'$ consisting of the sets $A_1, \dots, A_{p - 1}$ and $\bigsqcup_{j = p}^m{A_j}$. Clearly, the disjoint union has the same measure as $A_p$ and so all the sets in $\alpha'$ have strictly positive measure.
	
	This argument can be easily modified for countable partitions.
\end{remark}

\begin{definition}
	Suppose that $\alpha, \gamma$ are finite partitions of $(X, \B, \mu)$. If each element of $\alpha$ can be written as the union of elements of $\gamma$, then we write \key{$\alpha \leq \gamma$}. In particular, we have $\alpha \leq \gamma$ if and only if $\A(\alpha) \subset \A(\gamma)$, and $\A \subset \C$ if and only if $\alpha(\A) \leq \alpha(\C)$.
\end{definition}

\begin{definition}
	Let $\alpha = \{A_1, \dots, A_m\}$, $\gamma = \{C_1, \dots, C_n\}$ be two finite partitions of a measure space $(X, \B, \mu)$. We define their \key{join} $\alpha \join \gamma$ as the partition
	\[
		\alpha \join \gamma := \{A_j \cap C_k \mid 1 \leq j \leq m, 1 \leq k \leq n\}.
	\]
	If $\A, \C$ are finite sub-$\sigma$-algebras of $\B$, then we define the join $\A \join \C$ in the same way. If this is the case, then $\A \join \C$ is actually the smallest sub-$\sigma$-algebra of $\B$ containing both $\A$ and $\C$.
	
	It is clear that $\A \join \C$ is comprised of the unions of sets of the form $A \cap C$, where $A \in \A, C \in \C$.
	
	We also have the relations $\alpha(\A \join \C) = \alpha(\A) \join \alpha(\C)$ and $\A(\alpha \join \gamma) = \A(\alpha) \join \A(\gamma)$.
\end{definition}

\begin{remark}
	If $T : X \to X$ is a measure-preserving transformation and $n \geq 0$, then $T^{-n}$ preserves set theoretic operations and so we have
	\begin{enumerate}
		\item $\alpha(T^{-n}{\A}) = T^{-n}{\alpha(\A)}$,
		\item $\A(T^{-n}{\alpha}) = T^{-n}{\A(\alpha)}$,
		\item $T^{-n}(\A \join \C) = T^{-n}{\A} \join T^{-n}{\C}$,
		\item $T^{-n}(\alpha \join \gamma) = T^{-n}{\alpha} \join T^{-n}{\gamma}$,
		\item if $\alpha \leq \gamma$, then $T^{-n}{\alpha} \leq T^{-n}{\gamma}$,
		\item if $\A \subset \C$, then $T^{-n}{\A} \subset T^{-n}{\C}$.
	\end{enumerate}
\end{remark}

\begin{definition}
	Let $\alpha, \gamma$ be two partitions of $(X, \B, \mu)$. We say that $\alpha$ and $\gamma$ are \key{independent} if $\mu(A \cap C) = \mu(A)\mu(C)$ for all $A \in \alpha$, $C \in \gamma$.
\end{definition}

\subsection{Motivation for information and entropy}
The following motivation for information and entropy follows that of \cite[Lecture 23]{ergodic-lectures}.

Suppose that we want to locate a point $x \in X$. To do this, we can partition the state space $X$ by the finite partition $\alpha = \{A_1, \dots, A_k\}$. We will later show that we may also consider countable partitions. If we find that $x \in A_j$, then we have received some \key{information}, and we think of $\mu(A_j)$ to be the probability that this happens.

We would like to define a function $I_\mu(\alpha) : X \to \reals^+$ such that $I_\mu(\alpha)(x)$ is the information received upon observing that $x \in A_j$. We want $I_\mu(\alpha)$ to only depend on $\mu(A_j)$, and in particular, we should receive more information if $\mu(A_j)$ is small, and we should receive less information if $\mu(A_j)$ is large. So we want $I_\mu(\alpha)$ to be of the form
\[
	I_\mu(\alpha)(x) = \sum_{A \in \alpha}{\chi_A(x)\phi(\mu(A))},
\]
where $\phi : [0, 1] \to \reals^+$ is some nonnegative function.

For two independent partitions $\alpha, \gamma$, the information gained from observing that $x \in A \cap C$, where $A \in \alpha, C \in \gamma$, should be equal to the information we gain from observing $x \in A$ in addition to observing $x \in C$. In view of this, we would require that $I_\mu(\alpha \join \gamma) = I_\mu(\alpha) + I_\mu(\gamma)$.

Combining the above requirements, we get that $\phi(\mu(A \cap C)) = \phi(\mu(A)\mu(C)) = \phi(\mu(A)) + \phi(\mu(C))$. For $\phi$ to be a continuous function, we see that $\phi(t)$ must be a multiple of $-\log{t}$. This gives rise to the following definitions.

\subsection{Information and entropy of partitions}
\begin{definition}
	Let $\alpha$ be a partition of $(X, \B, \mu)$. We define the \key{information} $I_\mu(\alpha) : X \to \reals^+$ of the partition $\alpha$ (or of the sub-$\sigma$-algebra $\A(\alpha)$) by
	\[
		I_\mu(\A(\alpha))(x) = I_\mu(\alpha)(x) := -\sum_{A \in \alpha}{\chi_A(x) \log{\mu(A)}}.
	\]
	We define the \key{entropy} $H_\mu(\alpha)$ of the partition $\alpha$ (or of the sub-$\sigma$-algebra $\A(\alpha)$) to be the average of the information, i.e.
	\begin{align*}
		H_\mu(\A(\alpha)) = H_\mu(\alpha) &:= \int{I_\mu(\alpha)\ d\mu} \\
			&= \int{-\sum_{A \in \alpha}{\chi_A \log{\mu(A)}}\ d\mu} \\
			&= -\sum_{A \in \alpha}{\mu(A) \log{\mu(A)}}.
	\end{align*}
	Whenever we use this definition and those derived from it, we will use the convention that $x \log x = 0$ if $x = 0$.
\end{definition}

\begin{remark}
	If $\alpha = \{X, \emptyset\}$, then we don't gain any information from performing observations on this partition, so $H(\alpha) = 0$. This can also be easily verified from the definition above.
\end{remark}

It is useful to know that, given a partition of $(X, \B, \mu)$ into $k$ sets, we can find an upper bound for the entropy of the partition.

\begin{proposition} \label{prop:walters-cor-4-2-1}
	Let $\alpha = \{A_1, \dots, A_k\}$ be a partition of $(X, \B, \mu)$ into $k$ sets. Then $H_\mu(\alpha) \leq \log{k}$.
	
	In particular, we have $H_\mu(\alpha) = \log{k}$ if and only if $\mu(A_j) = 1 / k$ for all $j = 1, \dots k$.
	
	\begin{proof}
		By \thref{thm:walters-4-2-xlogx-convex}, $x \log{x}$ is strictly convex. This means that for any partition $\alpha = \{A_1, \dots, A_k\}$ of $(X, \B, \mu)$ and for any $\{\lambda_j \in [0, 1] \mid j \in \{1, \dots, k\},\ \sum_{j = 1}^k{\lambda_j} = 1\}$, we have
		\[
			\left(\sum_{j = 1}^k{\lambda_j \mu(A_j)}\right) \log{\left(\sum_{j = 1}^k{\lambda_j \mu(A_j)}\right)} \leq \sum_{j = 1}^k{\lambda_j \mu(A_j) \log{\mu(A_j)}},
		\]
		with equality if and only if $\mu(A_1) = \mu(A_2) = \dots = \mu(A_k)$ whenever $\lambda_j \neq 0$ for all $j = 1, \dots, k$.
		
		Substituting in $\lambda_j = 1 / k$ for all $j = 1, \dots, k$ and rearranging, we get
		\[
			H_\mu(\alpha) = -\sum_{j = 1}^k{\mu(A_j) \log{\mu(A_j)}} \leq -\log{\frac{1}{k}} = \log{k},
		\]
		with equality if and only if $\mu(A_j) = 1 / k$ for all $j = 1, \dots, k$.
	\end{proof}
\end{proposition}

\section{Conditional entropy}
\subsection{Conditional expectation}
The definitions and results in this subsection follow those in \cite[p8-9]{walters:intro-to-ergodic-theory}.
\begin{definition}
	Suppose that $\mu$, $\nu$ are probability measures on a measurable space $(X, \B)$. If all sets $B \in \B$ with $\mu$-measure zero are also sets of $\nu$-measure zero, then we say that $\nu$ is \key{absolutely continuous} with respect to $\mu$. If this is the case, we write $\nu \ll \mu$.
	
	Stated alternatively, we have $\nu \ll \mu$ if, for all $B \in \B$ such that $\mu(B) = 0$, then $\nu(B) = 0$.
	
	Note that there may be more sets of $\nu$-measure zero. In the case where $\nu \ll \mu$ and $\mu \ll \nu$, we say that $\mu$ and $\nu$ are \key{equivalent}.
\end{definition}

\begin{theorem}[Radon-Nikodym Theorem] \label{thm:radon-nikodym}
	Suppose that $\mu, \nu$ are probability measures on a measurable space $(X, \B)$. Then $\nu \ll \mu$ if and only if there exists a nonnegative $\mu$-integrable function $f \in L^1(X, \B, \mu)$ where $f \geq 0$, $\int{f\ d\mu} = 1$, such that $\nu(B) = \int_B{f\ d\mu}$ for all $B \in \B$.
	
	Moreover, the function $f$ is unique almost everywhere, i.e. if there exists another function $g$ which satisfies the above properties, then $f = g$ $\mu$-almost everywhere.
\end{theorem}

The Radon-Nikodym Theorem allows us to define the conditional expectation operator.

\begin{definition}
	Let $(X, \B, \mu)$ be a measure space and let $\C$ be a sub-$\sigma$-algebra of $\B$. The \key{conditional expectation} operator $E_\mu(\seedot \mid \C) : L^1(X, \B, \mu) \to L^1(X, \C, \mu)$ is defined as follows.
	
	If $f \in L^1(X, \B, \mu)$ is a nonnegative real-valued integrable function, then
	\[
		\nu_f(C) = a^{-1}\int_C{f\ d\mu},
	\]
	for $C \in \C$, where $a = \int_X{f\ d\mu}$, defines a probability measure $\nu_f$ on $(X, \C)$ with $\nu_f \ll \mu$. By \thref{thm:radon-nikodym}, there exists a nonnegative function $E_\mu(f \mid \C) \in L^1(X, \C, \mu)$ such that $\int_C{E_\mu(f \mid \C)\ d\mu} = \int_C{f\ d\mu}$ for all $C \in \C$. Furthermore, $E_\mu(f \mid \C)$ is unique almost everywhere.
	
	If $f$ is a real-valued function, we consider the positive and negative parts of $f = f^+ - f^-$, where $f^+, f^- \geq 0$, and define $E_\mu(f \mid \C) := E_\mu(f^+ \mid \C) - E_\mu(f^- \mid \C)$.
	
	If $f$ is complex-valued, we take the real and imaginary parts of $f$ and define $E_\mu(f \mid \C)$ linearly as above.
\end{definition}

The conditional expectation operator $E_\mu(f \mid \C)$ is uniquely determined by the requirement that $E_\mu(f \mid \C)$ is $\C$-measurable, and also that
\[
	\int_C{f\ d\mu} = \int_C{E_\mu(f \mid \C)\ d\mu},
\]
for all $C \in \C$. With this in mind, we can think of $E_\mu(f \mid \C)$ as the best approximation of $f$ in the smaller space $\C$ of measurable functions.~\cite[Lecture 21]{ergodic-lectures}

\subsubsection{Properties of \texorpdfstring{$E_\mu(\seedot \mid \C)$}{the conditional expectation operator}}
\begin{enumerate}
	\item Conditional expectation $E_\mu(\seedot \mid \C)$ is a linear operator. \label{cond-exp:1}
	\item If $f \geq 0$, then $E_\mu(f \mid \C)$. \label{cond-exp:2}
	\item If $f \in L^1(X, \B, \mu)$ and $g$ is a $\C$-measurable bounded function, then $E_\mu(fg \mid \C) = gE_\mu(f \mid \C)$. \label{cond-exp:3}
	\item For $f \in L^1(X, \B, \mu)$, we have $\left|E_\mu(f \mid \C)\right| \leq E_\mu(|f| \mid \C)$. \label{cond-exp:4}
	\item If $\C_2 \subset \C_1$, then for $f \in L^1(X, \B, \mu)$, we have $E_\mu(E_\mu(f \mid \C_1) \mid \C_2) = E_\mu(f \mid \C_2)$. \label{cond-exp:5}
\end{enumerate}

If $f$ is an integrable function, then we can find $E_\mu(f \mid \C)$ using the following formula.

\begin{proposition}
	Let $\C$ be a finite or countable sub-$\sigma$-algebra of $\B$. Then
	\[
		E_\mu(f \mid \C)(x) = \sum_{C \in \gamma}{\int_{C}{f\ d\mu}\frac{\chi_{C}(x)}{\mu(C)}}.
	\]
	
	\begin{proof}
		We follow the proof given in \cite[Example 10.1.2]{bogachev:measure}.
		
		The summation is clearly an integrable function, and the $\C$-measurable functions are exactly the characteristic functions of $C \in \C$. Therefore the result is equivalent to
		\[
			\int{\chi_B(x) \cdot E_\mu(f \mid \C)(x)\ d\mu} = \int{\left(\chi_B(x) \cdot \sum_{C \in \gamma}{\int_{C}{f\ d\mu}\frac{\chi_{C}(x)}{\mu(C)}}\right)\ d\mu},
		\]
		for any $B \in \C$. This is clearly true since by definition,
		\[
			\int{\chi_B(x) \cdot E_\mu(f \mid \C)(x)\ d\mu} = \int_B{f\ d\mu},
		\]
		and
		\[
			\int{\left(\chi_B(x) \cdot \sum_{C \in \gamma}{\int_{C}{f\ d\mu}\frac{\chi_{C}(x)}{\mu(C)}}\right)\ d\mu} = \int{\left({\int_{B}{f\ d\mu}\frac{\chi_{B}(x)}{\mu(B)}}\right)\ d\mu} = \int_{B}{f\ d\mu}.
		\]
	\end{proof}
\end{proposition}

\begin{definition}
	Let $\C \subset \B$ be a sub-$\sigma$-algebra of a $\sigma$-algebra $\B$. The \key{conditional probability} of $B \in \B$ given $\C$ is defined
	\[
		\mu(B \mid \C) := E_\mu(\chi_B \mid \C).
	\]
\end{definition}

\subsection{Conditional information and entropy}
We can define the conditional information and entropy of a partition $\alpha$, given that we know the information gained from the partition $\gamma$. Conditional entropy will prove to be useful later when look at the entropy of measure-preserving transformations.

\begin{definition}
	Let $(X, \B, \mu, T)$ be a measure-preserving transformation on a probability space. Let $\A, \C$ be finite sub-$\sigma$-algebras, where $\alpha(\A) = \{A_1, \dots, A_p\}$, $\alpha(\C) = \{C_1, \dots, C_q\}$. The \key{conditional entropy} of $\alpha$ given $\C$ is defined
	\begin{align*}
		H_\mu(\alpha(\A) \mid \alpha(\C)) = H_\mu(\A \mid \C) &:= -\sum_{k = 1}^q{\mu(C_k) \sum_{j = 1}^{p}{\frac{\mu(A_j \cap C_k)}{\mu(C_k)} \log{\frac{\mu(A_j \cap C_k)}{\mu(C_k)}}}} \\
			&= -\sum_{j, k}{\mu(A_j \cap C_k) \log{\frac{\mu(A_j \cap C_k)}{\mu(C_k)}}}.
	\end{align*}
\end{definition}

\begin{remark}
	If $\mathcal{N} = \{X, \emptyset\}$, then we have $H_\mu(\alpha \mid \mathcal{N}) = H_\mu(\alpha)$. Again, this is because we gain no information from $\mathcal{N}$.
\end{remark}

\begin{theorem} \label{thm:walters-4.3}
	Let $(X, \B, \mu)$ be a probability space and let $\A, \B, \D$ be finite sub-$\sigma$-algebras of $\B$. Suppose $T : X \to X$ is a measure-preserving transformation. Then
	\begin{enumerate}
		\item $H_\mu(\A \join \C \mid \D) = H_\mu(\A \mid \D) + H_\mu(\C \mid \A \join \D)$, \label{walters-thm-4.3:1}
		\item $H_\mu(\A \join \C) = H_\mu(\A) + H_\mu(\C \mid \A)$, \label{walters-thm-4.3:2}
		\item if $\A \subset \C$, then $H_\mu(\A \mid \D) \leq H_\mu(\C \mid \D)$, \label{walters-thm-4.3:3}
		\item if $\A \subset \C$, then $H_\mu(\A) \leq H_\mu(\C)$, \label{walters-thm-4.3:4}
		\item if $\C \subset \D$, then $H_\mu(\A \mid \C) \geq H_\mu(\A \mid \D)$, \label{walters-thm-4.3:5}
		\item $H_\mu(\A) \geq H_\mu(\A \mid \D)$, \label{walters-thm-4.3:6}
		\item $H_\mu(\A \join \C \mid \D) \leq H_\mu(\A \mid \D) + H_\mu(\C \mid \D)$, \label{walters-thm-4.3:7}
		\item $H_\mu(\A \join \C) \leq H_\mu(\A) + H_\mu(\C)$, \label{walters-thm-4.3:8}
		\item $H_\mu(T^{-1}\A \mid T^{-1}\C) = H_\mu(\A \mid \C)$, \label{walters-thm-4.3:9}
		\item $H_\mu(T^{-1}\A) = H_\mu(\A)$. \label{walters-thm-4.3:10}
	\end{enumerate}
	\begin{proof} \hfill
		\begin{enumerate}
			\item We have
				\[
					H_\mu(\A \join \C \mid \D) = -\sum_{j, k, m}{\mu(A_j \cap C_k \cap D_m) \log{\frac{\mu(A_j \cap C_k \cap D_m)}{\mu(D_m)}}}.
				\]
				If $\mu(A_j \cap D_m) \neq 0$, then
				\[
					\frac{\mu(A_j \cap C_k \cap D_m)}{\mu(D_m)} = \frac{\mu(A_j \cap C_k \cap D_m)}{\mu(A_j \cap D_m)} \frac{\mu(A_j \cap D_m)}{\mu(D_m)}.
				\]
				If $\mu(A_j \cap D_m) = 0$, then the above evaluates to zero anyway, so we ignore such terms. Therefore we have
				\begin{align*}
					H_\mu(\A \join \C \mid \D) &= -\sum_{j, k, m}{\mu(A_j \cap C_k \cap D_m) \log{\frac{\mu(A_j \cap D_m)}{\mu(D_m)}}} \\
						& \qquad - \sum_{j, k, m}{\mu(A_j \cap C_k \cap D_m) \log{\frac{\mu(A_j \cap C_k \cap D_m)}{\mu(A_j \cap D_m)}}} \\
						&= -\sum_{j, m}{\mu(A_j \cap D_m) \log{\frac{\mu(A_j \cap D_m)}{\mu(D_m)}}} + H_\mu(\C \mid \A \join \D) \\
						&= H_\mu(\A \mid \D) + H_\mu(\C \mid \A \join \D).
				\end{align*}
			\item We put $\D = \{X, \emptyset\}$ in \ref{walters-thm-4.3:1}. Then by the above remark the result follows immediately.
			\item Suppose that $\A \subset \C$. Then
				\begin{align*}
					H_\mu(\C \mid \D) &= H_\mu(\A \join \C \mid \D) & \text{(since } \A \subset \C) \\
						&= H_\mu(\A \mid \D) + H_\mu(\C \mid \A \join \D) & \text{(by \ref{walters-thm-4.3:1})} \\
						&\geq H_\mu(\A \mid \D).
				\end{align*}
			\item As with \ref{walters-thm-4.3:2}, we put $\D = \{X, \emptyset\}$ in \ref{walters-thm-4.3:3}.
			\item Suppose that $\C \subset \D$ and fix $j, k$. We have
				\[
					\sum_{m}{\frac{\mu(D_m \cap C_k)}{\mu(C_k)}} = 1
				\]
				and so we may apply Theorem \ref{thm:walters-4-2-xlogx-convex}. So with $f(x) = x\log{x}$, we have
				\begin{equation} \label{fml:walters-4-3-5-ineq}
					f\left(\sum_{m}{\frac{\mu(D_m \cap C_k)}{\mu(C_k)} \frac{\mu(A_j \cap D_m)}{\mu(D_m)}}\right) \leq \sum_{m}{\frac{\mu(D_m \cap C_k)}{\mu(C_k)} f\left(\frac{\mu(A_j \cap D_m)}{\mu(D_m)}\right)}.
				\end{equation}
				Since $\C \subset \D$, we have
				\begin{align*}
					f\left(\sum_{m}{\frac{\mu(D_m \cap C_k)}{\mu(C_k)} \frac{\mu(A_j \cap D_m)}{\mu(D_m)}}\right) &= f\left(\frac{\mu(A_j \cap C_k)}{\mu(C_k)}\right) \\
						&= \frac{\mu(A_j \cap C_k)}{\mu(C_k)} \log{\frac{\mu(A_j \cap C_k)}{\mu(C_k)}}.
				\end{align*}
				Then multiplying \eqref{fml:walters-4-3-5-ineq} by $\mu(C_k)$ and then summing over $j, k$, we get
				\begin{align*}
					-H_\mu(\A \mid \C) &= \sum_{j, k}{\mu(A_j \cap C_k) \log{\frac{\mu(A_j \cap C_k)}{\mu(C_k)}}} \\
						&\leq \sum_{j, k, m}{\mu(D_m \cap C_k) \frac{\mu(A_j \cap D_m)}{\mu(D_m)} \log{\frac{\mu(A_j \cap D_m)}{\mu(D_m)}}} \\
						&= \sum_{j, m}{\mu(A_j \cap D_m) \log{\frac{\mu(A_j \cap D_m)}{\mu(D_m)}}} \\
						&= -H_\mu(\A \mid \D).
				\end{align*}
				Hence $H_\mu(\A \mid \C) \geq H_\mu(\A \mid \D)$.
			\item We put $\C = \{X, \emptyset\}$ in \ref{walters-thm-4.3:5}.
			\item We have
				\begin{align*}
					H_\mu(\A \join \C \mid \D) &= H_\mu(\A \mid \D) + H_\mu(\C \mid \A \join \D) & \text{(by \ref{walters-thm-4.3:1})} \\
						&\leq H_\mu(\A \mid \D) + H_\mu(\C \mid \D) & \text{(by \ref{walters-thm-4.3:5})}.
				\end{align*}
			\item We put $\D = \{X, \emptyset\}$ in \ref{walters-thm-4.3:7}.
			\item This follows since $T$ is measure-preserving and by the definition conditional entropy.
			\item This is also immediate from the definitions.
		\end{enumerate}
	\end{proof}
\end{theorem}

Using conditional expectation, we can define conditional entropy for a finite sub-$\sigma$-algebra $\A$ of $\B$ given an arbitrary (not necessarily finite) sub-$\sigma$-algebra $\C$ of $\B$. We first suppose that $\C$ is finite so that $\alpha(\C) = \{C_1, \dots, C_q\}$, and also let $\alpha(\A) = \{A_1, \dots, A_p\}$. Note that
\[
		E_\mu(\chi_{A_j} \mid \C)(x) = \sum_{k = 1}^{q}{\int_{C_k}{\chi_{A_j}\ d\mu}\frac{\chi_{C_k}(x)}{\mu(C_k)}}.
\]

Then
\begin{align*}
	H_\mu(\alpha(\A) \mid \alpha(\C)) = H_\mu(\A \mid \C) &= -\sum_{j = 1}^{p}{\sum_{k = 1}^{q}{\mu(A_j \cap C_k) \log{\frac{\mu(A_j \cap C_k)}{\mu(C_k)}}}} \\
		&= -\sum_{j = 1}^{p}{\int{\chi_{A_j} \log{E_\mu(\chi_{A_j} \mid \C)}\ d\mu}} \\
		&= -\int{\sum_{j = 1}^{p}{E_\mu(\chi_{A_j} \mid \C) \log{E_\mu(\chi_{A_j} \mid \C)}}\ d\mu}.
\end{align*}

We can therefore make the following definition for countable sub-$\sigma$-algebras $\C$ of $\B$.

\begin{definition}
	Let $(X, \B, \mu)$ be a probability space. Suppose that $\A$ is a finite sub-$\sigma$-algebra of $\B$ and that $\C$ is an \emph{arbitrary} sub-$\sigma$-algebra of $\B$. Denote the partition $\alpha(\A) = \{A_1, \dots, A_p\}$. The \key{conditional entropy} of $\alpha$ given $\C$ is given by
	\[
		H_\mu(\alpha(\A) \mid \alpha(\C)) = H_\mu(\A \mid \C) := -\int{\sum_{j = 1}^{p}{\mu(A_j \mid \C) \log{\mu(A_j \mid \C)}}\ d\mu}.
	\]
\end{definition}

\begin{lemma} \label{lem:walters-4-6}
	Suppose that $\A_1 \subset \A_2 \subset \dots \subset \A_n \subset \dots$ is an increasing sequence of sub-$\sigma$-algebras of $\B$, and write $\A := \bigjoin_{n = 1}^\infty{\A_n}$. Then for all $f \in L^2(X, \B, \mu)$ we have $\|E_\mu(f \mid \A_n) - E_\mu(f \mid \A)\|_2 \to 0$, as $n \to \infty$.
	\begin{proof}
		By definition, the operator $E_\mu(\seedot \mid \A_n)$ maps functions from from $L^2(X, \B, \mu)$ to $L^2(X, \A_n, \mu)$. We let $A \in \A$ and choose a sequence $A_n \in \A_n$ such that $\mu(A_n \symdiff A) \to 0$, as $n \to +\infty$. (This is possible because $\A_n$ is an increasing sequence.)
		
		Since $E_\mu(\chi_A \mid \A_n)$ is a best approximation to $\chi_A$ in $L^2(X, \A_n, \mu)$, we have
		\[
			\|E_\mu(\chi_A \mid \A_n) - \chi_A\|_2^2 \leq \|\chi_{A_n} - \chi_A\|_2^2 = \mu(A_n \symdiff A) \to 0,
		\]
		as $n \to +\infty$.
		
		The set of all finite linear combinations of characteristic functions are dense in $L^2(X, \A, \mu)$ and so for all $g \in L^2(X, \A, \mu)$, we have
		\begin{equation} \label{fml:lem-4-6-star}
			\|E_\mu(g \mid \A_n) - g\|_2 \to 0,
		\end{equation}
		as $n \to +\infty$. Therefore if $f \in L^2(X, \B, \mu)$, then by property \ref{cond-exp:5} on page \pageref{cond-exp:5}, we have $E_\mu(E_\mu(f \mid \A) \mid \A_n) = E_\mu(f \mid \A_n)$ because $\A_n \subset \A$ for all $n \geq 1$. Hence by \eqref{fml:lem-4-6-star} we have
		\[
			\|E_\mu(f \mid \A_n) - E_\mu(f \mid \A)\|_2 = \|E_\mu(E_\mu(f \mid \A) \mid \A_n) - E_\mu(f \mid \A)\|_2 \to 0,
		\]
		as $n \to +\infty$, as required.
	\end{proof}
\end{lemma}

We also have the following theorem.

\begin{theorem}[Increasing Martingale Theorem] \label{thm:increasing-martingale}
	Suppose that $\A_1 \subset \A_2 \subset \dots \subset \A_n \subset \dots$ is an increasing sequence of sub-$\sigma$-algebras of $\B$ such that $\A_n \to \A$, as $n \to +\infty$. Then for all $f \in L^1(X, \B, \mu)$ we have
	\begin{enumerate}
		\item $E_\mu(f \mid \A_n) \to E_\mu(f \mid \A)$ $\mu$-almost everywhere, as $n \to +\infty$, and
		\item $E_\mu(f \mid \A_n) \to E_\mu(f \mid \A)$ in $L_1$, as $n \to +\infty$.
	\end{enumerate}
\end{theorem}

Note that \thref{thm:walters-4.3} was given in terms of \emph{finite} sub-$\sigma$-algebras and hence finite partitions. By the following theorem, we can in fact extend these results for \emph{countable} sub-$\sigma$-algebras and partitions.

\begin{theorem} \label{thm:walters-4-7}
	Suppose that $\A$ is a \emph{finite} sub-$\sigma$-algebra of $\B$. Furthermore, suppose that $\C_1 \subset \C_2 \subset \dots \subset \C_n \subset \dots$ is an increasing sequence of sub-$\sigma$-algebras of $\B$, and put $\C:= \bigvee_{n = 1}^\infty{\C_n}$. Then $H_\mu(\A \mid \C_n) \to H_\mu(\A \mid \C)$, as $n \to +\infty$.
	\begin{proof}
		Let $\alpha(\A) = \{A_1, \dots, \A_k\}$. By \thref{lem:walters-4-6}, $\|E_\mu(\chi_{A_j} \mid \C_n) - E_\mu(\chi_{A_j} \mid \C)\|_2 \to 0$, as $n \to +\infty$ for $j = 1, \dots, k$. So $E_\mu(\chi_{A_j} \mid \C_n)$ converges in measure to $E_\mu(\chi_{A_j} \mid \C)$, i.e. given $\varepsilon > 0$, we have that
		\[
			\lim_{n \to +\infty}{\mu\left\{x \in X \midmid \left|E_\mu(\chi_{A_j} \mid \C_n)(x) - E_\mu(\chi_{A_j} \mid \C)(x)\right| \geq \varepsilon\right\}} = 0.
		\]
		So it is clear that $-\sum_{j = 1}^k{E_\mu(\chi_{A_j} \mid \C_n) \log{E_\mu(\chi_{A_j} \mid \C_n)}}$ also converges in measure to $-\sum_{j = 1}^k{E_\mu(\chi_{A_j} \mid \C) \log{E_\mu(\chi_{A_j} \mid \C)}}$.
		
		Since $E_\mu(\seedot \mid \C)$ is a positive linear operator and since $\sum_{j = 1}^k{\chi_{A_j}} = 1$, we have $0 \leq E_\mu(\chi_{A_j} \mid \C)(x) \leq 1$ for $\mu$-almost every $x$. Hence
		\begin{align*}
			-\sum_{j = 1}^k{\mu(A_j \mid \C)(x) \log{\mu(A_j \mid \C)(x)}} &= -\sum_{j = 1}^k{E_\mu(\chi_{A_j} \mid \C)(x) \log{E_\mu(\chi_{A_j} \mid \C)(x)}} \\
				&\leq k \max_{t \in [0, 1]}(-t \log{t}) \\
				&= ke.
		\end{align*}
		So all functions of this form are bounded by $ke$ and hence converge in $L^1(\mu)$. Therefore, $H_\mu(\A \mid \C_n) \to H_\mu(\A \mid \C)$, as $n \to +\infty$.
	\end{proof}
\end{theorem}

As a result of this theorem, given a countable (not necessarily finite) sub-$\sigma$-algebra $\C$, we can find an increasing sequence $\C_1 \subset \C_2 \subset \dots \subset \C_n \subset \dots$ such that $\C_n \to \C$, as $n \to +\infty$. We then apply \thref{thm:walters-4-7} and we see that any result involving finite sub-$\sigma$-algebras can be extended for countable sub-$\sigma$-algebras.

\section{\texorpdfstring{\sloppy Entropy of measure-preserving transformations}{Entropy of measure-preserving transformations}}
So far, we have be focusing exclusively on the entropy of partitions and sub-$\sigma$-algebras, but we can now introduce a measure-preserving transformation $T : X \to X$. We can think of $T$ as the passing of a day in time, and so $H_\mu\left(\bigjoin_{j = 0}^{n - 1}{T^{-j}{\alpha}}\right)$ is the average information we gain after $n$ days. Given a partition $\alpha$, it is natural to define the entropy of $T$ by the average information we obtain \emph{per day}. First, we need to ensure that this is well-defined.

\begin{theorem} \label{thm:walters-4-9}
	Let $(a_n)_{n = 1}^\infty$ be a sequence of real numbers such that $a_{n + p} \leq a_n + a_p$ for all $n, p \geq 1$. Then $\lim_{n \to +\infty}(a_n / n)$ exists and equals $\inf_{n \geq 1}(a_n / n)$.
	
	This limit could be $-\infty$, but if $a_n$ is bounded below then, by the properties of the sequence, the limit is non-negative.
	\begin{proof}
		Fix $p \geq 1$. We can write $n = kp + j$ for some $0 \leq j < p$, and then
		\[
			\frac{a_n}{n} = \frac{a_{kp + j}}{kp + j} \leq \frac{a_j}{kp} + \frac{a_{kp}}{kp} \leq \frac{a_j}{kp} + \frac{ka_p}{kp} = \frac{a_j}{kp} + \frac{a_p}{p}.
		\]
		We have that $k \to +\infty$, as $n \to +\infty$, and so
		\[
			\frac{a_j}{kp} \to 0,
		\]
		as $n \to +\infty$. Putting the above results together, we have
		\[
			\limsup_{n \to +\infty}{\frac{a_n}{n}} \leq \frac{a_p}{p}.
		\]
		Since $p$ is fixed, we have
		\[
			\limsup_{n \to +\infty}{\frac{a_n}{n}} \leq \inf_{p \geq 1}{\frac{a_p}{p}}.
		\]
		On the other hand, it is clear that
		\[
			\inf_{p \geq 1}{\frac{a_p}{p}} \leq \liminf_{n \to +\infty}{\frac{a_n}{n}}.
		\]
		Therefore
		\[
			\lim_{n \to +\infty}{\frac{a_n}{n}}
		\]
		exists and is equal to the infimum.
	\end{proof}
\end{theorem}

\begin{corollary} \label{cor:walters-4-9-1}
	Let $T : X \to X$ be a measure-preserving transformation and suppose that $\A$ is a finite sub-$\sigma$-algebra of $\B$. Then
	\[
		\lim_{n \to +\infty}{\frac{1}{n} H_\mu\left(\bigjoin_{j = 0}^{n - 1}{T^{-j}{\A}}\right)}
	\]
	exists.
	\begin{proof}
		Let the sequence $(a_n)_{n = 1}^\infty$ be defined by $a_n = H_\mu\left(\bigjoin_{j = 0}^{n - 1}{T^{-j}{\A}}\right) \geq 0$. For any $n, p \geq 1$ we have
		\begin{align*}
			a_{n + p} &= H_\mu\left(\bigjoin_{j = 0}^{n + p - 1}{T^{-j}{\A}}\right) \\
				&\leq H_\mu\left(\bigjoin_{j = 0}^{n - 1}{T^{-j}{\A}}\right) + H_\mu\left(\bigjoin_{j = n}^{n + p - 1}{T^{-j}{\A}}\right) & \text{(by \thref{thm:walters-4.3} \ref{walters-thm-4.3:8})} \\
				&= a_n + H_\mu\left(\bigjoin_{j = 0}^{p - 1}{T^{-j}{\A}}\right) & \text{(by \thref{thm:walters-4.3} \ref{walters-thm-4.3:10})} \\
				&= a_n + a_p.
		\end{align*}
		By \thref{thm:walters-4-9}, the limit of $a_n$ exists and hence
		\[
			\lim_{n \to +\infty}{\frac{1}{n} H_\mu\left(\bigjoin_{j = 0}^{n - 1}{T^{-j}{\A}}\right)}
		\]
		exists.
	\end{proof}
\end{corollary}

This ensures that the following definition is well-defined.

\begin{definition}
	Let $(X, \B, \mu, T)$ be a measure-preserving transformation of a probability space and let $\alpha$ be a finite partition of $X$. The \key{entropy of $T$ with respect to $\alpha$} is defined
	\[
		h_\mu(T, \A(\alpha)) = h_\mu(T, \alpha) := \lim_{n \to +\infty}{\frac{1}{n} H_\mu\left(\bigjoin_{j = 0}^{n - 1}{}T^{-j}{\alpha}\right)}.
	\]
\end{definition}

\begin{theorem} \label{thm:walters-4.12}
	Let $\A$, $\C$ be finite sub-algebras of $\B$ and let $T$ be a measure-preserving transformation of a probability space $(X, \B, \mu)$ Then
	\begin{enumerate}
		\item $h_\mu(T, \A) \leq H_\mu(\A)$, \label{walters:thm-4-12:1}
		\item $h_\mu(T, \A \join \C) \leq h_\mu(T, \A) + h_\mu(T, \C)$, \label{walters:thm-4-12:2}
		\item if $\A \subset \C$ then $h_\mu(T, \A) \leq h_\mu(T, \C)$, \label{walters:thm-4-12:3}
		\item $h_\mu(T, \A) \leq h_\mu(T, \C) + H_\mu(\A \mid \C)$, \label{walters:thm-4-12:4}
		\item $h_\mu(T, T^{-1}{\A}) = h_\mu(T, \A)$, \label{walters:thm-4-12:5}
		\item if $m \geq 1$ then $h_\mu(T, \A) = h_\mu\left(T, \bigjoin\limits_{j = 0}^{m - 1}{T^{-j}{\A}}\right)$, \label{walters:thm-4-12:6}
		\item if $T$ is invertible and $m \geq 1$ then $h_\mu(T, \A) = h_\mu\left(T, \bigjoin\limits_{j = -m}^m{T^{-j}{\A}}\right)$. \label{walters:thm-4-12:7}
	\end{enumerate}
	\begin{proof} \hfill
		\begin{enumerate}
			\item For all $n \geq 1$,
				\begin{align*}
					\frac{1}{n} H_\mu\left(\bigjoin_{j = 0}^{n - 1}{T^{-j}{\A}}\right) &\leq \frac{1}{n} \sum_{j = 0}^{n - 1}{H_\mu({T^{-j}{\A}})} & \text{(by \thref{thm:walters-4.3} \ref{walters-thm-4.3:8})} \\
						&\leq \frac{1}{n} \sum_{j = 0}^{n - 1}{H_\mu(\A)} & \text{(by \thref{thm:walters-4.3} \ref{walters-thm-4.3:10})} \\
						&= H_\mu(\A).
				\end{align*}
			\item We have
				\begin{align*}
					H_\mu\left(\bigjoin_{j = 0}^{n - 1}{T^{-j}(\A \join \C)}\right) &= H_\mu\left(\bigjoin_{j = 0}^{n - 1}{T^{-j}{\A}} \join \bigjoin_{j = 0}^{n - 1}{T^{-j}{\C}}\right) \\
						&\leq H_\mu\left(\bigjoin_{j = 0}^{n - 1}{T^{-j}{\A}}\right) + H_\mu\left(\bigjoin_{j = 0}^{n - 1}{T^{-j}{\C}}\right)
				\end{align*}
				by \thref{thm:walters-4.3} \ref{walters-thm-4.3:8}.
			\item Since $T$ preserves set theoretic operations, if $\A \subset \C$, then for all $n \geq 1$ we have
				\[
					\bigjoin_{j = 0}^{n - 1}{T^{-j}{\A}} \subset \bigjoin_{j = 0}^{n - 1}{T^{-j}{\C}}.
				\]
				Applying \thref{thm:walters-4.3} \ref{walters-thm-4.3:4} we get
				\[
					H_\mu\left(\bigjoin_{j = 0}^{n - 1}{T^{-j}{\A}}\right) \leq H_\mu\left(\bigjoin_{j = 0}^{n - 1}{T^{-j}{\C}}\right).
				\]
			\item We have
				\begin{align*}
					H_\mu\left(\bigjoin_{j = 0}^{n - 1}{T^{-j}{\A}}\right) &\leq H_\mu\left(\bigjoin_{j = 0}^{n - 1}{T^{-j}{\A}} \join \bigjoin_{j = 0}^{n - 1}{T^{-j}{\C}}\right) \\ & \hspace{60mm} \text{(by \thref{thm:walters-4.3} \ref{walters-thm-4.3:4})} \\
						&= H_\mu\left(\bigjoin_{j = 0}^{n - 1}{T^{-j}{\C}}\right) + H_\mu\left(\bigjoin_{j = 0}^{n - 1}{T^{-j}{\A}} \midmid \bigjoin_{j = 0}^{n - 1}{T^{-j}{\C}}\right) \\ & \hspace{60mm} \text{(by \thref{thm:walters-4.3} \ref{walters-thm-4.3:2}).}
				\end{align*}
				By applying \thref{thm:walters-4.3} \ref{walters-thm-4.3:7} repeatedly, we have
				\begin{align*}
					H_\mu\left(\bigjoin_{j = 0}^{n - 1}{T^{-j}{\A}} \midmid \bigjoin_{j = 0}^{n - 1}{T^{-j}{\C}}\right) &\leq \sum_{j = 0}^{n - 1}{H_\mu\left(T^{-j}{\A} \midmid \bigjoin_{k = 0}^{n - 1}{T^{-k}{\C}}\right)} \\
						&\leq \sum_{j = 0}^{n - 1}{H_\mu\left(T^{-j}{\A} \midmid T^{-j}{\C}\right)} & \text{(by \thref{thm:walters-4.3} \ref{walters-thm-4.3:5})} \\
						&= nH_\mu(\A \mid \C) & \text{(by \thref{thm:walters-4.3} \ref{walters-thm-4.3:9}).}
				\end{align*}
				Combining this with the previous result, we have
				\begin{align*}
					h_\mu(T, \A) &\leq \lim_{n \to +\infty}{\left[\frac{1}{n} H_\mu\left(\bigjoin_{j = 0}^{n - 1}{T^{-j}{\C}}\right) + nH_\mu(\A \mid \C)\right]} \\
						&= h_\mu(T, \C) + H_\mu(\A \mid \C).
				\end{align*}
			\item By \thref{thm:walters-4.3} \ref{walters-thm-4.3:10} we have
				\begin{align*}
					h_\mu(T, T^{-1}{\A}) &= \lim_{n \to +\infty}{\frac{1}{n} H_\mu\left(\bigjoin_{j = 1}^{n - 1}{T^{-j}{\A}}\right)} \\
						&= \lim_{n \to +\infty}{\frac{1}{n} H_\mu\left(\bigjoin_{j = 0}^{n - 1}{T^{-j}{\A}}\right)} \\
						&= h_\mu(T, \A)
				\end{align*}
			\item Let $m \geq 1$ be fixed. Then
				\begin{align*}
					h_\mu\left(T, \bigjoin_{j = 0}^{m}{T^{-j}{\A}}\right) &= \lim_{n \to +\infty}{\frac{1}{n} H_\mu\left(\bigjoin_{j = 0}^{n - 1}{T^{-j}\left(\bigjoin_{k = 0}^{m}{T^{-k}{\A}}\right)}\right)} \\
						&= \lim_{n \to +\infty}{\frac{1}{n} H_\mu\left(\bigjoin_{j = 0}^{n + m - 1}{T^{-j}{\A}}\right)} \\
						&= \lim_{n \to +\infty}{\left(\frac{m + n}{n}\right) \frac{1}{m + n} H_\mu\left(\bigjoin_{j = 0}^{n + m - 1}{T^{-j}{\A}}\right)} \\
						&= h_\mu(T, \A)
				\end{align*}
			\item Suppose that $T$ is invertible and fix $m \geq 1$. We have
				\begin{align*}
					h_\mu\left(T, \bigjoin_{j = -m}^{m}{T^{-j}{\A}}\right) &= h_\mu\left(T, \bigjoin_{j = 0}^{2m}{T^{-j}{\A}}\right) & \text{(by \ref{walters:thm-4-12:5})} \\
						&= h_\mu(T, \A) & \text{(by \ref{walters:thm-4-12:6}).}
				\end{align*}
		\end{enumerate}
	\end{proof}
\end{theorem}

There is an alternative definition for $h_\mu(T, \alpha)$ which is useful if we want to utilise results relating to conditional entropy.

\begin{theorem}
	Suppose that $(X, \B, \mu, T)$ is a measure-preserving transformation of a probability space and that $\alpha$ be a finite partition of $X$. The entropy of $T$ with respect to $\alpha$ (or $\A(\alpha)$) may also be given by
	\[
		h_\mu(T, \A(\alpha)) = h_\mu(T, \alpha) = H_\mu\left(\alpha \midmid \bigjoin_{j = 1}^\infty {T^{-j}{\alpha}}\right).
	\]
	
	\begin{proof}
		We follow the proof given in \cite[Lecture 24]{ergodic-lectures}.
		
		Let $\A := \A(\alpha)$. We have
		\begin{align*}
			H_\mu\left(\bigjoin_{j = 0}^{n - 1}{T^{-j}{\A}}\right) &= H_\mu\left(\bigjoin_{j = 1}^{n - 1}{T^{-j}{\A}}\right) + H_\mu\left(\A \midmid \bigjoin_{j = 1}^{n - 1}{T^{-j}{\A}}\right) & \text{(by \thref{thm:walters-4.3} \ref{walters-thm-4.3:2})} \\
				&= H_\mu\left(\bigjoin_{j = 0}^{n - 2}{T^{-j}{\A}}\right) + H_\mu\left(\A \midmid \bigjoin_{j = 1}^{n - 1}{T^{-j}{\A}}\right) & \text{(by \thref{thm:walters-4.3} \ref{walters-thm-4.3:10})}.
		\end{align*}
		By induction, this means that
		\begin{align*}
			\frac{1}{n} H_\mu\left(\bigjoin_{j = 0}^{n - 1}{T^{-j}{\A}}\right) &= \frac{1}{n}\left[H_\mu\left(\A \midmid \bigjoin_{j = 1}^{n - 2}{T^{-j}{\A}}\right) + H_\mu\left(\A \midmid \bigjoin_{j = 1}^{n - 3}{T^{-j}{\A}}\right)\right. \\
				& \left. \vphantom{\bigjoin_{j = 1}^{n - 2}{T^{-j}{\A}}} \qquad + \dots + H_\mu(\A \mid T^{-1}{\A}) + H_\mu(\A)\right].
		\end{align*}
		By \thref{thm:walters-4.3} \ref{walters-thm-4.3:5} we have
		\[
			H_\mu\left(\A \midmid \bigjoin_{j = 1}^{n - 1}{T^{-j}{\A}}\right) \leq H_\mu\left(\A \midmid \bigjoin_{j = 1}^{n - 2}{T^{-j}{\A}}\right) \leq \dots \leq H_\mu(\A).
		\]
		We may now apply \thref{thm:increasing-martingale} to get that
		\[
			H_\mu\left(\A \midmid \bigjoin_{j = 1}^{n - 1}{T^{-j}{\A}}\right) \to H_\mu\left(\A \midmid \bigjoin_{j = 1}^\infty{T^{-j}{\A}}\right),
		\]
		as $n \to +\infty$, and therefore
		\begin{align*}
			h_\mu(T, \A) = \lim_{n \to +\infty}{\frac{1}{n} H_\mu\left(\bigjoin_{j = 0}^{n - 1}{}T^{-j}{\A}\right)} &= H_\mu\left(\A \midmid \bigjoin_{j = 1}^\infty{T^{-j}{\A}}\right).
		\end{align*}
	\end{proof}
\end{theorem}

Finally, we can think of the entropy of a measure-preserving transformation $T$, regardless of the choice of partition, to be maximal average information we can gain per day. We state this formally as follows.

\begin{definition}
	Let $(X, \B, \mu, T)$ be a measure-preserving transformation of a probability space. The \key{entropy of $T$} is defined
	\[
		h_\mu(T) := \sup_{\alpha}{h_\mu(T, \alpha)} = \sup_{\A}{h_\mu(T, \A)},
	\]
	where the supremum is taken over all finite measurable partitions $\alpha$ or finite sub-$\sigma$-algebras of $\B$, respectively.
\end{definition}

\begin{remark}
	If $T = \id_X$, then $h(T) = 0$. In general, if $h(T) = 0$, then $h(T, \alpha) = 0$ for all finite partitions $\alpha$. This means that we don't obtain `much' new information each day, i.e. $\bigvee_{j = 0}^{n - 1}{T^{-j}{\alpha}}$ doesn't change `much', as $n \to +\infty$. One such measure-preserving transformation which has this property is the identity transformation on $X$.
\end{remark}

\begin{theorem}
	Entropy is a conjugacy invariant and hence is also an isomorphism invariant.
	\begin{proof}
		Let $(X_1, \B_1, \mu_1, T_1), (X_2, \B_2, \mu_2, T_2)$ be measure-preserving transformations of probability spaces. Let $\Phi : (\tilde{B}_2, \tilde{\mu}_2) \to (\tilde{B}_1, \tilde{\mu}_1)$ be an isomorphism of measure algebras such that $\Phi \circ \tilde{T}_2^{-1} = \tilde{T}_1^{-1} \circ \Phi$. We aim to show that $h_{\mu_1}(T_1) = h_{\mu_2}(T_2)$.
		
		Let $\A_2$ be an arbitrary finite sub-$\sigma$-algebra of $\B_2$ and write $\alpha(\A_2) = \{A_1, \dots, A_r\}$. Since $\Phi$ is an isomorphism of measure algebras, we can choose $C_j \in \B_1$ such that $\tilde{C}_j = \Phi(\tilde{A}_j)$. Using this, we define $\gamma := \{C_1, \dots, C_r\}$, which we see is a partition of $(X_1, \B_1, \mu_1)$. We write $\A_1 := \A(\gamma)$.
		
		For any $(q_0, q_1, \dots, q_{n - 1})$, where $q_j \in \{1, \dots, r\}$ for each $j$, we have
		\begin{align*}
			\Phi\left(\bigcap_{j = 0}^{n - 1}{(T_2^{-j} A_{q_j})^\sim}\right) &= \Phi\left(\bigcap_{j = 0}^{n - 1}{\tilde{T}_2^{-j} \tilde{A}_{q_j}}\right) \\
				&= \bigcap_{j = 0}^{n - 1}{\tilde{T}_1^{-j} \Phi(\tilde{A}_{q_j})} \\
				&= \bigcap_{j = 0}^{n - 1}{\tilde{T}_1^{-j} \tilde{C}_{q_j}} \\
				&= \bigcap_{j = 0}^{n - 1}{(T_1^{-j} C_{q_j})^\sim}.
		\end{align*}
		Hence the sets $\bigcap_{j = 0}^{n - 1}{(T_2^{-j} A_{q_j})^\sim}$ and $\bigcap_{j = 0}^{n - 1}{(T_1^{-j} C_{q_j})^\sim}$ have the same measure. Recall that the entropy of a partition is completely determined by the measure of the elements in the partition. This means that
		\[
			H_{\mu_1}\left(\bigjoin_{j = 0}^{n - 1}{T_1^{-j}{\A_1}}\right) = H_{\mu_2}\left(\bigjoin_{j = 0}^{n - 1}{T_2^{-j}{\A_2}}\right),
		\]
		and hence $h_{\mu_1}(T_1, \A_1) = h_{\mu_2}(T_2, \A_2)$. Since $\A_2$ was chosen to be an arbitrary sub-$\sigma$-algebra of $\B_1$, this means that $h_{\mu_1}(T_1) \geq h_{\mu_2}(T_2)$.
		
		We repeat the proof, but choose an arbitrary finite sub-$\sigma$-algebra of $\B_1$ to get the reverse inequality $h_{\mu_1}(T_1) \leq h_{\mu_2}(T_2)$, and hence $h_{\mu_1}(T_1) = h_{\mu_2}(T_2)$.
	\end{proof}
\end{theorem}

\section{Calculating \texorpdfstring{$h_\mu(T)$}{h(T)}}
Recall that the entropy of a measure-preserving transformation $T$ is defined $h_\mu(T) := \sup_{\A}{h_\mu(T, \A)}$, where the supremum is taken over all finite sub-$\sigma$-algebras of $\B$ (or, equivalently, over all finite partitions of $(X, \B, \mu)$). Of course, it is difficult to consider all finite partitions, so we want to find criteria which guarantee that $h_\mu(T) = h_\mu(T, \A)$ instead.

One key result is the \key{Kolmogorov-Sinai Theorem}. We will prove this in \thref{thm:kolmogorov-sinai}, but to do this we need some preliminary results.

\begin{lemma} \label{lem:walters-4-15}
	Let $r \in \naturals$ be fixed and let $\varepsilon > 0$ be given. Then there exists $\delta > 0$ such that, if we have two partitions of $r$ sets $\alpha = \{A_1, \dots, A_r\}$, $\gamma = \{C_1, \dots, C_r\}$ of $(X, \B, \mu)$ such that
	\[
		\sum_{j = 1}^r{\mu(A_j \symdiff C_j)} < \delta,
	\]
	then we have $H_\mu(\alpha \mid \gamma) + H_\mu(\gamma \mid \alpha) < \varepsilon$.
	
	\begin{proof}
		Let $\varepsilon > 0$ be given. We choose $\delta > 0$ such that $\delta < 1 / 4$ and
		\[
			-r(r - 1) \delta \log{\delta} - (1 - \delta) \log(1 - \delta) < \frac{\varepsilon}{2}.
		\]
		
		Let $\beta$ be the partition of $(X, \B, \mu)$ consisting of the sets of the form $A_j \cap C_k$, where $j \neq k$, and the set $\bigcup_{j = 1}^r{A_j \cap C_j}$. It is then clear that $\alpha \join \gamma = \gamma \join \beta$. For $j \neq k$, we also have
		\[
			A_j \cap C_k \subset \bigcup_{n = 1}^r{A_n \symdiff C_n}.
		\]
		This gives, by the hypothesis, $\mu(A_j \cap C_k) < \delta$ for $j \neq k$. By the definition of symmetric difference, we also have
		\[
			\mu\left(\bigcup_{j = 1}^r{A_j \cap C_j}\right) > 1 - \delta.
		\]
		Hence
		\begin{align*}
			H_\mu(\beta) &= -\sum_{j \neq k}{\mu(A_j \cap C_k) \log{\mu(A_j \cap C_k)}} - \mu\left(\bigcup_{j = 1}^r{A_j \cap C_j}\right) \log{\mu\left(\bigcup_{j = 1}^r{A_j \cap C_j}\right)} \\
				&< -r(r - 1) \delta \log{\delta} - (1 - \delta) \log(1 - \delta) \\
				&< \frac{\varepsilon}{2}.
		\end{align*}
		We therefore have
		\begin{align*}
			H_\mu(\gamma) + H_\mu(\alpha \mid \gamma) &= H_\mu(\alpha \join \gamma) & \text{(by \thref{thm:walters-4.3} \ref{walters-thm-4.3:2})} \\
				&= H_\mu(\gamma \join \beta) \\
				&\leq H_\mu(\gamma) + H_\mu(\beta) & \text{(by \thref{thm:walters-4.3} \ref{walters-thm-4.3:8})} \\
				&< H_\mu(\gamma) + \frac{\varepsilon}{2},
		\end{align*}
		and hence $H_\mu(\alpha \mid \gamma) < \varepsilon / 2$.
		
		We repeat this argument using $\alpha \join \gamma = \alpha \join \beta$ to get $H_\mu(\gamma \mid \alpha) < \varepsilon / 2$. Combining these two results, we get $H_\mu(\alpha \mid \gamma) + H_\mu(\gamma \mid \alpha) < \varepsilon$.
	\end{proof}
\end{lemma}

\begin{theorem} \label{thm:walters-4-16}
	Suppose that $\C$ is a finite sub-$\sigma$-algebra of $\B$ and that $\B_0$ is an algebra such that $\B(\B_0) = \B$ $\mu$-almost everywhere. Then given any $\varepsilon > 0$, there exists a finite algebra $\D \subset \B_0$ such that
	\[
		H_\mu(\D \mid \C) + H_\mu(\C \mid \D) < \varepsilon.
	\]
	
	\begin{proof}
		Let $\varepsilon > 0$ be given and write $\alpha(\C) = \{C_1, \dots, C_r\}$. We choose $\delta > 0$ as in \thref{lem:walters-4-15}, where $r, \varepsilon$ here are as in the lemma. It suffices to show that, for each $\tau > 0$, there exists a partition $\D = \{D_1, \dots, D_r\}$, where $D_j \in \B_0$ and $\mu(C_j \symdiff D_j) < \tau$, for all $j = 1, \dots, r$. This is because we may then choose $\tau$ such that $r\tau \leq \delta$ and then apply \thref{lem:walters-4-15}.
		
		To begin, we choose $\lambda > 0$ such that $\lambda(r - 1)[1 + r(r - 1)] < \tau$. For each $j = 1, \dots, r$, choose $B_j \in \B_0$ such that $\mu(C_j \symdiff B_j) < \lambda$. Now if $j \neq k$, then $B_j \cap B_k \subset (B_j \symdiff C_j) \cup (\B_j \symdiff C_j)$. It follows that $\mu(B_j \cap B_k) < 2\lambda$. We let $N := \bigcup_{j \neq k}{(B_j \cap B_k)}$, so that $\mu(N) < r(r - 1)\lambda$.
		
		Now for $j = 1, \dots, r - 1$ we define $D_j = B_j \setminus N$, and $D_r = X \setminus \bigcup_{j = 1}^{r - 1}{D_j}$. This clearly defines a partition $\D := \{D_1, \dots, D_r\}$ of $X$, and each $D_j \in \B_0$, since $\B_0$ is an algebra (i.e. is closed under finite unions and complementation).
		
		If $j < r$, then $D_j \symdiff C_j \subset (B_j \symdiff C_j) \cup N$. Then by countable subadditivity,
		\begin{align*}
			\mu(D_j \symdiff C_j) &\leq \mu(B_j \symdiff C_j) + \mu(N) \\
				&< \lambda + r(r - 1)\lambda \\
				&= \lambda[1 + r(r - 1)] \\
				&< \tau.
		\end{align*}
		For the last part of $\D$, we use the fact that $D_r \symdiff C_r \subset \bigcup_{j = 1}^{r - 1}{(D_j \symdiff C_j)}$. Then $\mu(D_r \symdiff C_r) < (r - 1)\lambda[1 + r(r - 1)] < \tau$.
	\end{proof}
\end{theorem}

\begin{corollary} \label{cor:walters-4-16-1}
	Let $\A_1 \subset \A_2 \subset \dots \subset \A_n \subset \dots$ be an increasing sequence of finite sub-algebras of $\B$. Suppose that $\C$ is a finite sub-algebra of $\B$ such that $\C \subset \bigjoin_{n \geq 1}{\A_n}$ $\mu$-almost everywhere. Then $H_\mu(\C \mid \A_n) \to 0$, as $n \to +\infty$.
	
	Note: We say that $\A \subset \C$ $\mu$-almost everywhere if, for all $A \in \A$, there exists $C \in \C$ such that $\mu(A \symdiff C) = 0$.
	
	\begin{proof}
		Let $\varepsilon > 0$ be given. Write $\B_0 := \bigcup_{n = 1}^\infty{\A_n}$ so that $\B_0$ is an algebra. Since $\C \subset \bigjoin_{n \geq 1}{\A_n}$ $\mu$-almost everywhere, we have $\C \subset \B(\B_0)$ $\mu$-almost everywhere. By \thref{thm:walters-4-16}, there exists a finite sub-algebra $\D_\varepsilon$ of $\B_0$ such that $H_\mu(\C \mid \D_\varepsilon) < \varepsilon$.
		
		Since $\A_n$ is increasing and $\D_\varepsilon$ is finite, we have $\D_\varepsilon \subset \A_{n_0}$ for some $n_0 \in \naturals$. Then for all $n \geq n_0$, we have
		\[
			H_\mu(\C \mid \A_n) \leq H_\mu(\C \mid \A_{n_0}) \leq H_\mu(\C \mid \D_\varepsilon) < \varepsilon.
		\]
		Hence $H_\mu(\C \mid \A_n) \to 0$, as $n \to +\infty$.
	\end{proof}
\end{corollary}

We are ready to prove one of the main results of this chapter.

\begin{theorem}[Kolmogorov-Sinai Theorem] \label{thm:kolmogorov-sinai}
	Let $T : X \to X$ be an invertible measure-preserving transformation of a probability space $(X, \B, \mu)$. Suppose that $\A$ is a finite sub-$\sigma$-algebra of $\B$ such that
	\[
		\bigjoin_{n = -\infty}^\infty{T^{-n}{\A}} = \B
	\]
	$\mu$-almost everywhere. Then $h_\mu(T) = h_\mu(T, \A)$.
	
	\begin{proof}
		Let $\C$ be a finite sub-$\sigma$-algebra of $\B$. We want to show that $h_\mu(T, \C) \leq h_\mu(T, \A)$, i.e. $h_\mu(T, \A)$ achieves the supremum as in the definition of $h_\mu(T)$. We have
		\begin{align*}
			h_\mu(T, \C) &\leq h_\mu\left(T, \bigjoin_{j = -n}^n{T^{-j}{\A}}\right) + H_\mu\left(\C \midmid \bigjoin_{j = -n}^n{T^{-j}{\A}}\right) & \text{(by \thref{thm:walters-4.12} \ref{walters:thm-4-12:4})} \\
				&= h_\mu(T, \A) + H_\mu\left(\C \midmid \bigjoin_{j = -n}^n{T^{-j}{\A}} \right) & \text{(by \thref{thm:walters-4.12} \ref{walters:thm-4-12:7})}
		\end{align*}
		We let $\A_n := \bigjoin_{j = -n}^n{T^{-j}{\A}}$ so that $\A_n$ is an increasing sequence of finite sub-algebras of $\B$. Since $\bigjoin_{n = -\infty}^\infty{T^n{\A}} = \B$ $\mu$-almost everywhere, and $\C \subset \B$, we may apply \thref{cor:walters-4-16-1}. So $H_\mu(\C \mid \A_n) \to 0$, as $n \to +\infty$, and hence $h_\mu(T, \C) \leq h_\mu(T, \A)$.
	\end{proof}
\end{theorem}

There is a similar result which does not require that $T$ is invertible.

\begin{theorem} \label{thm:walters-4-18}
	Let $T : X \to X$ be a (not necessarily invertible) measure-preserving transformation of a probability space $(X, \B, \mu)$. Suppose that $\A$ is a finite sub-algebra of $\B$ such that
	\[
		\bigjoin_{n = 0}^\infty{T^{-n}{\A}} = \B
	\]
	$\mu$-almost everywhere. Then $h_\mu(T) = h_\mu(T, \A)$.
	
	\begin{proof}
		We repeat the same proof for \thref{thm:kolmogorov-sinai}, but replace $\bigjoin_{j = -n}^n{T^{-j}{\A}}$ with $\bigjoin_{j = 0}^n{T^{-j}{\A}}$ and apply \thref{thm:walters-4.12} \ref{walters:thm-4-12:6} instead of \ref{walters:thm-4-12:7}.
	\end{proof}
\end{theorem}

The following result gives a useful criterion for deciding if an invertible measure-preserving transformation has zero entropy.

\begin{corollary}
	Suppose that $T : X \to X$ is an invertible measure-preserving transformation of a probability space $(X, \B, \mu)$, and that
	\[
		\bigjoin_{n = 0}^\infty{T^{-n}{\A}} = \B
	\]
	$\mu$-almost everywhere for some finite sub-algebra $\A$ of $\B$. Then $h_\mu(T) = 0$.
	
	\begin{proof}
		By \thref{thm:walters-4-18}, we have
		\[
			h_\mu(T) = h_\mu(T, \A) = \lim_{n \to +\infty}{H_\mu\left(\A \midmid \bigjoin_{j = 1}^n{T^{-j}{\A}}\right)}.
		\]
		Since $T$ is a measure-preserving transformation, we have $\bigjoin_{j = 1}^\infty{T^{-j}{\A}} = T^{-1}{\B} = \B$ $\mu$-almost everywhere.
		
		We write $\A_n := \bigjoin_{j = 1}^\infty{T^{-j}{\A}}$, so that $\A_n$ is an increasing sequence of sub-algebras of $\B$, and $\bigjoin_{n = 1}^\infty{\A_n} = \B$ $\mu$-almost everywhere. In particular, for all $n \geq 1$ we have $\A \subset \A_n$ $\mu$-almost everywhere and so we may apply \thref{cor:walters-4-16-1}. So $H_\mu(\A \mid \A_n) \to 0$, as $n \to +\infty$. Hence $h_\mu(T) = 0$.
	\end{proof}
\end{corollary}

\section{Shifts of finite type} \label{sec:entropy:sft}
We now consider entropy for shifts of finite type, which we will be interested in later on. The following definitions appear in \cite{chazottes-maldonado:cbfee}.

Let $A$ be a finite alphabet and let $\Sigma := \{(x_j)_{j = 0}^\infty \mid x_j \in A\}$ denote the full (one-sided) shift. As usual, $\sigma : \Sigma \to \Sigma$ will be the (one-sided) shift map. For the sake of readability, if $(x_j)_{j = 0}^\infty \in \Sigma$, then we will write $x_m^n := (x_j)_{j = m}^n$.

In this setting, our measure-preserving transformation will always be $\sigma$ and so we are more interested in $\sigma$-invariant probability measures than the shift map itself. We will define the entropy of such measures in the remainder of this chapter and we will see that each definition is consistent with our previous discussion.

For the remainder of this section, we let $\alpha_k := \{[a_0^{k - 1}] \mid a_0^{k - 1} \in A^k\}$ denote the partition of $(\Sigma, \B, \nu)$ by cylinders of length $k$.

\begin{definition}
	Let $\nu$ be a $\sigma$-invariant probability measure on $\Sigma$. For shifts of finite type, for $k > 1$ we can define the \key{$k$-block entropy} of $\nu$ by
	\[
		H_k(\nu) := -\sum_{a_0^{k - 1} \in A^k}{\nu[a_0^{k - 1}] \log\nu[a_0^{k - 1}]},
	\]
	where the sum is taken over all sequences of length $k$.
\end{definition}

By definition, the entropy of $\alpha_k$ is
\[
	H_\nu(\alpha_k) = -\sum_{a_0^{k - 1} \in A^k}{\nu[a_0^{k - 1}] \log\nu[a_0^{k - 1}]} = H_k(\nu).
\]
In other words, $k$-block entropy is the entropy of the partition of $\Sigma$ by cylinders of length $k$.

\begin{definition}
	Let $\nu$ be a $\sigma$-invariant probability measure on $\Sigma$. The \key{entropy} of $\nu$ is given by
	\[
		h(\nu) := \lim_{k \to +\infty}{\frac{1}{k} H_k(\nu)}.
	\]
\end{definition}

Let $\A$ be the sub-algebra of $\B$ consisting of unions cylinders of length $1$. Then $\bigjoin_{k = 0}^\infty{\sigma^{-k}{\A}} = \B$ and hence \thref{thm:walters-4-18} applies. So
\[
	h_\nu(\sigma) = h_\nu(\sigma, \A) = \lim_{k \to +\infty}{\frac{1}{k} H_\nu\left(\bigjoin_{j = 0}^{k - 1}{\sigma^{-j}{\A}}\right)} = \lim_{k \to +\infty}{\frac{1}{k} H_k\left(\nu\right)} = h(\nu).
\]
Therefore this definition is consistent with our previous results.

\begin{definition}
	The \key{conditional $k$-block entropy}, where $k \geq 2$, is defined
	\[
		h_k(\nu) := -\sum_{a_0^{k - 1} \in A^k}{\nu[a_0^{k - 1}] \log{\frac{\nu[a_0^{k - 1}]}{\nu[a_0^{k - 2}]}}}.
	\]
\end{definition}

Note that $[a_0^{k - 1}] \subset [a_0^{k - 2}]$ and hence for $C \in \alpha_{k - 1}$,
\[
	[a_0^{k - 1}] \cap C =
	\begin{cases}
		\left[a_0^{k - 1}\right],	& \text{if } C = [a_0^{k - 2}]; \\
		\emptyset,	& \text{otherwise}.
	\end{cases}
\]
Then by the definition of conditional entropy we have
\begin{align*}
	H_\nu(\alpha_k \mid \alpha_{k - 1}) &= -\sum_{a_0^{k - 1} \in A^k}{\nu([a_0^{k - 2}] \cap [a_0^{k - 1}]) \log{\frac{[a_0^{k - 2}] \cap [a_0^{k - 1}]}{[a_0^{k - 2}]}}} \\
		&= -\sum_{a_0^{k - 1} \in A^k}{\nu[a_0^{k - 1}] \log{\frac{[a_0^{k - 1}]}{[a_0^{k - 2}]}}} \\
		&= h_k(\nu),
\end{align*}
and so the definitions are once again consistent. For $k \geq 2$, we also clearly have the relation
\[
	h_k(\nu) = H_k(\nu) - H_{k - 1}(\nu).
\]

Finally, we have \key{relative $k$-block entropy} which we will use later.

\begin{definition}
	For $k \geq 1$, the \key{$k$-block relative entropy} of a measure $\nu$ with respect to a measure $\mu$ is defined
	\[
		H_k(\nu \mid \mu) = \sum_{a_0^{k - 1} \in A^k}{\nu[a_0^{k - 1}] \log{\frac{\nu[a_0^{k - 1}]}{\mu[a_0^{k - 1}]}}}.
	\]
\end{definition}

\chapter{Gibbs measures} \label{chap:gibbs}
\section{Overview}
This chapter introduces Gibbs measures, a class of probability measures on shifts of finite type. Gibbs measures have properties which we make use of when we discuss \cite{chazottes-maldonado:cbfee} in Chapter \ref{chap:concentration-bounds}.

Throughout this chapter we will let $\Sigma = \Sigma_A^+$, where $A$ is an irreducible $k \times k$ matrix.

\section{The Ruelle operator}
\emph{This section predominantly follows material from \cite[Chapter 2]{parry-pollicott:zeta-fns-periodic-orbits}.}

We briefly introduce a theorem due to Ruelle. Later, it will be apparent that this theorem is useful for constructing Gibbs measures.

\begin{definition}
	Let $f \in F_\theta^+$. The \key{Ruelle operator} (or \key{transfer operator}) $L_f : F_\theta^+ \to F_\theta^+$ (or, more generally, $L_f : C(\Sigma, \complex) \to C(\Sigma, \complex)$) is defined
	\[
	(L_f{w})(x) = \sum_{y \in \Sigma \midcolon \sigma{y} = x}{e^{f(y)} w(y)} = \sum_{j \midcolon A_{j, x_0} = 1}{e^{f(j, x_0, x_1, \dots)} w(j, x_0, x_1, \dots)},
	\]
	where $x = (x_j)_{j = 0}^\infty \in \Sigma$. This is a bounded linear operator.
	
	The $n$-th iterate of $L_f$ is given by
	\[
	(L_f^n{w})(x) = \sum_{y \in \Sigma \midcolon \sigma^n{y} = x}{e^{f^n(y)} w(y)}.
	\]
	
	If $f$ is also real-valued and $L_f{1} = 1$, then we say that $f$ or $L_f$ is \key{normalised}.
\end{definition}

%\begin{proposition}
%	Let $f \in F_\theta^+$ with $f = u + iv$, where $u, v \in F_\theta^+$ are real-valued functions. If $L_u$ is normalised, i.e. $L_u{1} = 1$, then for all $n \geq 0$,
%	\[
%	|L_f^n{w}|_\theta \leq K|w|_\infty + \theta^n |w|_\theta
%	\]
%	for all $w \in F_\theta^+$, where $K > 0$ is a constant depending only on $f$ and $\theta$.
%\end{proposition}

\begin{theorem}[Ruelle's Perron-Frobenius Theorem] \label{thm:rpf}
	Suppose that $\Sigma = \Sigma_A^+$ is an aperiodic shift of finite type and let $f \in F_\theta^+$ be a real-valued function. Then
	\begin{enumerate}
		\item There is a simple maximal eigenvalue $\lambda$ of $L_f : C(\Sigma, \reals) \to C(\Sigma, \reals)$ with a corresponding eigenfunction $h \in C(\Sigma_A^+, \reals)$, with $h > 0$. \label{rpf:1}
		\item The remainder of the spectrum of $L_f$ is contained in a disc of radius strictly less than $\lambda$. \label{rpf:2}
		\item There is a unique probability measure $\mu$ such that $L_f^*{\mu} = \lambda\mu$. That is,
		\[
		\int{L_f{w}\ d\mu} = \lambda \int{w\ d\mu},
		\]
		for all $w \in C(\Sigma, \reals)$. Additionally, if $h$ is the eigenfunction as in \ref{rpf:1} and $\int{h\ d\mu} = 1$, then the measure $\nu$ defined by $d\nu = h\ d\mu$ is a $\sigma$-invariant probability measure. \label{rpf:3}
		\item If $h$ is the eigenfunction as in \ref{rpf:1} and $\int{h\ d\mu} = 1$, then for all $w \in C(\Sigma, \reals)$,
		\[
		\frac{1}{\lambda^n}L_f^n{w} \to h \int{w\ d\mu}
		\]
		uniformly, as $n \to +\infty$. \label{rpf:4}
	\end{enumerate}
\end{theorem}

\section{Gibbs measures and basic results}
\emph{The remainder of this chapter predominantly follows \cite[Chapter 3]{parry-pollicott:zeta-fns-periodic-orbits}.}

\begin{definition}
	Let $\phi \in C(\Sigma, \reals)$ be a continuous function. Let $\mu$ be a probability measure such that there exists a constants $C = C(\phi) > 1$, $P = P(\phi) \in \reals$, such that
	\begin{equation}
		C^{-1} \leq \frac{\mu[x_0, \dots, x_n]}{\exp\left(-Pn + \sum_{j = 0}^{n - 1}{\phi(\sigma^j{x})} \right)} \leq C,
	\end{equation}
	for all $n \geq 0$ and for all $x \in \Sigma$. Then the measure $\mu$ is a \key{Gibbs measure for $\phi$} or a \key{Gibbs measure with potential $\phi$}. We will write $\mu_\phi = \mu$.
\end{definition}

\begin{remark}
	The Gibbs measure $\mu_\phi$ is not necessarily $\sigma$-invariant.
\end{remark}

Given any $\phi \in F_\theta$, it can be shown that there exists a Gibbs measure $\mu_\phi$ for $\phi$. We prove this in the following results.

\begin{proposition} \label{prop:pp-3-2}
	Suppose that $\phi \in F_\theta$ is a normalised, real-valued function. Then for all $x \in \Sigma$ we have
	\begin{equation}
		e^{-|\phi|_\theta \theta^n} \leq \frac{\mu[x_0, \dots, x_n]}{\mu[x_1, \dots, x_n]} e^{-\phi(x)} \leq e^{|\phi|_\theta \theta^n},
	\end{equation}
	where $L_\phi^*\mu = \mu$ is the unique measure in Ruelle's Perron-Frobenius Theorem \ref{rpf:3} (\thref{thm:rpf}) with $\lambda = 1$.
	\begin{proof}
		Let $w \in C(\Sigma, \reals)$. We have
		\begin{align*}
			\int{w \circ \sigma\ d\mu} &= \int L_\phi(w \circ \sigma)\ d\mu \\
				&= \int{\sum_{y \in \Sigma \midcolon \sigma{y} = x}{e^{\phi(y)} w \circ \sigma(y)}\ d\mu} \\
				&= \int{\sum_{y \in \Sigma \midcolon \sigma{y} = x}{e^{\phi(y)} w(x)}\ d\mu} \\
				&= \int{w L_\phi{1}\ d\mu} \\
				&= \int{w\ d\mu}.
		\end{align*}
		By a basic result~\cite[Lemma 11.3]{ergodic-lectures} in ergodic theory, it follows that $\mu$ is a $\sigma$-invariant measure. Hence
		\begin{align}
			\mu[x_1, \dots, x_n] &= \mu\left(\bigsqcup_{x_0 \midcolon A_{x_0, x_1} = 1}{[x_0, x_1, \dots, x_n]}\right) \nonumber \\
				&= \int{\sum_{x_0 \midcolon A_{x_0, x_1} = 1} \chi_{[x_0, x_1, \dots, x_n]}(z)\ d\mu} \nonumber \\
				&= \int{\sum_{y \midcolon \sigma{y} = z} e^{\phi(y)} \chi_{[x_0, \dots, x_n]}(y) e^{-\phi(y)}\ d\mu} & (\text{since } \chi_B(y) = 0 \text{ or } 1) \nonumber \\
				&= \int{L_\phi\chi_{[x_0, x_1, \dots, x_n]}(z) e^{-\phi(z)}\ d\mu} \nonumber \\
				&= \int{\left(\chi_{[x_0, \dots, x_n]} e^{-\phi}\right)(z)\ d\mu} \nonumber \\
				&= \int_{[x_0, \dots, x_n]}{e^{-\phi}\ d\mu}. \label{fml:prop-3-2-mu-e}
		\end{align}
		Now let $z, w \in [x_0, \dots, x_n]$. Since $\phi \in F_\theta \subset C(\Sigma, \reals)$, we have that $\var_n(\phi) \leq |\phi|_\theta \theta^n$ and hence
		\[
			e^{-|\phi|_\theta \theta^n} \leq e^{\phi(z) - \phi(w)} \leq e^{|\phi|_\theta \theta^n}.
		\]
		Then by \eqref{fml:prop-3-2-mu-e} we have
		\[
			\mu[x_0, \dots, x_n] e^{-|\phi|_\theta \theta^n} \leq \mu[x_1, \dots, x_n] e^{\phi(x)} \leq \mu[x_0, \dots, x_n] e^{|\phi|_\theta \theta^n},
		\]
		and so
		\[
			e^{-|\phi|_\theta \theta^n} \leq \frac{\mu[x_0, \dots, x_n]}{\mu[x_1, \dots, x_n]} e^{-\phi(x)} \leq e^{|\phi|_\theta \theta^n}.
		\]
	\end{proof}
\end{proposition}

\begin{corollary} \label{cor:pp-3-2-1}
	The measure $\mu$ in \thref{prop:pp-3-2} is a Gibbs measure for $\phi$ with $P = 0$.
	\begin{proof}
		We apply \thref{prop:pp-3-2} to $\phi$, $\phi \circ \sigma$, \dots, $\phi \circ \sigma^n$ to get the sequence of inequalities
		\[
			\left.
			\begin{matrix}
				e^{-|\phi|_\theta \theta^n} &\leq& \dfrac{\mu[x_0, \dots, x_n]}{\mu[x_1, \dots, x_n]} e^{-\phi(x)} &\leq& e^{|\phi|_\theta \theta^n} \\
				e^{-|\phi|_\theta \theta^{n - 1}} &\leq& \dfrac{\mu[x_1, \dots, x_n]}{\mu[x_2, \dots, x_n]} e^{-\phi(\sigma{x})} &\leq& e^{|\phi|_\theta \theta^{n - 1}} \\
				\vdots & & \vdots & & \vdots \\
				e^{-|\phi|_\theta} &\leq& \mu[x_n] e^{-\phi(\sigma^n{x})} &\leq& e^{|\phi|_\theta}
			\end{matrix}
			\right\}
		\]
		for all $x \in \Sigma$. Multiplying these together, we get
		\[
			\exp\left(-\sum_{j = 0}^n{|\phi|_\theta \theta^j}\right) \leq \frac{\mu[x_0, \dots, x_n]}{\exp\left(-\sum_{j = 0}^n{\phi(\sigma^j{x})}\right)} \leq \exp\left(\sum_{j = 0}^n{|\phi|_\theta \theta^j}\right),
		\]
		and so
		\[
			\exp\left(-\frac{|\phi|_\theta}{1 - \theta}\right) \leq \frac{\mu[x_0, \dots, x_n]}{\exp\left(-\sum_{j = 0}^n{\phi(\sigma^j{x})}\right)} \leq \exp\left(\frac{|\phi|_\theta}{1 - \theta}\right).
		\]
		Hence $\mu$ is a Gibbs measure for $\phi$ with $P = 0$.
	\end{proof}
\end{corollary}

We can generalise these results for Lipschitz functions $\phi \in F_\theta$ which are not normalised.

\begin{corollary}
	Suppose that $\phi \in F_\theta$ is a real-valued function. Let $\mu$ be the unique measure where $L_\phi^*\mu = \lambda\mu$, where $\lambda > 0$ is the maximal eigenvalue in Ruelle's Perron-Frobenius Theorem \ref{rpf:3}. Then $\mu$ is a Gibbs measure for $\phi$ with $P = \log{\lambda}$.
	\begin{proof}
		First note that $\psi := \phi - \log{(h \circ \sigma)} + \log{h} - \log{\lambda}$ is normalised, where $h > 0$ is the eigenfunction corresponding to $\lambda$ in \thref{thm:rpf}. We have
		\begin{align*}
			\exp\left(\sum_{j = 0}^{n - 1}{\psi(\sigma^j{x})}\right) &= \exp\left(\sum_{j = 0}^{n - 1}{\phi(\sigma^j{x}) - \log{h(\sigma^{j + 1}{x})} + \log{h(\sigma^j{x})} - \log{\lambda}}\right) \\
				&= \exp\left(-n\log{\lambda} + \sum_{j = 0}^{n - 1}{\phi(\sigma^j{x})}\right) \frac{h(x)}{h(\sigma^n{x})}.
		\end{align*}
		We apply \thref{cor:pp-3-2-1} to $\psi = \phi - \log{(h \circ \sigma)} + \log{h} - \log{\lambda}$ to get
		\[
			C_0^{-1} \leq \frac{\mu[x_0, \dots, x_n]}{\exp\left(-n\log{\lambda} + \sum_{j = 0}^{n - 1}{\phi(\sigma^j{x})}\right)} \leq C_0,
		\]
		for some constant $C_0 > 1$. Hence $\mu$ is a Gibbs measure for $\phi$ with $P = \log{\lambda}$.
	\end{proof}
\end{corollary}

In view of this, if $\phi \in F_\theta$ has Gibbs measure $\mu_\phi$ with $P(\phi) > 0$, then the Gibbs measure for $\phi - P(\phi)$ gives $P(\phi - P(\phi)) = 0$. In both cases we have the same Gibbs measure $\mu_\phi$.

\section{The variational principle and pressure}
There is a characteristic which sets Gibbs measures apart from other $\sigma$-invariant probability measures. To prove this property, we need some concepts and results.

Let $\mu$ be a $\sigma$-invariant probability measure on $\Sigma$. For $n \geq 0$ we have that $\mu[x_0, \dots, x_{n - 1}] > 0$ for $\mu$-almost every $x \in \Sigma$. For $j = 1, \dots, k$ we define
\begin{align*}
	\mu_n[j \mid \sigma^{-1}{x}] &:= \frac{\mu[j, x_0, \dots, x_{n - 1}]}{\mu[x_0, \dots, x_{n - 1}]} \\
		&= \frac{\mu([j] \cap \sigma^{-1}[x_0, \dots, x_{n - 1}])}{\mu(\sigma^{-1}[x_0, \dots, x_{n - 1}])} \\
		&= \mu\left([j] \midmid \bigjoin_{r = 0}^{n - 1}{\sigma^{-r}{\alpha}}\right)(x),
\end{align*}
where $\alpha = \{[j] \mid j \in \{1, \dots, k\}\}$ is the partition of $\Sigma$ by cylinders of length 1. It is clear that this is a probability distribution for $\mu$-almost every $x$.

\begin{proposition}\label{prop:mu-sq-bkt-pd}
	For $n \geq 0$, we have that
	\[
		\mu[j \mid \sigma^{-1}{x}] := \lim_{n \to +\infty}{\mu_n[j \mid \sigma^{-1}{x}]}
	\]
	is a well-defined probability distribution on $\{1, \dots, k\}$ for $\mu$-almost every $x \in \Sigma$.
	\begin{proof}
		First note that $\bigjoin_{r = 0}^{n - 1}{\sigma^{-r}{\alpha}} \to \B$, as $n \to +\infty$. By definition we also have $\mu([j] \mid \bigjoin_{r = 0}^{n - 1}{\sigma^{-r}{\alpha}}) = E_\mu(\chi_{[j]} \mid \bigjoin_{r = 0}^{n - 1}{\sigma^{-r}{\alpha}})$. Since $\chi_{[j]} \in L^1(\Sigma, \B, \mu)$, we may apply the Increasing Martingale Theorem so that
		\[
			\lim_{n \to +\infty}{\mu\left([j] \midmid \bigjoin_{r = 0}^{n - 1}{\sigma^{-r}{\alpha}}\right)} = \mu([j] \mid \B),
		\]
		for $\mu$-almost every $x$. Hence
		\[
			\mu[j \mid \sigma^{-1}{x}] = \mu([j] \mid \B)(x)
		\]
		for $\mu$-almost every $x$, i.e. $\mu[j \mid \sigma^{-1}{x}]$ is a well-defined probability distribution on $\{1, \dots, k\}$ for $\mu$-almost every $x$.
	\end{proof}
\end{proposition}

\begin{lemma} \label{lem:pp-prop-p36}
	We have
	\begin{equation}
		\sum_{j = 1}^k{\int{\psi(j, x_0, x_1, \dots) \mu[j \mid \sigma^{-1}{x}] \ d\mu}} = \int{\psi\ d\mu},
	\end{equation}
	for all $\psi \in C(\Sigma, \reals)$.
	\begin{proof}
		Let $\psi = \chi_{[r_0, \dots, r_t]}$, where $t \in \naturals_0$. We have
		\begin{align*}
			\sum_{j = 1}^k{\int{\psi(j, x_0, x_1, \dots) \mu[j \mid \sigma^{-1}{x}] \ d\mu}} &= \lim_{n \to +\infty}{\sum_{j = 1}^k{\int{\chi_{[r_0, \dots, r_t]} \mu_n[j \mid \sigma^{-1}{x}] \ d\mu}}} \\
				&= \mu[r_0, \dots, r_t] \sum_{j = 1}^k{\frac{\mu[j, x_0, \dots, x_{n - 1}]}{\mu[x_0, \dots, x_{n - 1}]}} \\
				&= \int{\psi\ d\mu}.
		\end{align*}
		So the result holds for characteristic functions. We then apply the definitions from measure theory to show that the result holds for $\psi \in C(\Sigma, \reals)$.
	\end{proof}
\end{lemma}

\begin{lemma} \label{lem:pp-prop-3-4}
	Suppose that $\phi \in F_\theta$ is a real-valued function and that $L_\phi$ is normalised. Let $\mu$ be a probability measure such that $L_\phi^*{\mu} = \mu$. Then for any $\sigma$-invariant probability measure $m$, we have
	\[
		h_m(\sigma) + \int{\phi\ dm} \leq 0,
	\]
	with equality if and only if $m = \mu$.
	\begin{proof}
		Let $m \in M(\Sigma, \sigma)$ be a $\sigma$-invariant probability measure. We define a probability distribution on $\{1, \dots, k\}$ by $\mu[j \mid \sigma^{-1}{x}]$ as in \thref{prop:mu-sq-bkt-pd}. If $m = \mu$, then $L_\phi^*{m} = m$ and so we have the probability distribution
		\[
			m[j \mid \sigma^{-1}{x}] = \exp(\phi(j, x_0, x_1, \dots))
		\]
		for all $x \in \Sigma$. We apply \thref{lem:pp-3-3} so that
		\[
			-\sum_{j = 1}^k{m[j \mid \sigma^{-1}{x}] \log{m[j \mid \sigma^{-1}{x}]}} + \sum_{j = 1}^k{m[j \mid \sigma^{-1}{x}] \phi(j, x_0, x_1, \dots)} \leq 0,
		\]
		for $m$-almost every $x$, with equality if and only if $m[j \mid \sigma^{-1}{x}] = \phi(j, x_0, x_1, \dots)$ for all $j = 1, \dots, k$.
		
		We integrate with respect to $m$ and apply \thref{lem:pp-prop-p36} to get
		\[
			h_m(\sigma) + \sum_{j = 1}^k{\int{m[j \mid \sigma^{-1}{x}] \phi(j, x_0, x_1, \dots)\ dm}} = h_m(\sigma) + \int{\phi\ dm} \leq 0,
		\]
		with equality if and only if $m[j \mid \sigma^{-1}{x}] = \phi(j, x_0, x_1, \dots)$ for $m$-almost every $x$. By \thref{lem:pp-prop-p36}, this equality condition is equivalent to
		\begin{align*}
			\int{\sum_{j = 1}^k{m[j \mid \sigma^{-1}{x}] \psi(j, x_0, x_1, \dots)}\ dm} &= \int{\sum_{j = 1}^k{\phi(j, x_0, x_1, \dots) \psi(j, x_0, x_1, \dots)}\ dm} \\
				&= \int{\psi\ dm}
		\end{align*}
		for all $\psi \in C(\Sigma, \reals)$. In other words, $\int{L_\phi{\psi}\ dm} = \int{g\ dm}$ for all $\psi$, and so we have $L_\phi^*{m} = m$. By Ruelle's Perron-Frobenius Theorem \ref{rpf:3}, $\mu$ is the unique $\sigma$-invariant probability measure such that $L_\phi^*{\mu} = \mu$, so we have
		\[
			h_m(\sigma) + \int{\phi\ dm} \leq 0,
		\]
		with equality if and only if $m = \mu$.
	\end{proof}
\end{lemma}

It can be shown that similar results hold for 2-sided shifts of finite type, and also for $\phi \in F_\theta$ where $L_\phi$ is not necessarily normalised.

The following result shows one of the main distinguishing characteristics of Gibbs measures.

\begin{theorem}[Variational Principle] \label{thm:variational-principle}
	Suppose that $\phi \in F_\theta$ (or $F_\theta^+$). The Gibbs measure $\mu_\phi$ is the unique $\sigma$-invariant probability measure such that
	\[
		h_m(\sigma) + \int{\phi\ dm} \leq h_{\mu_\phi}(\sigma) + \int{\phi\ d\mu_\phi}
	\]
	for all $m \in M(\Sigma, \sigma)$, with equality if and only if $m = \mu_\phi$.
	\begin{proof}
		Let $\phi \in F_\theta$. By \thref{prop:pp-1-2}, there exists $g \in F_{\theta^{1 / 2}}^+$, $u \in F_{\theta^{1 / 2}}$ such that $\phi = g + (u \circ \sigma) - u$. By Ruelle's Perron-Frobenius Theorem, we can write $g = \log(h \circ \sigma) - \log{h} + \log{\lambda} + k$, for some $k \in F_{\theta^{1/2}}^+$ such that $L_k^*{\mu_\phi} = \mu_\phi$ and $L_k$ is normalised.
		
		Let $m$ be a $\sigma$-invariant probability measure. By \thref{lem:pp-prop-3-4} we have
		\[
			h_m(\sigma) + \int{k\ dm} \leq h_{\mu_\phi}(\sigma) + \int{k\ d\mu_\phi} = 0.
		\]
		Substituting in $k = \phi - (u \circ \sigma) + u - \log(h \circ \sigma) + \log{h} - \log{\lambda}$ and cancelling terms, we get
		\[
			h_m(\sigma) + \int{\phi\ dm} \leq h_{\mu_\phi}(\sigma) + \int{\phi\ d\mu_\phi},
		\]
		with equality if and only if $m = \mu_\phi$.
	\end{proof}
\end{theorem}

From the proof of \thref{thm:variational-principle}, we see that
\[
	P(\phi) = \sup_{m \in M(\Sigma, \sigma)}\left\{h_m(\sigma) + \int{\phi\ dm}\right\} = h_{\mu_\phi}(\sigma) + \int{\phi\ d\mu_\phi}.
\]
So $P(\phi) = \log{\lambda}$, where $\lambda$ is the maximal positive eigenvalue for $L_{\phi'}$, where $\phi$ is cohomologous to $\phi' \in F_{\theta^{1 / 2}}^+$.

\begin{definition}
	We call
	\[
		P(\phi) := \sup_{m \in M(\Sigma, \sigma)}\left\{h_m(\sigma) + \int{\phi\ dm}\right\}
	\]
	the \key{pressure} of $\phi$.
	
	If a measure a $\sigma$-invariant probability measure $m$ attains this supremum, i.e. $P(\phi) = h_m(\sigma) + \int{\phi\ dm}$, then we say that $m$ is an \key{equilibrium state}.
\end{definition}

The Variational Principle gives that if we have $\phi \in F_\theta$, then the equilibrium state is unique and we can also define the pressure of $\phi$ by $P(\phi) = \log{\lambda}$.

\section{Gibbs measures are weak Bernoulli}
We now describe a particular property of Gibbs measures which we will use in Subsection \ref{ssec:hitting-times}. The following definitions and results follow \cite[Section 1.E]{bowen:equilibrium}.

\begin{definition}
	Let $\beta, \gamma$ be two finite partitions of a measure space $(X, \B, \mu)$ and let $\varepsilon > 0$ be given. We say that $\beta$ and $\gamma$ are \key{$\varepsilon$-independent} if
	\[
		\sum_{B \in \beta,\ C \in \gamma}{|\mu(B \cap C) - \mu(B)\mu(C)|} < \varepsilon.
	\]
\end{definition}

\begin{definition}
	Let $\xi = \left\{[j] \mid j \in \{1, \dots, k\}\right\}$ be the partition of $(\Sigma, \B, \mu)$ by cylinders of length 1. We say that $\xi$ is \key{weak Bernoulli} (for $\sigma$ and $\mu$) if for all $\varepsilon > 0$ there exists $N(\varepsilon) > 0$ such that for all $n \geq 1$, then the partitions
	\[
		\beta = \bigjoin_{j = 0}^n{\sigma^{-j}(\xi)} \quad \text{and} \quad \gamma = \bigjoin_{j = t}^{t + r}{\sigma^{-j}(\xi)}
	\]
	are $\varepsilon$-independent for all $r \geq 0$ and for all $t \geq n + N(\varepsilon)$.
\end{definition}

Before we state the main result, we need an auxiliary lemma.

\begin{lemma}\label{bowen:lem-1-12}
	Let $r \geq 0$, $f \in C(\Sigma, \reals)$ and $\var_r(f) = 0$. Let
	\begin{align*}
		F \in \{g \in C(\Sigma, \reals) \mid g & \geq 0,\ \nu(g) = 1,\ g(x) \leq B_m g(x') \text{ whenever } x_j = x'_j \text{ for all } j = 0, \dots, m\},
	\end{align*}
	where $B_m := \exp\left(\sum_{k = m + 1}^\infty{2b\alpha^k}\right)$, where $b > 0$, $\alpha \in (0, 1)$ are any pair of constants which satisfy $\var_k(\phi) \leq b\alpha^k$ for all $k > 0$.
	
	Then for any $n \geq 0$ we have
	\[
		\|\lambda^{-n - r}L_\phi^{n + r}(fF) - \nu(fF)h\| \leq M\nu(|fF|)\rho^n,
	\]
	where $\nu$, $\lambda$, $h$ are as in the Ruelle's Perron Frobenius Theorem, and $M > 0$, $\rho \in (0, 1)$ are constants.
\end{lemma}

\begin{theorem}\label{thm:gibbs-is-weak-bernoulli}
	Let $\xi = \left\{[j] \mid j \in \{1, \dots, k\}\right\}$ be the partition of $(\Sigma, \B, \mu)$ by cylinders of length 1. Then $\xi$ is weak Bernoulli for the Gibbs measure $\mu_\phi$.
	\begin{proof}
		Let $\varepsilon > 0$ be given. Suppose that $\phi \in C(\Sigma, \reals)$, $n \geq 1$ and $t \geq n + N(\varepsilon)$, for some $N(\varepsilon)$. Let $\beta, \gamma$ be partitions of $\Sigma$ defined by
		\[
		\beta = \bigjoin_{j = 0}^n{\sigma^{-j}(\xi)} \quad \text{and} \quad \gamma = \bigjoin_{j = t}^{t + r}{\sigma^{-j}(\xi)}.
		\]
		For all $B \in \beta$ we clearly have $\chi_B \in C(\Sigma, \reals)$ and $\var_r(\chi_B) = 0$ for all $r \geq 0$.
		
		Now consider $C \in \gamma$. Since $t \geq n$, we know that $B$ consists of cylinders of lengths strictly less than the lengths of cylinders in $C$, and therefore $\sigma^{-t}C$. It follows that the intersection $B \cap C$ depends only on $B$, and hence
		\[
			\mu_\phi(B \cap C) = \mu_\phi(B \cap \sigma^{-t}{C}).
		\]
		We then have, where $\nu$, $h$ and $\lambda$ are as in the Ruelle's Perron Frobenius Theorem,
		\begin{align*}
			\mu_\phi(B \cap C) &= \mu_\phi(B \cap \sigma^{-t}C) \\
				&= \mu_\phi(\chi_B \cdot \chi_{\sigma^{-t}}{C}) \\
				&= \mu_\phi(\chi_B \cdot (\chi_{C} \circ \sigma^t)) \\
				&= \nu(h \chi_B \cdot (\chi_C \circ \sigma^t)) \\
				&= \lambda^{-t}(L_\phi^*)^t\nu(h\chi_B \cdot (\chi_C \circ \sigma^t)) \\
				&= \nu(\lambda^{-t} L_\phi^t(h \chi_B \cdot (\chi_C \circ \sigma^t))) \\
				&= \nu(\lambda^{-t} L_\phi^t(h \chi_B) \cdot \chi_C).
		\end{align*}
		Consequently,
		\begin{align*}
			|\mu_\phi(B \cap C) - \mu_\phi(B)\mu_\phi(C)| &= |\nu(\lambda^{-t} L_\phi^t(h \chi_B) \cdot \chi_C) - \nu(h \chi_B)\nu(h \chi_C)| \\
				&= |\nu((\lambda^{-t} L_\phi^t(h \chi_B) - \nu(h \chi_B)h)\chi_C)| \\
				&\leq \|\lambda^{-t} L_\phi^t(h \chi_B) - \nu(h \chi_B)h\| \nu(C).
		\end{align*}
		Since $\chi_B \in C(\Sigma, \reals)$ and $\var_r(\chi_B) = 0$ for all $r \geq 0$, we may apply \thref{bowen:lem-1-12}. So if $t \geq s$, then
		\[
			\|\lambda^{-t}L_\phi^{t}(h \chi_B) - \nu(h \chi_B)h\| \leq M\nu(h \chi_B)\rho^{t - s},
		\]
		where $M > 0$, $\rho \in (0, 1)$ are constants. We therefore have
		\begin{align*}
			|\mu_\phi(B \cap C) - \mu_\phi(B)\mu_\phi(C)| &\leq M\nu(h \chi_B)\rho^{t - s} \nu(C) \\
				&= M\mu_\phi(B)\rho^{t - s} \nu(C) \\
				&= M'\mu_\phi(B)\mu_\phi(C)\rho^{t - s},
		\end{align*}
		where $M' = M(\inf{h})^{-1}$. Summing over all elements in the partitions $\beta, \gamma$, we get
		\[
			\sum_{B \in \beta,\ C \in \gamma}{|\mu_\phi(B \cap C) - \mu_\phi(B)\mu_\phi(C)|} \leq M'\rho^{t - s} < \varepsilon
		\]
		for sufficiently large $t - s$.
		
		Hence $\mu_\phi$ is weak Bernoulli.
	\end{proof}
\end{theorem}

\chapter[Concentration bounds for entropy estimation]{Concentration bounds for entropy estimation of Gibbs measures}\label{chap:concentration-bounds}
\section{Overview}
We now have the required background knowledge to discuss \cite{chazottes-maldonado:cbfee}, which is concerned with methods for estimating the entropy of Gibbs measures. In particular, we will show that these methods yield estimated entropy values which concentrate around the actual value of entropy.

\section{Restrictions and notation}
\subsection{Full shifts}
Let $A$ be a finite alphabet. Throughout this chapter, $\Sigma = \{(x_j)_{j = 0}^\infty \mid x_j \in A\}$ will denote the full (one-sided) shift with the (one-sided) shift map $\sigma : \Sigma \to \Sigma$. Once again, we will write $x_m^n := (x_j)_{j = m}^n$. We will use the definitions from Section \ref{sec:entropy:sft}.

\subsection{Gibbs measures}
Let $\phi \in F_\theta$. Recall that if there exist constants $C = C(\phi) > 1$, $P = P(\phi)$ such that
\[
	C^{-1} \leq \frac{\mu_\phi[x_0^{n - 1}]}{\exp\left(-Pn + \sum_{j = 0}^{n - 1}{\phi(\sigma^j x)}\right)} \leq C,
\]
then $\mu_\phi$ is a Gibbs measure for $\phi$.

Throughout this chapter, we will assume that the pressure $P$ of $\phi$ is zero. With this in mind, recall that
Gibbs measures satisfy the Variational Principle (\thref{thm:variational-principle}), so we have
\[
	0 = P = h(\mu_\phi) + \int{\phi\ d\mu_\phi}.
\]
This gives the identity
\begin{equation}\label{fml:vp-identity}
	h(\mu_\phi) = -\int{\phi\ d\mu_\phi},
\end{equation}
which will be useful later in this chapter.

\subsection{Probability theory}
We will use and prove some results related to a couple of basic ideas in probability theory.
\begin{definition}
	The \key{expectation} of a continuous function $f : X \to \reals$ with respect to a probability measure $\mu$ is given by
	\[
		E_\mu(f) := \int{f\ d\mu}.
	\]
	The expectation is a weighted average, or mean, of the values $f$ takes.~\cite[p127]{gray:probability}
\end{definition}

\begin{definition}
	The \key{variance} of a function $f : X \to \reals$ with respect to a probability measure $\mu$ is given by
	\[
		\Var_{\mu}(f) := \int{\left(K(x) - \int{K(y)\ d\mu(y)}\right)^2\ d\mu(x)}.
	\]
	The variance gives a measurement of the spread of the values $f$ takes.
\end{definition}

\section{An exponential inequality}
Throughout the remainder of this chapter, we will assume that $\phi \in F_\theta$ and that $\mu_\phi$ is the unique Gibbs measure for $\phi$, which follows since $P(\phi) = 0$.

The results in Section \ref{sec:estimator-bounds} utilise an exponential inequality proved in \cite{collet-martinez-schmitt:exp-ineq}. We first introduce some definitions which are used in the inequality.

\begin{definition}
	Let $K : \Sigma^n \to \reals$ be a function of $n$ variables in $\Sigma$. For $j = 0, \dots, n - 1$, we define
	\[
		\Lip_j(K) := \sup_{\substack{x^{(0)}, x^{(1)}, \dots, x^{(n - 1)} \\ y^{(j)} \neq x^{(j)}}}{\frac{\left|K(x^{(0)}, \dots, x^{(n - 1)}) - K(x^{(0)}, \dots, x^{(j - 1)}, y^{(j)}, x^{(j + 1)}, \dots, x^{(n - 1)})\right|}{d_\theta(x^{(j)}, y^{(j)})}},
	\]
	where each $x^{(k)}, y^{(k)}$ is a sequence in $\Sigma$ for $k = 0, \dots, n - 1$.
	
	In other words, $\Lip_j(K)$ is a measurement of how much $K$ varies when the $j$-th variable is changed. There is an additional weighting which depends on how close the old $j$-th variable is to the new $j$-th variable.
	
	If $\Lip_j(K) < +\infty$ for all $j = 0, \dots, n - 1$, then we say that $K$ is a \key{separately Lipschitz function} of $n$ variables.
\end{definition}

We now present the exponential inequality, which appears in its original form in \cite[Theorem I.1]{collet-martinez-schmitt:exp-ineq}.

\begin{theorem}\label{thm:cm-3-1}
	Let $\mu_\phi$ be a Gibbs measure. Then there exists some constant $D = D(\phi) > 0$ such that
	\begin{equation}\label{fml:cms-exp-ineq}
		\int{e^{K(x, \dots, \sigma^{n - 1}{x})}\ d\mu_\phi(x)} \leq e^{\int{K(y, \dots, \sigma^{n - 1}{y})\ d\mu_\phi(y)}} \; e^{D \sum_{j = 0}^{n - 1}{\Lip_j^2(K)}}.
	\end{equation}
	 for all $n \in \naturals$ and any separately Lipschitz function $K : \Sigma^n \to \reals$.
\end{theorem}

We may write \eqref{fml:cms-exp-ineq} more succinctly by defining a measure $\mu_\phi^{(n)}$ on $\Sigma^n$ by
\[
	d\mu_\phi^{(n)}(x^{(0)}, \dots, x^{(n - 1)}) = d\mu_\phi(x^{(0)}) \prod_{j = 1}^{n - 1}{\delta(x^{(j)} - \sigma^j{x^{(0)}})},
\]
where $\delta(x - a) = \delta_a(x)$ is the Dirac measure at $a$. Using this, \eqref{fml:cms-exp-ineq} becomes
\[
	\int{e^K\ d\mu_\phi^{(n)}} \leq e^{\int{K\ d\mu_\phi^{(n)}}} \; e^{D \sum_{j = 0}^{n - 1}{\Lip_j^2(K)}}.
\]

\subsection{Important results}
We will mostly use the results in this subsection which follow from \thref{thm:cm-3-1}. To begin, we need Markov's Inequality.

\begin{lemma}[Markov's Inequality]\label{lem:markov-ineq}
	Let $f \geq 0$ be a nonnegative measurable function on a measure space $(X, \B, \mu)$. Then for all $t > 0$ we have
	\[
		\mu\{x \in X \mid f(x) \geq t\} \leq \frac{1}{t}\int{f\ d\mu}.
	\]
	
	\begin{proof}
		We follow the proof given in \cite[Theorem 3.1.1]{athreya-lahiri:measure-theory}.
		
		We have $f \geq 0$ and so
		\begin{align*}
			\int{f\ d\mu} &\geq \int_{\{f \geq t\}}{f\ d\mu} \\ &= \mu\{f(x) \mid x \in X \text{ and } f(x) \geq t \} \\ &\geq t\mu\{x \in X \mid f(x) \geq t \}.
		\end{align*}
		The result follows by dividing by $t$.
	\end{proof}
\end{lemma}

\begin{corollary}\label{cor:cm-3-1}
	For all $t > 0$ we have
	\begin{align}\label{fml:cm-4}
		\mu_\phi&\left\{x \midmid K(x, \sigma{x}, \dots, \sigma^{n - 1}{x}) \geq \int{K(y, \sigma{y}, \dots, \sigma^{n - 1}{y})\ d\mu_\phi(y)} + t\right\} \nonumber \\
			&\leq \exp\left(-\frac{t^2}{4D\sum_{j = 0}^{n - 1}{\Lip_j^2(K)}}\right),
	\end{align}
	for all $n \in \naturals$ and all separately Lipschitz $K$.
	\begin{proof}
		Since $K$ is a separately Lipschitz function, it is clear that $\lambda K$ is also a separately Lipschitz function for all $\lambda > 0$.
		
		Now, for all $\lambda > 0$ and $t > 0$ we have
		\begin{align}
			\mu_\phi&\left\{x \midmid K(x, \sigma{x}, \dots, \sigma^{n - 1}{x}) \geq \int{K(y, \sigma{y}, \dots, \sigma^{n - 1}{y})\ d\mu_\phi(y)} + t\right\} \nonumber \\
				&= \mu_\phi\left\{x \midmid e^{\lambda\left[K(x, \sigma{x}, \dots, \sigma^{n - 1}{x}) - \int{K(y, \sigma{y}, \dots, \sigma^{n - 1}{y})\ d\mu_\phi(y)}\right]} \geq e^{\lambda t}\right\} \nonumber \\
				&\leq \frac{1}{e^{\lambda t}} \int{e^{\lambda\left[K(x, \sigma{x}, \dots, \sigma^{n - 1}{x}) - \int{K(y, \sigma{y}, \dots, \sigma^{n - 1}{y})\ d\mu_\phi(y)}\right]}\ d\mu_\phi(x)} \nonumber \\
				& \leq \exp(-\lambda t) \exp\left(\lambda^2 D \sum_{j = 0}^{n - 1}{\Lip_j^2(K)}\right). \label{fml:cor-3-1-exp}
		\end{align}
		(We have used \thref{lem:markov-ineq} on the penultimate line and \thref{thm:cm-3-1} on the last line.)
		
		We now optimise \eqref{fml:cor-3-1-exp} over $\lambda$, that is, we find the value of $\lambda$ for which \eqref{fml:cor-3-1-exp} has derivative zero with respect to $\lambda$. We have
		\begin{align*}
			0 &= \frac{d}{d\lambda} \exp\left(-\lambda t + \lambda^2 D \sum_{j = 0}^{n - 1}{\Lip_j^2(K)}\right) \\
				&= \left(-t + 2\lambda D \sum_{j = 0}^{n - 1}{\Lip_j^2(K)}\right) \exp\left(-\lambda t + \lambda^2 D \sum_{j = 0}^{n - 1}{\Lip_j^2(K)}\right).
		\end{align*}
		Since $\exp\left(-\lambda t + \lambda^2 D \sum_{j = 0}^{n - 1}{\Lip_j^2(K)}\right) > 0$ for all $\lambda > 0$, we get that
		\[
			\lambda = \frac{t}{2D \sum_{j = 0}^{n - 1}{\Lip_j^2(K)}}.
		\]
		Substituting this value of $\lambda$ into \eqref{fml:cor-3-1-exp} gives
		\begin{align*}
			\mu_\phi&\left\{x \midmid K(x, \sigma{x}, \dots, \sigma^{n - 1}{x}) \geq \int{K(y, \sigma{y}, \dots, \sigma^{n - 1}{y})\ d\mu_\phi(y)} + t\right\} \\
			 &\leq \exp\left(-\frac{t^2}{2D \sum_{j = 0}^{n - 1}{\Lip_j^2(K)}} + \frac{t^2 D \sum_{j = 0}^{n - 1}{\Lip_j^2(K)}}{4D^2 \left(\sum_{j = 0}^{n - 1}{\Lip_j^2(K)}\right)^2}\right) \\
			 &= \exp\left(-\frac{t^2}{4D\sum_{j = 0}^{n - 1}{\Lip_j^2(K)}}\right),
		\end{align*}
		and the result holds.
	\end{proof}
\end{corollary}

The following result is an immediate consequence of \thref{cor:cm-3-1}.

\begin{corollary}\label{cor:cm-3-1-5}
	For all $t > 0$ we have
	\begin{align}\label{fml:cm-4-5}
		\mu_\phi&\left\{x \midmid \left| K(x, \sigma{x}, \dots, \sigma^{n - 1}{x}) - \int{K(y, \sigma{y}, \dots, \sigma^{n - 1}{y})\ d\mu_\phi(y)} \right| \geq t\right\} \nonumber \\
		&\leq 2\exp\left(-\frac{t^2}{4D\sum_{j = 0}^{n - 1}{\Lip_j^2(K)}}\right),
	\end{align}
	for all $n \in \naturals$ and all separately Lipschitz $K$.
	\begin{proof}
		Apply \thref{cor:cm-3-1} for $K$ and $-K$ and then the result follows by countable subadditivity.
	\end{proof}
\end{corollary}

We can also find an upper bound for the variance of $K$ with respect to $\mu_\phi$.

\begin{corollary} \label{cor:cm-3-2}
	For all $n \in \naturals$ and all separately Lipschitz functions $K$, we have
	\[
		\Var_{\mu_\phi}(K) := \int{\left[K(x, \dots, \sigma^{n - 1}{x}) - \int{K(y, \dots, \sigma^{n - 1}{y})\ d\mu_\phi(y)}\right]^2\ d\mu_\phi(x)} \leq 2D\sum_{j = 0}^{n - 1}{\Lip_j^2(K)}.
	\]
	
	\begin{proof}
		Suppose that $\lambda \neq 0$. Applying \thref{thm:cm-3-1} to $\lambda K$, subtracting $1$ and then dividing by $\lambda^2$, we get
		\begin{equation} \label{fml:cor-3-2-variance}
			\frac{1}{\lambda^2}\left(\int{e^{\lambda\left[K(x, \dots, \sigma^{n - 1}{x}) - \int{K(y, \dots, \sigma^{n - 1}{y})\ d\mu_\phi(y)}\right]}\ d\mu_\phi(x)} - 1\right) \leq \frac{1}{\lambda^2}\left(e^{\lambda^2 D \sum_{j = 0}^{n - 1}{\Lip_j^2(K)}} - 1\right).
		\end{equation}
		Now put $L := K(x, \dots, \sigma^{n - 1}{x}) - \int{K(y, \dots, \sigma^{n - 1}{y})\ d\mu_\phi(y)}$ and note that $\int{L\ d\mu_\phi(x)} = 0$. The Taylor expansion of the left-hand side of \eqref{fml:cor-3-2-variance} is
		\begin{align*}
			\frac{1}{\lambda^2}&\left[\int{\left(1 + \lambda L + \frac{\lambda^2 L^2}{2!} + \frac{\lambda^3 L^3}{3!} + \dots\right)\ d\mu_\phi(x)} - 1\right] \\
				&= \frac{1}{\lambda^2}\left(\int{1 + \lambda L\ d\mu_\phi(x)} - 1\right) + \frac{L^2}{2} + \int{O(\lambda)\ d\mu_\phi(x)} \\
				&= \frac{L^2}{2} + \int{O(\lambda)\ d\mu_\phi(x)} \\
				&\to \frac{L^2}{2},
		\end{align*}
		as $\lambda \to 0$.
		
		The Taylor expansion of the right-hand side of \eqref{fml:cor-3-2-variance} is
		\begin{align*}
			\frac{1}{\lambda^2}&\left[\left(1 + \lambda^2 D \sum_{j = 0}^{n - 1}{\Lip_j^2(K)} + \lambda^4 D^2 \left(\sum_{j = 0}^{n - 1}{\Lip_j^2(K)} \right)^2 + \dots\right) - 1 \right] \\
				&= D \sum_{j = 0}^{n - 1}{\Lip_j^2(K)} + O(\lambda^2) \\
				&\to D \sum_{j = 0}^{n - 1}{\Lip_j^2(K)},
		\end{align*}
		as $\lambda \to 0$. The result then follows by combining the two sides.
	\end{proof}
\end{corollary}

We can also apply the above results to find results about ergodic averages.

\begin{corollary}\label{cor:cm-3-3}
	Let $f : \Sigma \to \reals$ be a Lipschitz function. Then for all $t > 0$ and all $n \geq 1$ we have
	\begin{equation}
		\mu_\phi\left\{x \midmid \frac{1}{n}\sum_{j = 0}^{n - 1}{f(\sigma^j{x})} - \int{f\ d\mu_\phi} \geq t\right\} \leq \exp(-Bnt^2),
	\end{equation}
	where $B := (4D|f|_\theta^2)^{-1}$.
	\begin{proof}
		Let $K_0(x^{(0)}, x^{(1)}, \dots, x^{(n - 1)}) := f(x^{(0)}) + f(x^{(1)}) + \dots + f(x^{(n - 1)})$. Since $f$ is Lipschitz, it follows that $K_0$ is separately Lipschitz. In particular, note that for all $j = 0, \dots, n - 1$, we have
		\begin{align*}
			\Lip_j(K_0) &= \sup_{y^{(j)} \neq x^{(j)}}{\frac{|f(x^{(j)}) - f(y^{(j)})|}{d_\theta(x^{(j)}, y^{(j)})}} \\
				&\leq \sup_{m \geq 0}\left\{\frac{\var_m(f)}{\theta^m}\right\} \\
				&= |f|_\theta.
		\end{align*}
		
		Now we apply \thref{cor:cm-3-1} to $\frac{1}{n}K_0$, so that
		\begin{align*}
			\mu_\phi&\left\{x \midmid \frac{1}{n}\sum_{j = 0}^{n - 1}{f(\sigma^j{x})} - \frac{1}{n}\int{\sum_{j = 0}^{n - 1}{f(\sigma^j{y})}\ d\mu_\phi(y)} \geq t\right\} \\
			&=\mu_\phi\left\{x \midmid \frac{1}{n}\sum_{j = 0}^{n - 1}{f(\sigma^j{x})} - \int{f\ d\mu_\phi} \geq t\right\} \\
			&\leq \exp\left(-\frac{t^2}{4D\frac{1}{n^2}\sum_{j = 0}^{n - 1}{\Lip_j^2(K_0)}}\right) \\
			&\leq \exp\left(-\frac{nt^2}{4D|f|_\theta^2}\right) \\
			&= \exp(-Bnt^2).
		\end{align*}
	\end{proof}
\end{corollary}

This result implies that, as $t$ increases, the ergodic average $\frac{1}{n}\sum_{j = 0}^{n - 1}{f(\sigma^j{x})}$ is exponentially less likely to deviate from the space average $\int{f\ d\mu_\phi}$. In other words, the ergodic average concentrates around the space average, hence the term \emph{concentration bound}.

\subsection{Functions of \texorpdfstring{$n$}{n} symbols}
We will consider entropy estimators that utilise functions of $n$ \emph{symbols}, as opposed to functions of $n$ variables in $\Sigma$. We will discuss the estimators in more detail in the next section.

It is clear that we can identify a function $\tilde{K} : A^n \to \reals$ of $n$ symbols with a function $K : \Sigma^n \to \reals$ of $n$ sequences. We can therefore apply \thref{thm:cm-3-1} and its corollaries to $\tilde{K}$. However, for each $j = 0, \dots, n - 1$, we must replace $\Lip_j(K)$ in the above results with
\[
	\delta_j(\tilde{K}) := \sup_{\substack{a_0^{n - 1} \in A^k \\ b_j \neq a_j}}{|\tilde{K}(a_0, \dots, a_{n - 1}) - \tilde{K}(a_0, \dots, a_{j - 1}, b_j, a_{j + 1}, \dots, a_{n - 1})|},
\]
the \key{oscillation at the $j$-th coordinate}.

\section{Entropy estimators}\label{sec:estimator-bounds}
\subsection{Motivation}
Suppose we have an ergodic source with a typical sample output $(x_0, x_1, \dots, x_{n - 1}) \in A^n$. If we do not know the measure-preserving transformation which yields this output, we have to use alternative methods to estimate its entropy. To do this, we will consider \key{plug-in estimators} and the \key{hitting-time estimator}.

There are theorems which show that these estimators tend to the entropy $h(\nu)$, as $n \to +\infty$, for almost every sample sequence $(x_0, \dots, x_{n - 1})$. However, as we would expect, the values given by these estimators have some margin of error when $n$ is relatively small, i.e. they \key{fluctuate} around $h(\nu)$. Our aim is to find concentration bounds for these fluctuations.

\subsection{Plug-in estimators}
Before we define any \key{plug-in estimators}, we first introduce some definitions. The main concept employed by plug-in estimators is the \key{empirical frequency} with which a word $a_0^{k - 1}$ of length $k$ occurs in a sample path $x_0^{n - 1}$ of length $n$.\footnote{The word ``empirical'' relates to information gained by observations. In our case, empirical frequency is the frequency observed by taking a sample path and matching it with a word of length $k$.}

\begin{definition}
	The \key{empirical frequency} with which the word $a_0^{k - 1}$ occurs in $x_0^{n - 1}$ is given by
	\[
		\E_k(a_0^{k - 1}; x_0^{n - 1}) := \frac{1}{n}\card\left\{j \in \{0, \dots, n - 1\} \midmid \tilde{x}_j^{j + k - 1} = a_0^{k - 1} \right\},
	\]
	where $\tilde{x} = x_0^{n - 1} x_0^{n - 1} x_0^{n - 1} \dots$ is the sequence of period $n$ obtained by concatenating $x_0^{n - 1}$ continually, i.e. $\tilde{x}_j = x_{j \bmod n}$.
\end{definition}

The definition of $\tilde{x}$ means that $\E_k(\seedot; x_0^{n - 1})$ is a locally $\sigma$-invariant probability measure on $A^k$. That is, for all words $a_1^k \in A^k$ there exists some $a_0 \in A$ such that
\[
	\E_k(a_0^{k - 1}; x_0^{n - 1}) = \E_k(a_1^k; x_0^{n - 1}).
\]

Given an ergodic measure $\nu$, Birkhoff's Ergodic Theorem gives that for $\nu$-almost every $x \in \Sigma$ and for all $k \geq 1$, we have
\[
	\E_k(a_0^{k - 1}; x_0^{n - 1}) = \frac{1}{n}\card\left\{j \in \{0, \dots, n - 1\} \midmid \tilde{x}_j^{j + k - 1} = a_0^{k - 1} \right\} \to \nu[a_0^{k - 1}],
\]
as $n \to +\infty$.

We are now in the position to define the following examples of plug-in entropy estimators, which can be found in \cite[Definition 2.1]{chazottes-gabrielle:large-deviations}.

\begin{definition}
	Suppose $x_0^{n - 1} \in A^n$ is a word of length $n$.
	
	For $k \geq 1$, the \key{$k$-block empirical entropy} is defined
	\[
		\hat{H}_k(x_0^{n - 1}) := H_k(\E_k(\seedot; x_0^{n - 1})).
	\]
	
	For $k \geq 2$, the \key{$k$-block conditional empirical entropy} is defined
	\[
		\hat{h}_k(x_0^{n - 1}) := h_k(\E_k(\seedot; x_0^{n - 1})).
	\]
\end{definition}

We have that
\[
	\frac{1}{k} \hat{H}_k(x_0^{n - 1}) = \frac{1}{k} H_k(\E_k(\seedot; x_0^{n - 1})) \to \frac{1}{k} H_k(\nu),
\]
as $n \to +\infty$, for $\nu$-almost every $x \in \Sigma$. Recalling that $\frac{1}{k} H_k(\nu) \to h(\nu)$, as $k \to +\infty$, we have
\[
	\lim_{k \to +\infty} \lim_{n \to +\infty}{\frac{1}{k} \hat{H}_k(x_0^{n - 1})} = h(\nu).
\]

We can actually remove one of these limits: By \thref{prop:walters-cor-4-2-1}, we easily see that $0 \leq h(\nu) \leq \log{|A|}$ and so we have
\[
	\frac{\log{n}}{\log{|A|}} \leq \frac{\log{n}}{h(\nu)}.
\]
Hence we can define a monotonically increasing function $k : \naturals \to \naturals$ such that $k(n) \leq \frac{\log{n}}{h(\nu)}$. Then a result by Ornstein and Weiss~\cite{shields:ergodic} shows that
\[
	\lim_{n \to +\infty}{\frac{1}{k(n)}\hat{H}_{k(n)}(x_0^{n -1})} = h(\nu),
\]
for $\nu$-almost every $x \in \Sigma$.

For conditional empirical entropy, we have the following result proved in \cite[Theorem II.3.5]{shields:ergodic}.

\begin{theorem}[Entropy-Estimation Theorem] \label{thm:entropy-estimation}
	 Let $\nu$ be ergodic measure and let $\alpha \in (0, 1)$. Let $k : \naturals \to \naturals$ be a function such that $k(n) \leq \frac{(1 - \alpha)}{\log{|A|}}\log{n}$. Then
	 \[
	 	\lim_{n \to +\infty}{\hat{h}_{k(n)}(x_0^{n - 1})} = h(\nu),
	 \]
	 for $\nu$-almost every $x \in \Sigma$.
\end{theorem}

Therefore both kinds of empirical entropy can be used for estimating the actual entropy $h(\nu)$.

We now present a concentration bound for the $k$-block empirical entropy about its expectation value.

\begin{theorem}\label{thm:cm-4-1}
	For all $\alpha \in (0, 1)$, $t > 0$ and $n \geq 2$, if
	\[
		k(n) \leq \frac{\alpha}{2 \log{|A|}}\log{n},
	\]
	then
	\[
		\mu_\phi\left\{\left|\frac{\hat{H}_{k(n)}}{k(n)} - E_{\mu_\phi}\left(\frac{\hat{H}_{k(n)}}{k(n)}\right)\right| \geq t\right\} \leq \exp\left(-\frac{n^{1 - \alpha} t^2}{16D(\log{n})^2}\right),
	\]
	where $D > 0$ is as in \thref{thm:cm-3-1}.
	
	Furthermore, for all $n \geq 2$ we have
	\[
		\Var_{\mu_\phi}\left(\frac{\hat{H}_{k(n)}}{k(n)}\right) \leq 8D\frac{(\log{n})^2}{n^{1 - \alpha}}.
	\]
	\begin{proof}
		Let $k \in \naturals$ and define a function $\tilde{K} : A^n \to \reals$ by $\tilde{K}(s_0, \dots, s_{n - 1}) = \hat{H}_k(s_0^{n - 1})$. Recall that
		\[
			\delta_j(\tilde{K}) := \sup_{\substack{s_0^{n - 1} \in A^k \\ r_j \neq s_j}}{|\tilde{K}(s_0, \dots, s_{n - 1}) - \tilde{K}(s_0, \dots, s_{j - 1}, r_j, s_{j + 1}, \dots, s_{n - 1})|}
		\]
		replaces $\Lip_j$ in \thref{thm:cm-3-1} and its corollaries. To utilise these results, we must estimate $\delta_j$.
		\begin{claim}
			We claim that
			\begin{equation}
				\delta_j(\tilde{K}) \leq 2k|A|^k\frac{\log{n}}{n}, \label{fml:oscil-est}
			\end{equation}
			for $j = 0, \dots, n - 1$.
			
			Indeed, first suppose that $a_0^{k - 1} \in A^k$ is given. By the nature of how $\delta_j(\tilde{K})$ is defined, we want to consider the effect on the value of $\tilde{K}$ when one coordinate in $s_0^{n - 1}$ is changed. Since $\tilde{K}$ is defined in terms of the empirical frequency, we consider the largest effect on $\E_k(a_0^{k - 1}; s_0^{n - 1})$.
			
			If we change exactly one symbol $s_j$ in $s_0^{n - 1}$, then the value of $\E_k(a_0^{k - 1}; s_0^{n - 1})$ can decrease by \emph{at most} $k / n$. This is because, in the worst case scenario, $s_j$ matches all $k$ characters in $a_0^{k - 1}$ (and hence $\tilde{a}$). Likewise, changing one symbol in $s_0^{n - 1}$ can increase the value of $\E_k(a_0^{k - 1}; s_0^{n - 1})$ by at most $k / n$.
			
			By \thref{prop:logs-thm-4-1}, for $\ell, k \in \naturals$ with $\ell+ k \leq n$, we have that
			\[
				\left|\left(\frac{\ell}{n}\right)\log\left(\frac{\ell}{n}\right) - \left(\frac{\ell + k}{n}\right)\log\left(\frac{\ell + k}{n}\right)\right| \leq \frac{k}{n}\log{n}.
			\]
			Since $\tilde{K}(s_0^{n - 1}) = H_k(\E_k(\seedot; s_0^{n - 1}))$, by all the above results we have
			\begin{align*}
				\delta_j(\tilde{K}) &\leq 2\sum_{a_0^{k - 1} \in A^k}{\left|\left(\frac{\ell}{n}\right)\log\left(\frac{\ell}{n}\right) - \left(\frac{\ell + k}{n}\right)\log\left(\frac{\ell + k}{n}\right)\right|} \\
				 &\leq 2k|A|^k \frac{\log{n}}{n},
			\end{align*}
			for all $j = 0, \dots, n - 1$. Hence the claim holds.
		\end{claim}
		
		Now take $k(n) \leq \frac{\alpha}{2\log{|A|}}\log{n}$, where $\alpha \in (0, 1)$. By rearranging this, we have the relation $|A|^{2k(n)} \leq n^\alpha$. We now apply \thref{cor:cm-3-1-5} so that for all $t > 0$,
		\begin{align*}
			\mu_\phi\left\{\left| \frac{\hat{H}_{k(n)}}{k(n)} - E_{\mu_\phi}\left(\frac{\hat{H}_{k(n)}}{k(n)}\right) \right| \geq t\right\} &= \mu_\phi\left\{\left| \frac{\tilde{K}}{k(n)} - \int{\frac{\tilde{K}}{k(n)}\ d\mu_\phi} \right| \geq t\right\} \\
				&\leq 2\exp\left(-\frac{t^2 (k(n))^2}{4D\sum_{j = 0}^{n - 1}{\delta_j^2(\tilde{K})}}\right) \\
				&\leq 2\exp\left(-\frac{nt^2}{16D|A|^{2k(n)} (\log{n})^2}\right) \\
				&\leq 2\exp\left(-\frac{n^{1 - \alpha} t^2}{16D (\log{n})^2}\right).
		\end{align*}
		This completes the proof of the concentration bound.
		
		For the variance, we apply \thref{cor:cm-3-2} to get
		\begin{align*}
			\Var_{\mu_\phi}\left(\frac{\hat{H}_{k(n)}}{k(n)}\right) &= \Var_{\mu_\phi}\left(\frac{\tilde{K}}{k(n)}\right) \\
				&= \frac{1}{(k(n))^2}\int{\left[\tilde{K} - E_{\mu_\phi}(\tilde{K})\right]^2\ d\mu_\phi} \\
				&\leq \frac{2D}{(k(n))^2}\sum_{j = 0}^{n - 1}{\delta_j^2(\tilde{K})} \\
				&\leq \frac{8D (\log{n})^2}{n^{1 - \alpha}}.
		\end{align*}
	\end{proof}
\end{theorem}

This result gives a concentration bound about the expectation value. However, to quantify the accuracy of an entropy estimator, it makes more sense to find a concentration bound about the entropy $h(\mu_\phi)$. For this, we use $k$-block conditional empirical entropy.

\begin{theorem} \label{thm:cm-4-2}
	Suppose that $\theta < |A|^{-1}$. If $k(n) \leq \frac{\log{n}}{2\log{|A|}}$, then there exist strictly positive constants $c, \gamma, \Gamma, \xi > 0$ such that for all $t > 0$ we have
	\begin{equation}
		\mu_\phi\left\{\left|\hat{h}_{k(n)} - h(\mu_\phi)\right| \geq t + \frac{c}{n^\gamma}\right\} \leq 2\exp\left(-\frac{\Gamma n^\xi t^2}{(\log{n})^4}\right),
	\end{equation}
	for sufficiently large $n$.
	
	%Here, this upper bound decays superexponentially as $t \to +\infty$.
	If we think of $t + (c / n^\gamma)$ as the error of the $k$-block conditional empirical entropy, we see that the number of sequences which yield a `bad' estimate decays superexponentially as $t \to +\infty$. On the other hand, if we let $n \to +\infty$, then the upper bound will tend to zero for all $t > 0$. Hence $\mu_\phi$-almost every sequence gives an estimate which converges to $h(\mu_\phi)$ and this agrees with \thref{thm:entropy-estimation}
	
	\begin{proof}
		Recall that $\hat{h}_k = \hat{H}_k - \hat{H}_{k - 1}$. We put
		\[
			\tilde{K}'(s_0, \dots, s_{n - 1}) := \hat{h}_k(s_0^{n - 1}) = \hat{H}_k(s_0^{n - 1}) - \hat{H}_{k - 1}(s_0^{n - 1}).
		\]
		We estimate $\delta_j(\tilde{K}')$ by $2\delta_j(\tilde{K})$, where $\tilde{K}(s_0, \dots, s_{n - 1}) = \hat{H}_k(s_0^{n - 1})$ and using the estimate for $\delta_j(\tilde{K})$ in Formula \eqref{fml:oscil-est}.
		
		By \thref{lem:cm-4-1}, we have
		\begin{align}
			E_{\mu_\phi}&\left(\hat{h}_{k(n)}(x_0^{n - 1}) - h(\mu_\phi)\right) \nonumber \\
				&= E_{\mu_\phi}\left(\frac{1}{n}\sum_{j = 0}^{n - 1}(-\phi \circ \sigma^j) + \hat{\Delta}_{k(n)} + O(\theta^{k(n)}) - h(\mu_\phi)\right), \label{fml:cm-3-9-lem-ref}
		\end{align}
		where $\hat{\Delta}_{k(n)}(x_0^{n - 1}) = -H_{k(n)}(\E_{k(n)}(\seedot; x_0^{n - 1})) + H_{k(n) - 1}(\E_{k(n) - 1}(\seedot; x_0^{n - 1}))$. 
		Recalling that $h(\mu_\phi) = -\int{\phi\ d\mu_\phi}$, \eqref{fml:cm-3-9-lem-ref} becomes
		\begin{align*}
			E_{\mu_\phi}\left(\hat{h}_{k(n)}\right) - h(\mu_\phi) &= -\int{\phi\ d\mu_\phi} + E_{\mu_\phi}\left(\hat{\Delta}_{k(n)}\right) + O(\theta^{k(n)}) - h(\mu_\phi) \\
				&= h(\mu_\phi) + E_{\mu_\phi}\left(\hat{\Delta}_{k(n)}\right) + O(\theta^{k(n)}) - h(\mu_\phi) \\
				&= E_{\mu_\phi}\left(\hat{\Delta}_{k(n)}\right) + O(\theta^{k(n)}).
		\end{align*}
		
		We now put $k(n) = \frac{q\log{n}}{\log{|A|}}$, where $0 < q < 1$. If we choose $q$ to be
		\[
			q = \frac{1}{1 + \frac{\log{\theta^{-1}}}{\log{|A|}}},
		\]
		then using
		\[
			|E_{\mu_\phi}(\hat{h}_{k(n)}) - h(\mu_\phi)| \leq \frac{M|A|^{k(n)}}{n}
		\]
		from \thref{lem:cm-4-1}, a straightforward but tedious calculation shows that
		\begin{equation}\label{fml:cm-8}
			|E_{\mu_\phi}(\hat{h}_{k(n)}) - h(\mu_\phi)| \leq \frac{c}{n^\gamma},
		\end{equation}
		where $c > 0$ is a constant and $\gamma = \left(1 + \frac{\log{|A|}}{\log{\theta^{-1}}}\right)^{-1} > 0$. We then have
		\begin{align*}
			|\hat{h}_{k(n)} - h(\mu_\phi)| &= |\hat{h}_{k(n)} - E_{\mu_\phi}(\hat{k}_{k(n)}) - h(\mu_\phi) + E_{\mu_\phi}(\hat{k}_{k(n)})| \\
				&\leq |\hat{h}_{k(n)} - E_{\mu_\phi}(\hat{k}_{k(n)})| + |h(\mu_\phi) - E_{\mu_\phi}(\hat{k}_{k(n)})| \\
				&\leq |\hat{h}_{k(n)} - E_{\mu_\phi}(\hat{k}_{k(n)})| + \frac{c}{n^\gamma}.
		\end{align*}
		
		We now apply \thref{cor:cm-3-1-5} to get
		\begin{align*}
			\mu_\phi\left\{\left|\hat{h}_{k(n)} - h(\mu_\phi)\right| \geq t + \frac{c}{n^\gamma}\right\} &\leq \mu_\phi\left\{\left|\hat{h}_{k(n)} - E_{\mu_\phi}(\hat{k}_{k(n)})\right| \geq t\right\} \\
				&\leq 2\exp\left(-\frac{t^2}{4D\sum_{j = 0}^{n - 1}{\delta_j^2(\tilde{K}')}}\right) \\
				&\leq 2\exp\left(-\frac{nt^2}{64D(k(n))^2|A|^{2k(n)}(\log{n})^2}\right) \\
				&\leq 2\exp\left(-\frac{(\log{|A|})^2}{16D} \frac{nt^2}{|A|^{2k(n)}(\log{n})^4}\right) \\
				&\leq 2\exp\left(-\frac{(\log{|A|})^2}{16D} \frac{\theta^{2k(n)} nt^2}{(\log{n})^4}\right) \\
				&\leq 2\exp\left(-\frac{\Gamma n^\xi t^2}{(\log{n})^4}\right),
		\end{align*}
		where
		\[
			\Gamma = \frac{(\log{|A|})^2}{16D} \quad \text{and} \quad \xi = 1 - \frac{\log{\theta^{-1}}}{\log{|A|}}.
		\]
		The value of $\xi$ is found using the assumption that $k(n) \leq \frac{\log{n}}{2\log{|A|}}$. Furthermore, to guarantee that $\xi > 0$ we must have
		\[
			\xi = 1 - \frac{\log{\theta^{-1}}}{\log{|A|}} > 0 \iff \log{|A|} > \log{\theta^{-1}} \iff |A| > \theta^{-1},
		\]
		which is the first hypothesis in the theorem. This completes the proof.
	\end{proof}
\end{theorem}

\subsection{Hitting time estimators}\label{ssec:hitting-times}
We now focus on \key{hitting time entropy estimators}. Instead of using empirical frequencies, hitting time estimators use the first occurrence of the first $n$ symbols of a sequence $x \in \Sigma$ in another sequence $y \in \Sigma$. The following formal definition can be found in \cite[Definition 2.1]{chazottes-ugalde:hitting-times}.

\begin{definition}
	Let $x, y \in \Sigma$, let $a_0^{n - 1} \in A^n$. The (first) \key{hitting time of $x$ to a cylinder $[a_0^{n - 1}]$} is defined
	\[
		\tau_{[a_0^{n - 1}]}(x) := \inf\{j \geq 1 \mid x_j^{j + n - 1} = a_0^{n - 1}\}.
	\]
	
	The \key{following hitting time} or \key{waiting-time}~\cite[Section III.5]{shields:ergodic} is defined
	\[
		W_n(x, y) := \tau_{[x_0^{n - 1}]}(y) = \inf\{j \geq 1 \mid y_j^{j + n - 1} = x_0^{n - 1}\}.
	\]
\end{definition}

Note that, while the above definitions are largely the same, $\tau_{[a_0^{n - 1}]}$ is defined using a word $a_0^{n - 1}$ of length $n$, whereas $W_n$ is defined using sequences in $\Sigma$. In order to analyse the limiting behaviour as $n \to +\infty$, it makes more sense to work with $W_n$.

The following result relates the waiting-time function to the entropy $h(\nu)$ of certain measures $\nu$. This result is proved in \cite[Theorem III.5.1]{shields:ergodic}.

\begin{theorem}[Exact-Match Theorem] \label{thm:exact-match}
	If $\nu$ is weak Bernoulli, then
	\[
		\lim_{n \to +\infty}{\frac{1}{n}\log{W_n(x, y)}} = h(\nu),
	\]
	for $\nu \otimes \nu$-almost every $(x, y) \in \Sigma \otimes \Sigma$.
\end{theorem}

That is, if $\nu$ is weak Bernoulli, then the time it takes for one sequence to hit another tends to the entropy $h(\nu)$, for almost all pairs of sequences. This is the \key{hitting time entropy estimator}.

By \thref{thm:gibbs-is-weak-bernoulli}, Gibbs measures are weak Bernoulli and hence the Exact-Match Theorem applies. As with plug-in estimators, we also have concentration bounds for the hitting time estimator. To prove the main theorem, we will need two lemmata which follow from \cite[Theorem 1]{abadi:sharp}.

\begin{lemma}\label{lem:cm-4-2}
	There exists constants $C, c, \lambda_1, \lambda_2 > 0$ with $\lambda_1 < \lambda_2$ such that, for all $n \geq 1$ and for all $a_0^{n - 1} \in A^n$, there exists $\Lambda(a_0^{n - 1}) \in [\lambda_1, \lambda_2]$ such that for all $u > 0$ we have
	\[
		\left| \mu_\phi\left\{x \midmid \tau_{[a_0^{n - 1}]}(x) > \frac{u}{\Lambda(a_0^{n - 1}) \mu_\phi[a_0^{n - 1}]}\right\} - e^{-u} \right| \leq Ce^{-cu}.
	\]
\end{lemma}

\begin{lemma}\label{lem:cm-4-3}
	For all $v > 0$ and any word $a_0^{n - 1} \in A^n$ such that $v\mu_\phi[a_0^{n - 1}] \leq \frac{1}{2}$ we have
	\[
		\lambda_1 \leq -\frac{\log{\mu_\phi\{\tau_{[a_0^{n - 1}]} > v\}}}{v\mu_\phi[a_0^{n - 1}]} \leq \lambda_2,
	\]
	where $\lambda_1, \lambda_2 > 0$ are the same constants as in \thref{lem:cm-4-2}.
\end{lemma}

We are now ready to state and prove concentration bounds for the hitting time estimator.

\begin{theorem} \label{thm:cm-4-3}
	There exist strictly positive constants $C_1, C_2 > 0$, $t_0 > 0$ such that, for all $n \geq 1$ and for all $t > t_0$, we have
	\begin{equation}
		(\mu_\phi \otimes \mu_\phi)\left\{(x, y) \midmid \frac{1}{n}\log{W_n(x, y)} > h(\mu_\phi) + t\right\} \leq C_1 e^{-C_2 nt^2} \label{fml:cm-9}
	\end{equation}
	and
	\begin{equation}
		(\mu_\phi \otimes \mu_\phi)\left\{(x, y) \midmid \frac{1}{n}\log{W_n(x, y)} < h(\mu_\phi) - t\right\} \leq C_1 e^{-C_2 nt}. \label{fml:cm-10}
	\end{equation}	
	Essentially, \eqref{fml:cm-9} tells us how likely the hitting time estimator is to \emph{overestimate} the entropy, while \eqref{fml:cm-10} gives how likely it is to \emph{underestimate} the entropy. Note that the upper bound in \eqref{fml:cm-10} decays exponentially as $t \to +\infty$, whereas in \eqref{fml:cm-9} the bound decays superexponentially as $t \to +\infty$. In either case, this means that fewer pairs $(x, y)$ yield estimates which deviate very far from $h(\mu_\phi)$.
	
	If we let $n \to +\infty$, both \eqref{fml:cm-9} and \eqref{fml:cm-10} are bounded above by $0$ for any $t > t_0$. This means that $\mu_\phi \otimes \mu_\phi$-almost every $(x, y)$ gives an estimate which deviates by at most $t_0$ from $h(\mu_\phi)$, and this largely agrees with \thref{thm:exact-match}.
	
	Unfortunately, to prove this result we cannot use \thref{thm:cm-3-1} and its corollaries directly. Our approach to the proof of this theorem involves approximating the value of $W_n(x, y)$, and to do this we need to use the fact that $\mu_\phi$ is a Gibbs measure.
	
	\begin{proof}
		We concentrate on \eqref{fml:cm-9} first. We have
		\begin{align*}
			&(\mu_\phi \otimes \mu_\phi)\left\{(x, y) \midmid \frac{1}{n}\log{W_n(x, y)} > h(\mu_\phi) + t\right\} \\
				&= (\mu_\phi \otimes \mu_\phi)\left\{(x, y) \midmid \frac{1}{n}\log{W_n(x, y)} + \frac{1}{n}\log{\mu_\phi[x_0^{n - 1}]} - \frac{1}{n}\log{\mu_\phi[x_0^{n - 1}]} - h(\mu_\phi) > t\right\} \\
				&\leq T_1 + T_2, % (\mu_\phi \otimes \mu_\phi)\left\{(x, y) \midmid \frac{1}{n}\log{W_n(x, y)} + \frac{1}{n}\log{\mu_\phi[x_0^{n - 1}]} > \frac{t}{2}\right\}, \\
				%& \quad + \mu_\phi\left\{x \midmid - \frac{1}{n}\log{\mu_\phi[x_0^{n - 1}]} - h(\mu_\phi) > \frac{t}{2}\right\} \\
		\end{align*}
		where
		\begin{align*}
			T_1 &:= (\mu_\phi \otimes \mu_\phi)\left\{(x, y) \midmid \frac{1}{n}\log{W_n(x, y)} + \frac{1}{n}\log{\mu_\phi[x_0^{n - 1}]} > \frac{t}{2}\right\} \\
				&= (\mu_\phi \otimes \mu_\phi)\left\{(x, y) \midmid \frac{1}{n}\log\left(W_n(x, y) \mu_\phi[x_0^{n - 1}]\right) > \frac{t}{2}\right\}
		\end{align*}
		and
		\[
			T_2 := \mu_\phi\left\{x \midmid - \frac{1}{n}\log{\mu_\phi[x_0^{n - 1}]} - h(\mu_\phi) > \frac{t}{2}\right\}.
		\]
		
		We find an upper bound for $T_2$ first. By the definition of Gibbs measures, we have
		\begin{equation}\label{fml:gibbs-property}
			-\log{C} + \sum_{j = 0}^{n - 1}{\phi(\sigma^j{x})} \leq \log{\mu_\phi[x_0^{n - 1}]} \leq \log{C} + \sum_{j = 0}^{n - 1}{\phi(\sigma^j{x})}.
		\end{equation}
		Once again, recalling that $h(\mu_\phi) = -\int{\phi\ d\mu_\phi}$, we apply \thref{cor:cm-3-3} with $f = -\phi$ to get
		\begin{align*}
			T_2 &= \mu_\phi\left\{x \midmid - \frac{1}{n}\log{\mu_\phi[x_0^{n - 1}]} - h(\mu_\phi) > \frac{t}{2}\right\} \\
				&\leq \mu_\phi\left\{x \midmid -\frac{1}{n}\sum_{j = 0}^{n - 1}{\phi(\sigma^j{x})} -h(\mu_\phi) > \frac{t}{2} -\frac{1}{n}\log{C}\right\} \\
				&= \mu_\phi\left\{x \midmid -\frac{1}{n}\sum_{j = 0}^{n - 1}{\phi(\sigma^j{x})} + \int{\phi\ d\mu_\phi} > \frac{t}{2} -\frac{1}{n}\log{C}\right\} \\
				& \leq e^{-Bnt^2},
		\end{align*}
		for all $t > 2\log{C}$, where $B = (4D|f|_\theta^2)^{-1}$.
		
		We now find an upper bound for $T_1$. We have
		\begin{align*}
			T_1 &= (\mu_\phi \otimes \mu_\phi)\left\{(x, y) \midmid \frac{1}{n}\log\left(W_n(x, y) \mu_\phi[x_0^{n - 1}]\right) > \frac{t}{2}\right\} \\
				&= \sum_{a_0^{n - 1} \in A^n}{(\mu_\phi \otimes \mu_\phi)\left\{(x, y) \midmid x \in [a_0^{n - 1}],\ \frac{1}{n}\log\left(W_n(x, y) \mu_\phi[x_0^{n - 1}]\right) > \frac{t}{2}\right\}} \\
				&= \sum_{a_0^{n - 1} \in A^n}{\mu_\phi[a_0^{n - 1}] \mu_\phi\left\{y \midmid \tau_{[a_0^{n - 1}]}(y) \mu_\phi[a_0^{n - 1}] > e^\frac{nt}{2}\right\}}
		\end{align*}
		By \thref{lem:cm-4-2} with $u = e^\frac{nt}{2}$ we have
		\begin{align*}
			\left|T_1 - e^{-e^\frac{nt}{2}}\right| &= \left|\sum_{a_0^{n - 1} \in A^n}\mu_\phi[a_0^{n - 1}]\left(\mu_\phi\left\{y \midmid \tau_{[a_0^{n - 1}]}(y) \mu_\phi[a_0^{n - 1}] > e^\frac{nt}{2}\right\} - e^{-e^\frac{nt}{2}}\right)\right| \\
				& \leq \sum_{a_0^{n - 1} \in A^n}\mu_\phi[a_0^{n - 1}] \hat{C}e^{-\hat{c}e^\frac{nt}{2}} \\
				& = \hat{C}e^{-\hat{c}e^\frac{nt}{2}},
		\end{align*}
		for some constants $\hat{C}, \hat{c} > 0$. So we have $T_1 \leq C'e^{-c'e^\frac{nt}{2}}$, for some constants $C', c' > 0$.
		
		Finally, we combine $T_1$ and $T_2$ to get
		\begin{align*}
			(\mu_\phi \otimes \mu_\phi)\left\{(x, y) \midmid \frac{1}{n}\log{W_n(x, y)} > h(\mu_\phi) + t\right\} &\leq C'e^{-c'e^\frac{nt}{2}} + e^{-Bnt^2} \\
				& \leq C_1 e^{Bnt^2},
		\end{align*}
		for some constant $C_1 > 0$. This proves \eqref{fml:cm-9}.
		
		We now turn our attention to \eqref{fml:cm-10}. We have
		\begin{align*}
			&(\mu_\phi \otimes \mu_\phi)\left\{(x, y) \midmid \frac{1}{n}\log{W_n(x, y)} < h(\mu_\phi) - t\right\} \\
				&= (\mu_\phi \otimes \mu_\phi)\left\{(x, y) \midmid -\frac{1}{n}\log{W_n(x, y)} - \frac{1}{n}\log{\mu_\phi[x_0^{n - 1}]} + \frac{1}{n}\log{\mu_\phi[x_0^{n - 1}]} + h(\mu_\phi) > t\right\} \\
				&\leq T'_1 + T'_2,
		\end{align*}
		where
		\begin{align*}
			T'_1 &:= (\mu_\phi \otimes \mu_\phi)\left\{(x, y) \midmid -\frac{1}{n}\log{W_n(x, y)} - \frac{1}{n}\log{\mu_\phi[x_0^{n - 1}]} > \frac{t}{2}\right\} \\
				&= (\mu_\phi \otimes \mu_\phi)\left\{(x, y) \midmid -\frac{1}{n}\log\left(W_n(x, y) \mu_\phi[x_0^{n - 1}]\right) > \frac{t}{2}\right\}
		\end{align*}
		and
		\[
			T'_2 := \mu_\phi \left\{x \midmid \frac{1}{n}\log{\mu_\phi[x_0^{n - 1}]} + h(\mu_\phi) > \frac{t}{2}\right\}.
		\]
		
		To find an upper bound for $T'_2$, we once again use the Gibbs property in \eqref{fml:gibbs-property} and apply \thref{cor:cm-3-3} with $f = \phi$ to get
		\begin{align*}
			T'_2 &\leq \mu_\phi\left\{x \midmid \frac{1}{n}\sum_{j = 0}^{n - 1}{\phi(\sigma^j{x})} + h(\mu_\phi) > \frac{t}{2} - \frac{1}{n}\log{C}\right\} \\
				&= \mu_\phi\left\{x \midmid \frac{1}{n}\sum_{j = 0}^{n - 1}{\phi(\sigma^j{x})} - \int{\phi\ d\mu_\phi} > \frac{t}{2} - \frac{1}{n}\log{C}\right\} \\
				&\leq e^{-\frac{Bnt^2}{4}},
		\end{align*}
		for all $t > 2\log{C}$. (We have applied Corollary 4.6 for $t / 2$ instead of $t$.)
		
		As for $T_1$ above, for $T'_1$ we have
		\begin{align*}
			T'_1 &= (\mu_\phi \otimes \mu_\phi)\left\{(x, y) \midmid -\frac{1}{n}\log\left(W_n(x, y) \mu_\phi[x_0^{n - 1}]\right) > \frac{t}{2}\right\} \\
				&= \sum_{a_0^{n - 1} \in A^n}{(\mu_\phi \otimes \mu_\phi)\left\{(x, y) \midmid x \in [a_0^{n - 1}],\ -\frac{1}{n}\log\left(W_n(x, y) \mu_\phi[x_0^{n - 1}]\right) > \frac{t}{2}\right\}} \\
				&= \sum_{a_0^{n - 1} \in A^n} \mu_\phi[a_0^{n - 1}]\mu_\phi\left\{y \midmid \tau_{[a_0^{n - 1}]}(y) \mu_\phi[a_0^{n - 1}] < e^{-\frac{nt}{2}}\right\}.
		\end{align*}
		Now by \thref{lem:cm-4-3}, if $v\mu_\phi[a_0^{n - 1}] \leq \frac{1}{2}$, then we have
		\[
			-\frac{\log{\mu_\phi\{\tau_{[a_0^{n - 1}]} > v\}}}{v\mu_\phi[a_0^{n - 1}]} \leq -\frac{\log{\mu_\phi\{\tau_{[a_0^{n - 1}]} > v\}}}{v} \leq \lambda_2.
		\]
		Rearranging this, we get
		\[
			-\log\left(1 - \mu_\phi\{\tau_{[a_0^{n - 1}]} < v\}\right) = -\log{\mu_\phi\{\tau_{[a_0^{n - 1}]} > v\}} \leq \lambda_2 v,
		\]
		which gives
		\[
			\mu_\phi\{\tau_{[a_0^{n - 1}]} < v\} \leq 1 - e^{-\lambda_2 v} \leq \lambda_2 v,
		\]
		because $1 - e^{-u} \leq u$ for all $u \in \reals$.
		
		Since $0 \leq \mu_\phi[a_0^{n - 1}] \leq 1$ for all $a_0^{n - 1}$, if we let $v = e^{-\frac{nt}{2}}$, then we have
		\begin{align*}
			T'_1 &\leq \sum_{a_0^{n - 1} \in A^n} \mu_\phi[a_0^{n - 1}]\mu_\phi\left\{y \midmid \tau_{[a_0^{n - 1}]}(y) < e^{-\frac{nt}{2}}\right\} \\
				&\leq \sum_{a_0^{n - 1} \in A^n} \mu_\phi[a_0^{n - 1}] \lambda_2 e^{-\frac{nt}{2}} \\
				&= \lambda_2 e^{-\frac{nt}{2}},
		\end{align*}
		provided that $e^{-\frac{nt}{2}}\mu_\phi[a_0^{n - 1}] \leq \frac{1}{2}$. We want this to be true for all $n$ and all words $a_0^{n - 1}$ of length $n$, so this restriction is equivalent to $e^{-\frac{t}{2}} \leq \frac{1}{2}$ which is the same as $t \geq 2\log{2}$. So we take $t_0 = \max\{2\log{C}, 2 \log{2}\}$.
		
		Finally, we combine $T'_1$ and $T'_2$ to get
		\[
			(\mu_\phi \otimes \mu_\phi)\left\{(x, y) \midmid \frac{1}{n}\log{W_n(x, y)} < h(\mu_\phi) - t\right\} \leq \lambda_2 e^{-\frac{nt}{2}} + e^{-\frac{Bnt^2}{4}} \leq C_1e^{-C_2 nt},
		\]
		for some constants $C_1, C_2 > 0$.
	\end{proof}
\end{theorem}


\appendix
\chapter{Auxiliary results}
This appendix contains results which are used throughout the dissertation. We separate the appendix into sections depending on which chapter the result is first used.

\section{Entropy}
We use the following result in \thref{prop:walters-cor-4-2-1} and also \thref{lem:pp-3-3}.

\begin{definition}
	Let $f : X \to \reals$ be a function of a convex set $X$. We say that $f$ is \key{strictly convex} if, for all $x, y \in X$ and for all $\alpha \in [0, 1]$, we have
	\[
		f(\alpha x + (1 - \alpha)y) \leq \alpha f(x) + (1 - \alpha)f(y),
	\]
	with equality if and only if $x = y$ or $\alpha = 0$ or $1$.
	
	In general, $f$ is strictly convex if, for any set $\{x_j \in X \mid j \in \{1, \dots, k\}\}$ and for any $\{\alpha_j \in [0, 1] \mid j \in \{1, \dots, k\},\ \sum_{j = 1}^k{\alpha_j} = 1\}$, we have
	\[
		f\left(\sum_{j = 1}^k{\alpha_j x_j}\right) \leq \sum_{j = 1}^k{\alpha_j} f(x_j),
	\]
	with equality if and only if $x_1 = x_2 = \dots = x_k$ whenever $\alpha_j \neq 0$ for all $j = 1, \dots, k$.
\end{definition}

\begin{theorem} \label{thm:walters-4-2-xlogx-convex}
	Let $f : [0, +\infty) \to \reals$ be the function defined by
	\[
		f(x) =
		\begin{cases}
			0, & \text{if } x = 0; \\
			x\log{x}, & \text{if } x \neq 0.
		\end{cases}
	\]
	Then $f$ is \emph{strictly convex}.
	\begin{proof}
		For $x > 0$ we have $f'(x) = 1 + \log{x}$ and $f''(x) = 1 / x$. Suppose that $y > x$ and let $\alpha \in (0, 1)$. In particular, this means that $x < \alpha x + (1 - \alpha)y < y$. Now $f$ is clearly continuous, and hence we may apply the Mean Value Theorem. So there exists $z \in (\alpha x + (1 - \alpha)y, y)$ such that
		\[
			f'(z) = \frac{f(y) - f(\alpha x + (1 - \alpha)y)}{y - \alpha x - (1 - \alpha)y},
		\]
		and so
		\[
			f(y) - f(\alpha x + (1 - \alpha)y) = f'(z)\alpha(y - x).
		\]
		
		Similarly, there exists $w \in (x, \alpha x + (1 - \alpha)y)$ such that
		\[
			f(\alpha x + (1 - \alpha)y) - f(x) = f'(w)(1 - \alpha)(y - x).
		\]
		
		Since $x > 0$, we have $f''(x) = 1 / x > 0$ and so $f'(z) > f'(w)$. Hence
		\begin{align*}
			&(1 - \alpha)(f(y) - f(\alpha x + (1 - \alpha)y)) = f'(z)\alpha(1 - \alpha)(y - x) \\
			& \quad > f'(w)\alpha(1 - \alpha)(y - x) = \alpha(f(\alpha x + (1 - \alpha)y) - f(x)).
		\end{align*}
		Cancelling terms, this gives
		\[
			f(\alpha x + (1 - \alpha)y) < \alpha f(x) + (1 - \alpha)f(y)
		\]
		for $y > x > 0$. We may apply the same argument for $x, y \geq 0$ and $x \neq y$.
	\end{proof}
\end{theorem}

\section{Gibbs measures}
The follow result is used in the proof of \thref{lem:pp-prop-3-4}. First, we need to a definition.

\begin{definition}
	Let $X$ be a convex set and suppose $f : X \to \reals$ (or $\complex$). We say that the function $f$ is \key{concave} on $X$ if for all $x, y \in X$ we have
	\[
		f(\alpha x + (1 - \alpha)y) \geq \alpha f(x) + (1 - \alpha)f(y),
	\]
	for all $\alpha \in [0, 1]$. The function $f$ is \key{strictly concave} if the inequality is strict.~\cite[p11]{cambini-martein:generalized}
\end{definition}

\begin{lemma} \label{lem:pp-3-3}
	Suppose that $(p_1, \dots, p_k), (q_1, \dots, q_k)$ are probability vectors with $p_j > 0$ for all $j = 1, \dots, k$. Then
	\[
		-\sum_{j = 1}^k{q_j \log{q_j}} + \sum_{j = 1}^k{q_j \log{p_j}} \leq 0,
	\]
	with equality if and only if $p_j = q_j$ for all $j = 1, \dots, k$.
	\begin{proof}
		We have
		\begin{align}
			-\sum_{j = 1}^k{q_j \log{q_j}} + \sum_{j = 1}^k{q_j \log{p_j}} &= \sum_{j = 1}^k{q_j \log\left(\frac{p_j}{q_j}\right)} \nonumber \\
				&= \sum_{j = 1}^k{-p_j \frac{q_j}{p_j} \log\left(\frac{q_j}{p_j}\right)} \nonumber \\
				&= \sum_{j = 1}^k{p_j \phi\left(\frac{q_j}{p_j}\right)}, \label{fml:pp-3-3-sums}
		\end{align}
		where $\phi(x) = -x \log{x}$, with the convention that $\phi(0) = 0$. By \thref{thm:walters-4-2-xlogx-convex}, we have that $x \log{x}$ is a strictly convex function, and so $\phi$ is strictly concave. Continuing from \eqref{fml:pp-3-3-sums}, we have
		\[
			-\sum_{j = 1}^k{q_j \log{q_j}} + \sum_{j = 1}^k{q_j \log{p_j}} \leq \phi\left(\sum_{j = 1}^k{p_j \frac{q_j}{p_j}}\right) = \phi(1) = 0,
		\]
		with equality if and only if all the $(q_j / p_j)$ terms are equal.
	\end{proof}
\end{lemma}

\section{Concentration bounds}

The following result is used in the proof of \thref{thm:cm-4-1}.

\begin{proposition}\label{prop:logs-thm-4-1}
	Let $\ell, k \in \naturals$ be such that $\ell + k \leq n$. Then
	\[
		\left|\left(\frac{\ell}{n}\right)\log\left(\frac{\ell}{n}\right) - \left(\frac{\ell + k}{n}\right)\log\left(\frac{\ell + k}{n}\right)\right| \leq \left(\frac{k}{n}\right)\log{n}.
	\]
	
	\begin{proof}
		We have
		\begin{align*}
			&\left(\frac{\ell}{n}\right)\log\left(\frac{\ell}{n}\right) - \left(\frac{\ell + k}{n}\right)\log\left(\frac{\ell + k}{n}\right) \\
				&\quad = \left(\frac{k}{n}\right)\log{n} - \left[\left(\frac{\ell + k}{n}\right)\log(\ell + k) - \left(\frac{\ell}{n}\right)\log{\ell}\right].
		\end{align*}
		Since $k \geq 1$, we have
		\begin{align*}
			0 &\leq \left(\frac{\ell + k}{n}\right)\log(\ell + k) - \left(\frac{\ell}{n}\right)\log{\ell} \\
				&\leq \left(\frac{\ell + k}{n}\right)\log(\ell + k) + \left(\frac{k - \ell}{n}\right)\log{\ell} \\
				&\leq \left(\frac{\ell + k}{n}\right)\log{n} + \left(\frac{k - \ell}{n}\right)\log{n} \\
				&\leq 2\left(\frac{k}{n}\right)\log{n}.
		\end{align*}
		Hence
		\[
			-\left(\frac{k}{n}\right)\log{n} \leq \left(\frac{\ell}{n}\right)\log\left(\frac{\ell}{n}\right) - \left(\frac{\ell + k}{n}\right)\log\left(\frac{\ell + k}{n}\right) \leq \left(\frac{k}{n}\right)\log{n},
		\]
		in other words,
		\[
			\left|\left(\frac{\ell}{n}\right)\log\left(\frac{\ell}{n}\right) - \left(\frac{\ell + k}{n}\right)\log\left(\frac{\ell + k}{n}\right)\right| \leq \left(\frac{k}{n}\right)\log{n}.
		\]
	\end{proof}
\end{proposition}

The following result allows us to write the $k$-block conditional empirical entropy $\hat{h}_{k(n)}(x_0^{n - 1})$ in a more useful form. It is used in the proof of \thref{thm:cm-4-2}.

\begin{lemma}\label{lem:cm-4-1}
	Let $\phi \in F_\theta$. We have
	\begin{equation}
		\hat{h}_{k(n)}(x_0^{n - 1}) = \frac{1}{n}\sum_{j = 0}^{n - 1}(-\phi \circ \sigma^j(x)) + \hat{\Delta}_{k(n)}(x_0^{n - 1}) + O(\theta^{k(n)}),
	\end{equation}
	where
	\[
		|E(\hat{\Delta}_{k(n)})| \leq \frac{M|A|^{k(n)}}{n},
	\]
	for some $M > 0$.
	\begin{proof}
		We follow the proof given in \cite[p10-11]{chazottes-maldonado:cbfee}.
		
		We have
		\begin{align}
			\hat{h}_k(x_0^{n - 1}) &= h_k(\E_k(\seedot; x_0^{n - 1})) \nonumber \\
				&= -\sum_{a_0^{k - 1} \in A^k}{\E_k(a_0^{k - 1}; x_0^{n - 1}) \log{\frac{\E_k(a_0^{k - 1}; x_0^{n - 1})}{\E_{k - 1}(a_0^{k - 2}; x_0^{n - 1})}}} \nonumber \\
				&= \hat{\Delta}_k(x_0^{n - 1}) -\sum_{a_0^{k - 1} \in A^k}\E_k(a_0^{k - 1}; x_0^{n - 1}) \log{\frac{\mu_\phi[a_0^{k - 1}]}{\mu_\phi[a_1^{k - 1}]}}, \label{fml:cm-11}
		\end{align}
		where
		\begin{align*}
			\hat{\Delta}_k(x_0^{n - 1}) &=-\sum_{a_0^{k - 1} \in A^k}{\E_k(a_0^{k - 1}; x_0^{n - 1}) \log{\frac{\E_k(a_0^{k - 1}; x_0^{n - 1})}{\E_{k - 1}(a_0^{k - 2}; x_0^{n - 1})}}} \\
				& \qquad + \sum_{a_0^{k - 1} \in A^k}\E_k(a_0^{k - 1}; x_0^{n - 1}) \log{\frac{\mu_\phi[a_0^{k - 1}]}{\mu_\phi[a_1^{k - 1}]}} \\
				&=-\sum_{a_0^{k - 1} \in A^k}{\E_k(a_0^{k - 1}; x_0^{n - 1}) \log{\frac{\E_k(a_0^{k - 1}; x_0^{n - 1})}{\mu_\phi[a_0^{k - 1}]}}} \\
				& \qquad + \sum_{a_0^{k - 1} \in A^k}\E_k(a_0^{k - 1}; x_0^{n - 1}) \log{\frac{\E_{k - 1}(a_0^{k - 2}; x_0^{n - 1})}{\mu_\phi[a_1^{k - 1}]}}.
		\end{align*}
		Since $\E_k(\seedot; x_0^{n - 1})$ is locally $\sigma$-invariant, we have
		\[
			\sum_{a_0 \in A}{\E_k(a_0^{k - 1}; x_0^{n - 1})} = \E_{k - 1}(a_1^{k - 1}; x_0^{n - 1}).
		\]
		It is also clear that
		\[
			\sum_{a_{k - 1} \in A}{\mathcal{E}_k(a_0^{k - 1}; x_0^{n - 1})} = \mathcal{E}_{k - 1}(a_0^{k - 2}; x_0^{n - 1}).
		\]
		This means that
		\begin{align*}
			\sum_{a_0^{k - 1} \in A^k}&\E_k(a_0^{k - 1}; x_0^{n - 1}) \log{\frac{\E_{k - 1}(a_0^{k - 2}; x_0^{n - 1})}{\mu_\phi[a_1^{k - 1}]}} \\
				&= \sum_{a_0^{k - 1} \in A^k}\E_k(a_0^{k - 1}; x_0^{n - 1}) \log{\frac{\sum_{a_{k} \in A}{\mathcal{E}_k(a_0^{k - 1}; x_0^{n - 1})}}{\mu_\phi[a_1^{k - 1}]}} \\
				&= \sum_{a_0^{k - 1} \in A^k}\E_k(a_0^{k - 1}; x_0^{n - 1}) \log{\frac{\mathcal{E}_k(a_0^{k - 1}; x_0^{n - 1})}{\mu_\phi[a_1^{k - 1}]}} \\
				&= \sum_{a_1^{k - 1} \in A^{k - 1}}\sum_{a_0 \in A}\E_k(a_0^{k - 1}; x_0^{n - 1}) \log{\frac{\mathcal{E}_k(a_0^{k - 1}; x_0^{n - 1})}{\mu_\phi[a_1^{k - 1}]}} \\
				&= \sum_{a_1^{k - 1} \in A^{k - 1}}\E_{k - 1}(a_1^{k - 1}; x_0^{n - 1}) \log{\frac{\mathcal{E}_{k - 1}(a_1^{k - 1}; x_0^{n - 1})}{\mu_\phi[a_1^{k - 1}]}} \\
				&= H_{k - 1}(\E_{k - 1}(\seedot; x_0^{n - 1})).
		\end{align*}
		Hence $\hat{\Delta}_k(x_0^{n - 1}) = -H_k(\E_k(\seedot; x_0^{n - 1})) + H_{k - 1}(\E_{k - 1}(\seedot; x_0^{n - 1}))$.
		
		There is a formula due to \cite[Formulae (4.15), (4.16)]{gabrielli-galves-guiol:fluctuations} which gives the bound
		\[
			|E_{\mu_\phi}(\hat{\Delta}_{k(n)})| \leq \frac{M|A|^k}{n}
		\]
		for all $n \geq 1$, where $M > 0$ is a strictly positive constant.
		
		We now deal with the summation in Formula \eqref{fml:cm-11}. For $y \in \Sigma$ we put
		\[
			\phi_k(y) = \log\frac{\mu_\phi[y_0^{k - 1}]}{\mu_\phi[y_1^{k - 1}]}.
		\]
		By \thref{prop:pp-3-2}, for all $y \in \Sigma$ we have
		\[
			\|\phi_k(y) - \phi(y)\|_\infty = \left\|\log\frac{\mu_\phi[y_0^{k - 1}]}{\mu_\phi[y_1^{k - 1}]} - \phi(y)\right\|_\infty \leq |\phi|_\theta \theta^k.
		\]
		So we may reasonably replace any instance of $\phi_k(y)$ with $\phi(y) + O(\theta^k)$. By the way $\E_k(\seedot; x_0^{n - 1})$ is defined, we have
		\begin{align*}
			-\sum_{a_0^{k - 1} \in A^k}\E_k(a_0^{k - 1}; x_0^{n - 1}) \log{\frac{\mu_\phi[a_0^{k - 1}]}{\mu_\phi[a_1^{k - 1}]}} &= -\sum_{a_0^{k - 1} \in A^k}\E_k(a_0^{k - 1}; x_0^{n - 1}) \phi_k(a) \\
				&= \frac{1}{n}\sum_{j = 0}^{n - 1}{(-\phi \circ \sigma^j(x))} + O(\theta^k).
		\end{align*}
		The result follows by substituting this back into Formula \eqref{fml:cm-11}.
	\end{proof}
\end{lemma}


\bibliographystyle{alpha}
\bibliography{references}

% If you need more than one appendix, then just use another \chapter command
%\chapter{Yet Another Appendix}

\end{document}
}

% Uncomment the line below to suppress the `List of Tables' page (optional)
\tablespagefalse

% Uncomment the line below to suppress the `List of Figures' page (optional)
\figurespagefalse

\beforeabstract

Abstract here.

\afterabstract

% The next part is optional; however it is a good place to thank your
% supervisor and the people responsible for providing computer support ;-)
\prefacesection{Acknowledgements}
I would like to thank...

% The next line is NOT optional and MUST appear
\afterpreface

% Finally, you can start writing about all the new theorems you have proved
% and all the new results that you have discovered

\prefacesection{Conventions}
The following outlines the conventions used throughout this dissertation.

\section*{Notation}
\begin{trivlist}
	\item $\naturals = \{1, 2, 3, \ldots\}$, the set of natural numbers -- positive integers \emph{not} including $0$.
	\item $\naturals_0 = \{0, 1, 2, \ldots\}$, the set of nonnegative integers.
	\item $\reals^+ = \{a \in \reals \mid a \geq 0\}$, the set of nonnegative real numbers including $0$.
\end{trivlist}

\section*{Propositions, theorems, examples, etc.}
Throughout this dissertation a black square $\blacksquare$ marks the end of each definition, remark and results where the proof has been omitted. If we have a claim within a proof, then the proof of the claim is also marked with a black square.

A white square $\square$ is used to mark the end of all other proofs.

\chapter{Introduction}
\section{Overview}
The main aim of this dissertation is to provide a more accessible account of \cite{chazottes-maldonado:cbfee}, which we will do in Chapter \ref{chap:concentration-bounds}. To achieve this, the following chapters are devoted to describing the key concepts required to provide the reader with the relevant background knowledge.

A property of measure-preserving transformations called \key{entropy} makes up a crucial part of this dissertation. We will show that, if two measure-preserving transformations are `the same', then they have the same entropy. (In Chapter \ref{chap:entropy} we will formally define `the same'.)

The main ideas in \cite{chazottes-maldonado:cbfee} focus on methods for estimating entropy. For example, there is a theorem which says that, for almost all $(x, y)$ we have
\[
	\frac{1}{n} \log{W_n(x, y)} \to h(\nu),
\]
as $n \to +\infty$, where $W_n$ is a function we will define later and $h(\nu)$ is the entropy of a measure $\nu$. We will see later that this is called the hitting time entropy estimator.

Although it may seem obvious, it is worth emphasising that entropy estimators give \emph{estimates} for the entropy. For example, consider two sample pairs $(x_1, y_1)$, $(x_2, y_2)$ which achieve convergence for the above function. If we fix $n \geq 1$, it is possible that $\frac{1}{n} \log{W_n(x_1, y_1)}$ gives a value which is close to $h(\nu)$, whereas $\frac{1}{n} \log{W_n(x_2, y_2)}$ gives a value which is far away from $h(\nu)$. In the final part of this dissertation, we will be particularly interested in these fluctuation properties of entropy estimators.

We will work with \key{Gibbs measures}, which is a class of measures on shifts of finite type with a distinguishing property. Therefore Chapter \ref{chap:sft} will provide the relevant background for shifts of finite type, and Chapter \ref{chap:gibbs} will define Gibbs measures and its main properties.

\section{Preliminaries}
Before we begin with the main background material, this section briefly introduces some concepts and definitions which will be used throughout this dissertation.

\begin{definition}
	Let $(X, d_X)$ and $(Y, d_Y)$ be metric spaces. A function $f : X \to Y$ is a \key{Lipschitz function} if there exists a constant $K > 0$ such that
	\[
		d_Y(f(x), f(y)) \leq Kd_X(x, y)
	\]
	for all $x, y \in X$. If this is the case, we say that $f$ is a Lipschitz function with \key{Lipschitz constant} $K$.~\cite[p154]{searcoid:metric-spaces}
\end{definition}

\begin{definition}
	A transformation $T : (X_1, \B_1, \mu_1) \to (X_2, \B_2, \mu_2)$ is \key{measure-preserving} if:
	\begin{enumerate}
		\item $T$ is measurable, i.e. if $B_2 \in \B_2$, then $T^{-1}{B_2} \in \B_1$, and
		\item $\mu_1(T^{-1}{B_2}) = \mu_2(B_2)$ for all $B_2 \in \B_2$.
	\end{enumerate}
	This agrees with our usual definition when $X_1 = X_2$.
\end{definition}

\begin{definition}
	Let $X$ be a compact metric space with Borel $\sigma$-algebra $\B$. We let $M(X)$ denote the set of all probability measures on $(X, \B)$.
	
	Let $T : X \to X$ be a continuous mapping on $X$. We let $M(X, T)$ denote the set of $T$-invariant probability measures on $(X, B)$.
\end{definition}

\begin{definition}
	The \key{symmetric difference} of two sets $A, B$ is defined $(A \setminus B) \cup (B \setminus A)$. We will write this as
	\[
		A \symdiff B := (A \setminus B) \cup (B \setminus A).
	\]
\end{definition}
\chapter{Background}
\section{Preliminaries}
This section briefly introduces some concepts which will be used throughout this dissertation.

\begin{definition}
	Let $(X, d_X)$ and $(Y, d_Y)$ be metric spaces. A function $f : X \to Y$ is a \key{Lipschitz function} if there exists a constant $K > 0$ such that
	\[
		d_Y(f(x), f(y)) \leq Kd_X(x, y)
	\]
	for all $x, y \in X$. We say that $f$ is a Lipschitz function with \key{Lipschitz constant} $K$.~\cite[p154]{searcoid:metric-spaces}
%	
%	More generally, we say that $f$ is \key{H\"older continuous} if there exists some constants $K > 0$ and $\alpha \in (0, 1]$ such that
%	\[
%	d_Y(f(x), f(y)) \leq K(d_X(x, y))^\alpha
%	\]
%	for all $x, y \in X$. In this case, we say that $f$ is H\"older continuous with \key{H\"older exponent} $\alpha$ and \key{H\"older constant} $K$.~\cite[p143]{brin-stuck:dynsys}
\end{definition}

\begin{definition}
	A transformation $T : (X_1, \B_1, \mu_1) \to (X_2, \B_2, \mu_2)$ is \key{measure-preserving} if:
	\begin{enumerate}
		\item $T$ is measurable, i.e. if $B_2 \in \B_2$, then $T^{-1}{B_2} \in \B_1$, and
		\item $\mu_1(T^{-1}{B_2}) = \mu_2(B_2)$ for all $B_2 \in \B_2$.
	\end{enumerate}
	This agrees with our usual definition when $X_1 = X_2$.
\end{definition}

\begin{definition}
	Let $X$ be a compact metric space with Borel $\sigma$-algebra $\B$. We let $M(X)$ denote the set of all probability measures on $(X, \B)$.
	
	Let $T : X \to X$ be a continuous mapping on $X$. We let $M(X, T)$ denote the set of $T$-invariant probability measures on $(X, B)$.
\end{definition}

\begin{definition}
	The \key{symmetric difference} of two sets $A, B$ is defined $(A \setminus B) \cup (B \setminus A)$. We will write this as
	\begin{equation*}
		A \symdiff B := (A \setminus B) \cup (B \setminus A).
	\end{equation*}
\end{definition}

\section{Shifts of finite type}
\emph{A large portion of this section follows \cite[Chapter 1]{parry-pollicott:zeta-fns-periodic-orbits}.}
\subsection{The basics}
Let $A$ be a $k \times k$ matrix with entries in $\{0, 1\}$. A \key{(two-sided) shift of finite type} $\Sigma_A$ is defined by
\[
	\Sigma_A = \{(x_j)_{j = -\infty}^\infty \mid A_{x_j, x_{j + 1}} = 1,\ j \in \integers\}.
\]
Similarly, a \key{(one-sided) shift of finite type} $\Sigma_A^+$ is defined by
\[
	\Sigma_A^+ = \{(x_j)_{j = 0}^\infty \mid A_{x_j, x_{j + 1}} = 1,\ j \in \naturals_0\}.
\]

Let $x = (x_j)_{j = -\infty}^\infty \in \Sigma_A$. We define the \key{(two-sided, left) shift map} $\sigma : \Sigma_A \to \Sigma_A$ by
\[
	(\sigma(x))_j = x_{j + 1}.
\]
which shifts each coordinate of $x$ one position to the left.

Now let $x = (x_j)_{j = 0}^\infty \in \Sigma_A^+$. We similarly define the \key{(one-sided, left) shift map} $\sigma^+ : \Sigma_A^+ \to \Sigma_A^+$ by
\[
	(\sigma^+(x))_j = x_{j + 1}.
\]
As with the one-sided case, this shifts the coordinates of $x$ one position to the left but also deletes the first coordinate $x_0$. It is clear that $\sigma^+$ is not invertible, whereas $\sigma$ is invertible.

To avoid excessive use of subscripts and superscripts, we will often write $\sigma$ for both the one-sided and two-sided shift maps. It should be clear from the context which of these maps $\sigma$ denotes.

If $x = (x_j)_{j = -\infty}^\infty \in \Sigma_A$, then we call $(x_j)_{j = -\infty}^0$ the \key{past}, $x_0$ the \key{present}, and $(x_j)_{j = 0}^\infty$ the \key{future}.

\subsubsection{Irreducibility, aperiodicity and cylinders}
Let $A$ is a $k \times k$ matrix with entries in $\{0, 1\}$, and let $\Sigma_A$ (or $\Sigma_A^+$) be the associated shift of finite type. We may consider $A$ to be the adjacency matrix of a directed graph $G_A$ with $k$ vertices.

We say that $A$ is \key{irreducible} if, for each $i, j \in \{1, \dots, k\}$, there exists $n = n(i, j) > 0$ such that $(A^n)_{i, j} > 0$. Alternatively, $A$ is irreducible if there exists an edge-path between any two vertices in the corresponding graph $G_A$. In this case, we say that the shift of finite type $\Sigma_A$ (or $\Sigma_A^+$) is irreducible.

If there exists $n > 0$ such that $(A^n)_{i, j} > 0$ for all $i, j \in \{1, \dots, k\}$, then we say that $A$ is \key{aperiodic}. That is, $A$ is aperiodic if all edge-paths between any two vertices in $G_A$ can be chosen to be of the same length. As before, this means that $\Sigma_A$ (or $\Sigma_A^+$) is aperiodic.

A \key{cylinder} $C$ on $\Sigma_A$ is defined
\[
	C = [i_{-m}, \dots, i_{-1}, i_0, i_1, \dots, i_n]_{-m, n} = \{(x_j)_{j = -\infty}^\infty \in \Sigma \mid x_j = i_j \text{ for } -m \leq j \leq n\}.
\]
Similarly, a cylinder $C^+$ on $\Sigma_A^+$ is given by
\[
	C^+ = [i_0, i_1, \dots, i_{n - 1}, i_n]_{0, n} = \{(x_j)_{j = 0}^\infty \in \Sigma \mid x_j = i_j \text{ for } 0 \leq j \leq n\}.
\]
In other words, a cylinder is the set of all sequences which agree in the given positions.

\subsection{Function spaces for shifts of finite type}
Let $\Sigma_A$ and $\Sigma_A^+$ be two-sided and one-sided shifts of finite type, respectively, and let $\theta \in (0, 1)$ be fixed.

\subsubsection{Metrics for shifts of finite type}
Let $x = (x_j)_{j = -\infty}^\infty, y = (y_j)_{j = -\infty}^\infty \in \Sigma_A$. We define $n = n(x, y) \geq 0$ to be the largest integer such that $x_j = y_j$ for all $|j| < n$, but $x_n \neq y_n$ or $x_{-n} \neq y_{-n}$. If $x_j = y_j$ for all $j \in \integers$ then we define $n = +\infty$.

We define the map $d_\theta : \Sigma_A \times \Sigma_A \to \reals^+$ by
\[
	d_\theta(x, y) =
	\begin{cases}
		\theta^n, & \text{if } x \neq y; \\
		0 & \text{if } x = y.
	\end{cases}
\]
It can be shown that $d_\theta$ is a \key{metric} on $\Sigma_A$, so sequences in $\Sigma_A$ are `close' if they agree for a large number of leading coordinates.

Similarly, if $x = (x_j)_{j = 0}^\infty, y = (y_j)_{j = 0}^\infty \in \Sigma_A^+$, then $n = n(x, y) \geq 0$ is defined to be the largest integer such that $x_j = y_j$ for all $0 \leq j < n$ but $x_n \neq y_n$. We define the map $d_\theta : \Sigma_A^+ \times \Sigma_A^+ \to \reals^+$ in the same way as the two-sided case, and it can be shown that $d_\theta$ is also a metric.

\subsubsection{The space of Lipschitz functions}
\begin{definition}
	Let $f : \Sigma_A \to \complex$ be a continuous function and let $n \geq 0$. We define the \key{$n$-th variation of $f$} by
	\[
		\var_n(f) = \sup\{|f(x) - f(y)| \mid x, y \in \Sigma_A,\ x_j = y_j \text{ for } |j| < n\}.
	\]
	
	Similarly, if $g : \Sigma_A^+ \to \complex$ is a continuous function, then the \key{$n$-th variation of $g$} is given by
	\[
	\var_n(g) = \sup\{|g(x) - g(y)| \mid x, y \in \Sigma_A^+,\ x_j = y_j \text{ for } 0 \leq j < n\}.
	\]
	That is, $\var_n(g)$ indicates how much $g$ varies on cylinders of length $n$.~\cite[Lecture 8]{magic-ergodic}
\end{definition}

It is easy to see that $\var_n(f) \leq K\theta^n$ for all $n \geq 0$ if and only if $|f(x) - f(y)| \leq Kd_\theta(x, y)$, i.e. $f$ is a Lipschitz function. This allows the following definition to be given in terms of $n$-th variations of continuous functions.

\begin{definition}
	Define
	\[
		F_\theta = F_\theta(\Sigma_A) = \{f \in C(\Sigma_A, \complex) \mid \var_n(f) \leq K\theta^n \text{ for all } n \geq 0, \text{ for some } K > 0\}
	\]
	to be the space of Lipschitz functions with respect to the metric $d_\theta$.
	
	We define $F_\theta^+ = F_\theta^+(\Sigma_A^+)$ in the same way, replacing $\Sigma_A$ with $\Sigma_A^+$.
\end{definition}

\begin{definition}
	Let $f \in F_\theta$ (or $F_\theta^+)$. Define
	\[
		|f|_\theta = \sup_{n \geq 0}\left\{\frac{\var_n(f)}{\theta^n}\right\}
	\]
	to be the least Lipschitz constant of $f$. In other words, $|f|_\theta$ is the smallest $K > 0$ such that $\var_n(f) \leq K\theta^n$ for all $n \geq 0$.
	
	We can then define a \key{norm} on $F_\theta$ (or $F_\theta^+$) by
	\[
		\|f\|_\theta = |f|_\infty+ |f|_\theta,
	\]
	where $|f|_\infty = \sup_{x \in \Sigma}\{|f(x)|\}$.
\end{definition}

\begin{proposition}
	The spaces $(F_\theta, \|\cdot\|_\theta)$ and $(F_\theta^+, \|\cdot\|_\theta)$ are Banach spaces.
\end{proposition}

\begin{definition}
	Let $f, g \in F_\theta$ (or $F_\theta^+$). We say $f$ and $g$ are \key{cohomologous} if there exists a continuous function $h$ such that $f = g + h \circ \sigma - h$. In this case, we write $f \sim g$.
	
	If $f$ is cohomologous to $0$, then we say $f$ is a \key{coboundary}.
\end{definition}

\begin{remark}
	It is clear that $\sim$ is an equivalence relation.
\end{remark}

\begin{proposition} \label{prop:pp-1-2}
	Suppose that $f \in F_\theta$. Then there exists $g, h \in F_{\theta^{1 / 2}}$ such that $f = g + h - h \circ \sigma$, where $g(x) = g(y)$ if $x_j = y_j$ for all $j \geq 0$. In other words, the value of $g(x)$ is determined only by the future coordinates of $x$.
	\begin{proof}
		For each $j = 1, \dots, k$ we choose a sequence $(a_n^{(j)})_{n = -\infty}^0$ from the past such that $a_0^{(j)} = j$. Now we define a function $\phi : \Sigma_A \to \Sigma_A : x \mapsto x'$, where
		\[
			(x')_n
			\begin{cases}
				x_n, & \text{if } n \geq 0; \\
				a_n^{(x_0)}, & \text{if } n \leq 0.
			\end{cases}
		\]
		So $\phi$ fixes the future coordinates of $x$, but replaces the past with $(a_n^{(x_0)})_{n = -\infty}^0$.
		
		Let $h(x) = \sum_{n = 0}^\infty{(f(\sigma^n{x}) - f(\sigma^n \phi{x}))}$. Note that $h$ converges because for all $n \geq 0$,
		\[
			|f(\sigma^n{x}) - f(\sigma^n \phi{x})| \leq \var_n(f) \leq |f|_\theta \theta^n,
		\]
		We have
		\begin{align*}
			h(x) - h(\sigma{x}) &= \sum_{n = 0}^\infty{(f(\sigma^n{x}) - f(\sigma^n \phi{x}))} - \sum_{n = 0}^\infty{(f(\sigma^{n + 1}{x}) - f(\sigma^n \phi \sigma{x}))} \\
				&= f(x) - \left(f(\phi{x}) + \sum_{n = 0}^\infty{(f(\sigma^{n + 1} \phi{x}) - f(\sigma^n \phi \sigma{x}))}\right) \\
				&= f(x) - g(x),
		\end{align*}
		where $g(x) := f(\phi{x}) + \sum_{n = 0}^\infty{(f(\sigma^{n + 1} \phi{x}) - f(\sigma^n \phi \sigma{x}))}$. Since all terms in $g$ contain $\phi$, we see that $g$ depends only on its future coordinates.
		
		It remains to show that $h \in F_{\theta^{1 / 2}}$ and it we will immediately have $g \in F_{\theta^{1 / 2}}$. It is sufficient to show that $\var_{2N}(h) \leq K\theta^N$ for all $N \geq 0$, for some constant $K > 0$, because this gives that
		\[
			\var_{2N + 1}(h) \leq K\theta^N = \frac{K}{\theta^{1 / 2}}(\theta^{1 / 2})^{2N + 1}.
		\]
		
		Let $x, y \in \Sigma_A$ be such that $x_j = y_j$ for $|j| \leq 2N$. Then for all $n = 0, \dots, N$, we have
		\[
			|f(\sigma^n{x}) - f(\sigma^n{y})| \leq |f|_\theta \theta^{2N - n} \quad \text{and} \quad |f(\sigma^n \phi{x}) - f(\sigma^n \phi{y})| \leq |f|_\theta \theta^{2N - n}.
		\]
		By definition, for all $n \geq 0$ we have
		\[
			|f(\sigma^n{x}) - f(\sigma^n \phi{x})| \leq |f|_\theta \theta^n \quad \text{and} \quad |f(\sigma^n {y}) - f(\sigma^n \phi{y})| \leq |f|_\theta \theta^n.
		\]
		Hence
		\begin{align*}
			|h(x) - h(y)| &= \left|\sum_{n = 0}^\infty{(f(\sigma^n{x}) - f(\sigma^n \phi{x}) - f(\sigma^n{y}) + f(\sigma^n \phi{y}))}\right| \\
				&\leq \sum_{n = 0}^N{|f(\sigma^n{x}) - f(\sigma^n{y})| + | f(\sigma^n \phi{x}) -  f(\sigma^n \phi{y})|} \\
				&\quad + \sum_{n = N + 1}^\infty{|f(\sigma^n{x}) - f(\sigma^n \phi{x})| + | f(\sigma^n \phi{y}) -  f(\sigma^n {y})|} \\
				&\leq 2|f|_\theta \sum_{n = 0}^N{\theta^{2N - n}} + 2|f|_\theta \sum_{n + N + 1}^\infty{\theta^n} \\
				&= 2|f|_theta \theta^{2N} \left(\frac{\theta^{-N - 1} - 1}{\theta^{-1} - 1}\right) + |f|_\theta \frac{\theta^{N + 1}}{1 - \theta} \\
				&\leq 4|f|_\theta \frac{\theta^N}{1 - \theta}.
		\end{align*}
		Therefore
		\[
			\var_{2N}(h) \leq \left(\frac{4|f|_\theta}{1 - \theta}\right)\theta^N
		\]
		and so $h \in F_{\theta^{1 / 2}}$.
	\end{proof}
\end{proposition}

The above result allows us to write $f \in F_\theta$ using $g \in F_{\theta^{1 / 2}}^+$. This allows us to apply results to $g$ which hold for $F_\theta^+$ but not necessarily for $F_\theta$. We will encounter such uses for this proposition in due course.

\begin{comment}
If a function $f : \Sigma \to \complex$ is $\alpha$-H\"older continuous, $\alpha \in (0, 1]$, then it is a Lipschitz function with respect to $d_{\theta^\alpha}$.

Suppose we have $0 < \theta < \theta' < 1$. Then
\[
	F_{\theta'}(\Sigma) \supset F_\theta(\Sigma) \quad \text{and} \quad F_{\theta'}^+(\Sigma^+) \supset F_\theta^+(\Sigma^+).
\]
We can therefore define
\[
	F = \bigcup_{0 < \theta < 1}{F_\theta(\Sigma)} \quad \text{and} \quad F^+ = \bigcup_{0 < \theta < 1}{F_\theta^+(\Sigma^+)},
\]
the spaces of all H\"older continuous functions.


We now consider a class of functions lying in $F_\theta^+$ for all $0 < \theta < 1$. For all $m \geq 1$ we define
\[
	F_m^+ = \{f : \Sigma^+ \to \complex \mid f(x) = f(y) \text{ if } x_j = y_j, \text{ for all } 0 \leq j < m\},
\]
the set of all locally constant functions which depend on the first $m$ terms of $x \in \Sigma^+$. It is clear that
\[
	F_1^+ \subset F_2^+ \subset F_3^+ \subset \dots.
\]
and, for $f \in F_m^+$, $\var_m(f) = 0$. Hence
\[
	\bigcup_{m = 1}^\infty{F_m^+} \subset \bigcap_{0 < \theta < 1}{F_\theta^+}.
\]

\begin{proposition}
	Suppose $0 < \theta < \theta' < 1$. Then for all $m \geq 0$ we have
	\begin{equation*}
		|f - f_m|_{\theta'} \leq |f|_\theta \left(\frac{\theta}{\theta'}\right)^m.
	\end{equation*}
\end{proposition}
\end{comment}

\section{The Ruelle operator}
Throughout this section, let $\Sigma = \Sigma_A^+$ be a (one-sided) shift of finite type.

\begin{definition}
	Let $f \in F_\theta^+$. The \key{Ruelle operator} (or \key{transfer operator}) $L_f : F_\theta^+ \to F_\theta^+$ (or, more generally, $L_f : C(\Sigma, \complex) \to C(\Sigma, \complex)$) is defined
	\[
		(L_f{w})(x) = \sum_{y \in \Sigma \midcolon \sigma{y} = x}{e^{f(y)} w(y)} = \sum_{j \midcolon A_{j, x_0} = 1}{e^{f(j, x_0, x_1, \dots)} w(j, x_0, x_1, \dots)},
	\]
	where $x = (x_j)_{j = 0}^\infty \in \Sigma$. This is a bounded linear operator.
	
	The $n$-th iterate of $L_f$ is given by
	\[
		(L_f^n{w})(x) = \sum_{y \in \Sigma \midcolon \sigma^n{y} = x}{e^{f^n(y)} w(y)}.
	\]
	
	If $f$ is also real-valued and $L_f{1} = 1$, then we say that $f$ or $L_f$ is \key{normalised}.
\end{definition}

\begin{proposition}
	Let $f \in F_\theta^+$ with $f = u + iv$, where $u, v \in F_\theta^+$ are real-valued functions. If $L_u$ is normalised, i.e. $L_u{1} = 1$, then for all $n \geq 0$,
	\[
		|L_f^n{w}|_\theta \leq K|w|_\infty + \theta^n |w|_\theta
	\]
	for all $w \in F_\theta^+$, where $K > 0$ is a constant depending only on $f$ and $\theta$.
\end{proposition}

\begin{theorem}[Ruelle's Perron-Frobenius Theorem] \label{thm:rpf}
	Suppose $\Sigma = \Sigma_A^+$ is an aperiodic shift of finite type and let $f \in F_\theta^+$ be a real-valued function. Then
	\begin{enumerate}
		\item There is a simple maximal eigenvalue $\lambda$ of $L_f : C(\Sigma, \reals) \to C(\Sigma, \reals)$ with a corresponding eigenfunction $h \in C(\Sigma_A^+, \reals)$, with $h > 0$. \label{rpf:1}
		\item The remainder of the spectrum of $L_f$ is contained in a disc of radius strictly less than $\lambda$. \label{rpf:2}
		\item There is a unique probability measure $\mu$ such that $L_f^*{\mu} = \lambda\mu$. That is,
		\[
			\int{L_f{v}\ d\mu} = \lambda \int{v\ d\mu},
		\]
		for all $v \in C(\Sigma, \reals)$. Additionally, if $h$ is the eigenfunction as in \ref{rpf:1} and $\int{h\ d\mu} = 1$, then the measure $\nu$ defined by $d\nu = h\ d\mu$ is a $\sigma$-invariant probability measure. \label{rpf:3}
		\item If $h$ is the eigenfunction as in \ref{rpf:1} and $\int{h\ d\mu} = 1$, then for all $v \in C(\Sigma, \reals)$,
		\[
			\frac{1}{\lambda^n}L_f^n{v} \to h \int{v\ d\mu}
		\]
		uniformly. \label{rpf:4}
	\end{enumerate}
\end{theorem}

\chapter{Entropy} \label{chap:entropy}
\section{Overview}
Entropy is an important property used to distinguish measure-preserving transformations from each other and is used extensively in ergodic theory. Chapter \ref{chap:concentration-bounds} looks at methods for estimating entropy and finding inequalities to describe how these `estimators' behave. This chapter focuses on defining entropy and explaining why it is a useful property.

Throughout this chapter $(X, \B, \mu)$ will denote a probability space.

\section{Isomorphisms of measure-preserving transformations}\label{sec:isos-of-mpts}
One of the main problems in ergodic theory is to classify measure-preserving transformations. To this end, we want to decide the conditions required for two measure-preserving transformations to be `the same' -- up to sets of measure zero.

\emph{This section predominantly follows material in \cite[Chapter 2]{walters:intro-to-ergodic-theory}.}

\subsection{Isomorphism and conjugacy of measure spaces}

We begin by defining when two probability spaces are isomorphic or conjugate.

\begin{definition}
	Two probability spaces $(X_1, \B_1, \mu_1), (X_2, \B_2, \mu_2)$ are \key{isomorphic} if there exists $M_1 \in \B_1$, $M_2 \in \B_2$ such that $\mu_1(M_1) = 1 = \mu_2(M_2)$ and if there exists an invertible measure-preserving transformation $\phi: M_1 \to M_2$.
\end{definition}

Let $A, C \subset \B$. We define an equivalence relation on $\B$: we have $A \sim C$ if and only if $\mu(A \symdiff C) = 0$. In other words, $A$ and $C$ belong to the same equivalence class if they are equal almost everywhere. It can be easily checked that $\sim$ is indeed an equivalence relation.

Let $\tilde{\B}$ denote the collection of all equivalence classes in $\B$. Since $\B$ is a $\sigma$-algebra, it is clear that $\tilde{\B}$ is also a $\sigma$-algebra. We can define a measure $\tilde{\mu} : \tilde{\B} \to \reals^+$ by $\tilde{\mu}(\tilde{B}) = \mu(B)$, where $B$ belongs to the equivalence class $\tilde{B}$.

\begin{definition}
	A \key{measure algebra} is a Boolean $\sigma$-algebra equipped with a measure.
\end{definition}

In view of this definition, we see that $(\tilde{\B}, \tilde{\mu})$ is a \key{measure algebra}.

\begin{definition}
	Let $(X_1, \B_1, \mu_1), (X_2, \B_2, \mu_2)$ be probability spaces with corresponding measure algebras $(\tilde{\B}_1, \tilde{\mu}_1), (\tilde{\B}_2, \tilde{\mu}_2)$, respectively.
	
	We say $(\tilde{\B}_1, \tilde{\mu}_1)$ and $(\tilde{\B}_2, \tilde{\mu}_2)$ are \key{isomorphic} if there exists a bijection $\phi : \tilde{\B}_2 \to \tilde{\B}_1$ which preserves complementation and countable unions and intersections such that $\tilde{\mu}_1(\phi \tilde{B}) = \tilde{\mu}_2(\tilde{B})$ for all $\tilde{B} \in \tilde{\B}_2$.
	
	The probability spaces $(X_1, \B_1, \mu_1)$ and $(X_2, \B_2, \mu_2)$ are \key{conjugate} if their corresponding measure algebras are isomorphic.
\end{definition}

\begin{proposition}
	If two probability spaces are isomorphic, then they are also conjugate.
	\begin{proof}
		Suppose $(X_1, \B_1, \mu_1), (X_2, \B_2, \mu_2)$ are isomorphic probability spaces with corresponding measure algebras $(\tilde{\B}_1, \tilde{\mu}_1), (\tilde{\B}_2, \tilde{\mu}_2)$. By definition, this means there exists $M_1 \in \B_1$, $M_2 \in \B_2$ such that $\mu_1(M_1) = 1 = \mu_2(M_2)$ and there exists an invertible measure-preserving transformation $\phi: M_1 \to M_2$.
		
		Now we can define the map
		\[
			\psi : \tilde{\B}_2 \to \tilde{\B}_1 : \tilde{B} \mapsto (\phi^{-1}(M_2 \cap B))^\sim.
		\]
		This is clearly a bijection and, since $\phi$ is measure-preserving and $M_2 = X_2$ almost everywhere, we have
		\[
			\tilde{\mu}_1(\psi\tilde{B}) = \tilde{\mu}_1(\phi^{-1}(M_2 \cap B))^\sim = \tilde{\mu}_2(M_2 \cap B)^\sim = \tilde{\mu}_2(\tilde{B}),
		\]
		for all $\tilde{B} \in \tilde{\B}_2$. Therefore the measure algebras are isomorphic and hence the corresponding measure spaces are conjugate.
	\end{proof}
\end{proposition}

The converse statement is not necessarily true. Indeed, suppose we have the probability space $(X_1, \B_1, \mu_1)$ consisting of exactly one point, and another probability space $(X_2, \B_2, \mu_2)$ consisting of exactly two points, with $\B_2 = \{\emptyset, X_2\}$. It is easy to see that the measure algebras are isomorphic and hence the measure spaces are conjugate.

We need to choose $M_1 \in \B_1$, $M_2 \in \B_2$ such that $\mu_1(M_1) = 1 = \mu_2(M_2)$; the only possibility is $M_1 = X_1$ and $M_2 = X_2$. However there does not exist bijection between these two sets, so the probability spaces are \emph{isomorphic}.

\subsection{A motivational example}
We describe a scenario when two measure-preserving transformations could be considered `the same'. We follow the example in \cite[p58]{walters:intro-to-ergodic-theory}.

We first introduce a new probability space.

\begin{comment}
Let $Y = \{0, 1\}$ and let $(p_0, p_1)$ be a probability vector with no zero entries. Then $(Y, 2^Y, \nu)$ is a measure space, with measure $\nu$ defined by $\nu(y) = p_y$ for $y \in Y$. Now let $X = \{(x_j)_{j = 0}^\infty \mid x_j \in Y\}$, the space of infinite sequences with entries in $Y = \{0, 1\}$.
\end{comment}
\subsubsection{Bernoulli shifts}
Let $Y = \{0, 1, \dots, k - 1\}$ be a set of $k - 1$ symbols and let $p = (p_0, p_1, \dots, p_{k - 1})$ be a probability vector with no zero entries. Let $X = \{(x_j)_{j = 0}^\infty \mid x_j \in Y \text{ for all } j \geq 0\}$ be the space of infinite sequences with entries in $Y$. We may define a measure $\nu$ on cylinders of length $n$ by
\[
	\nu[x_0, x_1, \dots, x_{n - 1}] = p_{x_0} p_{x_1} \dots p_{x_{n - 1}}.
\]
Such measures are known as \key{Bernoulli measures}. Let $\sigma : X \to X$ be the one-sided, left shift map on $X$.

\begin{proposition}
	The measure $\nu$ is $\sigma$-invariant.
	\begin{proof}
		We have
		\begin{align*}
			\nu(\sigma^{-1}[x_1, \dots, x_n]) &= \nu\left(\bigsqcup_{j = 0}^{k - 1}{[j, x_1, \dots, x_n]}\right) \\
				&= \sum_{j = 0}^{k - 1}{\nu[j, x_1, \dots, x_n]} \\
				&= \sum_{j = 0}^{k - 1}{p_j p_{x_1} \dots p_{x_n}} \\
				&= p_{x_1} \dots p_{x_n} \\
				&= \nu[x_1, \dots, x_n].
		\end{align*}
		(We have used the fact that $\sum_{j = 0}^{k - 1}{p_j} = 1$ on the penultimate line.)
	\end{proof}
\end{proposition}

The shift map $\sigma : (X, \nu) \to (X, \nu)$ is called the one-sided $(p_0, p_1, \dots, p_{k - 1})$-shift.

We are now ready to present two measure-preserving transformations which we argue are `the same'.

\subsubsection{The \texorpdfstring{$\mathbf{\left(\frac{1}{2}, \frac{1}{2}\right)}$}{(1/2, 1/2)}-shift and the doubling map}
Let $T : ([0, 1), \B, \mu) \to ([0, 1), \B, \mu) : x \mapsto 2x \bmod 1$ be the doubling map, where $\B$ is the Borel $\sigma$-algebra on $[0, 1)$ and $\mu$ is Lebesgue measure.

Let $\sigma : (X, \C, \nu) \to (X, \C, \nu)$ be the $\left(\frac{1}{2}, \frac{1}{2}\right)$-shift, where
\[
	X := \{(x_j)_{j = 0}^\infty \mid x_j \in \{0, 1\} \text{ for all } j \geq 0\},
\]
$\C$ is the $\sigma$-algebra generated by all cylinders in $X$, and $\nu$ is the Bernoulli measure as described above with $p = \left(\frac{1}{2}, \frac{1}{2}\right)$.

Define the map $\phi : X \to [0, 1)$ by
\[
	\phi(x_0, x_1, \dots) = \sum_{j = 0}^\infty{\frac{x_j}{2^{j + 1}}} = \frac{x_0}{2^1} + \frac{x_1}{2^2} + \frac{x_2}{2^3} + \dots.
\]
It is easy to see that $\phi$ maps the binary expansion of a number to the actual number itself.

Let $E := \{(x_j)_{j = 0}^\infty \in X \mid (x_j)_{j = N}^\infty \text{ is constant for some } N \geq 0\}$ be the set of sequences in $X$ whose coordinates are eventually constant. Now, if the binary expansion of a number is \emph{not} eventually constant, then this binary expansion is unique. Therefore $\phi$ is \emph{injective} on $X \setminus E$. It is also clear that $\phi$ is \emph{surjective}, since every number in $[0, 1)$ has at least one binary expansion. In addition, it is easy to that $\phi \circ \sigma = T \circ \phi$.

We now show that $\phi$ is measure-preserving. A dyadic interval is an interval of the form $\left[\frac{a}{2^s}, \frac{a + 1}{2^s}\right] \subset [0, 1)$, where $s \in \naturals$. We can write
\[
	\frac{a}{2^s} = \sum_{j = 0}^{s - 1}{\frac{a_j}{2^j}} \quad \text{and} \quad \frac{a + 1}{2^s} = \sum_{j = 0}^\infty{\frac{a_j}{2^j}},
\]
where $a_j \in \{0, 1\}$ for $j = 0, 1, \dots, s - 2$ and $a_k = 1$ for $k \geq s - 1$. In other words, the binary expansion of all numbers in the interval $\left[\frac{a}{2^s}, \frac{a + 1}{2^s}\right]$ agree in the first $s$ positions. Thus,
\begin{align*}
	\nu\left(\phi^{-1}\left[\frac{a}{2^s}, \frac{a + 1}{2^s}\right]\right) &= \nu[a_0, a_1, \dots, a_{s - 1}] \\
		&= \frac{1}{2^s} \\
		&= \mu\left[\frac{a}{2^s}, \frac{a + 1}{2^s}\right].
\end{align*}
Hence $\phi$ is measure-preserving on dyadic intervals, which generate the Borel $\sigma$-algebra $\B$ on $[0, 1)$. We may therefore apply the Kolmogorov Extension Theorem and it follows that $\phi$ is \emph{measure-preserving} on all Borel sets $B \in \B$.

Let $D := \left\{\frac{a}{2^s} \in [0, 1) \mid s \in \naturals,\ 0 \leq a < 2^s\right\}$ be the set of all dyadic rationals in $[0, 1)$. Clearly, $T^{-1}D = D$ and this means that $T^{-1}([0, 1) \setminus D) = [0, 1) \setminus D$. It is also clear that $\sigma^{-1}E = E$ and so $\sigma^{-1}(X \setminus E) = X \setminus E$. So by the above observations, we see that $\phi: X \setminus E \to [0, 1) \setminus D$ is a bijection. It is also clear that $\phi \circ \sigma(x) = T \circ \phi(x)$ for all $x \in X \setminus E$.

Finally, we have $D \subset \rationals$ which gives $\mu(D) = 0$, and we also note that there are countably many sequences in $E$, thus $\nu(E) = 0$. Therefore $\phi$ is an invertible measure-preserving transformation between $X$ and $[0, 1)$ (modulo sets of measure zero), that is, the measure-preserving transformations are \emph{isomorphic}. Therefore it makes sense to say that these measure-preserving transformations are `the same'.

\subsection{\texorpdfstring{\sloppy Isomorphism and conjugacy of measure-preserving transformations}{Isomorphism and conjugacy of measure-preserving transformations}}
We now formalise the ideas illustrated in the above example.

\begin{definition}
	\sloppy Let $(X_1, \B_1, \mu_1, T_1)$, $(X_2, \B_2, \mu_2, T_2)$ be measure-preserving transformations of probability spaces. We say that $T_1$ is \key{isomorphic} to $T_2$ if there exists $M_1 \in \B_1$, $M_2 \in \B_2$ such that $\mu_1(M_1) = 1 = \mu_2(M_2)$ with
	\begin{enumerate}
		\item $T_1{M_1} \subset M_1$ and $T_2{M_2} \subset M_2$, and \label{mpt-iso-i}
		\item there exists an invertible measure-preserving transformation $\phi : M_1 \to M_2$ such that $\phi \circ T_1(x) = T_2 \circ \phi(x)$ for all $x \in M_1$. \label{mpt-iso-ii}
	\end{enumerate}
	If this is the case, we write $T_1 \simeq T_2$.
\end{definition}

Now suppose that $T_1 \simeq T_2$ with $M_1$, $M_2$ and $\phi : M_1 \to M_2$ as in the above definition. Then for $n \geq 1$ we clearly have $T_1^n{M_1} \subset M_1$ and $T_2^n{M_2} \subset M_2$, satisfying condition \ref{mpt-iso-i}. This in turn gives that $\phi \circ T_1^n(x) = T_2^n \circ \phi(x)$ for all $x \in M_1$, satisfying condition \ref{mpt-iso-ii}. In other words, if $T_1 \simeq T_2$, then $T_1^n \simeq T_2^n$ for all $n \geq 1$.

We also have the notion of conjugacy of measure-preserving transformations.

\begin{definition}
	Let $(X_1, \B_1, \mu_1, T_1)$, $(X_2, \B_2, \mu_2, T_2)$ be measure-preserving transformations of probability spaces. We say that $T_1$ is \key{conjugate} to $T_2$ if there exists an isomorphism $\Phi : (\tilde{\B}_2, \tilde{\mu}_2) \to (\tilde{\B}_1, \tilde{\mu}_1)$ of measure algebras such that $\Phi \circ \tilde{T}_2^{-1} = \tilde{T}_1^{-1} \circ \Phi$.
\end{definition}

It can be easily checked that isomorphism and conjugacy are equivalence relations on the set of all measure-preserving transformations.

As with probability spaces, isomorphic measure-preserving transformations are also conjugate. We show this in the following result.

\begin{theorem}\label{thm:walters-2-5}
	Let $(X_1, \B_1, \mu_1, T_1)$, $(X_2, \B_2, \mu_2, T_2)$ be measure-preserving transformations of probability spaces and suppose that $T_1 \simeq T_2$. Then $T_1$ is conjugate to $T_2$.
	
	\begin{proof}
		Suppose that $T_1 \simeq T_2$, so there exists a measure-preserving transformation $\phi : M_1 \to M_2$ such that $\phi \circ T_1(x) = T_2 \circ \phi(x)$ for all $x \in M_1$, where $M_1, M_2$ are as in the definition.
		
		Define $\Phi : (\tilde{\B}_2, \tilde{\mu}_2) \to (\tilde{\B}_1, \tilde{\mu}_1)$ by $\Phi(\tilde{B}) \mapsto (\phi^{-1}(B \cap M_2))^\sim$ for $B \in B_2$. Recall that $\tilde{B}$ is an equivalence class, so it is easy to see that $\Phi$ is an isomorphism. We also have
		\[
			\tilde{T}_1^{-1} \circ \Phi(\tilde{B}) = \tilde{T}_1^{-1} \circ (\phi^{-1}(B \cap M_2))^\sim = \phi^{-1} \circ \tilde{T}_2^{-1} (B \cap M_2)^\sim = \Phi \circ \tilde{T}_2^{-1}(B)
		\]
		for all $B \in \B_2$. Hence $T_1$ is conjugate to $T_2$.
	\end{proof}
\end{theorem}

The converse of this theorem is not necessarily true. However, we will find it useful to know the conditions for which the converse holds. We need the following definition from \cite[Definition A.21]{einsiedler-ward:ergodic-nt}.

\begin{definition}
	Let $Y$ be a set of countably or finitely many points, where each $y \in Y$ has positive measure $p_y > 0$ such that $\sum_{y \in Y}{p_y} \leq 1$. Put $s := 1 - \sum_{y \in Y}{p_y}$ and let $\mathcal{L}[0, s]$ denote the $\sigma$-algebra of Lebesgue measurable sets on the closed interval $[0, s]$. Let $\lambda_{[0, s]}$ denote Lebesgue measure on $[0, s]$.
	
	If the probability space $(X, \B, \mu)$ is isomorphic to the probability space
	\[
		\left([0, s] \sqcup Y,\ \mathcal{L}[0, s],\ \lambda_{[0, s]} + \sum_{y \in Y}{p_y \delta_y} \right),
	\]
	where $\delta_y$ is the Dirac measure at $y$, then we say that $(X, \B, \mu)$ is a \key{Lebesgue space}.
\end{definition}

We will also use the following result, which is proved in \cite[Theorem 12]{royden:real-analysis}.

\begin{lemma} \label{lem:walters-thm-2-2}
	For $j = 1, 2$, let $(X_j, \B(X_j), \mu_j)$ be complete separable metric spaces endowed with Borel $\sigma$-algebra $\B(X_j)$ and probability measure $\mu_j$. Suppose that $\Phi: \tilde{\B}(X_2) \to \tilde{\B}(X_1)$ is an isomorphism of measure algebras. Then there exists $M_1 \in \B(X_1)$, $M_2 \in \B(X_2)$ such that $\mu_1(M_1) = 1 = \mu_2(M_2)$, and an invertible measure-preserving transformation $\phi: M_1 \to M_2$ such that $\Phi(\tilde{B}) = (\phi^{-1}(B \cap M_2))^\sim$ for all $B \in \B(X_2)$.
	
	If $\psi$ is any other isomorphism $(X_1, \B(X_1), \mu_1)$ to $(X_2, \B(X_2), \mu_2)$ which induces $\Phi$, then $\mu_1\{x \in X_1 \mid \phi(x) \neq \psi(x)\} = 0$.
\end{lemma}

The following result gives the conditions for which the converse of \thref{thm:walters-2-5} is true.

\begin{theorem} \label{thm:walters-2-6}
	Suppose that either $(X_1, \B_1, \mu_1)$, $(X_2, \B_2, \mu_2)$ are Lebesgue spaces, or that $X_1, X_2$ are each complete separable metric spaces with corresponding Borel $\sigma$-algebras $\B_1, \B_2$. Suppose that $T_1 : X_1 \to X_1$, $T_2 : X_2 \to X_2$ are measure-preserving transformations and that $T_1$ is conjugate to $T_2$. Then $T_1 \simeq T_2$.
	\begin{proof}
		Suppose that $\Phi : (\tilde{B}_2, \tilde{\mu}_2) \to (\tilde{B}_1, \tilde{\mu}_1)$ is an isomorphism of measure algebras such that $\Phi \circ \tilde{T}_2^{-1} = \tilde{T}_1^{-1} \circ \Phi$. By \thref{lem:walters-thm-2-2} there exists sets $X'_1 \in \B_1$, $X'_2 \in \B_2$ such that $\mu_1(X'_1) = 1 = \mu(X'_2)$, and there exists an invertible measure-preserving transformation $\phi : X'_1 \to X'_2$ such that $\Phi(\tilde{B}) = (\phi^{-1}(B \cap X'_2))^\sim$ for all $B \in \B_2$. Then we have $\tilde{\phi}^{-1} \circ \tilde{T}_2^{-1} = \tilde{T}_1^{-1} \circ \tilde{\phi}^{-1}$, i.e. $T_2 \circ \phi = \phi \circ T_1$ almost everywhere.
		
		Now put
		\[
			A_1 := \{x \in X_1 \mid T_2 \circ \phi(x) = \phi \circ T_1(x)\} \quad \text{and} \quad M_1 := \bigcap_{n = 0}^\infty{T_1^{-n}{A_1}}.
		\]
		Then $\mu_1(M_1) = 1$ and $T_1^{-1}{M_1} \supset M_1$ which means that $M_1 \supset T_1 M_1$. We then define $M_2 := \phi M_1$ so that $T_2 M_2 \subset M_2$. Hence $T_1 \simeq T_2$.
	\end{proof}
\end{theorem}

As we mentioned briefly at the beginning of Section \ref{sec:isos-of-mpts}, we want to be able to decide when two measure-preserving transformations are `the same'. In view of the above discussion, `the same' can be replaced with `conjugate' or `isomorphic'. \key{Entropy} is one of the main conjugacy and isomorphism invariants studied in ergodic theory, and the remainder of this chapter will describe how the entropy of a measure-preserving transformation is defined.

The rest of this chapter predominantly follows \cite[Chapter 4]{walters:intro-to-ergodic-theory} unless otherwise stated. In particular, any definitions relating to \emph{information} is derived from \cite[p33-34]{parry-pollicott:zeta-fns-periodic-orbits}

\section{Entropy of partitions and sub-\texorpdfstring{$\sigma$}{sigma}-algebras}
\subsection{Partitions and sub-\texorpdfstring{$\sigma$}{sigma}-algebras}

We begin with a finite partition $\alpha = \{A_1, \dots, A_m\}$ of $(X, \B, \mu)$, i.e. the $A_j$ are pairwise disjoint and $X = \bigsqcup_{j = 1}^m{A_j}$. For clarity, we will denote partitions by the Greek letters, usually $\alpha, \beta$ or $\gamma$. Consider the collection of all elements of $\B$ such that their unions are elements of $\alpha$. Such a collection is a sub-$\sigma$-algebra of $\B$ and we will denote it by $\A(\alpha)$.

On the other hand, consider a finite sub-$\sigma$-algebra $\C = \{C_1, \dots, C_n\}$ of $\B$. We will use script uppercase letters to denote sub-$\sigma$-algebras, usually $\A, \C$ or $\D$. We can form a partition of $X$ by $\{B_1, \dots, B_n\}$, where $B_j = C_j$ or $X \setminus C_j$. We denote this partition by $\alpha(\C)$.

Note that if $\C$ is a sub-$\sigma$-algebra of $\B$ and $\gamma$ is a partition of $X$, then $\A(\alpha(\C)) = \C$ and $\alpha(\A(\gamma)) = \gamma$. This means that there is a one-to-one correspondence between finite partitions of $X$ and finite sub-$\sigma$-algebras of $\B$. Hence, in a lot of cases, we may use ``partition'' and ``sub-$\sigma$-algebra'' interchangeably.

If $T: X \to X$ is a measure-preserving transformation and $n \geq 0$, then $T^{-n}{\alpha}$ denotes the partition $\{T^{-n}{A_1}, \dots, T^{-n}{A_k}\}$.

\begin{remark}
	Let $\alpha = \{A_1, \dots, A_m\}$ be a finite partition of $(X, \B, \mu)$. Throughout this chapter, we may assume without loss of generality that $\mu(A_j) > 0$ for all $j = 1, \dots, m$. Indeed, we may index $\alpha$ so that
	\[
		\mu(A_j)
		\begin{cases}
			> 0, & \text{if } 1 \leq j \leq p; \\
			= 0, & \text{if } p + 1 \leq j \leq m.
		\end{cases}
	\]
	Then we may form a new partition $\alpha'$ consisting of the sets $A_1, \dots, A_{p - 1}$ and $\bigsqcup_{j = p}^m{A_j}$. Clearly, the disjoint union has the same measure as $A_p$ and so all the sets in $\alpha'$ have strictly positive measure.
	
	This argument can be easily modified for countable partitions.
\end{remark}

\begin{definition}
	Suppose that $\alpha, \gamma$ are finite partitions of $(X, \B, \mu)$. If each element of $\alpha$ can be written as the union of elements of $\gamma$, then we write \key{$\alpha \leq \gamma$}. In particular, we have $\alpha \leq \gamma$ if and only if $\A(\alpha) \subset \A(\gamma)$, and $\A \subset \C$ if and only if $\alpha(\A) \leq \alpha(\C)$.
\end{definition}

\begin{definition}
	Let $\alpha = \{A_1, \dots, A_m\}$, $\gamma = \{C_1, \dots, C_n\}$ be two finite partitions of a measure space $(X, \B, \mu)$. We define their \key{join} $\alpha \join \gamma$ as the partition
	\[
		\alpha \join \gamma := \{A_j \cap C_k \mid 1 \leq j \leq m, 1 \leq k \leq n\}.
	\]
	If $\A, \C$ are finite sub-$\sigma$-algebras of $\B$, then we define the join $\A \join \C$ in the same way. If this is the case, then $\A \join \C$ is actually the smallest sub-$\sigma$-algebra of $\B$ containing both $\A$ and $\C$.
	
	It is clear that $\A \join \C$ is comprised of the unions of sets of the form $A \cap C$, where $A \in \A, C \in \C$.
	
	We also have the relations $\alpha(\A \join \C) = \alpha(\A) \join \alpha(\C)$ and $\A(\alpha \join \gamma) = \A(\alpha) \join \A(\gamma)$.
\end{definition}

\begin{remark}
	If $T : X \to X$ is a measure-preserving transformation and $n \geq 0$, then $T^{-n}$ preserves set theoretic operations and so we have
	\begin{enumerate}
		\item $\alpha(T^{-n}{\A}) = T^{-n}{\alpha(\A)}$,
		\item $\A(T^{-n}{\alpha}) = T^{-n}{\A(\alpha)}$,
		\item $T^{-n}(\A \join \C) = T^{-n}{\A} \join T^{-n}{\C}$,
		\item $T^{-n}(\alpha \join \gamma) = T^{-n}{\alpha} \join T^{-n}{\gamma}$,
		\item if $\alpha \leq \gamma$, then $T^{-n}{\alpha} \leq T^{-n}{\gamma}$,
		\item if $\A \subset \C$, then $T^{-n}{\A} \subset T^{-n}{\C}$.
	\end{enumerate}
\end{remark}

\begin{definition}
	Let $\alpha, \gamma$ be two partitions of $(X, \B, \mu)$. We say that $\alpha$ and $\gamma$ are \key{independent} if $\mu(A \cap C) = \mu(A)\mu(C)$ for all $A \in \alpha$, $C \in \gamma$.
\end{definition}

\subsection{Motivation for information and entropy}
The following motivation for information and entropy follows that of \cite[Lecture 23]{ergodic-lectures}.

Suppose that we want to locate a point $x \in X$. To do this, we can partition the state space $X$ by the finite partition $\alpha = \{A_1, \dots, A_k\}$. We will later show that we may also consider countable partitions. If we find that $x \in A_j$, then we have received some \key{information}, and we think of $\mu(A_j)$ to be the probability that this happens.

We would like to define a function $I_\mu(\alpha) : X \to \reals^+$ such that $I_\mu(\alpha)(x)$ is the information received upon observing that $x \in A_j$. We want $I_\mu(\alpha)$ to only depend on $\mu(A_j)$, and in particular, we should receive more information if $\mu(A_j)$ is small, and we should receive less information if $\mu(A_j)$ is large. So we want $I_\mu(\alpha)$ to be of the form
\[
	I_\mu(\alpha)(x) = \sum_{A \in \alpha}{\chi_A(x)\phi(\mu(A))},
\]
where $\phi : [0, 1] \to \reals^+$ is some nonnegative function.

For two independent partitions $\alpha, \gamma$, the information gained from observing that $x \in A \cap C$, where $A \in \alpha, C \in \gamma$, should be equal to the information we gain from observing $x \in A$ in addition to observing $x \in C$. In view of this, we would require that $I_\mu(\alpha \join \gamma) = I_\mu(\alpha) + I_\mu(\gamma)$.

Combining the above requirements, we get that $\phi(\mu(A \cap C)) = \phi(\mu(A)\mu(C)) = \phi(\mu(A)) + \phi(\mu(C))$. For $\phi$ to be a continuous function, we see that $\phi(t)$ must be a multiple of $-\log{t}$. This gives rise to the following definitions.

\subsection{Information and entropy of partitions}
\begin{definition}
	Let $\alpha$ be a partition of $(X, \B, \mu)$. We define the \key{information} $I_\mu(\alpha) : X \to \reals^+$ of the partition $\alpha$ (or of the sub-$\sigma$-algebra $\A(\alpha)$) by
	\[
		I_\mu(\A(\alpha))(x) = I_\mu(\alpha)(x) := -\sum_{A \in \alpha}{\chi_A(x) \log{\mu(A)}}.
	\]
	We define the \key{entropy} $H_\mu(\alpha)$ of the partition $\alpha$ (or of the sub-$\sigma$-algebra $\A(\alpha)$) to be the average of the information, i.e.
	\begin{align*}
		H_\mu(\A(\alpha)) = H_\mu(\alpha) &:= \int{I_\mu(\alpha)\ d\mu} \\
			&= \int{-\sum_{A \in \alpha}{\chi_A \log{\mu(A)}}\ d\mu} \\
			&= -\sum_{A \in \alpha}{\mu(A) \log{\mu(A)}}.
	\end{align*}
	Whenever we use this definition and those derived from it, we will use the convention that $x \log x = 0$ if $x = 0$.
\end{definition}

\begin{remark}
	If $\alpha = \{X, \emptyset\}$, then we don't gain any information from performing observations on this partition, so $H(\alpha) = 0$. This can also be easily verified from the definition above.
\end{remark}

It is useful to know that, given a partition of $(X, \B, \mu)$ into $k$ sets, we can find an upper bound for the entropy of the partition.

\begin{proposition} \label{prop:walters-cor-4-2-1}
	Let $\alpha = \{A_1, \dots, A_k\}$ be a partition of $(X, \B, \mu)$ into $k$ sets. Then $H_\mu(\alpha) \leq \log{k}$.
	
	In particular, we have $H_\mu(\alpha) = \log{k}$ if and only if $\mu(A_j) = 1 / k$ for all $j = 1, \dots k$.
	
	\begin{proof}
		By \thref{thm:walters-4-2-xlogx-convex}, $x \log{x}$ is strictly convex. This means that for any partition $\alpha = \{A_1, \dots, A_k\}$ of $(X, \B, \mu)$ and for any $\{\lambda_j \in [0, 1] \mid j \in \{1, \dots, k\},\ \sum_{j = 1}^k{\lambda_j} = 1\}$, we have
		\[
			\left(\sum_{j = 1}^k{\lambda_j \mu(A_j)}\right) \log{\left(\sum_{j = 1}^k{\lambda_j \mu(A_j)}\right)} \leq \sum_{j = 1}^k{\lambda_j \mu(A_j) \log{\mu(A_j)}},
		\]
		with equality if and only if $\mu(A_1) = \mu(A_2) = \dots = \mu(A_k)$ whenever $\lambda_j \neq 0$ for all $j = 1, \dots, k$.
		
		Substituting in $\lambda_j = 1 / k$ for all $j = 1, \dots, k$ and rearranging, we get
		\[
			H_\mu(\alpha) = -\sum_{j = 1}^k{\mu(A_j) \log{\mu(A_j)}} \leq -\log{\frac{1}{k}} = \log{k},
		\]
		with equality if and only if $\mu(A_j) = 1 / k$ for all $j = 1, \dots, k$.
	\end{proof}
\end{proposition}

\section{Conditional entropy}
\subsection{Conditional expectation}
The definitions and results in this subsection follow those in \cite[p8-9]{walters:intro-to-ergodic-theory}.
\begin{definition}
	Suppose that $\mu$, $\nu$ are probability measures on a measurable space $(X, \B)$. If all sets $B \in \B$ with $\mu$-measure zero are also sets of $\nu$-measure zero, then we say that $\nu$ is \key{absolutely continuous} with respect to $\mu$. If this is the case, we write $\nu \ll \mu$.
	
	Stated alternatively, we have $\nu \ll \mu$ if, for all $B \in \B$ such that $\mu(B) = 0$, then $\nu(B) = 0$.
	
	Note that there may be more sets of $\nu$-measure zero. In the case where $\nu \ll \mu$ and $\mu \ll \nu$, we say that $\mu$ and $\nu$ are \key{equivalent}.
\end{definition}

\begin{theorem}[Radon-Nikodym Theorem] \label{thm:radon-nikodym}
	Suppose that $\mu, \nu$ are probability measures on a measurable space $(X, \B)$. Then $\nu \ll \mu$ if and only if there exists a nonnegative $\mu$-integrable function $f \in L^1(X, \B, \mu)$ where $f \geq 0$, $\int{f\ d\mu} = 1$, such that $\nu(B) = \int_B{f\ d\mu}$ for all $B \in \B$.
	
	Moreover, the function $f$ is unique almost everywhere, i.e. if there exists another function $g$ which satisfies the above properties, then $f = g$ $\mu$-almost everywhere.
\end{theorem}

The Radon-Nikodym Theorem allows us to define the conditional expectation operator.

\begin{definition}
	Let $(X, \B, \mu)$ be a measure space and let $\C$ be a sub-$\sigma$-algebra of $\B$. The \key{conditional expectation} operator $E_\mu(\seedot \mid \C) : L^1(X, \B, \mu) \to L^1(X, \C, \mu)$ is defined as follows.
	
	If $f \in L^1(X, \B, \mu)$ is a nonnegative real-valued integrable function, then
	\[
		\nu_f(C) = a^{-1}\int_C{f\ d\mu},
	\]
	for $C \in \C$, where $a = \int_X{f\ d\mu}$, defines a probability measure $\nu_f$ on $(X, \C)$ with $\nu_f \ll \mu$. By \thref{thm:radon-nikodym}, there exists a nonnegative function $E_\mu(f \mid \C) \in L^1(X, \C, \mu)$ such that $\int_C{E_\mu(f \mid \C)\ d\mu} = \int_C{f\ d\mu}$ for all $C \in \C$. Furthermore, $E_\mu(f \mid \C)$ is unique almost everywhere.
	
	If $f$ is a real-valued function, we consider the positive and negative parts of $f = f^+ - f^-$, where $f^+, f^- \geq 0$, and define $E_\mu(f \mid \C) := E_\mu(f^+ \mid \C) - E_\mu(f^- \mid \C)$.
	
	If $f$ is complex-valued, we take the real and imaginary parts of $f$ and define $E_\mu(f \mid \C)$ linearly as above.
\end{definition}

The conditional expectation operator $E_\mu(f \mid \C)$ is uniquely determined by the requirement that $E_\mu(f \mid \C)$ is $\C$-measurable, and also that
\[
	\int_C{f\ d\mu} = \int_C{E_\mu(f \mid \C)\ d\mu},
\]
for all $C \in \C$. With this in mind, we can think of $E_\mu(f \mid \C)$ as the best approximation of $f$ in the smaller space $\C$ of measurable functions.~\cite[Lecture 21]{ergodic-lectures}

\subsubsection{Properties of \texorpdfstring{$E_\mu(\seedot \mid \C)$}{the conditional expectation operator}}
\begin{enumerate}
	\item Conditional expectation $E_\mu(\seedot \mid \C)$ is a linear operator. \label{cond-exp:1}
	\item If $f \geq 0$, then $E_\mu(f \mid \C)$. \label{cond-exp:2}
	\item If $f \in L^1(X, \B, \mu)$ and $g$ is a $\C$-measurable bounded function, then $E_\mu(fg \mid \C) = gE_\mu(f \mid \C)$. \label{cond-exp:3}
	\item For $f \in L^1(X, \B, \mu)$, we have $\left|E_\mu(f \mid \C)\right| \leq E_\mu(|f| \mid \C)$. \label{cond-exp:4}
	\item If $\C_2 \subset \C_1$, then for $f \in L^1(X, \B, \mu)$, we have $E_\mu(E_\mu(f \mid \C_1) \mid \C_2) = E_\mu(f \mid \C_2)$. \label{cond-exp:5}
\end{enumerate}

If $f$ is an integrable function, then we can find $E_\mu(f \mid \C)$ using the following formula.

\begin{proposition}
	Let $\C$ be a finite or countable sub-$\sigma$-algebra of $\B$. Then
	\[
		E_\mu(f \mid \C)(x) = \sum_{C \in \gamma}{\int_{C}{f\ d\mu}\frac{\chi_{C}(x)}{\mu(C)}}.
	\]
	
	\begin{proof}
		We follow the proof given in \cite[Example 10.1.2]{bogachev:measure}.
		
		The summation is clearly an integrable function, and the $\C$-measurable functions are exactly the characteristic functions of $C \in \C$. Therefore the result is equivalent to
		\[
			\int{\chi_B(x) \cdot E_\mu(f \mid \C)(x)\ d\mu} = \int{\left(\chi_B(x) \cdot \sum_{C \in \gamma}{\int_{C}{f\ d\mu}\frac{\chi_{C}(x)}{\mu(C)}}\right)\ d\mu},
		\]
		for any $B \in \C$. This is clearly true since by definition,
		\[
			\int{\chi_B(x) \cdot E_\mu(f \mid \C)(x)\ d\mu} = \int_B{f\ d\mu},
		\]
		and
		\[
			\int{\left(\chi_B(x) \cdot \sum_{C \in \gamma}{\int_{C}{f\ d\mu}\frac{\chi_{C}(x)}{\mu(C)}}\right)\ d\mu} = \int{\left({\int_{B}{f\ d\mu}\frac{\chi_{B}(x)}{\mu(B)}}\right)\ d\mu} = \int_{B}{f\ d\mu}.
		\]
	\end{proof}
\end{proposition}

\begin{definition}
	Let $\C \subset \B$ be a sub-$\sigma$-algebra of a $\sigma$-algebra $\B$. The \key{conditional probability} of $B \in \B$ given $\C$ is defined
	\[
		\mu(B \mid \C) := E_\mu(\chi_B \mid \C).
	\]
\end{definition}

\subsection{Conditional information and entropy}
We can define the conditional information and entropy of a partition $\alpha$, given that we know the information gained from the partition $\gamma$. Conditional entropy will prove to be useful later when look at the entropy of measure-preserving transformations.

\begin{definition}
	Let $(X, \B, \mu, T)$ be a measure-preserving transformation on a probability space. Let $\A, \C$ be finite sub-$\sigma$-algebras, where $\alpha(\A) = \{A_1, \dots, A_p\}$, $\alpha(\C) = \{C_1, \dots, C_q\}$. The \key{conditional entropy} of $\alpha$ given $\C$ is defined
	\begin{align*}
		H_\mu(\alpha(\A) \mid \alpha(\C)) = H_\mu(\A \mid \C) &:= -\sum_{k = 1}^q{\mu(C_k) \sum_{j = 1}^{p}{\frac{\mu(A_j \cap C_k)}{\mu(C_k)} \log{\frac{\mu(A_j \cap C_k)}{\mu(C_k)}}}} \\
			&= -\sum_{j, k}{\mu(A_j \cap C_k) \log{\frac{\mu(A_j \cap C_k)}{\mu(C_k)}}}.
	\end{align*}
\end{definition}

\begin{remark}
	If $\mathcal{N} = \{X, \emptyset\}$, then we have $H_\mu(\alpha \mid \mathcal{N}) = H_\mu(\alpha)$. Again, this is because we gain no information from $\mathcal{N}$.
\end{remark}

\begin{theorem} \label{thm:walters-4.3}
	Let $(X, \B, \mu)$ be a probability space and let $\A, \B, \D$ be finite sub-$\sigma$-algebras of $\B$. Suppose $T : X \to X$ is a measure-preserving transformation. Then
	\begin{enumerate}
		\item $H_\mu(\A \join \C \mid \D) = H_\mu(\A \mid \D) + H_\mu(\C \mid \A \join \D)$, \label{walters-thm-4.3:1}
		\item $H_\mu(\A \join \C) = H_\mu(\A) + H_\mu(\C \mid \A)$, \label{walters-thm-4.3:2}
		\item if $\A \subset \C$, then $H_\mu(\A \mid \D) \leq H_\mu(\C \mid \D)$, \label{walters-thm-4.3:3}
		\item if $\A \subset \C$, then $H_\mu(\A) \leq H_\mu(\C)$, \label{walters-thm-4.3:4}
		\item if $\C \subset \D$, then $H_\mu(\A \mid \C) \geq H_\mu(\A \mid \D)$, \label{walters-thm-4.3:5}
		\item $H_\mu(\A) \geq H_\mu(\A \mid \D)$, \label{walters-thm-4.3:6}
		\item $H_\mu(\A \join \C \mid \D) \leq H_\mu(\A \mid \D) + H_\mu(\C \mid \D)$, \label{walters-thm-4.3:7}
		\item $H_\mu(\A \join \C) \leq H_\mu(\A) + H_\mu(\C)$, \label{walters-thm-4.3:8}
		\item $H_\mu(T^{-1}\A \mid T^{-1}\C) = H_\mu(\A \mid \C)$, \label{walters-thm-4.3:9}
		\item $H_\mu(T^{-1}\A) = H_\mu(\A)$. \label{walters-thm-4.3:10}
	\end{enumerate}
	\begin{proof} \hfill
		\begin{enumerate}
			\item We have
				\[
					H_\mu(\A \join \C \mid \D) = -\sum_{j, k, m}{\mu(A_j \cap C_k \cap D_m) \log{\frac{\mu(A_j \cap C_k \cap D_m)}{\mu(D_m)}}}.
				\]
				If $\mu(A_j \cap D_m) \neq 0$, then
				\[
					\frac{\mu(A_j \cap C_k \cap D_m)}{\mu(D_m)} = \frac{\mu(A_j \cap C_k \cap D_m)}{\mu(A_j \cap D_m)} \frac{\mu(A_j \cap D_m)}{\mu(D_m)}.
				\]
				If $\mu(A_j \cap D_m) = 0$, then the above evaluates to zero anyway, so we ignore such terms. Therefore we have
				\begin{align*}
					H_\mu(\A \join \C \mid \D) &= -\sum_{j, k, m}{\mu(A_j \cap C_k \cap D_m) \log{\frac{\mu(A_j \cap D_m)}{\mu(D_m)}}} \\
						& \qquad - \sum_{j, k, m}{\mu(A_j \cap C_k \cap D_m) \log{\frac{\mu(A_j \cap C_k \cap D_m)}{\mu(A_j \cap D_m)}}} \\
						&= -\sum_{j, m}{\mu(A_j \cap D_m) \log{\frac{\mu(A_j \cap D_m)}{\mu(D_m)}}} + H_\mu(\C \mid \A \join \D) \\
						&= H_\mu(\A \mid \D) + H_\mu(\C \mid \A \join \D).
				\end{align*}
			\item We put $\D = \{X, \emptyset\}$ in \ref{walters-thm-4.3:1}. Then by the above remark the result follows immediately.
			\item Suppose that $\A \subset \C$. Then
				\begin{align*}
					H_\mu(\C \mid \D) &= H_\mu(\A \join \C \mid \D) & \text{(since } \A \subset \C) \\
						&= H_\mu(\A \mid \D) + H_\mu(\C \mid \A \join \D) & \text{(by \ref{walters-thm-4.3:1})} \\
						&\geq H_\mu(\A \mid \D).
				\end{align*}
			\item As with \ref{walters-thm-4.3:2}, we put $\D = \{X, \emptyset\}$ in \ref{walters-thm-4.3:3}.
			\item Suppose that $\C \subset \D$ and fix $j, k$. We have
				\[
					\sum_{m}{\frac{\mu(D_m \cap C_k)}{\mu(C_k)}} = 1
				\]
				and so we may apply Theorem \ref{thm:walters-4-2-xlogx-convex}. So with $f(x) = x\log{x}$, we have
				\begin{equation} \label{fml:walters-4-3-5-ineq}
					f\left(\sum_{m}{\frac{\mu(D_m \cap C_k)}{\mu(C_k)} \frac{\mu(A_j \cap D_m)}{\mu(D_m)}}\right) \leq \sum_{m}{\frac{\mu(D_m \cap C_k)}{\mu(C_k)} f\left(\frac{\mu(A_j \cap D_m)}{\mu(D_m)}\right)}.
				\end{equation}
				Since $\C \subset \D$, we have
				\begin{align*}
					f\left(\sum_{m}{\frac{\mu(D_m \cap C_k)}{\mu(C_k)} \frac{\mu(A_j \cap D_m)}{\mu(D_m)}}\right) &= f\left(\frac{\mu(A_j \cap C_k)}{\mu(C_k)}\right) \\
						&= \frac{\mu(A_j \cap C_k)}{\mu(C_k)} \log{\frac{\mu(A_j \cap C_k)}{\mu(C_k)}}.
				\end{align*}
				Then multiplying \eqref{fml:walters-4-3-5-ineq} by $\mu(C_k)$ and then summing over $j, k$, we get
				\begin{align*}
					-H_\mu(\A \mid \C) &= \sum_{j, k}{\mu(A_j \cap C_k) \log{\frac{\mu(A_j \cap C_k)}{\mu(C_k)}}} \\
						&\leq \sum_{j, k, m}{\mu(D_m \cap C_k) \frac{\mu(A_j \cap D_m)}{\mu(D_m)} \log{\frac{\mu(A_j \cap D_m)}{\mu(D_m)}}} \\
						&= \sum_{j, m}{\mu(A_j \cap D_m) \log{\frac{\mu(A_j \cap D_m)}{\mu(D_m)}}} \\
						&= -H_\mu(\A \mid \D).
				\end{align*}
				Hence $H_\mu(\A \mid \C) \geq H_\mu(\A \mid \D)$.
			\item We put $\C = \{X, \emptyset\}$ in \ref{walters-thm-4.3:5}.
			\item We have
				\begin{align*}
					H_\mu(\A \join \C \mid \D) &= H_\mu(\A \mid \D) + H_\mu(\C \mid \A \join \D) & \text{(by \ref{walters-thm-4.3:1})} \\
						&\leq H_\mu(\A \mid \D) + H_\mu(\C \mid \D) & \text{(by \ref{walters-thm-4.3:5})}.
				\end{align*}
			\item We put $\D = \{X, \emptyset\}$ in \ref{walters-thm-4.3:7}.
			\item This follows since $T$ is measure-preserving and by the definition conditional entropy.
			\item This is also immediate from the definitions.
		\end{enumerate}
	\end{proof}
\end{theorem}

Using conditional expectation, we can define conditional entropy for a finite sub-$\sigma$-algebra $\A$ of $\B$ given an arbitrary (not necessarily finite) sub-$\sigma$-algebra $\C$ of $\B$. We first suppose that $\C$ is finite so that $\alpha(\C) = \{C_1, \dots, C_q\}$, and also let $\alpha(\A) = \{A_1, \dots, A_p\}$. Note that
\[
		E_\mu(\chi_{A_j} \mid \C)(x) = \sum_{k = 1}^{q}{\int_{C_k}{\chi_{A_j}\ d\mu}\frac{\chi_{C_k}(x)}{\mu(C_k)}}.
\]

Then
\begin{align*}
	H_\mu(\alpha(\A) \mid \alpha(\C)) = H_\mu(\A \mid \C) &= -\sum_{j = 1}^{p}{\sum_{k = 1}^{q}{\mu(A_j \cap C_k) \log{\frac{\mu(A_j \cap C_k)}{\mu(C_k)}}}} \\
		&= -\sum_{j = 1}^{p}{\int{\chi_{A_j} \log{E_\mu(\chi_{A_j} \mid \C)}\ d\mu}} \\
		&= -\int{\sum_{j = 1}^{p}{E_\mu(\chi_{A_j} \mid \C) \log{E_\mu(\chi_{A_j} \mid \C)}}\ d\mu}.
\end{align*}

We can therefore make the following definition for countable sub-$\sigma$-algebras $\C$ of $\B$.

\begin{definition}
	Let $(X, \B, \mu)$ be a probability space. Suppose that $\A$ is a finite sub-$\sigma$-algebra of $\B$ and that $\C$ is an \emph{arbitrary} sub-$\sigma$-algebra of $\B$. Denote the partition $\alpha(\A) = \{A_1, \dots, A_p\}$. The \key{conditional entropy} of $\alpha$ given $\C$ is given by
	\[
		H_\mu(\alpha(\A) \mid \alpha(\C)) = H_\mu(\A \mid \C) := -\int{\sum_{j = 1}^{p}{\mu(A_j \mid \C) \log{\mu(A_j \mid \C)}}\ d\mu}.
	\]
\end{definition}

\begin{lemma} \label{lem:walters-4-6}
	Suppose that $\A_1 \subset \A_2 \subset \dots \subset \A_n \subset \dots$ is an increasing sequence of sub-$\sigma$-algebras of $\B$, and write $\A := \bigjoin_{n = 1}^\infty{\A_n}$. Then for all $f \in L^2(X, \B, \mu)$ we have $\|E_\mu(f \mid \A_n) - E_\mu(f \mid \A)\|_2 \to 0$, as $n \to \infty$.
	\begin{proof}
		By definition, the operator $E_\mu(\seedot \mid \A_n)$ maps functions from from $L^2(X, \B, \mu)$ to $L^2(X, \A_n, \mu)$. We let $A \in \A$ and choose a sequence $A_n \in \A_n$ such that $\mu(A_n \symdiff A) \to 0$, as $n \to +\infty$. (This is possible because $\A_n$ is an increasing sequence.)
		
		Since $E_\mu(\chi_A \mid \A_n)$ is a best approximation to $\chi_A$ in $L^2(X, \A_n, \mu)$, we have
		\[
			\|E_\mu(\chi_A \mid \A_n) - \chi_A\|_2^2 \leq \|\chi_{A_n} - \chi_A\|_2^2 = \mu(A_n \symdiff A) \to 0,
		\]
		as $n \to +\infty$.
		
		The set of all finite linear combinations of characteristic functions are dense in $L^2(X, \A, \mu)$ and so for all $g \in L^2(X, \A, \mu)$, we have
		\begin{equation} \label{fml:lem-4-6-star}
			\|E_\mu(g \mid \A_n) - g\|_2 \to 0,
		\end{equation}
		as $n \to +\infty$. Therefore if $f \in L^2(X, \B, \mu)$, then by property \ref{cond-exp:5} on page \pageref{cond-exp:5}, we have $E_\mu(E_\mu(f \mid \A) \mid \A_n) = E_\mu(f \mid \A_n)$ because $\A_n \subset \A$ for all $n \geq 1$. Hence by \eqref{fml:lem-4-6-star} we have
		\[
			\|E_\mu(f \mid \A_n) - E_\mu(f \mid \A)\|_2 = \|E_\mu(E_\mu(f \mid \A) \mid \A_n) - E_\mu(f \mid \A)\|_2 \to 0,
		\]
		as $n \to +\infty$, as required.
	\end{proof}
\end{lemma}

We also have the following theorem.

\begin{theorem}[Increasing Martingale Theorem] \label{thm:increasing-martingale}
	Suppose that $\A_1 \subset \A_2 \subset \dots \subset \A_n \subset \dots$ is an increasing sequence of sub-$\sigma$-algebras of $\B$ such that $\A_n \to \A$, as $n \to +\infty$. Then for all $f \in L^1(X, \B, \mu)$ we have
	\begin{enumerate}
		\item $E_\mu(f \mid \A_n) \to E_\mu(f \mid \A)$ $\mu$-almost everywhere, as $n \to +\infty$, and
		\item $E_\mu(f \mid \A_n) \to E_\mu(f \mid \A)$ in $L_1$, as $n \to +\infty$.
	\end{enumerate}
\end{theorem}

Note that \thref{thm:walters-4.3} was given in terms of \emph{finite} sub-$\sigma$-algebras and hence finite partitions. By the following theorem, we can in fact extend these results for \emph{countable} sub-$\sigma$-algebras and partitions.

\begin{theorem} \label{thm:walters-4-7}
	Suppose that $\A$ is a \emph{finite} sub-$\sigma$-algebra of $\B$. Furthermore, suppose that $\C_1 \subset \C_2 \subset \dots \subset \C_n \subset \dots$ is an increasing sequence of sub-$\sigma$-algebras of $\B$, and put $\C:= \bigvee_{n = 1}^\infty{\C_n}$. Then $H_\mu(\A \mid \C_n) \to H_\mu(\A \mid \C)$, as $n \to +\infty$.
	\begin{proof}
		Let $\alpha(\A) = \{A_1, \dots, \A_k\}$. By \thref{lem:walters-4-6}, $\|E_\mu(\chi_{A_j} \mid \C_n) - E_\mu(\chi_{A_j} \mid \C)\|_2 \to 0$, as $n \to +\infty$ for $j = 1, \dots, k$. So $E_\mu(\chi_{A_j} \mid \C_n)$ converges in measure to $E_\mu(\chi_{A_j} \mid \C)$, i.e. given $\varepsilon > 0$, we have that
		\[
			\lim_{n \to +\infty}{\mu\left\{x \in X \midmid \left|E_\mu(\chi_{A_j} \mid \C_n)(x) - E_\mu(\chi_{A_j} \mid \C)(x)\right| \geq \varepsilon\right\}} = 0.
		\]
		So it is clear that $-\sum_{j = 1}^k{E_\mu(\chi_{A_j} \mid \C_n) \log{E_\mu(\chi_{A_j} \mid \C_n)}}$ also converges in measure to $-\sum_{j = 1}^k{E_\mu(\chi_{A_j} \mid \C) \log{E_\mu(\chi_{A_j} \mid \C)}}$.
		
		Since $E_\mu(\seedot \mid \C)$ is a positive linear operator and since $\sum_{j = 1}^k{\chi_{A_j}} = 1$, we have $0 \leq E_\mu(\chi_{A_j} \mid \C)(x) \leq 1$ for $\mu$-almost every $x$. Hence
		\begin{align*}
			-\sum_{j = 1}^k{\mu(A_j \mid \C)(x) \log{\mu(A_j \mid \C)(x)}} &= -\sum_{j = 1}^k{E_\mu(\chi_{A_j} \mid \C)(x) \log{E_\mu(\chi_{A_j} \mid \C)(x)}} \\
				&\leq k \max_{t \in [0, 1]}(-t \log{t}) \\
				&= ke.
		\end{align*}
		So all functions of this form are bounded by $ke$ and hence converge in $L^1(\mu)$. Therefore, $H_\mu(\A \mid \C_n) \to H_\mu(\A \mid \C)$, as $n \to +\infty$.
	\end{proof}
\end{theorem}

As a result of this theorem, given a countable (not necessarily finite) sub-$\sigma$-algebra $\C$, we can find an increasing sequence $\C_1 \subset \C_2 \subset \dots \subset \C_n \subset \dots$ such that $\C_n \to \C$, as $n \to +\infty$. We then apply \thref{thm:walters-4-7} and we see that any result involving finite sub-$\sigma$-algebras can be extended for countable sub-$\sigma$-algebras.

\section{\texorpdfstring{\sloppy Entropy of measure-preserving transformations}{Entropy of measure-preserving transformations}}
So far, we have be focusing exclusively on the entropy of partitions and sub-$\sigma$-algebras, but we can now introduce a measure-preserving transformation $T : X \to X$. We can think of $T$ as the passing of a day in time, and so $H_\mu\left(\bigjoin_{j = 0}^{n - 1}{T^{-j}{\alpha}}\right)$ is the average information we gain after $n$ days. Given a partition $\alpha$, it is natural to define the entropy of $T$ by the average information we obtain \emph{per day}. First, we need to ensure that this is well-defined.

\begin{theorem} \label{thm:walters-4-9}
	Let $(a_n)_{n = 1}^\infty$ be a sequence of real numbers such that $a_{n + p} \leq a_n + a_p$ for all $n, p \geq 1$. Then $\lim_{n \to +\infty}(a_n / n)$ exists and equals $\inf_{n \geq 1}(a_n / n)$.
	
	This limit could be $-\infty$, but if $a_n$ is bounded below then, by the properties of the sequence, the limit is non-negative.
	\begin{proof}
		Fix $p \geq 1$. We can write $n = kp + j$ for some $0 \leq j < p$, and then
		\[
			\frac{a_n}{n} = \frac{a_{kp + j}}{kp + j} \leq \frac{a_j}{kp} + \frac{a_{kp}}{kp} \leq \frac{a_j}{kp} + \frac{ka_p}{kp} = \frac{a_j}{kp} + \frac{a_p}{p}.
		\]
		We have that $k \to +\infty$, as $n \to +\infty$, and so
		\[
			\frac{a_j}{kp} \to 0,
		\]
		as $n \to +\infty$. Putting the above results together, we have
		\[
			\limsup_{n \to +\infty}{\frac{a_n}{n}} \leq \frac{a_p}{p}.
		\]
		Since $p$ is fixed, we have
		\[
			\limsup_{n \to +\infty}{\frac{a_n}{n}} \leq \inf_{p \geq 1}{\frac{a_p}{p}}.
		\]
		On the other hand, it is clear that
		\[
			\inf_{p \geq 1}{\frac{a_p}{p}} \leq \liminf_{n \to +\infty}{\frac{a_n}{n}}.
		\]
		Therefore
		\[
			\lim_{n \to +\infty}{\frac{a_n}{n}}
		\]
		exists and is equal to the infimum.
	\end{proof}
\end{theorem}

\begin{corollary} \label{cor:walters-4-9-1}
	Let $T : X \to X$ be a measure-preserving transformation and suppose that $\A$ is a finite sub-$\sigma$-algebra of $\B$. Then
	\[
		\lim_{n \to +\infty}{\frac{1}{n} H_\mu\left(\bigjoin_{j = 0}^{n - 1}{T^{-j}{\A}}\right)}
	\]
	exists.
	\begin{proof}
		Let the sequence $(a_n)_{n = 1}^\infty$ be defined by $a_n = H_\mu\left(\bigjoin_{j = 0}^{n - 1}{T^{-j}{\A}}\right) \geq 0$. For any $n, p \geq 1$ we have
		\begin{align*}
			a_{n + p} &= H_\mu\left(\bigjoin_{j = 0}^{n + p - 1}{T^{-j}{\A}}\right) \\
				&\leq H_\mu\left(\bigjoin_{j = 0}^{n - 1}{T^{-j}{\A}}\right) + H_\mu\left(\bigjoin_{j = n}^{n + p - 1}{T^{-j}{\A}}\right) & \text{(by \thref{thm:walters-4.3} \ref{walters-thm-4.3:8})} \\
				&= a_n + H_\mu\left(\bigjoin_{j = 0}^{p - 1}{T^{-j}{\A}}\right) & \text{(by \thref{thm:walters-4.3} \ref{walters-thm-4.3:10})} \\
				&= a_n + a_p.
		\end{align*}
		By \thref{thm:walters-4-9}, the limit of $a_n$ exists and hence
		\[
			\lim_{n \to +\infty}{\frac{1}{n} H_\mu\left(\bigjoin_{j = 0}^{n - 1}{T^{-j}{\A}}\right)}
		\]
		exists.
	\end{proof}
\end{corollary}

This ensures that the following definition is well-defined.

\begin{definition}
	Let $(X, \B, \mu, T)$ be a measure-preserving transformation of a probability space and let $\alpha$ be a finite partition of $X$. The \key{entropy of $T$ with respect to $\alpha$} is defined
	\[
		h_\mu(T, \A(\alpha)) = h_\mu(T, \alpha) := \lim_{n \to +\infty}{\frac{1}{n} H_\mu\left(\bigjoin_{j = 0}^{n - 1}{}T^{-j}{\alpha}\right)}.
	\]
\end{definition}

\begin{theorem} \label{thm:walters-4.12}
	Let $\A$, $\C$ be finite sub-algebras of $\B$ and let $T$ be a measure-preserving transformation of a probability space $(X, \B, \mu)$ Then
	\begin{enumerate}
		\item $h_\mu(T, \A) \leq H_\mu(\A)$, \label{walters:thm-4-12:1}
		\item $h_\mu(T, \A \join \C) \leq h_\mu(T, \A) + h_\mu(T, \C)$, \label{walters:thm-4-12:2}
		\item if $\A \subset \C$ then $h_\mu(T, \A) \leq h_\mu(T, \C)$, \label{walters:thm-4-12:3}
		\item $h_\mu(T, \A) \leq h_\mu(T, \C) + H_\mu(\A \mid \C)$, \label{walters:thm-4-12:4}
		\item $h_\mu(T, T^{-1}{\A}) = h_\mu(T, \A)$, \label{walters:thm-4-12:5}
		\item if $m \geq 1$ then $h_\mu(T, \A) = h_\mu\left(T, \bigjoin\limits_{j = 0}^{m - 1}{T^{-j}{\A}}\right)$, \label{walters:thm-4-12:6}
		\item if $T$ is invertible and $m \geq 1$ then $h_\mu(T, \A) = h_\mu\left(T, \bigjoin\limits_{j = -m}^m{T^{-j}{\A}}\right)$. \label{walters:thm-4-12:7}
	\end{enumerate}
	\begin{proof} \hfill
		\begin{enumerate}
			\item For all $n \geq 1$,
				\begin{align*}
					\frac{1}{n} H_\mu\left(\bigjoin_{j = 0}^{n - 1}{T^{-j}{\A}}\right) &\leq \frac{1}{n} \sum_{j = 0}^{n - 1}{H_\mu({T^{-j}{\A}})} & \text{(by \thref{thm:walters-4.3} \ref{walters-thm-4.3:8})} \\
						&\leq \frac{1}{n} \sum_{j = 0}^{n - 1}{H_\mu(\A)} & \text{(by \thref{thm:walters-4.3} \ref{walters-thm-4.3:10})} \\
						&= H_\mu(\A).
				\end{align*}
			\item We have
				\begin{align*}
					H_\mu\left(\bigjoin_{j = 0}^{n - 1}{T^{-j}(\A \join \C)}\right) &= H_\mu\left(\bigjoin_{j = 0}^{n - 1}{T^{-j}{\A}} \join \bigjoin_{j = 0}^{n - 1}{T^{-j}{\C}}\right) \\
						&\leq H_\mu\left(\bigjoin_{j = 0}^{n - 1}{T^{-j}{\A}}\right) + H_\mu\left(\bigjoin_{j = 0}^{n - 1}{T^{-j}{\C}}\right)
				\end{align*}
				by \thref{thm:walters-4.3} \ref{walters-thm-4.3:8}.
			\item Since $T$ preserves set theoretic operations, if $\A \subset \C$, then for all $n \geq 1$ we have
				\[
					\bigjoin_{j = 0}^{n - 1}{T^{-j}{\A}} \subset \bigjoin_{j = 0}^{n - 1}{T^{-j}{\C}}.
				\]
				Applying \thref{thm:walters-4.3} \ref{walters-thm-4.3:4} we get
				\[
					H_\mu\left(\bigjoin_{j = 0}^{n - 1}{T^{-j}{\A}}\right) \leq H_\mu\left(\bigjoin_{j = 0}^{n - 1}{T^{-j}{\C}}\right).
				\]
			\item We have
				\begin{align*}
					H_\mu\left(\bigjoin_{j = 0}^{n - 1}{T^{-j}{\A}}\right) &\leq H_\mu\left(\bigjoin_{j = 0}^{n - 1}{T^{-j}{\A}} \join \bigjoin_{j = 0}^{n - 1}{T^{-j}{\C}}\right) \\ & \hspace{60mm} \text{(by \thref{thm:walters-4.3} \ref{walters-thm-4.3:4})} \\
						&= H_\mu\left(\bigjoin_{j = 0}^{n - 1}{T^{-j}{\C}}\right) + H_\mu\left(\bigjoin_{j = 0}^{n - 1}{T^{-j}{\A}} \midmid \bigjoin_{j = 0}^{n - 1}{T^{-j}{\C}}\right) \\ & \hspace{60mm} \text{(by \thref{thm:walters-4.3} \ref{walters-thm-4.3:2}).}
				\end{align*}
				By applying \thref{thm:walters-4.3} \ref{walters-thm-4.3:7} repeatedly, we have
				\begin{align*}
					H_\mu\left(\bigjoin_{j = 0}^{n - 1}{T^{-j}{\A}} \midmid \bigjoin_{j = 0}^{n - 1}{T^{-j}{\C}}\right) &\leq \sum_{j = 0}^{n - 1}{H_\mu\left(T^{-j}{\A} \midmid \bigjoin_{k = 0}^{n - 1}{T^{-k}{\C}}\right)} \\
						&\leq \sum_{j = 0}^{n - 1}{H_\mu\left(T^{-j}{\A} \midmid T^{-j}{\C}\right)} & \text{(by \thref{thm:walters-4.3} \ref{walters-thm-4.3:5})} \\
						&= nH_\mu(\A \mid \C) & \text{(by \thref{thm:walters-4.3} \ref{walters-thm-4.3:9}).}
				\end{align*}
				Combining this with the previous result, we have
				\begin{align*}
					h_\mu(T, \A) &\leq \lim_{n \to +\infty}{\left[\frac{1}{n} H_\mu\left(\bigjoin_{j = 0}^{n - 1}{T^{-j}{\C}}\right) + nH_\mu(\A \mid \C)\right]} \\
						&= h_\mu(T, \C) + H_\mu(\A \mid \C).
				\end{align*}
			\item By \thref{thm:walters-4.3} \ref{walters-thm-4.3:10} we have
				\begin{align*}
					h_\mu(T, T^{-1}{\A}) &= \lim_{n \to +\infty}{\frac{1}{n} H_\mu\left(\bigjoin_{j = 1}^{n - 1}{T^{-j}{\A}}\right)} \\
						&= \lim_{n \to +\infty}{\frac{1}{n} H_\mu\left(\bigjoin_{j = 0}^{n - 1}{T^{-j}{\A}}\right)} \\
						&= h_\mu(T, \A)
				\end{align*}
			\item Let $m \geq 1$ be fixed. Then
				\begin{align*}
					h_\mu\left(T, \bigjoin_{j = 0}^{m}{T^{-j}{\A}}\right) &= \lim_{n \to +\infty}{\frac{1}{n} H_\mu\left(\bigjoin_{j = 0}^{n - 1}{T^{-j}\left(\bigjoin_{k = 0}^{m}{T^{-k}{\A}}\right)}\right)} \\
						&= \lim_{n \to +\infty}{\frac{1}{n} H_\mu\left(\bigjoin_{j = 0}^{n + m - 1}{T^{-j}{\A}}\right)} \\
						&= \lim_{n \to +\infty}{\left(\frac{m + n}{n}\right) \frac{1}{m + n} H_\mu\left(\bigjoin_{j = 0}^{n + m - 1}{T^{-j}{\A}}\right)} \\
						&= h_\mu(T, \A)
				\end{align*}
			\item Suppose that $T$ is invertible and fix $m \geq 1$. We have
				\begin{align*}
					h_\mu\left(T, \bigjoin_{j = -m}^{m}{T^{-j}{\A}}\right) &= h_\mu\left(T, \bigjoin_{j = 0}^{2m}{T^{-j}{\A}}\right) & \text{(by \ref{walters:thm-4-12:5})} \\
						&= h_\mu(T, \A) & \text{(by \ref{walters:thm-4-12:6}).}
				\end{align*}
		\end{enumerate}
	\end{proof}
\end{theorem}

There is an alternative definition for $h_\mu(T, \alpha)$ which is useful if we want to utilise results relating to conditional entropy.

\begin{theorem}
	Suppose that $(X, \B, \mu, T)$ is a measure-preserving transformation of a probability space and that $\alpha$ be a finite partition of $X$. The entropy of $T$ with respect to $\alpha$ (or $\A(\alpha)$) may also be given by
	\[
		h_\mu(T, \A(\alpha)) = h_\mu(T, \alpha) = H_\mu\left(\alpha \midmid \bigjoin_{j = 1}^\infty {T^{-j}{\alpha}}\right).
	\]
	
	\begin{proof}
		We follow the proof given in \cite[Lecture 24]{ergodic-lectures}.
		
		Let $\A := \A(\alpha)$. We have
		\begin{align*}
			H_\mu\left(\bigjoin_{j = 0}^{n - 1}{T^{-j}{\A}}\right) &= H_\mu\left(\bigjoin_{j = 1}^{n - 1}{T^{-j}{\A}}\right) + H_\mu\left(\A \midmid \bigjoin_{j = 1}^{n - 1}{T^{-j}{\A}}\right) & \text{(by \thref{thm:walters-4.3} \ref{walters-thm-4.3:2})} \\
				&= H_\mu\left(\bigjoin_{j = 0}^{n - 2}{T^{-j}{\A}}\right) + H_\mu\left(\A \midmid \bigjoin_{j = 1}^{n - 1}{T^{-j}{\A}}\right) & \text{(by \thref{thm:walters-4.3} \ref{walters-thm-4.3:10})}.
		\end{align*}
		By induction, this means that
		\begin{align*}
			\frac{1}{n} H_\mu\left(\bigjoin_{j = 0}^{n - 1}{T^{-j}{\A}}\right) &= \frac{1}{n}\left[H_\mu\left(\A \midmid \bigjoin_{j = 1}^{n - 2}{T^{-j}{\A}}\right) + H_\mu\left(\A \midmid \bigjoin_{j = 1}^{n - 3}{T^{-j}{\A}}\right)\right. \\
				& \left. \vphantom{\bigjoin_{j = 1}^{n - 2}{T^{-j}{\A}}} \qquad + \dots + H_\mu(\A \mid T^{-1}{\A}) + H_\mu(\A)\right].
		\end{align*}
		By \thref{thm:walters-4.3} \ref{walters-thm-4.3:5} we have
		\[
			H_\mu\left(\A \midmid \bigjoin_{j = 1}^{n - 1}{T^{-j}{\A}}\right) \leq H_\mu\left(\A \midmid \bigjoin_{j = 1}^{n - 2}{T^{-j}{\A}}\right) \leq \dots \leq H_\mu(\A).
		\]
		We may now apply \thref{thm:increasing-martingale} to get that
		\[
			H_\mu\left(\A \midmid \bigjoin_{j = 1}^{n - 1}{T^{-j}{\A}}\right) \to H_\mu\left(\A \midmid \bigjoin_{j = 1}^\infty{T^{-j}{\A}}\right),
		\]
		as $n \to +\infty$, and therefore
		\begin{align*}
			h_\mu(T, \A) = \lim_{n \to +\infty}{\frac{1}{n} H_\mu\left(\bigjoin_{j = 0}^{n - 1}{}T^{-j}{\A}\right)} &= H_\mu\left(\A \midmid \bigjoin_{j = 1}^\infty{T^{-j}{\A}}\right).
		\end{align*}
	\end{proof}
\end{theorem}

Finally, we can think of the entropy of a measure-preserving transformation $T$, regardless of the choice of partition, to be maximal average information we can gain per day. We state this formally as follows.

\begin{definition}
	Let $(X, \B, \mu, T)$ be a measure-preserving transformation of a probability space. The \key{entropy of $T$} is defined
	\[
		h_\mu(T) := \sup_{\alpha}{h_\mu(T, \alpha)} = \sup_{\A}{h_\mu(T, \A)},
	\]
	where the supremum is taken over all finite measurable partitions $\alpha$ or finite sub-$\sigma$-algebras of $\B$, respectively.
\end{definition}

\begin{remark}
	If $T = \id_X$, then $h(T) = 0$. In general, if $h(T) = 0$, then $h(T, \alpha) = 0$ for all finite partitions $\alpha$. This means that we don't obtain `much' new information each day, i.e. $\bigvee_{j = 0}^{n - 1}{T^{-j}{\alpha}}$ doesn't change `much', as $n \to +\infty$. One such measure-preserving transformation which has this property is the identity transformation on $X$.
\end{remark}

\begin{theorem}
	Entropy is a conjugacy invariant and hence is also an isomorphism invariant.
	\begin{proof}
		Let $(X_1, \B_1, \mu_1, T_1), (X_2, \B_2, \mu_2, T_2)$ be measure-preserving transformations of probability spaces. Let $\Phi : (\tilde{B}_2, \tilde{\mu}_2) \to (\tilde{B}_1, \tilde{\mu}_1)$ be an isomorphism of measure algebras such that $\Phi \circ \tilde{T}_2^{-1} = \tilde{T}_1^{-1} \circ \Phi$. We aim to show that $h_{\mu_1}(T_1) = h_{\mu_2}(T_2)$.
		
		Let $\A_2$ be an arbitrary finite sub-$\sigma$-algebra of $\B_2$ and write $\alpha(\A_2) = \{A_1, \dots, A_r\}$. Since $\Phi$ is an isomorphism of measure algebras, we can choose $C_j \in \B_1$ such that $\tilde{C}_j = \Phi(\tilde{A}_j)$. Using this, we define $\gamma := \{C_1, \dots, C_r\}$, which we see is a partition of $(X_1, \B_1, \mu_1)$. We write $\A_1 := \A(\gamma)$.
		
		For any $(q_0, q_1, \dots, q_{n - 1})$, where $q_j \in \{1, \dots, r\}$ for each $j$, we have
		\begin{align*}
			\Phi\left(\bigcap_{j = 0}^{n - 1}{(T_2^{-j} A_{q_j})^\sim}\right) &= \Phi\left(\bigcap_{j = 0}^{n - 1}{\tilde{T}_2^{-j} \tilde{A}_{q_j}}\right) \\
				&= \bigcap_{j = 0}^{n - 1}{\tilde{T}_1^{-j} \Phi(\tilde{A}_{q_j})} \\
				&= \bigcap_{j = 0}^{n - 1}{\tilde{T}_1^{-j} \tilde{C}_{q_j}} \\
				&= \bigcap_{j = 0}^{n - 1}{(T_1^{-j} C_{q_j})^\sim}.
		\end{align*}
		Hence the sets $\bigcap_{j = 0}^{n - 1}{(T_2^{-j} A_{q_j})^\sim}$ and $\bigcap_{j = 0}^{n - 1}{(T_1^{-j} C_{q_j})^\sim}$ have the same measure. Recall that the entropy of a partition is completely determined by the measure of the elements in the partition. This means that
		\[
			H_{\mu_1}\left(\bigjoin_{j = 0}^{n - 1}{T_1^{-j}{\A_1}}\right) = H_{\mu_2}\left(\bigjoin_{j = 0}^{n - 1}{T_2^{-j}{\A_2}}\right),
		\]
		and hence $h_{\mu_1}(T_1, \A_1) = h_{\mu_2}(T_2, \A_2)$. Since $\A_2$ was chosen to be an arbitrary sub-$\sigma$-algebra of $\B_1$, this means that $h_{\mu_1}(T_1) \geq h_{\mu_2}(T_2)$.
		
		We repeat the proof, but choose an arbitrary finite sub-$\sigma$-algebra of $\B_1$ to get the reverse inequality $h_{\mu_1}(T_1) \leq h_{\mu_2}(T_2)$, and hence $h_{\mu_1}(T_1) = h_{\mu_2}(T_2)$.
	\end{proof}
\end{theorem}

\section{Calculating \texorpdfstring{$h_\mu(T)$}{h(T)}}
Recall that the entropy of a measure-preserving transformation $T$ is defined $h_\mu(T) := \sup_{\A}{h_\mu(T, \A)}$, where the supremum is taken over all finite sub-$\sigma$-algebras of $\B$ (or, equivalently, over all finite partitions of $(X, \B, \mu)$). Of course, it is difficult to consider all finite partitions, so we want to find criteria which guarantee that $h_\mu(T) = h_\mu(T, \A)$ instead.

One key result is the \key{Kolmogorov-Sinai Theorem}. We will prove this in \thref{thm:kolmogorov-sinai}, but to do this we need some preliminary results.

\begin{lemma} \label{lem:walters-4-15}
	Let $r \in \naturals$ be fixed and let $\varepsilon > 0$ be given. Then there exists $\delta > 0$ such that, if we have two partitions of $r$ sets $\alpha = \{A_1, \dots, A_r\}$, $\gamma = \{C_1, \dots, C_r\}$ of $(X, \B, \mu)$ such that
	\[
		\sum_{j = 1}^r{\mu(A_j \symdiff C_j)} < \delta,
	\]
	then we have $H_\mu(\alpha \mid \gamma) + H_\mu(\gamma \mid \alpha) < \varepsilon$.
	
	\begin{proof}
		Let $\varepsilon > 0$ be given. We choose $\delta > 0$ such that $\delta < 1 / 4$ and
		\[
			-r(r - 1) \delta \log{\delta} - (1 - \delta) \log(1 - \delta) < \frac{\varepsilon}{2}.
		\]
		
		Let $\beta$ be the partition of $(X, \B, \mu)$ consisting of the sets of the form $A_j \cap C_k$, where $j \neq k$, and the set $\bigcup_{j = 1}^r{A_j \cap C_j}$. It is then clear that $\alpha \join \gamma = \gamma \join \beta$. For $j \neq k$, we also have
		\[
			A_j \cap C_k \subset \bigcup_{n = 1}^r{A_n \symdiff C_n}.
		\]
		This gives, by the hypothesis, $\mu(A_j \cap C_k) < \delta$ for $j \neq k$. By the definition of symmetric difference, we also have
		\[
			\mu\left(\bigcup_{j = 1}^r{A_j \cap C_j}\right) > 1 - \delta.
		\]
		Hence
		\begin{align*}
			H_\mu(\beta) &= -\sum_{j \neq k}{\mu(A_j \cap C_k) \log{\mu(A_j \cap C_k)}} - \mu\left(\bigcup_{j = 1}^r{A_j \cap C_j}\right) \log{\mu\left(\bigcup_{j = 1}^r{A_j \cap C_j}\right)} \\
				&< -r(r - 1) \delta \log{\delta} - (1 - \delta) \log(1 - \delta) \\
				&< \frac{\varepsilon}{2}.
		\end{align*}
		We therefore have
		\begin{align*}
			H_\mu(\gamma) + H_\mu(\alpha \mid \gamma) &= H_\mu(\alpha \join \gamma) & \text{(by \thref{thm:walters-4.3} \ref{walters-thm-4.3:2})} \\
				&= H_\mu(\gamma \join \beta) \\
				&\leq H_\mu(\gamma) + H_\mu(\beta) & \text{(by \thref{thm:walters-4.3} \ref{walters-thm-4.3:8})} \\
				&< H_\mu(\gamma) + \frac{\varepsilon}{2},
		\end{align*}
		and hence $H_\mu(\alpha \mid \gamma) < \varepsilon / 2$.
		
		We repeat this argument using $\alpha \join \gamma = \alpha \join \beta$ to get $H_\mu(\gamma \mid \alpha) < \varepsilon / 2$. Combining these two results, we get $H_\mu(\alpha \mid \gamma) + H_\mu(\gamma \mid \alpha) < \varepsilon$.
	\end{proof}
\end{lemma}

\begin{theorem} \label{thm:walters-4-16}
	Suppose that $\C$ is a finite sub-$\sigma$-algebra of $\B$ and that $\B_0$ is an algebra such that $\B(\B_0) = \B$ $\mu$-almost everywhere. Then given any $\varepsilon > 0$, there exists a finite algebra $\D \subset \B_0$ such that
	\[
		H_\mu(\D \mid \C) + H_\mu(\C \mid \D) < \varepsilon.
	\]
	
	\begin{proof}
		Let $\varepsilon > 0$ be given and write $\alpha(\C) = \{C_1, \dots, C_r\}$. We choose $\delta > 0$ as in \thref{lem:walters-4-15}, where $r, \varepsilon$ here are as in the lemma. It suffices to show that, for each $\tau > 0$, there exists a partition $\D = \{D_1, \dots, D_r\}$, where $D_j \in \B_0$ and $\mu(C_j \symdiff D_j) < \tau$, for all $j = 1, \dots, r$. This is because we may then choose $\tau$ such that $r\tau \leq \delta$ and then apply \thref{lem:walters-4-15}.
		
		To begin, we choose $\lambda > 0$ such that $\lambda(r - 1)[1 + r(r - 1)] < \tau$. For each $j = 1, \dots, r$, choose $B_j \in \B_0$ such that $\mu(C_j \symdiff B_j) < \lambda$. Now if $j \neq k$, then $B_j \cap B_k \subset (B_j \symdiff C_j) \cup (\B_j \symdiff C_j)$. It follows that $\mu(B_j \cap B_k) < 2\lambda$. We let $N := \bigcup_{j \neq k}{(B_j \cap B_k)}$, so that $\mu(N) < r(r - 1)\lambda$.
		
		Now for $j = 1, \dots, r - 1$ we define $D_j = B_j \setminus N$, and $D_r = X \setminus \bigcup_{j = 1}^{r - 1}{D_j}$. This clearly defines a partition $\D := \{D_1, \dots, D_r\}$ of $X$, and each $D_j \in \B_0$, since $\B_0$ is an algebra (i.e. is closed under finite unions and complementation).
		
		If $j < r$, then $D_j \symdiff C_j \subset (B_j \symdiff C_j) \cup N$. Then by countable subadditivity,
		\begin{align*}
			\mu(D_j \symdiff C_j) &\leq \mu(B_j \symdiff C_j) + \mu(N) \\
				&< \lambda + r(r - 1)\lambda \\
				&= \lambda[1 + r(r - 1)] \\
				&< \tau.
		\end{align*}
		For the last part of $\D$, we use the fact that $D_r \symdiff C_r \subset \bigcup_{j = 1}^{r - 1}{(D_j \symdiff C_j)}$. Then $\mu(D_r \symdiff C_r) < (r - 1)\lambda[1 + r(r - 1)] < \tau$.
	\end{proof}
\end{theorem}

\begin{corollary} \label{cor:walters-4-16-1}
	Let $\A_1 \subset \A_2 \subset \dots \subset \A_n \subset \dots$ be an increasing sequence of finite sub-algebras of $\B$. Suppose that $\C$ is a finite sub-algebra of $\B$ such that $\C \subset \bigjoin_{n \geq 1}{\A_n}$ $\mu$-almost everywhere. Then $H_\mu(\C \mid \A_n) \to 0$, as $n \to +\infty$.
	
	Note: We say that $\A \subset \C$ $\mu$-almost everywhere if, for all $A \in \A$, there exists $C \in \C$ such that $\mu(A \symdiff C) = 0$.
	
	\begin{proof}
		Let $\varepsilon > 0$ be given. Write $\B_0 := \bigcup_{n = 1}^\infty{\A_n}$ so that $\B_0$ is an algebra. Since $\C \subset \bigjoin_{n \geq 1}{\A_n}$ $\mu$-almost everywhere, we have $\C \subset \B(\B_0)$ $\mu$-almost everywhere. By \thref{thm:walters-4-16}, there exists a finite sub-algebra $\D_\varepsilon$ of $\B_0$ such that $H_\mu(\C \mid \D_\varepsilon) < \varepsilon$.
		
		Since $\A_n$ is increasing and $\D_\varepsilon$ is finite, we have $\D_\varepsilon \subset \A_{n_0}$ for some $n_0 \in \naturals$. Then for all $n \geq n_0$, we have
		\[
			H_\mu(\C \mid \A_n) \leq H_\mu(\C \mid \A_{n_0}) \leq H_\mu(\C \mid \D_\varepsilon) < \varepsilon.
		\]
		Hence $H_\mu(\C \mid \A_n) \to 0$, as $n \to +\infty$.
	\end{proof}
\end{corollary}

We are ready to prove one of the main results of this chapter.

\begin{theorem}[Kolmogorov-Sinai Theorem] \label{thm:kolmogorov-sinai}
	Let $T : X \to X$ be an invertible measure-preserving transformation of a probability space $(X, \B, \mu)$. Suppose that $\A$ is a finite sub-$\sigma$-algebra of $\B$ such that
	\[
		\bigjoin_{n = -\infty}^\infty{T^{-n}{\A}} = \B
	\]
	$\mu$-almost everywhere. Then $h_\mu(T) = h_\mu(T, \A)$.
	
	\begin{proof}
		Let $\C$ be a finite sub-$\sigma$-algebra of $\B$. We want to show that $h_\mu(T, \C) \leq h_\mu(T, \A)$, i.e. $h_\mu(T, \A)$ achieves the supremum as in the definition of $h_\mu(T)$. We have
		\begin{align*}
			h_\mu(T, \C) &\leq h_\mu\left(T, \bigjoin_{j = -n}^n{T^{-j}{\A}}\right) + H_\mu\left(\C \midmid \bigjoin_{j = -n}^n{T^{-j}{\A}}\right) & \text{(by \thref{thm:walters-4.12} \ref{walters:thm-4-12:4})} \\
				&= h_\mu(T, \A) + H_\mu\left(\C \midmid \bigjoin_{j = -n}^n{T^{-j}{\A}} \right) & \text{(by \thref{thm:walters-4.12} \ref{walters:thm-4-12:7})}
		\end{align*}
		We let $\A_n := \bigjoin_{j = -n}^n{T^{-j}{\A}}$ so that $\A_n$ is an increasing sequence of finite sub-algebras of $\B$. Since $\bigjoin_{n = -\infty}^\infty{T^n{\A}} = \B$ $\mu$-almost everywhere, and $\C \subset \B$, we may apply \thref{cor:walters-4-16-1}. So $H_\mu(\C \mid \A_n) \to 0$, as $n \to +\infty$, and hence $h_\mu(T, \C) \leq h_\mu(T, \A)$.
	\end{proof}
\end{theorem}

There is a similar result which does not require that $T$ is invertible.

\begin{theorem} \label{thm:walters-4-18}
	Let $T : X \to X$ be a (not necessarily invertible) measure-preserving transformation of a probability space $(X, \B, \mu)$. Suppose that $\A$ is a finite sub-algebra of $\B$ such that
	\[
		\bigjoin_{n = 0}^\infty{T^{-n}{\A}} = \B
	\]
	$\mu$-almost everywhere. Then $h_\mu(T) = h_\mu(T, \A)$.
	
	\begin{proof}
		We repeat the same proof for \thref{thm:kolmogorov-sinai}, but replace $\bigjoin_{j = -n}^n{T^{-j}{\A}}$ with $\bigjoin_{j = 0}^n{T^{-j}{\A}}$ and apply \thref{thm:walters-4.12} \ref{walters:thm-4-12:6} instead of \ref{walters:thm-4-12:7}.
	\end{proof}
\end{theorem}

The following result gives a useful criterion for deciding if an invertible measure-preserving transformation has zero entropy.

\begin{corollary}
	Suppose that $T : X \to X$ is an invertible measure-preserving transformation of a probability space $(X, \B, \mu)$, and that
	\[
		\bigjoin_{n = 0}^\infty{T^{-n}{\A}} = \B
	\]
	$\mu$-almost everywhere for some finite sub-algebra $\A$ of $\B$. Then $h_\mu(T) = 0$.
	
	\begin{proof}
		By \thref{thm:walters-4-18}, we have
		\[
			h_\mu(T) = h_\mu(T, \A) = \lim_{n \to +\infty}{H_\mu\left(\A \midmid \bigjoin_{j = 1}^n{T^{-j}{\A}}\right)}.
		\]
		Since $T$ is a measure-preserving transformation, we have $\bigjoin_{j = 1}^\infty{T^{-j}{\A}} = T^{-1}{\B} = \B$ $\mu$-almost everywhere.
		
		We write $\A_n := \bigjoin_{j = 1}^\infty{T^{-j}{\A}}$, so that $\A_n$ is an increasing sequence of sub-algebras of $\B$, and $\bigjoin_{n = 1}^\infty{\A_n} = \B$ $\mu$-almost everywhere. In particular, for all $n \geq 1$ we have $\A \subset \A_n$ $\mu$-almost everywhere and so we may apply \thref{cor:walters-4-16-1}. So $H_\mu(\A \mid \A_n) \to 0$, as $n \to +\infty$. Hence $h_\mu(T) = 0$.
	\end{proof}
\end{corollary}

\section{Shifts of finite type} \label{sec:entropy:sft}
We now consider entropy for shifts of finite type, which we will be interested in later on. The following definitions appear in \cite{chazottes-maldonado:cbfee}.

Let $A$ be a finite alphabet and let $\Sigma := \{(x_j)_{j = 0}^\infty \mid x_j \in A\}$ denote the full (one-sided) shift. As usual, $\sigma : \Sigma \to \Sigma$ will be the (one-sided) shift map. For the sake of readability, if $(x_j)_{j = 0}^\infty \in \Sigma$, then we will write $x_m^n := (x_j)_{j = m}^n$.

In this setting, our measure-preserving transformation will always be $\sigma$ and so we are more interested in $\sigma$-invariant probability measures than the shift map itself. We will define the entropy of such measures in the remainder of this chapter and we will see that each definition is consistent with our previous discussion.

For the remainder of this section, we let $\alpha_k := \{[a_0^{k - 1}] \mid a_0^{k - 1} \in A^k\}$ denote the partition of $(\Sigma, \B, \nu)$ by cylinders of length $k$.

\begin{definition}
	Let $\nu$ be a $\sigma$-invariant probability measure on $\Sigma$. For shifts of finite type, for $k > 1$ we can define the \key{$k$-block entropy} of $\nu$ by
	\[
		H_k(\nu) := -\sum_{a_0^{k - 1} \in A^k}{\nu[a_0^{k - 1}] \log\nu[a_0^{k - 1}]},
	\]
	where the sum is taken over all sequences of length $k$.
\end{definition}

By definition, the entropy of $\alpha_k$ is
\[
	H_\nu(\alpha_k) = -\sum_{a_0^{k - 1} \in A^k}{\nu[a_0^{k - 1}] \log\nu[a_0^{k - 1}]} = H_k(\nu).
\]
In other words, $k$-block entropy is the entropy of the partition of $\Sigma$ by cylinders of length $k$.

\begin{definition}
	Let $\nu$ be a $\sigma$-invariant probability measure on $\Sigma$. The \key{entropy} of $\nu$ is given by
	\[
		h(\nu) := \lim_{k \to +\infty}{\frac{1}{k} H_k(\nu)}.
	\]
\end{definition}

Let $\A$ be the sub-algebra of $\B$ consisting of unions cylinders of length $1$. Then $\bigjoin_{k = 0}^\infty{\sigma^{-k}{\A}} = \B$ and hence \thref{thm:walters-4-18} applies. So
\[
	h_\nu(\sigma) = h_\nu(\sigma, \A) = \lim_{k \to +\infty}{\frac{1}{k} H_\nu\left(\bigjoin_{j = 0}^{k - 1}{\sigma^{-j}{\A}}\right)} = \lim_{k \to +\infty}{\frac{1}{k} H_k\left(\nu\right)} = h(\nu).
\]
Therefore this definition is consistent with our previous results.

\begin{definition}
	The \key{conditional $k$-block entropy}, where $k \geq 2$, is defined
	\[
		h_k(\nu) := -\sum_{a_0^{k - 1} \in A^k}{\nu[a_0^{k - 1}] \log{\frac{\nu[a_0^{k - 1}]}{\nu[a_0^{k - 2}]}}}.
	\]
\end{definition}

Note that $[a_0^{k - 1}] \subset [a_0^{k - 2}]$ and hence for $C \in \alpha_{k - 1}$,
\[
	[a_0^{k - 1}] \cap C =
	\begin{cases}
		\left[a_0^{k - 1}\right],	& \text{if } C = [a_0^{k - 2}]; \\
		\emptyset,	& \text{otherwise}.
	\end{cases}
\]
Then by the definition of conditional entropy we have
\begin{align*}
	H_\nu(\alpha_k \mid \alpha_{k - 1}) &= -\sum_{a_0^{k - 1} \in A^k}{\nu([a_0^{k - 2}] \cap [a_0^{k - 1}]) \log{\frac{[a_0^{k - 2}] \cap [a_0^{k - 1}]}{[a_0^{k - 2}]}}} \\
		&= -\sum_{a_0^{k - 1} \in A^k}{\nu[a_0^{k - 1}] \log{\frac{[a_0^{k - 1}]}{[a_0^{k - 2}]}}} \\
		&= h_k(\nu),
\end{align*}
and so the definitions are once again consistent. For $k \geq 2$, we also clearly have the relation
\[
	h_k(\nu) = H_k(\nu) - H_{k - 1}(\nu).
\]

Finally, we have \key{relative $k$-block entropy} which we will use later.

\begin{definition}
	For $k \geq 1$, the \key{$k$-block relative entropy} of a measure $\nu$ with respect to a measure $\mu$ is defined
	\[
		H_k(\nu \mid \mu) = \sum_{a_0^{k - 1} \in A^k}{\nu[a_0^{k - 1}] \log{\frac{\nu[a_0^{k - 1}]}{\mu[a_0^{k - 1}]}}}.
	\]
\end{definition}

\chapter{Gibbs measures} \label{chap:gibbs}
\section{Overview}
This chapter introduces Gibbs measures, a class of probability measures on shifts of finite type. Gibbs measures have properties which we make use of when we discuss \cite{chazottes-maldonado:cbfee} in Chapter \ref{chap:concentration-bounds}.

Throughout this chapter we will let $\Sigma = \Sigma_A^+$, where $A$ is an irreducible $k \times k$ matrix.

\section{The Ruelle operator}
\emph{This section predominantly follows material from \cite[Chapter 2]{parry-pollicott:zeta-fns-periodic-orbits}.}

We briefly introduce a theorem due to Ruelle. Later, it will be apparent that this theorem is useful for constructing Gibbs measures.

\begin{definition}
	Let $f \in F_\theta^+$. The \key{Ruelle operator} (or \key{transfer operator}) $L_f : F_\theta^+ \to F_\theta^+$ (or, more generally, $L_f : C(\Sigma, \complex) \to C(\Sigma, \complex)$) is defined
	\[
	(L_f{w})(x) = \sum_{y \in \Sigma \midcolon \sigma{y} = x}{e^{f(y)} w(y)} = \sum_{j \midcolon A_{j, x_0} = 1}{e^{f(j, x_0, x_1, \dots)} w(j, x_0, x_1, \dots)},
	\]
	where $x = (x_j)_{j = 0}^\infty \in \Sigma$. This is a bounded linear operator.
	
	The $n$-th iterate of $L_f$ is given by
	\[
	(L_f^n{w})(x) = \sum_{y \in \Sigma \midcolon \sigma^n{y} = x}{e^{f^n(y)} w(y)}.
	\]
	
	If $f$ is also real-valued and $L_f{1} = 1$, then we say that $f$ or $L_f$ is \key{normalised}.
\end{definition}

%\begin{proposition}
%	Let $f \in F_\theta^+$ with $f = u + iv$, where $u, v \in F_\theta^+$ are real-valued functions. If $L_u$ is normalised, i.e. $L_u{1} = 1$, then for all $n \geq 0$,
%	\[
%	|L_f^n{w}|_\theta \leq K|w|_\infty + \theta^n |w|_\theta
%	\]
%	for all $w \in F_\theta^+$, where $K > 0$ is a constant depending only on $f$ and $\theta$.
%\end{proposition}

\begin{theorem}[Ruelle's Perron-Frobenius Theorem] \label{thm:rpf}
	Suppose that $\Sigma = \Sigma_A^+$ is an aperiodic shift of finite type and let $f \in F_\theta^+$ be a real-valued function. Then
	\begin{enumerate}
		\item There is a simple maximal eigenvalue $\lambda$ of $L_f : C(\Sigma, \reals) \to C(\Sigma, \reals)$ with a corresponding eigenfunction $h \in C(\Sigma_A^+, \reals)$, with $h > 0$. \label{rpf:1}
		\item The remainder of the spectrum of $L_f$ is contained in a disc of radius strictly less than $\lambda$. \label{rpf:2}
		\item There is a unique probability measure $\mu$ such that $L_f^*{\mu} = \lambda\mu$. That is,
		\[
		\int{L_f{w}\ d\mu} = \lambda \int{w\ d\mu},
		\]
		for all $w \in C(\Sigma, \reals)$. Additionally, if $h$ is the eigenfunction as in \ref{rpf:1} and $\int{h\ d\mu} = 1$, then the measure $\nu$ defined by $d\nu = h\ d\mu$ is a $\sigma$-invariant probability measure. \label{rpf:3}
		\item If $h$ is the eigenfunction as in \ref{rpf:1} and $\int{h\ d\mu} = 1$, then for all $w \in C(\Sigma, \reals)$,
		\[
		\frac{1}{\lambda^n}L_f^n{w} \to h \int{w\ d\mu}
		\]
		uniformly, as $n \to +\infty$. \label{rpf:4}
	\end{enumerate}
\end{theorem}

\section{Gibbs measures and basic results}
\emph{The remainder of this chapter predominantly follows \cite[Chapter 3]{parry-pollicott:zeta-fns-periodic-orbits}.}

\begin{definition}
	Let $\phi \in C(\Sigma, \reals)$ be a continuous function. Let $\mu$ be a probability measure such that there exists a constants $C = C(\phi) > 1$, $P = P(\phi) \in \reals$, such that
	\begin{equation}
		C^{-1} \leq \frac{\mu[x_0, \dots, x_n]}{\exp\left(-Pn + \sum_{j = 0}^{n - 1}{\phi(\sigma^j{x})} \right)} \leq C,
	\end{equation}
	for all $n \geq 0$ and for all $x \in \Sigma$. Then the measure $\mu$ is a \key{Gibbs measure for $\phi$} or a \key{Gibbs measure with potential $\phi$}. We will write $\mu_\phi = \mu$.
\end{definition}

\begin{remark}
	The Gibbs measure $\mu_\phi$ is not necessarily $\sigma$-invariant.
\end{remark}

Given any $\phi \in F_\theta$, it can be shown that there exists a Gibbs measure $\mu_\phi$ for $\phi$. We prove this in the following results.

\begin{proposition} \label{prop:pp-3-2}
	Suppose that $\phi \in F_\theta$ is a normalised, real-valued function. Then for all $x \in \Sigma$ we have
	\begin{equation}
		e^{-|\phi|_\theta \theta^n} \leq \frac{\mu[x_0, \dots, x_n]}{\mu[x_1, \dots, x_n]} e^{-\phi(x)} \leq e^{|\phi|_\theta \theta^n},
	\end{equation}
	where $L_\phi^*\mu = \mu$ is the unique measure in Ruelle's Perron-Frobenius Theorem \ref{rpf:3} (\thref{thm:rpf}) with $\lambda = 1$.
	\begin{proof}
		Let $w \in C(\Sigma, \reals)$. We have
		\begin{align*}
			\int{w \circ \sigma\ d\mu} &= \int L_\phi(w \circ \sigma)\ d\mu \\
				&= \int{\sum_{y \in \Sigma \midcolon \sigma{y} = x}{e^{\phi(y)} w \circ \sigma(y)}\ d\mu} \\
				&= \int{\sum_{y \in \Sigma \midcolon \sigma{y} = x}{e^{\phi(y)} w(x)}\ d\mu} \\
				&= \int{w L_\phi{1}\ d\mu} \\
				&= \int{w\ d\mu}.
		\end{align*}
		By a basic result~\cite[Lemma 11.3]{ergodic-lectures} in ergodic theory, it follows that $\mu$ is a $\sigma$-invariant measure. Hence
		\begin{align}
			\mu[x_1, \dots, x_n] &= \mu\left(\bigsqcup_{x_0 \midcolon A_{x_0, x_1} = 1}{[x_0, x_1, \dots, x_n]}\right) \nonumber \\
				&= \int{\sum_{x_0 \midcolon A_{x_0, x_1} = 1} \chi_{[x_0, x_1, \dots, x_n]}(z)\ d\mu} \nonumber \\
				&= \int{\sum_{y \midcolon \sigma{y} = z} e^{\phi(y)} \chi_{[x_0, \dots, x_n]}(y) e^{-\phi(y)}\ d\mu} & (\text{since } \chi_B(y) = 0 \text{ or } 1) \nonumber \\
				&= \int{L_\phi\chi_{[x_0, x_1, \dots, x_n]}(z) e^{-\phi(z)}\ d\mu} \nonumber \\
				&= \int{\left(\chi_{[x_0, \dots, x_n]} e^{-\phi}\right)(z)\ d\mu} \nonumber \\
				&= \int_{[x_0, \dots, x_n]}{e^{-\phi}\ d\mu}. \label{fml:prop-3-2-mu-e}
		\end{align}
		Now let $z, w \in [x_0, \dots, x_n]$. Since $\phi \in F_\theta \subset C(\Sigma, \reals)$, we have that $\var_n(\phi) \leq |\phi|_\theta \theta^n$ and hence
		\[
			e^{-|\phi|_\theta \theta^n} \leq e^{\phi(z) - \phi(w)} \leq e^{|\phi|_\theta \theta^n}.
		\]
		Then by \eqref{fml:prop-3-2-mu-e} we have
		\[
			\mu[x_0, \dots, x_n] e^{-|\phi|_\theta \theta^n} \leq \mu[x_1, \dots, x_n] e^{\phi(x)} \leq \mu[x_0, \dots, x_n] e^{|\phi|_\theta \theta^n},
		\]
		and so
		\[
			e^{-|\phi|_\theta \theta^n} \leq \frac{\mu[x_0, \dots, x_n]}{\mu[x_1, \dots, x_n]} e^{-\phi(x)} \leq e^{|\phi|_\theta \theta^n}.
		\]
	\end{proof}
\end{proposition}

\begin{corollary} \label{cor:pp-3-2-1}
	The measure $\mu$ in \thref{prop:pp-3-2} is a Gibbs measure for $\phi$ with $P = 0$.
	\begin{proof}
		We apply \thref{prop:pp-3-2} to $\phi$, $\phi \circ \sigma$, \dots, $\phi \circ \sigma^n$ to get the sequence of inequalities
		\[
			\left.
			\begin{matrix}
				e^{-|\phi|_\theta \theta^n} &\leq& \dfrac{\mu[x_0, \dots, x_n]}{\mu[x_1, \dots, x_n]} e^{-\phi(x)} &\leq& e^{|\phi|_\theta \theta^n} \\
				e^{-|\phi|_\theta \theta^{n - 1}} &\leq& \dfrac{\mu[x_1, \dots, x_n]}{\mu[x_2, \dots, x_n]} e^{-\phi(\sigma{x})} &\leq& e^{|\phi|_\theta \theta^{n - 1}} \\
				\vdots & & \vdots & & \vdots \\
				e^{-|\phi|_\theta} &\leq& \mu[x_n] e^{-\phi(\sigma^n{x})} &\leq& e^{|\phi|_\theta}
			\end{matrix}
			\right\}
		\]
		for all $x \in \Sigma$. Multiplying these together, we get
		\[
			\exp\left(-\sum_{j = 0}^n{|\phi|_\theta \theta^j}\right) \leq \frac{\mu[x_0, \dots, x_n]}{\exp\left(-\sum_{j = 0}^n{\phi(\sigma^j{x})}\right)} \leq \exp\left(\sum_{j = 0}^n{|\phi|_\theta \theta^j}\right),
		\]
		and so
		\[
			\exp\left(-\frac{|\phi|_\theta}{1 - \theta}\right) \leq \frac{\mu[x_0, \dots, x_n]}{\exp\left(-\sum_{j = 0}^n{\phi(\sigma^j{x})}\right)} \leq \exp\left(\frac{|\phi|_\theta}{1 - \theta}\right).
		\]
		Hence $\mu$ is a Gibbs measure for $\phi$ with $P = 0$.
	\end{proof}
\end{corollary}

We can generalise these results for Lipschitz functions $\phi \in F_\theta$ which are not normalised.

\begin{corollary}
	Suppose that $\phi \in F_\theta$ is a real-valued function. Let $\mu$ be the unique measure where $L_\phi^*\mu = \lambda\mu$, where $\lambda > 0$ is the maximal eigenvalue in Ruelle's Perron-Frobenius Theorem \ref{rpf:3}. Then $\mu$ is a Gibbs measure for $\phi$ with $P = \log{\lambda}$.
	\begin{proof}
		First note that $\psi := \phi - \log{(h \circ \sigma)} + \log{h} - \log{\lambda}$ is normalised, where $h > 0$ is the eigenfunction corresponding to $\lambda$ in \thref{thm:rpf}. We have
		\begin{align*}
			\exp\left(\sum_{j = 0}^{n - 1}{\psi(\sigma^j{x})}\right) &= \exp\left(\sum_{j = 0}^{n - 1}{\phi(\sigma^j{x}) - \log{h(\sigma^{j + 1}{x})} + \log{h(\sigma^j{x})} - \log{\lambda}}\right) \\
				&= \exp\left(-n\log{\lambda} + \sum_{j = 0}^{n - 1}{\phi(\sigma^j{x})}\right) \frac{h(x)}{h(\sigma^n{x})}.
		\end{align*}
		We apply \thref{cor:pp-3-2-1} to $\psi = \phi - \log{(h \circ \sigma)} + \log{h} - \log{\lambda}$ to get
		\[
			C_0^{-1} \leq \frac{\mu[x_0, \dots, x_n]}{\exp\left(-n\log{\lambda} + \sum_{j = 0}^{n - 1}{\phi(\sigma^j{x})}\right)} \leq C_0,
		\]
		for some constant $C_0 > 1$. Hence $\mu$ is a Gibbs measure for $\phi$ with $P = \log{\lambda}$.
	\end{proof}
\end{corollary}

In view of this, if $\phi \in F_\theta$ has Gibbs measure $\mu_\phi$ with $P(\phi) > 0$, then the Gibbs measure for $\phi - P(\phi)$ gives $P(\phi - P(\phi)) = 0$. In both cases we have the same Gibbs measure $\mu_\phi$.

\section{The variational principle and pressure}
There is a characteristic which sets Gibbs measures apart from other $\sigma$-invariant probability measures. To prove this property, we need some concepts and results.

Let $\mu$ be a $\sigma$-invariant probability measure on $\Sigma$. For $n \geq 0$ we have that $\mu[x_0, \dots, x_{n - 1}] > 0$ for $\mu$-almost every $x \in \Sigma$. For $j = 1, \dots, k$ we define
\begin{align*}
	\mu_n[j \mid \sigma^{-1}{x}] &:= \frac{\mu[j, x_0, \dots, x_{n - 1}]}{\mu[x_0, \dots, x_{n - 1}]} \\
		&= \frac{\mu([j] \cap \sigma^{-1}[x_0, \dots, x_{n - 1}])}{\mu(\sigma^{-1}[x_0, \dots, x_{n - 1}])} \\
		&= \mu\left([j] \midmid \bigjoin_{r = 0}^{n - 1}{\sigma^{-r}{\alpha}}\right)(x),
\end{align*}
where $\alpha = \{[j] \mid j \in \{1, \dots, k\}\}$ is the partition of $\Sigma$ by cylinders of length 1. It is clear that this is a probability distribution for $\mu$-almost every $x$.

\begin{proposition}\label{prop:mu-sq-bkt-pd}
	For $n \geq 0$, we have that
	\[
		\mu[j \mid \sigma^{-1}{x}] := \lim_{n \to +\infty}{\mu_n[j \mid \sigma^{-1}{x}]}
	\]
	is a well-defined probability distribution on $\{1, \dots, k\}$ for $\mu$-almost every $x \in \Sigma$.
	\begin{proof}
		First note that $\bigjoin_{r = 0}^{n - 1}{\sigma^{-r}{\alpha}} \to \B$, as $n \to +\infty$. By definition we also have $\mu([j] \mid \bigjoin_{r = 0}^{n - 1}{\sigma^{-r}{\alpha}}) = E_\mu(\chi_{[j]} \mid \bigjoin_{r = 0}^{n - 1}{\sigma^{-r}{\alpha}})$. Since $\chi_{[j]} \in L^1(\Sigma, \B, \mu)$, we may apply the Increasing Martingale Theorem so that
		\[
			\lim_{n \to +\infty}{\mu\left([j] \midmid \bigjoin_{r = 0}^{n - 1}{\sigma^{-r}{\alpha}}\right)} = \mu([j] \mid \B),
		\]
		for $\mu$-almost every $x$. Hence
		\[
			\mu[j \mid \sigma^{-1}{x}] = \mu([j] \mid \B)(x)
		\]
		for $\mu$-almost every $x$, i.e. $\mu[j \mid \sigma^{-1}{x}]$ is a well-defined probability distribution on $\{1, \dots, k\}$ for $\mu$-almost every $x$.
	\end{proof}
\end{proposition}

\begin{lemma} \label{lem:pp-prop-p36}
	We have
	\begin{equation}
		\sum_{j = 1}^k{\int{\psi(j, x_0, x_1, \dots) \mu[j \mid \sigma^{-1}{x}] \ d\mu}} = \int{\psi\ d\mu},
	\end{equation}
	for all $\psi \in C(\Sigma, \reals)$.
	\begin{proof}
		Let $\psi = \chi_{[r_0, \dots, r_t]}$, where $t \in \naturals_0$. We have
		\begin{align*}
			\sum_{j = 1}^k{\int{\psi(j, x_0, x_1, \dots) \mu[j \mid \sigma^{-1}{x}] \ d\mu}} &= \lim_{n \to +\infty}{\sum_{j = 1}^k{\int{\chi_{[r_0, \dots, r_t]} \mu_n[j \mid \sigma^{-1}{x}] \ d\mu}}} \\
				&= \mu[r_0, \dots, r_t] \sum_{j = 1}^k{\frac{\mu[j, x_0, \dots, x_{n - 1}]}{\mu[x_0, \dots, x_{n - 1}]}} \\
				&= \int{\psi\ d\mu}.
		\end{align*}
		So the result holds for characteristic functions. We then apply the definitions from measure theory to show that the result holds for $\psi \in C(\Sigma, \reals)$.
	\end{proof}
\end{lemma}

\begin{lemma} \label{lem:pp-prop-3-4}
	Suppose that $\phi \in F_\theta$ is a real-valued function and that $L_\phi$ is normalised. Let $\mu$ be a probability measure such that $L_\phi^*{\mu} = \mu$. Then for any $\sigma$-invariant probability measure $m$, we have
	\[
		h_m(\sigma) + \int{\phi\ dm} \leq 0,
	\]
	with equality if and only if $m = \mu$.
	\begin{proof}
		Let $m \in M(\Sigma, \sigma)$ be a $\sigma$-invariant probability measure. We define a probability distribution on $\{1, \dots, k\}$ by $\mu[j \mid \sigma^{-1}{x}]$ as in \thref{prop:mu-sq-bkt-pd}. If $m = \mu$, then $L_\phi^*{m} = m$ and so we have the probability distribution
		\[
			m[j \mid \sigma^{-1}{x}] = \exp(\phi(j, x_0, x_1, \dots))
		\]
		for all $x \in \Sigma$. We apply \thref{lem:pp-3-3} so that
		\[
			-\sum_{j = 1}^k{m[j \mid \sigma^{-1}{x}] \log{m[j \mid \sigma^{-1}{x}]}} + \sum_{j = 1}^k{m[j \mid \sigma^{-1}{x}] \phi(j, x_0, x_1, \dots)} \leq 0,
		\]
		for $m$-almost every $x$, with equality if and only if $m[j \mid \sigma^{-1}{x}] = \phi(j, x_0, x_1, \dots)$ for all $j = 1, \dots, k$.
		
		We integrate with respect to $m$ and apply \thref{lem:pp-prop-p36} to get
		\[
			h_m(\sigma) + \sum_{j = 1}^k{\int{m[j \mid \sigma^{-1}{x}] \phi(j, x_0, x_1, \dots)\ dm}} = h_m(\sigma) + \int{\phi\ dm} \leq 0,
		\]
		with equality if and only if $m[j \mid \sigma^{-1}{x}] = \phi(j, x_0, x_1, \dots)$ for $m$-almost every $x$. By \thref{lem:pp-prop-p36}, this equality condition is equivalent to
		\begin{align*}
			\int{\sum_{j = 1}^k{m[j \mid \sigma^{-1}{x}] \psi(j, x_0, x_1, \dots)}\ dm} &= \int{\sum_{j = 1}^k{\phi(j, x_0, x_1, \dots) \psi(j, x_0, x_1, \dots)}\ dm} \\
				&= \int{\psi\ dm}
		\end{align*}
		for all $\psi \in C(\Sigma, \reals)$. In other words, $\int{L_\phi{\psi}\ dm} = \int{g\ dm}$ for all $\psi$, and so we have $L_\phi^*{m} = m$. By Ruelle's Perron-Frobenius Theorem \ref{rpf:3}, $\mu$ is the unique $\sigma$-invariant probability measure such that $L_\phi^*{\mu} = \mu$, so we have
		\[
			h_m(\sigma) + \int{\phi\ dm} \leq 0,
		\]
		with equality if and only if $m = \mu$.
	\end{proof}
\end{lemma}

It can be shown that similar results hold for 2-sided shifts of finite type, and also for $\phi \in F_\theta$ where $L_\phi$ is not necessarily normalised.

The following result shows one of the main distinguishing characteristics of Gibbs measures.

\begin{theorem}[Variational Principle] \label{thm:variational-principle}
	Suppose that $\phi \in F_\theta$ (or $F_\theta^+$). The Gibbs measure $\mu_\phi$ is the unique $\sigma$-invariant probability measure such that
	\[
		h_m(\sigma) + \int{\phi\ dm} \leq h_{\mu_\phi}(\sigma) + \int{\phi\ d\mu_\phi}
	\]
	for all $m \in M(\Sigma, \sigma)$, with equality if and only if $m = \mu_\phi$.
	\begin{proof}
		Let $\phi \in F_\theta$. By \thref{prop:pp-1-2}, there exists $g \in F_{\theta^{1 / 2}}^+$, $u \in F_{\theta^{1 / 2}}$ such that $\phi = g + (u \circ \sigma) - u$. By Ruelle's Perron-Frobenius Theorem, we can write $g = \log(h \circ \sigma) - \log{h} + \log{\lambda} + k$, for some $k \in F_{\theta^{1/2}}^+$ such that $L_k^*{\mu_\phi} = \mu_\phi$ and $L_k$ is normalised.
		
		Let $m$ be a $\sigma$-invariant probability measure. By \thref{lem:pp-prop-3-4} we have
		\[
			h_m(\sigma) + \int{k\ dm} \leq h_{\mu_\phi}(\sigma) + \int{k\ d\mu_\phi} = 0.
		\]
		Substituting in $k = \phi - (u \circ \sigma) + u - \log(h \circ \sigma) + \log{h} - \log{\lambda}$ and cancelling terms, we get
		\[
			h_m(\sigma) + \int{\phi\ dm} \leq h_{\mu_\phi}(\sigma) + \int{\phi\ d\mu_\phi},
		\]
		with equality if and only if $m = \mu_\phi$.
	\end{proof}
\end{theorem}

From the proof of \thref{thm:variational-principle}, we see that
\[
	P(\phi) = \sup_{m \in M(\Sigma, \sigma)}\left\{h_m(\sigma) + \int{\phi\ dm}\right\} = h_{\mu_\phi}(\sigma) + \int{\phi\ d\mu_\phi}.
\]
So $P(\phi) = \log{\lambda}$, where $\lambda$ is the maximal positive eigenvalue for $L_{\phi'}$, where $\phi$ is cohomologous to $\phi' \in F_{\theta^{1 / 2}}^+$.

\begin{definition}
	We call
	\[
		P(\phi) := \sup_{m \in M(\Sigma, \sigma)}\left\{h_m(\sigma) + \int{\phi\ dm}\right\}
	\]
	the \key{pressure} of $\phi$.
	
	If a measure a $\sigma$-invariant probability measure $m$ attains this supremum, i.e. $P(\phi) = h_m(\sigma) + \int{\phi\ dm}$, then we say that $m$ is an \key{equilibrium state}.
\end{definition}

The Variational Principle gives that if we have $\phi \in F_\theta$, then the equilibrium state is unique and we can also define the pressure of $\phi$ by $P(\phi) = \log{\lambda}$.

\section{Gibbs measures are weak Bernoulli}
We now describe a particular property of Gibbs measures which we will use in Subsection \ref{ssec:hitting-times}. The following definitions and results follow \cite[Section 1.E]{bowen:equilibrium}.

\begin{definition}
	Let $\beta, \gamma$ be two finite partitions of a measure space $(X, \B, \mu)$ and let $\varepsilon > 0$ be given. We say that $\beta$ and $\gamma$ are \key{$\varepsilon$-independent} if
	\[
		\sum_{B \in \beta,\ C \in \gamma}{|\mu(B \cap C) - \mu(B)\mu(C)|} < \varepsilon.
	\]
\end{definition}

\begin{definition}
	Let $\xi = \left\{[j] \mid j \in \{1, \dots, k\}\right\}$ be the partition of $(\Sigma, \B, \mu)$ by cylinders of length 1. We say that $\xi$ is \key{weak Bernoulli} (for $\sigma$ and $\mu$) if for all $\varepsilon > 0$ there exists $N(\varepsilon) > 0$ such that for all $n \geq 1$, then the partitions
	\[
		\beta = \bigjoin_{j = 0}^n{\sigma^{-j}(\xi)} \quad \text{and} \quad \gamma = \bigjoin_{j = t}^{t + r}{\sigma^{-j}(\xi)}
	\]
	are $\varepsilon$-independent for all $r \geq 0$ and for all $t \geq n + N(\varepsilon)$.
\end{definition}

Before we state the main result, we need an auxiliary lemma.

\begin{lemma}\label{bowen:lem-1-12}
	Let $r \geq 0$, $f \in C(\Sigma, \reals)$ and $\var_r(f) = 0$. Let
	\begin{align*}
		F \in \{g \in C(\Sigma, \reals) \mid g & \geq 0,\ \nu(g) = 1,\ g(x) \leq B_m g(x') \text{ whenever } x_j = x'_j \text{ for all } j = 0, \dots, m\},
	\end{align*}
	where $B_m := \exp\left(\sum_{k = m + 1}^\infty{2b\alpha^k}\right)$, where $b > 0$, $\alpha \in (0, 1)$ are any pair of constants which satisfy $\var_k(\phi) \leq b\alpha^k$ for all $k > 0$.
	
	Then for any $n \geq 0$ we have
	\[
		\|\lambda^{-n - r}L_\phi^{n + r}(fF) - \nu(fF)h\| \leq M\nu(|fF|)\rho^n,
	\]
	where $\nu$, $\lambda$, $h$ are as in the Ruelle's Perron Frobenius Theorem, and $M > 0$, $\rho \in (0, 1)$ are constants.
\end{lemma}

\begin{theorem}\label{thm:gibbs-is-weak-bernoulli}
	Let $\xi = \left\{[j] \mid j \in \{1, \dots, k\}\right\}$ be the partition of $(\Sigma, \B, \mu)$ by cylinders of length 1. Then $\xi$ is weak Bernoulli for the Gibbs measure $\mu_\phi$.
	\begin{proof}
		Let $\varepsilon > 0$ be given. Suppose that $\phi \in C(\Sigma, \reals)$, $n \geq 1$ and $t \geq n + N(\varepsilon)$, for some $N(\varepsilon)$. Let $\beta, \gamma$ be partitions of $\Sigma$ defined by
		\[
		\beta = \bigjoin_{j = 0}^n{\sigma^{-j}(\xi)} \quad \text{and} \quad \gamma = \bigjoin_{j = t}^{t + r}{\sigma^{-j}(\xi)}.
		\]
		For all $B \in \beta$ we clearly have $\chi_B \in C(\Sigma, \reals)$ and $\var_r(\chi_B) = 0$ for all $r \geq 0$.
		
		Now consider $C \in \gamma$. Since $t \geq n$, we know that $B$ consists of cylinders of lengths strictly less than the lengths of cylinders in $C$, and therefore $\sigma^{-t}C$. It follows that the intersection $B \cap C$ depends only on $B$, and hence
		\[
			\mu_\phi(B \cap C) = \mu_\phi(B \cap \sigma^{-t}{C}).
		\]
		We then have, where $\nu$, $h$ and $\lambda$ are as in the Ruelle's Perron Frobenius Theorem,
		\begin{align*}
			\mu_\phi(B \cap C) &= \mu_\phi(B \cap \sigma^{-t}C) \\
				&= \mu_\phi(\chi_B \cdot \chi_{\sigma^{-t}}{C}) \\
				&= \mu_\phi(\chi_B \cdot (\chi_{C} \circ \sigma^t)) \\
				&= \nu(h \chi_B \cdot (\chi_C \circ \sigma^t)) \\
				&= \lambda^{-t}(L_\phi^*)^t\nu(h\chi_B \cdot (\chi_C \circ \sigma^t)) \\
				&= \nu(\lambda^{-t} L_\phi^t(h \chi_B \cdot (\chi_C \circ \sigma^t))) \\
				&= \nu(\lambda^{-t} L_\phi^t(h \chi_B) \cdot \chi_C).
		\end{align*}
		Consequently,
		\begin{align*}
			|\mu_\phi(B \cap C) - \mu_\phi(B)\mu_\phi(C)| &= |\nu(\lambda^{-t} L_\phi^t(h \chi_B) \cdot \chi_C) - \nu(h \chi_B)\nu(h \chi_C)| \\
				&= |\nu((\lambda^{-t} L_\phi^t(h \chi_B) - \nu(h \chi_B)h)\chi_C)| \\
				&\leq \|\lambda^{-t} L_\phi^t(h \chi_B) - \nu(h \chi_B)h\| \nu(C).
		\end{align*}
		Since $\chi_B \in C(\Sigma, \reals)$ and $\var_r(\chi_B) = 0$ for all $r \geq 0$, we may apply \thref{bowen:lem-1-12}. So if $t \geq s$, then
		\[
			\|\lambda^{-t}L_\phi^{t}(h \chi_B) - \nu(h \chi_B)h\| \leq M\nu(h \chi_B)\rho^{t - s},
		\]
		where $M > 0$, $\rho \in (0, 1)$ are constants. We therefore have
		\begin{align*}
			|\mu_\phi(B \cap C) - \mu_\phi(B)\mu_\phi(C)| &\leq M\nu(h \chi_B)\rho^{t - s} \nu(C) \\
				&= M\mu_\phi(B)\rho^{t - s} \nu(C) \\
				&= M'\mu_\phi(B)\mu_\phi(C)\rho^{t - s},
		\end{align*}
		where $M' = M(\inf{h})^{-1}$. Summing over all elements in the partitions $\beta, \gamma$, we get
		\[
			\sum_{B \in \beta,\ C \in \gamma}{|\mu_\phi(B \cap C) - \mu_\phi(B)\mu_\phi(C)|} \leq M'\rho^{t - s} < \varepsilon
		\]
		for sufficiently large $t - s$.
		
		Hence $\mu_\phi$ is weak Bernoulli.
	\end{proof}
\end{theorem}

\chapter[Concentration bounds for entropy estimation]{Concentration bounds for entropy estimation of Gibbs measures}\label{chap:concentration-bounds}
\section{Overview}
We now have the required background knowledge to discuss \cite{chazottes-maldonado:cbfee}, which is concerned with methods for estimating the entropy of Gibbs measures. In particular, we will show that these methods yield estimated entropy values which concentrate around the actual value of entropy.

\section{Restrictions and notation}
\subsection{Full shifts}
Let $A$ be a finite alphabet. Throughout this chapter, $\Sigma = \{(x_j)_{j = 0}^\infty \mid x_j \in A\}$ will denote the full (one-sided) shift with the (one-sided) shift map $\sigma : \Sigma \to \Sigma$. Once again, we will write $x_m^n := (x_j)_{j = m}^n$. We will use the definitions from Section \ref{sec:entropy:sft}.

\subsection{Gibbs measures}
Let $\phi \in F_\theta$. Recall that if there exist constants $C = C(\phi) > 1$, $P = P(\phi)$ such that
\[
	C^{-1} \leq \frac{\mu_\phi[x_0^{n - 1}]}{\exp\left(-Pn + \sum_{j = 0}^{n - 1}{\phi(\sigma^j x)}\right)} \leq C,
\]
then $\mu_\phi$ is a Gibbs measure for $\phi$.

Throughout this chapter, we will assume that the pressure $P$ of $\phi$ is zero. With this in mind, recall that
Gibbs measures satisfy the Variational Principle (\thref{thm:variational-principle}), so we have
\[
	0 = P = h(\mu_\phi) + \int{\phi\ d\mu_\phi}.
\]
This gives the identity
\begin{equation}\label{fml:vp-identity}
	h(\mu_\phi) = -\int{\phi\ d\mu_\phi},
\end{equation}
which will be useful later in this chapter.

\subsection{Probability theory}
We will use and prove some results related to a couple of basic ideas in probability theory.
\begin{definition}
	The \key{expectation} of a continuous function $f : X \to \reals$ with respect to a probability measure $\mu$ is given by
	\[
		E_\mu(f) := \int{f\ d\mu}.
	\]
	The expectation is a weighted average, or mean, of the values $f$ takes.~\cite[p127]{gray:probability}
\end{definition}

\begin{definition}
	The \key{variance} of a function $f : X \to \reals$ with respect to a probability measure $\mu$ is given by
	\[
		\Var_{\mu}(f) := \int{\left(K(x) - \int{K(y)\ d\mu(y)}\right)^2\ d\mu(x)}.
	\]
	The variance gives a measurement of the spread of the values $f$ takes.
\end{definition}

\section{An exponential inequality}
Throughout the remainder of this chapter, we will assume that $\phi \in F_\theta$ and that $\mu_\phi$ is the unique Gibbs measure for $\phi$, which follows since $P(\phi) = 0$.

The results in Section \ref{sec:estimator-bounds} utilise an exponential inequality proved in \cite{collet-martinez-schmitt:exp-ineq}. We first introduce some definitions which are used in the inequality.

\begin{definition}
	Let $K : \Sigma^n \to \reals$ be a function of $n$ variables in $\Sigma$. For $j = 0, \dots, n - 1$, we define
	\[
		\Lip_j(K) := \sup_{\substack{x^{(0)}, x^{(1)}, \dots, x^{(n - 1)} \\ y^{(j)} \neq x^{(j)}}}{\frac{\left|K(x^{(0)}, \dots, x^{(n - 1)}) - K(x^{(0)}, \dots, x^{(j - 1)}, y^{(j)}, x^{(j + 1)}, \dots, x^{(n - 1)})\right|}{d_\theta(x^{(j)}, y^{(j)})}},
	\]
	where each $x^{(k)}, y^{(k)}$ is a sequence in $\Sigma$ for $k = 0, \dots, n - 1$.
	
	In other words, $\Lip_j(K)$ is a measurement of how much $K$ varies when the $j$-th variable is changed. There is an additional weighting which depends on how close the old $j$-th variable is to the new $j$-th variable.
	
	If $\Lip_j(K) < +\infty$ for all $j = 0, \dots, n - 1$, then we say that $K$ is a \key{separately Lipschitz function} of $n$ variables.
\end{definition}

We now present the exponential inequality, which appears in its original form in \cite[Theorem I.1]{collet-martinez-schmitt:exp-ineq}.

\begin{theorem}\label{thm:cm-3-1}
	Let $\mu_\phi$ be a Gibbs measure. Then there exists some constant $D = D(\phi) > 0$ such that
	\begin{equation}\label{fml:cms-exp-ineq}
		\int{e^{K(x, \dots, \sigma^{n - 1}{x})}\ d\mu_\phi(x)} \leq e^{\int{K(y, \dots, \sigma^{n - 1}{y})\ d\mu_\phi(y)}} \; e^{D \sum_{j = 0}^{n - 1}{\Lip_j^2(K)}}.
	\end{equation}
	 for all $n \in \naturals$ and any separately Lipschitz function $K : \Sigma^n \to \reals$.
\end{theorem}

We may write \eqref{fml:cms-exp-ineq} more succinctly by defining a measure $\mu_\phi^{(n)}$ on $\Sigma^n$ by
\[
	d\mu_\phi^{(n)}(x^{(0)}, \dots, x^{(n - 1)}) = d\mu_\phi(x^{(0)}) \prod_{j = 1}^{n - 1}{\delta(x^{(j)} - \sigma^j{x^{(0)}})},
\]
where $\delta(x - a) = \delta_a(x)$ is the Dirac measure at $a$. Using this, \eqref{fml:cms-exp-ineq} becomes
\[
	\int{e^K\ d\mu_\phi^{(n)}} \leq e^{\int{K\ d\mu_\phi^{(n)}}} \; e^{D \sum_{j = 0}^{n - 1}{\Lip_j^2(K)}}.
\]

\subsection{Important results}
We will mostly use the results in this subsection which follow from \thref{thm:cm-3-1}. To begin, we need Markov's Inequality.

\begin{lemma}[Markov's Inequality]\label{lem:markov-ineq}
	Let $f \geq 0$ be a nonnegative measurable function on a measure space $(X, \B, \mu)$. Then for all $t > 0$ we have
	\[
		\mu\{x \in X \mid f(x) \geq t\} \leq \frac{1}{t}\int{f\ d\mu}.
	\]
	
	\begin{proof}
		We follow the proof given in \cite[Theorem 3.1.1]{athreya-lahiri:measure-theory}.
		
		We have $f \geq 0$ and so
		\begin{align*}
			\int{f\ d\mu} &\geq \int_{\{f \geq t\}}{f\ d\mu} \\ &= \mu\{f(x) \mid x \in X \text{ and } f(x) \geq t \} \\ &\geq t\mu\{x \in X \mid f(x) \geq t \}.
		\end{align*}
		The result follows by dividing by $t$.
	\end{proof}
\end{lemma}

\begin{corollary}\label{cor:cm-3-1}
	For all $t > 0$ we have
	\begin{align}\label{fml:cm-4}
		\mu_\phi&\left\{x \midmid K(x, \sigma{x}, \dots, \sigma^{n - 1}{x}) \geq \int{K(y, \sigma{y}, \dots, \sigma^{n - 1}{y})\ d\mu_\phi(y)} + t\right\} \nonumber \\
			&\leq \exp\left(-\frac{t^2}{4D\sum_{j = 0}^{n - 1}{\Lip_j^2(K)}}\right),
	\end{align}
	for all $n \in \naturals$ and all separately Lipschitz $K$.
	\begin{proof}
		Since $K$ is a separately Lipschitz function, it is clear that $\lambda K$ is also a separately Lipschitz function for all $\lambda > 0$.
		
		Now, for all $\lambda > 0$ and $t > 0$ we have
		\begin{align}
			\mu_\phi&\left\{x \midmid K(x, \sigma{x}, \dots, \sigma^{n - 1}{x}) \geq \int{K(y, \sigma{y}, \dots, \sigma^{n - 1}{y})\ d\mu_\phi(y)} + t\right\} \nonumber \\
				&= \mu_\phi\left\{x \midmid e^{\lambda\left[K(x, \sigma{x}, \dots, \sigma^{n - 1}{x}) - \int{K(y, \sigma{y}, \dots, \sigma^{n - 1}{y})\ d\mu_\phi(y)}\right]} \geq e^{\lambda t}\right\} \nonumber \\
				&\leq \frac{1}{e^{\lambda t}} \int{e^{\lambda\left[K(x, \sigma{x}, \dots, \sigma^{n - 1}{x}) - \int{K(y, \sigma{y}, \dots, \sigma^{n - 1}{y})\ d\mu_\phi(y)}\right]}\ d\mu_\phi(x)} \nonumber \\
				& \leq \exp(-\lambda t) \exp\left(\lambda^2 D \sum_{j = 0}^{n - 1}{\Lip_j^2(K)}\right). \label{fml:cor-3-1-exp}
		\end{align}
		(We have used \thref{lem:markov-ineq} on the penultimate line and \thref{thm:cm-3-1} on the last line.)
		
		We now optimise \eqref{fml:cor-3-1-exp} over $\lambda$, that is, we find the value of $\lambda$ for which \eqref{fml:cor-3-1-exp} has derivative zero with respect to $\lambda$. We have
		\begin{align*}
			0 &= \frac{d}{d\lambda} \exp\left(-\lambda t + \lambda^2 D \sum_{j = 0}^{n - 1}{\Lip_j^2(K)}\right) \\
				&= \left(-t + 2\lambda D \sum_{j = 0}^{n - 1}{\Lip_j^2(K)}\right) \exp\left(-\lambda t + \lambda^2 D \sum_{j = 0}^{n - 1}{\Lip_j^2(K)}\right).
		\end{align*}
		Since $\exp\left(-\lambda t + \lambda^2 D \sum_{j = 0}^{n - 1}{\Lip_j^2(K)}\right) > 0$ for all $\lambda > 0$, we get that
		\[
			\lambda = \frac{t}{2D \sum_{j = 0}^{n - 1}{\Lip_j^2(K)}}.
		\]
		Substituting this value of $\lambda$ into \eqref{fml:cor-3-1-exp} gives
		\begin{align*}
			\mu_\phi&\left\{x \midmid K(x, \sigma{x}, \dots, \sigma^{n - 1}{x}) \geq \int{K(y, \sigma{y}, \dots, \sigma^{n - 1}{y})\ d\mu_\phi(y)} + t\right\} \\
			 &\leq \exp\left(-\frac{t^2}{2D \sum_{j = 0}^{n - 1}{\Lip_j^2(K)}} + \frac{t^2 D \sum_{j = 0}^{n - 1}{\Lip_j^2(K)}}{4D^2 \left(\sum_{j = 0}^{n - 1}{\Lip_j^2(K)}\right)^2}\right) \\
			 &= \exp\left(-\frac{t^2}{4D\sum_{j = 0}^{n - 1}{\Lip_j^2(K)}}\right),
		\end{align*}
		and the result holds.
	\end{proof}
\end{corollary}

The following result is an immediate consequence of \thref{cor:cm-3-1}.

\begin{corollary}\label{cor:cm-3-1-5}
	For all $t > 0$ we have
	\begin{align}\label{fml:cm-4-5}
		\mu_\phi&\left\{x \midmid \left| K(x, \sigma{x}, \dots, \sigma^{n - 1}{x}) - \int{K(y, \sigma{y}, \dots, \sigma^{n - 1}{y})\ d\mu_\phi(y)} \right| \geq t\right\} \nonumber \\
		&\leq 2\exp\left(-\frac{t^2}{4D\sum_{j = 0}^{n - 1}{\Lip_j^2(K)}}\right),
	\end{align}
	for all $n \in \naturals$ and all separately Lipschitz $K$.
	\begin{proof}
		Apply \thref{cor:cm-3-1} for $K$ and $-K$ and then the result follows by countable subadditivity.
	\end{proof}
\end{corollary}

We can also find an upper bound for the variance of $K$ with respect to $\mu_\phi$.

\begin{corollary} \label{cor:cm-3-2}
	For all $n \in \naturals$ and all separately Lipschitz functions $K$, we have
	\[
		\Var_{\mu_\phi}(K) := \int{\left[K(x, \dots, \sigma^{n - 1}{x}) - \int{K(y, \dots, \sigma^{n - 1}{y})\ d\mu_\phi(y)}\right]^2\ d\mu_\phi(x)} \leq 2D\sum_{j = 0}^{n - 1}{\Lip_j^2(K)}.
	\]
	
	\begin{proof}
		Suppose that $\lambda \neq 0$. Applying \thref{thm:cm-3-1} to $\lambda K$, subtracting $1$ and then dividing by $\lambda^2$, we get
		\begin{equation} \label{fml:cor-3-2-variance}
			\frac{1}{\lambda^2}\left(\int{e^{\lambda\left[K(x, \dots, \sigma^{n - 1}{x}) - \int{K(y, \dots, \sigma^{n - 1}{y})\ d\mu_\phi(y)}\right]}\ d\mu_\phi(x)} - 1\right) \leq \frac{1}{\lambda^2}\left(e^{\lambda^2 D \sum_{j = 0}^{n - 1}{\Lip_j^2(K)}} - 1\right).
		\end{equation}
		Now put $L := K(x, \dots, \sigma^{n - 1}{x}) - \int{K(y, \dots, \sigma^{n - 1}{y})\ d\mu_\phi(y)}$ and note that $\int{L\ d\mu_\phi(x)} = 0$. The Taylor expansion of the left-hand side of \eqref{fml:cor-3-2-variance} is
		\begin{align*}
			\frac{1}{\lambda^2}&\left[\int{\left(1 + \lambda L + \frac{\lambda^2 L^2}{2!} + \frac{\lambda^3 L^3}{3!} + \dots\right)\ d\mu_\phi(x)} - 1\right] \\
				&= \frac{1}{\lambda^2}\left(\int{1 + \lambda L\ d\mu_\phi(x)} - 1\right) + \frac{L^2}{2} + \int{O(\lambda)\ d\mu_\phi(x)} \\
				&= \frac{L^2}{2} + \int{O(\lambda)\ d\mu_\phi(x)} \\
				&\to \frac{L^2}{2},
		\end{align*}
		as $\lambda \to 0$.
		
		The Taylor expansion of the right-hand side of \eqref{fml:cor-3-2-variance} is
		\begin{align*}
			\frac{1}{\lambda^2}&\left[\left(1 + \lambda^2 D \sum_{j = 0}^{n - 1}{\Lip_j^2(K)} + \lambda^4 D^2 \left(\sum_{j = 0}^{n - 1}{\Lip_j^2(K)} \right)^2 + \dots\right) - 1 \right] \\
				&= D \sum_{j = 0}^{n - 1}{\Lip_j^2(K)} + O(\lambda^2) \\
				&\to D \sum_{j = 0}^{n - 1}{\Lip_j^2(K)},
		\end{align*}
		as $\lambda \to 0$. The result then follows by combining the two sides.
	\end{proof}
\end{corollary}

We can also apply the above results to find results about ergodic averages.

\begin{corollary}\label{cor:cm-3-3}
	Let $f : \Sigma \to \reals$ be a Lipschitz function. Then for all $t > 0$ and all $n \geq 1$ we have
	\begin{equation}
		\mu_\phi\left\{x \midmid \frac{1}{n}\sum_{j = 0}^{n - 1}{f(\sigma^j{x})} - \int{f\ d\mu_\phi} \geq t\right\} \leq \exp(-Bnt^2),
	\end{equation}
	where $B := (4D|f|_\theta^2)^{-1}$.
	\begin{proof}
		Let $K_0(x^{(0)}, x^{(1)}, \dots, x^{(n - 1)}) := f(x^{(0)}) + f(x^{(1)}) + \dots + f(x^{(n - 1)})$. Since $f$ is Lipschitz, it follows that $K_0$ is separately Lipschitz. In particular, note that for all $j = 0, \dots, n - 1$, we have
		\begin{align*}
			\Lip_j(K_0) &= \sup_{y^{(j)} \neq x^{(j)}}{\frac{|f(x^{(j)}) - f(y^{(j)})|}{d_\theta(x^{(j)}, y^{(j)})}} \\
				&\leq \sup_{m \geq 0}\left\{\frac{\var_m(f)}{\theta^m}\right\} \\
				&= |f|_\theta.
		\end{align*}
		
		Now we apply \thref{cor:cm-3-1} to $\frac{1}{n}K_0$, so that
		\begin{align*}
			\mu_\phi&\left\{x \midmid \frac{1}{n}\sum_{j = 0}^{n - 1}{f(\sigma^j{x})} - \frac{1}{n}\int{\sum_{j = 0}^{n - 1}{f(\sigma^j{y})}\ d\mu_\phi(y)} \geq t\right\} \\
			&=\mu_\phi\left\{x \midmid \frac{1}{n}\sum_{j = 0}^{n - 1}{f(\sigma^j{x})} - \int{f\ d\mu_\phi} \geq t\right\} \\
			&\leq \exp\left(-\frac{t^2}{4D\frac{1}{n^2}\sum_{j = 0}^{n - 1}{\Lip_j^2(K_0)}}\right) \\
			&\leq \exp\left(-\frac{nt^2}{4D|f|_\theta^2}\right) \\
			&= \exp(-Bnt^2).
		\end{align*}
	\end{proof}
\end{corollary}

This result implies that, as $t$ increases, the ergodic average $\frac{1}{n}\sum_{j = 0}^{n - 1}{f(\sigma^j{x})}$ is exponentially less likely to deviate from the space average $\int{f\ d\mu_\phi}$. In other words, the ergodic average concentrates around the space average, hence the term \emph{concentration bound}.

\subsection{Functions of \texorpdfstring{$n$}{n} symbols}
We will consider entropy estimators that utilise functions of $n$ \emph{symbols}, as opposed to functions of $n$ variables in $\Sigma$. We will discuss the estimators in more detail in the next section.

It is clear that we can identify a function $\tilde{K} : A^n \to \reals$ of $n$ symbols with a function $K : \Sigma^n \to \reals$ of $n$ sequences. We can therefore apply \thref{thm:cm-3-1} and its corollaries to $\tilde{K}$. However, for each $j = 0, \dots, n - 1$, we must replace $\Lip_j(K)$ in the above results with
\[
	\delta_j(\tilde{K}) := \sup_{\substack{a_0^{n - 1} \in A^k \\ b_j \neq a_j}}{|\tilde{K}(a_0, \dots, a_{n - 1}) - \tilde{K}(a_0, \dots, a_{j - 1}, b_j, a_{j + 1}, \dots, a_{n - 1})|},
\]
the \key{oscillation at the $j$-th coordinate}.

\section{Entropy estimators}\label{sec:estimator-bounds}
\subsection{Motivation}
Suppose we have an ergodic source with a typical sample output $(x_0, x_1, \dots, x_{n - 1}) \in A^n$. If we do not know the measure-preserving transformation which yields this output, we have to use alternative methods to estimate its entropy. To do this, we will consider \key{plug-in estimators} and the \key{hitting-time estimator}.

There are theorems which show that these estimators tend to the entropy $h(\nu)$, as $n \to +\infty$, for almost every sample sequence $(x_0, \dots, x_{n - 1})$. However, as we would expect, the values given by these estimators have some margin of error when $n$ is relatively small, i.e. they \key{fluctuate} around $h(\nu)$. Our aim is to find concentration bounds for these fluctuations.

\subsection{Plug-in estimators}
Before we define any \key{plug-in estimators}, we first introduce some definitions. The main concept employed by plug-in estimators is the \key{empirical frequency} with which a word $a_0^{k - 1}$ of length $k$ occurs in a sample path $x_0^{n - 1}$ of length $n$.\footnote{The word ``empirical'' relates to information gained by observations. In our case, empirical frequency is the frequency observed by taking a sample path and matching it with a word of length $k$.}

\begin{definition}
	The \key{empirical frequency} with which the word $a_0^{k - 1}$ occurs in $x_0^{n - 1}$ is given by
	\[
		\E_k(a_0^{k - 1}; x_0^{n - 1}) := \frac{1}{n}\card\left\{j \in \{0, \dots, n - 1\} \midmid \tilde{x}_j^{j + k - 1} = a_0^{k - 1} \right\},
	\]
	where $\tilde{x} = x_0^{n - 1} x_0^{n - 1} x_0^{n - 1} \dots$ is the sequence of period $n$ obtained by concatenating $x_0^{n - 1}$ continually, i.e. $\tilde{x}_j = x_{j \bmod n}$.
\end{definition}

The definition of $\tilde{x}$ means that $\E_k(\seedot; x_0^{n - 1})$ is a locally $\sigma$-invariant probability measure on $A^k$. That is, for all words $a_1^k \in A^k$ there exists some $a_0 \in A$ such that
\[
	\E_k(a_0^{k - 1}; x_0^{n - 1}) = \E_k(a_1^k; x_0^{n - 1}).
\]

Given an ergodic measure $\nu$, Birkhoff's Ergodic Theorem gives that for $\nu$-almost every $x \in \Sigma$ and for all $k \geq 1$, we have
\[
	\E_k(a_0^{k - 1}; x_0^{n - 1}) = \frac{1}{n}\card\left\{j \in \{0, \dots, n - 1\} \midmid \tilde{x}_j^{j + k - 1} = a_0^{k - 1} \right\} \to \nu[a_0^{k - 1}],
\]
as $n \to +\infty$.

We are now in the position to define the following examples of plug-in entropy estimators, which can be found in \cite[Definition 2.1]{chazottes-gabrielle:large-deviations}.

\begin{definition}
	Suppose $x_0^{n - 1} \in A^n$ is a word of length $n$.
	
	For $k \geq 1$, the \key{$k$-block empirical entropy} is defined
	\[
		\hat{H}_k(x_0^{n - 1}) := H_k(\E_k(\seedot; x_0^{n - 1})).
	\]
	
	For $k \geq 2$, the \key{$k$-block conditional empirical entropy} is defined
	\[
		\hat{h}_k(x_0^{n - 1}) := h_k(\E_k(\seedot; x_0^{n - 1})).
	\]
\end{definition}

We have that
\[
	\frac{1}{k} \hat{H}_k(x_0^{n - 1}) = \frac{1}{k} H_k(\E_k(\seedot; x_0^{n - 1})) \to \frac{1}{k} H_k(\nu),
\]
as $n \to +\infty$, for $\nu$-almost every $x \in \Sigma$. Recalling that $\frac{1}{k} H_k(\nu) \to h(\nu)$, as $k \to +\infty$, we have
\[
	\lim_{k \to +\infty} \lim_{n \to +\infty}{\frac{1}{k} \hat{H}_k(x_0^{n - 1})} = h(\nu).
\]

We can actually remove one of these limits: By \thref{prop:walters-cor-4-2-1}, we easily see that $0 \leq h(\nu) \leq \log{|A|}$ and so we have
\[
	\frac{\log{n}}{\log{|A|}} \leq \frac{\log{n}}{h(\nu)}.
\]
Hence we can define a monotonically increasing function $k : \naturals \to \naturals$ such that $k(n) \leq \frac{\log{n}}{h(\nu)}$. Then a result by Ornstein and Weiss~\cite{shields:ergodic} shows that
\[
	\lim_{n \to +\infty}{\frac{1}{k(n)}\hat{H}_{k(n)}(x_0^{n -1})} = h(\nu),
\]
for $\nu$-almost every $x \in \Sigma$.

For conditional empirical entropy, we have the following result proved in \cite[Theorem II.3.5]{shields:ergodic}.

\begin{theorem}[Entropy-Estimation Theorem] \label{thm:entropy-estimation}
	 Let $\nu$ be ergodic measure and let $\alpha \in (0, 1)$. Let $k : \naturals \to \naturals$ be a function such that $k(n) \leq \frac{(1 - \alpha)}{\log{|A|}}\log{n}$. Then
	 \[
	 	\lim_{n \to +\infty}{\hat{h}_{k(n)}(x_0^{n - 1})} = h(\nu),
	 \]
	 for $\nu$-almost every $x \in \Sigma$.
\end{theorem}

Therefore both kinds of empirical entropy can be used for estimating the actual entropy $h(\nu)$.

We now present a concentration bound for the $k$-block empirical entropy about its expectation value.

\begin{theorem}\label{thm:cm-4-1}
	For all $\alpha \in (0, 1)$, $t > 0$ and $n \geq 2$, if
	\[
		k(n) \leq \frac{\alpha}{2 \log{|A|}}\log{n},
	\]
	then
	\[
		\mu_\phi\left\{\left|\frac{\hat{H}_{k(n)}}{k(n)} - E_{\mu_\phi}\left(\frac{\hat{H}_{k(n)}}{k(n)}\right)\right| \geq t\right\} \leq \exp\left(-\frac{n^{1 - \alpha} t^2}{16D(\log{n})^2}\right),
	\]
	where $D > 0$ is as in \thref{thm:cm-3-1}.
	
	Furthermore, for all $n \geq 2$ we have
	\[
		\Var_{\mu_\phi}\left(\frac{\hat{H}_{k(n)}}{k(n)}\right) \leq 8D\frac{(\log{n})^2}{n^{1 - \alpha}}.
	\]
	\begin{proof}
		Let $k \in \naturals$ and define a function $\tilde{K} : A^n \to \reals$ by $\tilde{K}(s_0, \dots, s_{n - 1}) = \hat{H}_k(s_0^{n - 1})$. Recall that
		\[
			\delta_j(\tilde{K}) := \sup_{\substack{s_0^{n - 1} \in A^k \\ r_j \neq s_j}}{|\tilde{K}(s_0, \dots, s_{n - 1}) - \tilde{K}(s_0, \dots, s_{j - 1}, r_j, s_{j + 1}, \dots, s_{n - 1})|}
		\]
		replaces $\Lip_j$ in \thref{thm:cm-3-1} and its corollaries. To utilise these results, we must estimate $\delta_j$.
		\begin{claim}
			We claim that
			\begin{equation}
				\delta_j(\tilde{K}) \leq 2k|A|^k\frac{\log{n}}{n}, \label{fml:oscil-est}
			\end{equation}
			for $j = 0, \dots, n - 1$.
			
			Indeed, first suppose that $a_0^{k - 1} \in A^k$ is given. By the nature of how $\delta_j(\tilde{K})$ is defined, we want to consider the effect on the value of $\tilde{K}$ when one coordinate in $s_0^{n - 1}$ is changed. Since $\tilde{K}$ is defined in terms of the empirical frequency, we consider the largest effect on $\E_k(a_0^{k - 1}; s_0^{n - 1})$.
			
			If we change exactly one symbol $s_j$ in $s_0^{n - 1}$, then the value of $\E_k(a_0^{k - 1}; s_0^{n - 1})$ can decrease by \emph{at most} $k / n$. This is because, in the worst case scenario, $s_j$ matches all $k$ characters in $a_0^{k - 1}$ (and hence $\tilde{a}$). Likewise, changing one symbol in $s_0^{n - 1}$ can increase the value of $\E_k(a_0^{k - 1}; s_0^{n - 1})$ by at most $k / n$.
			
			By \thref{prop:logs-thm-4-1}, for $\ell, k \in \naturals$ with $\ell+ k \leq n$, we have that
			\[
				\left|\left(\frac{\ell}{n}\right)\log\left(\frac{\ell}{n}\right) - \left(\frac{\ell + k}{n}\right)\log\left(\frac{\ell + k}{n}\right)\right| \leq \frac{k}{n}\log{n}.
			\]
			Since $\tilde{K}(s_0^{n - 1}) = H_k(\E_k(\seedot; s_0^{n - 1}))$, by all the above results we have
			\begin{align*}
				\delta_j(\tilde{K}) &\leq 2\sum_{a_0^{k - 1} \in A^k}{\left|\left(\frac{\ell}{n}\right)\log\left(\frac{\ell}{n}\right) - \left(\frac{\ell + k}{n}\right)\log\left(\frac{\ell + k}{n}\right)\right|} \\
				 &\leq 2k|A|^k \frac{\log{n}}{n},
			\end{align*}
			for all $j = 0, \dots, n - 1$. Hence the claim holds.
		\end{claim}
		
		Now take $k(n) \leq \frac{\alpha}{2\log{|A|}}\log{n}$, where $\alpha \in (0, 1)$. By rearranging this, we have the relation $|A|^{2k(n)} \leq n^\alpha$. We now apply \thref{cor:cm-3-1-5} so that for all $t > 0$,
		\begin{align*}
			\mu_\phi\left\{\left| \frac{\hat{H}_{k(n)}}{k(n)} - E_{\mu_\phi}\left(\frac{\hat{H}_{k(n)}}{k(n)}\right) \right| \geq t\right\} &= \mu_\phi\left\{\left| \frac{\tilde{K}}{k(n)} - \int{\frac{\tilde{K}}{k(n)}\ d\mu_\phi} \right| \geq t\right\} \\
				&\leq 2\exp\left(-\frac{t^2 (k(n))^2}{4D\sum_{j = 0}^{n - 1}{\delta_j^2(\tilde{K})}}\right) \\
				&\leq 2\exp\left(-\frac{nt^2}{16D|A|^{2k(n)} (\log{n})^2}\right) \\
				&\leq 2\exp\left(-\frac{n^{1 - \alpha} t^2}{16D (\log{n})^2}\right).
		\end{align*}
		This completes the proof of the concentration bound.
		
		For the variance, we apply \thref{cor:cm-3-2} to get
		\begin{align*}
			\Var_{\mu_\phi}\left(\frac{\hat{H}_{k(n)}}{k(n)}\right) &= \Var_{\mu_\phi}\left(\frac{\tilde{K}}{k(n)}\right) \\
				&= \frac{1}{(k(n))^2}\int{\left[\tilde{K} - E_{\mu_\phi}(\tilde{K})\right]^2\ d\mu_\phi} \\
				&\leq \frac{2D}{(k(n))^2}\sum_{j = 0}^{n - 1}{\delta_j^2(\tilde{K})} \\
				&\leq \frac{8D (\log{n})^2}{n^{1 - \alpha}}.
		\end{align*}
	\end{proof}
\end{theorem}

This result gives a concentration bound about the expectation value. However, to quantify the accuracy of an entropy estimator, it makes more sense to find a concentration bound about the entropy $h(\mu_\phi)$. For this, we use $k$-block conditional empirical entropy.

\begin{theorem} \label{thm:cm-4-2}
	Suppose that $\theta < |A|^{-1}$. If $k(n) \leq \frac{\log{n}}{2\log{|A|}}$, then there exist strictly positive constants $c, \gamma, \Gamma, \xi > 0$ such that for all $t > 0$ we have
	\begin{equation}
		\mu_\phi\left\{\left|\hat{h}_{k(n)} - h(\mu_\phi)\right| \geq t + \frac{c}{n^\gamma}\right\} \leq 2\exp\left(-\frac{\Gamma n^\xi t^2}{(\log{n})^4}\right),
	\end{equation}
	for sufficiently large $n$.
	
	%Here, this upper bound decays superexponentially as $t \to +\infty$.
	If we think of $t + (c / n^\gamma)$ as the error of the $k$-block conditional empirical entropy, we see that the number of sequences which yield a `bad' estimate decays superexponentially as $t \to +\infty$. On the other hand, if we let $n \to +\infty$, then the upper bound will tend to zero for all $t > 0$. Hence $\mu_\phi$-almost every sequence gives an estimate which converges to $h(\mu_\phi)$ and this agrees with \thref{thm:entropy-estimation}
	
	\begin{proof}
		Recall that $\hat{h}_k = \hat{H}_k - \hat{H}_{k - 1}$. We put
		\[
			\tilde{K}'(s_0, \dots, s_{n - 1}) := \hat{h}_k(s_0^{n - 1}) = \hat{H}_k(s_0^{n - 1}) - \hat{H}_{k - 1}(s_0^{n - 1}).
		\]
		We estimate $\delta_j(\tilde{K}')$ by $2\delta_j(\tilde{K})$, where $\tilde{K}(s_0, \dots, s_{n - 1}) = \hat{H}_k(s_0^{n - 1})$ and using the estimate for $\delta_j(\tilde{K})$ in Formula \eqref{fml:oscil-est}.
		
		By \thref{lem:cm-4-1}, we have
		\begin{align}
			E_{\mu_\phi}&\left(\hat{h}_{k(n)}(x_0^{n - 1}) - h(\mu_\phi)\right) \nonumber \\
				&= E_{\mu_\phi}\left(\frac{1}{n}\sum_{j = 0}^{n - 1}(-\phi \circ \sigma^j) + \hat{\Delta}_{k(n)} + O(\theta^{k(n)}) - h(\mu_\phi)\right), \label{fml:cm-3-9-lem-ref}
		\end{align}
		where $\hat{\Delta}_{k(n)}(x_0^{n - 1}) = -H_{k(n)}(\E_{k(n)}(\seedot; x_0^{n - 1})) + H_{k(n) - 1}(\E_{k(n) - 1}(\seedot; x_0^{n - 1}))$. 
		Recalling that $h(\mu_\phi) = -\int{\phi\ d\mu_\phi}$, \eqref{fml:cm-3-9-lem-ref} becomes
		\begin{align*}
			E_{\mu_\phi}\left(\hat{h}_{k(n)}\right) - h(\mu_\phi) &= -\int{\phi\ d\mu_\phi} + E_{\mu_\phi}\left(\hat{\Delta}_{k(n)}\right) + O(\theta^{k(n)}) - h(\mu_\phi) \\
				&= h(\mu_\phi) + E_{\mu_\phi}\left(\hat{\Delta}_{k(n)}\right) + O(\theta^{k(n)}) - h(\mu_\phi) \\
				&= E_{\mu_\phi}\left(\hat{\Delta}_{k(n)}\right) + O(\theta^{k(n)}).
		\end{align*}
		
		We now put $k(n) = \frac{q\log{n}}{\log{|A|}}$, where $0 < q < 1$. If we choose $q$ to be
		\[
			q = \frac{1}{1 + \frac{\log{\theta^{-1}}}{\log{|A|}}},
		\]
		then using
		\[
			|E_{\mu_\phi}(\hat{h}_{k(n)}) - h(\mu_\phi)| \leq \frac{M|A|^{k(n)}}{n}
		\]
		from \thref{lem:cm-4-1}, a straightforward but tedious calculation shows that
		\begin{equation}\label{fml:cm-8}
			|E_{\mu_\phi}(\hat{h}_{k(n)}) - h(\mu_\phi)| \leq \frac{c}{n^\gamma},
		\end{equation}
		where $c > 0$ is a constant and $\gamma = \left(1 + \frac{\log{|A|}}{\log{\theta^{-1}}}\right)^{-1} > 0$. We then have
		\begin{align*}
			|\hat{h}_{k(n)} - h(\mu_\phi)| &= |\hat{h}_{k(n)} - E_{\mu_\phi}(\hat{k}_{k(n)}) - h(\mu_\phi) + E_{\mu_\phi}(\hat{k}_{k(n)})| \\
				&\leq |\hat{h}_{k(n)} - E_{\mu_\phi}(\hat{k}_{k(n)})| + |h(\mu_\phi) - E_{\mu_\phi}(\hat{k}_{k(n)})| \\
				&\leq |\hat{h}_{k(n)} - E_{\mu_\phi}(\hat{k}_{k(n)})| + \frac{c}{n^\gamma}.
		\end{align*}
		
		We now apply \thref{cor:cm-3-1-5} to get
		\begin{align*}
			\mu_\phi\left\{\left|\hat{h}_{k(n)} - h(\mu_\phi)\right| \geq t + \frac{c}{n^\gamma}\right\} &\leq \mu_\phi\left\{\left|\hat{h}_{k(n)} - E_{\mu_\phi}(\hat{k}_{k(n)})\right| \geq t\right\} \\
				&\leq 2\exp\left(-\frac{t^2}{4D\sum_{j = 0}^{n - 1}{\delta_j^2(\tilde{K}')}}\right) \\
				&\leq 2\exp\left(-\frac{nt^2}{64D(k(n))^2|A|^{2k(n)}(\log{n})^2}\right) \\
				&\leq 2\exp\left(-\frac{(\log{|A|})^2}{16D} \frac{nt^2}{|A|^{2k(n)}(\log{n})^4}\right) \\
				&\leq 2\exp\left(-\frac{(\log{|A|})^2}{16D} \frac{\theta^{2k(n)} nt^2}{(\log{n})^4}\right) \\
				&\leq 2\exp\left(-\frac{\Gamma n^\xi t^2}{(\log{n})^4}\right),
		\end{align*}
		where
		\[
			\Gamma = \frac{(\log{|A|})^2}{16D} \quad \text{and} \quad \xi = 1 - \frac{\log{\theta^{-1}}}{\log{|A|}}.
		\]
		The value of $\xi$ is found using the assumption that $k(n) \leq \frac{\log{n}}{2\log{|A|}}$. Furthermore, to guarantee that $\xi > 0$ we must have
		\[
			\xi = 1 - \frac{\log{\theta^{-1}}}{\log{|A|}} > 0 \iff \log{|A|} > \log{\theta^{-1}} \iff |A| > \theta^{-1},
		\]
		which is the first hypothesis in the theorem. This completes the proof.
	\end{proof}
\end{theorem}

\subsection{Hitting time estimators}\label{ssec:hitting-times}
We now focus on \key{hitting time entropy estimators}. Instead of using empirical frequencies, hitting time estimators use the first occurrence of the first $n$ symbols of a sequence $x \in \Sigma$ in another sequence $y \in \Sigma$. The following formal definition can be found in \cite[Definition 2.1]{chazottes-ugalde:hitting-times}.

\begin{definition}
	Let $x, y \in \Sigma$, let $a_0^{n - 1} \in A^n$. The (first) \key{hitting time of $x$ to a cylinder $[a_0^{n - 1}]$} is defined
	\[
		\tau_{[a_0^{n - 1}]}(x) := \inf\{j \geq 1 \mid x_j^{j + n - 1} = a_0^{n - 1}\}.
	\]
	
	The \key{following hitting time} or \key{waiting-time}~\cite[Section III.5]{shields:ergodic} is defined
	\[
		W_n(x, y) := \tau_{[x_0^{n - 1}]}(y) = \inf\{j \geq 1 \mid y_j^{j + n - 1} = x_0^{n - 1}\}.
	\]
\end{definition}

Note that, while the above definitions are largely the same, $\tau_{[a_0^{n - 1}]}$ is defined using a word $a_0^{n - 1}$ of length $n$, whereas $W_n$ is defined using sequences in $\Sigma$. In order to analyse the limiting behaviour as $n \to +\infty$, it makes more sense to work with $W_n$.

The following result relates the waiting-time function to the entropy $h(\nu)$ of certain measures $\nu$. This result is proved in \cite[Theorem III.5.1]{shields:ergodic}.

\begin{theorem}[Exact-Match Theorem] \label{thm:exact-match}
	If $\nu$ is weak Bernoulli, then
	\[
		\lim_{n \to +\infty}{\frac{1}{n}\log{W_n(x, y)}} = h(\nu),
	\]
	for $\nu \otimes \nu$-almost every $(x, y) \in \Sigma \otimes \Sigma$.
\end{theorem}

That is, if $\nu$ is weak Bernoulli, then the time it takes for one sequence to hit another tends to the entropy $h(\nu)$, for almost all pairs of sequences. This is the \key{hitting time entropy estimator}.

By \thref{thm:gibbs-is-weak-bernoulli}, Gibbs measures are weak Bernoulli and hence the Exact-Match Theorem applies. As with plug-in estimators, we also have concentration bounds for the hitting time estimator. To prove the main theorem, we will need two lemmata which follow from \cite[Theorem 1]{abadi:sharp}.

\begin{lemma}\label{lem:cm-4-2}
	There exists constants $C, c, \lambda_1, \lambda_2 > 0$ with $\lambda_1 < \lambda_2$ such that, for all $n \geq 1$ and for all $a_0^{n - 1} \in A^n$, there exists $\Lambda(a_0^{n - 1}) \in [\lambda_1, \lambda_2]$ such that for all $u > 0$ we have
	\[
		\left| \mu_\phi\left\{x \midmid \tau_{[a_0^{n - 1}]}(x) > \frac{u}{\Lambda(a_0^{n - 1}) \mu_\phi[a_0^{n - 1}]}\right\} - e^{-u} \right| \leq Ce^{-cu}.
	\]
\end{lemma}

\begin{lemma}\label{lem:cm-4-3}
	For all $v > 0$ and any word $a_0^{n - 1} \in A^n$ such that $v\mu_\phi[a_0^{n - 1}] \leq \frac{1}{2}$ we have
	\[
		\lambda_1 \leq -\frac{\log{\mu_\phi\{\tau_{[a_0^{n - 1}]} > v\}}}{v\mu_\phi[a_0^{n - 1}]} \leq \lambda_2,
	\]
	where $\lambda_1, \lambda_2 > 0$ are the same constants as in \thref{lem:cm-4-2}.
\end{lemma}

We are now ready to state and prove concentration bounds for the hitting time estimator.

\begin{theorem} \label{thm:cm-4-3}
	There exist strictly positive constants $C_1, C_2 > 0$, $t_0 > 0$ such that, for all $n \geq 1$ and for all $t > t_0$, we have
	\begin{equation}
		(\mu_\phi \otimes \mu_\phi)\left\{(x, y) \midmid \frac{1}{n}\log{W_n(x, y)} > h(\mu_\phi) + t\right\} \leq C_1 e^{-C_2 nt^2} \label{fml:cm-9}
	\end{equation}
	and
	\begin{equation}
		(\mu_\phi \otimes \mu_\phi)\left\{(x, y) \midmid \frac{1}{n}\log{W_n(x, y)} < h(\mu_\phi) - t\right\} \leq C_1 e^{-C_2 nt}. \label{fml:cm-10}
	\end{equation}	
	Essentially, \eqref{fml:cm-9} tells us how likely the hitting time estimator is to \emph{overestimate} the entropy, while \eqref{fml:cm-10} gives how likely it is to \emph{underestimate} the entropy. Note that the upper bound in \eqref{fml:cm-10} decays exponentially as $t \to +\infty$, whereas in \eqref{fml:cm-9} the bound decays superexponentially as $t \to +\infty$. In either case, this means that fewer pairs $(x, y)$ yield estimates which deviate very far from $h(\mu_\phi)$.
	
	If we let $n \to +\infty$, both \eqref{fml:cm-9} and \eqref{fml:cm-10} are bounded above by $0$ for any $t > t_0$. This means that $\mu_\phi \otimes \mu_\phi$-almost every $(x, y)$ gives an estimate which deviates by at most $t_0$ from $h(\mu_\phi)$, and this largely agrees with \thref{thm:exact-match}.
	
	Unfortunately, to prove this result we cannot use \thref{thm:cm-3-1} and its corollaries directly. Our approach to the proof of this theorem involves approximating the value of $W_n(x, y)$, and to do this we need to use the fact that $\mu_\phi$ is a Gibbs measure.
	
	\begin{proof}
		We concentrate on \eqref{fml:cm-9} first. We have
		\begin{align*}
			&(\mu_\phi \otimes \mu_\phi)\left\{(x, y) \midmid \frac{1}{n}\log{W_n(x, y)} > h(\mu_\phi) + t\right\} \\
				&= (\mu_\phi \otimes \mu_\phi)\left\{(x, y) \midmid \frac{1}{n}\log{W_n(x, y)} + \frac{1}{n}\log{\mu_\phi[x_0^{n - 1}]} - \frac{1}{n}\log{\mu_\phi[x_0^{n - 1}]} - h(\mu_\phi) > t\right\} \\
				&\leq T_1 + T_2, % (\mu_\phi \otimes \mu_\phi)\left\{(x, y) \midmid \frac{1}{n}\log{W_n(x, y)} + \frac{1}{n}\log{\mu_\phi[x_0^{n - 1}]} > \frac{t}{2}\right\}, \\
				%& \quad + \mu_\phi\left\{x \midmid - \frac{1}{n}\log{\mu_\phi[x_0^{n - 1}]} - h(\mu_\phi) > \frac{t}{2}\right\} \\
		\end{align*}
		where
		\begin{align*}
			T_1 &:= (\mu_\phi \otimes \mu_\phi)\left\{(x, y) \midmid \frac{1}{n}\log{W_n(x, y)} + \frac{1}{n}\log{\mu_\phi[x_0^{n - 1}]} > \frac{t}{2}\right\} \\
				&= (\mu_\phi \otimes \mu_\phi)\left\{(x, y) \midmid \frac{1}{n}\log\left(W_n(x, y) \mu_\phi[x_0^{n - 1}]\right) > \frac{t}{2}\right\}
		\end{align*}
		and
		\[
			T_2 := \mu_\phi\left\{x \midmid - \frac{1}{n}\log{\mu_\phi[x_0^{n - 1}]} - h(\mu_\phi) > \frac{t}{2}\right\}.
		\]
		
		We find an upper bound for $T_2$ first. By the definition of Gibbs measures, we have
		\begin{equation}\label{fml:gibbs-property}
			-\log{C} + \sum_{j = 0}^{n - 1}{\phi(\sigma^j{x})} \leq \log{\mu_\phi[x_0^{n - 1}]} \leq \log{C} + \sum_{j = 0}^{n - 1}{\phi(\sigma^j{x})}.
		\end{equation}
		Once again, recalling that $h(\mu_\phi) = -\int{\phi\ d\mu_\phi}$, we apply \thref{cor:cm-3-3} with $f = -\phi$ to get
		\begin{align*}
			T_2 &= \mu_\phi\left\{x \midmid - \frac{1}{n}\log{\mu_\phi[x_0^{n - 1}]} - h(\mu_\phi) > \frac{t}{2}\right\} \\
				&\leq \mu_\phi\left\{x \midmid -\frac{1}{n}\sum_{j = 0}^{n - 1}{\phi(\sigma^j{x})} -h(\mu_\phi) > \frac{t}{2} -\frac{1}{n}\log{C}\right\} \\
				&= \mu_\phi\left\{x \midmid -\frac{1}{n}\sum_{j = 0}^{n - 1}{\phi(\sigma^j{x})} + \int{\phi\ d\mu_\phi} > \frac{t}{2} -\frac{1}{n}\log{C}\right\} \\
				& \leq e^{-Bnt^2},
		\end{align*}
		for all $t > 2\log{C}$, where $B = (4D|f|_\theta^2)^{-1}$.
		
		We now find an upper bound for $T_1$. We have
		\begin{align*}
			T_1 &= (\mu_\phi \otimes \mu_\phi)\left\{(x, y) \midmid \frac{1}{n}\log\left(W_n(x, y) \mu_\phi[x_0^{n - 1}]\right) > \frac{t}{2}\right\} \\
				&= \sum_{a_0^{n - 1} \in A^n}{(\mu_\phi \otimes \mu_\phi)\left\{(x, y) \midmid x \in [a_0^{n - 1}],\ \frac{1}{n}\log\left(W_n(x, y) \mu_\phi[x_0^{n - 1}]\right) > \frac{t}{2}\right\}} \\
				&= \sum_{a_0^{n - 1} \in A^n}{\mu_\phi[a_0^{n - 1}] \mu_\phi\left\{y \midmid \tau_{[a_0^{n - 1}]}(y) \mu_\phi[a_0^{n - 1}] > e^\frac{nt}{2}\right\}}
		\end{align*}
		By \thref{lem:cm-4-2} with $u = e^\frac{nt}{2}$ we have
		\begin{align*}
			\left|T_1 - e^{-e^\frac{nt}{2}}\right| &= \left|\sum_{a_0^{n - 1} \in A^n}\mu_\phi[a_0^{n - 1}]\left(\mu_\phi\left\{y \midmid \tau_{[a_0^{n - 1}]}(y) \mu_\phi[a_0^{n - 1}] > e^\frac{nt}{2}\right\} - e^{-e^\frac{nt}{2}}\right)\right| \\
				& \leq \sum_{a_0^{n - 1} \in A^n}\mu_\phi[a_0^{n - 1}] \hat{C}e^{-\hat{c}e^\frac{nt}{2}} \\
				& = \hat{C}e^{-\hat{c}e^\frac{nt}{2}},
		\end{align*}
		for some constants $\hat{C}, \hat{c} > 0$. So we have $T_1 \leq C'e^{-c'e^\frac{nt}{2}}$, for some constants $C', c' > 0$.
		
		Finally, we combine $T_1$ and $T_2$ to get
		\begin{align*}
			(\mu_\phi \otimes \mu_\phi)\left\{(x, y) \midmid \frac{1}{n}\log{W_n(x, y)} > h(\mu_\phi) + t\right\} &\leq C'e^{-c'e^\frac{nt}{2}} + e^{-Bnt^2} \\
				& \leq C_1 e^{Bnt^2},
		\end{align*}
		for some constant $C_1 > 0$. This proves \eqref{fml:cm-9}.
		
		We now turn our attention to \eqref{fml:cm-10}. We have
		\begin{align*}
			&(\mu_\phi \otimes \mu_\phi)\left\{(x, y) \midmid \frac{1}{n}\log{W_n(x, y)} < h(\mu_\phi) - t\right\} \\
				&= (\mu_\phi \otimes \mu_\phi)\left\{(x, y) \midmid -\frac{1}{n}\log{W_n(x, y)} - \frac{1}{n}\log{\mu_\phi[x_0^{n - 1}]} + \frac{1}{n}\log{\mu_\phi[x_0^{n - 1}]} + h(\mu_\phi) > t\right\} \\
				&\leq T'_1 + T'_2,
		\end{align*}
		where
		\begin{align*}
			T'_1 &:= (\mu_\phi \otimes \mu_\phi)\left\{(x, y) \midmid -\frac{1}{n}\log{W_n(x, y)} - \frac{1}{n}\log{\mu_\phi[x_0^{n - 1}]} > \frac{t}{2}\right\} \\
				&= (\mu_\phi \otimes \mu_\phi)\left\{(x, y) \midmid -\frac{1}{n}\log\left(W_n(x, y) \mu_\phi[x_0^{n - 1}]\right) > \frac{t}{2}\right\}
		\end{align*}
		and
		\[
			T'_2 := \mu_\phi \left\{x \midmid \frac{1}{n}\log{\mu_\phi[x_0^{n - 1}]} + h(\mu_\phi) > \frac{t}{2}\right\}.
		\]
		
		To find an upper bound for $T'_2$, we once again use the Gibbs property in \eqref{fml:gibbs-property} and apply \thref{cor:cm-3-3} with $f = \phi$ to get
		\begin{align*}
			T'_2 &\leq \mu_\phi\left\{x \midmid \frac{1}{n}\sum_{j = 0}^{n - 1}{\phi(\sigma^j{x})} + h(\mu_\phi) > \frac{t}{2} - \frac{1}{n}\log{C}\right\} \\
				&= \mu_\phi\left\{x \midmid \frac{1}{n}\sum_{j = 0}^{n - 1}{\phi(\sigma^j{x})} - \int{\phi\ d\mu_\phi} > \frac{t}{2} - \frac{1}{n}\log{C}\right\} \\
				&\leq e^{-\frac{Bnt^2}{4}},
		\end{align*}
		for all $t > 2\log{C}$. (We have applied Corollary 4.6 for $t / 2$ instead of $t$.)
		
		As for $T_1$ above, for $T'_1$ we have
		\begin{align*}
			T'_1 &= (\mu_\phi \otimes \mu_\phi)\left\{(x, y) \midmid -\frac{1}{n}\log\left(W_n(x, y) \mu_\phi[x_0^{n - 1}]\right) > \frac{t}{2}\right\} \\
				&= \sum_{a_0^{n - 1} \in A^n}{(\mu_\phi \otimes \mu_\phi)\left\{(x, y) \midmid x \in [a_0^{n - 1}],\ -\frac{1}{n}\log\left(W_n(x, y) \mu_\phi[x_0^{n - 1}]\right) > \frac{t}{2}\right\}} \\
				&= \sum_{a_0^{n - 1} \in A^n} \mu_\phi[a_0^{n - 1}]\mu_\phi\left\{y \midmid \tau_{[a_0^{n - 1}]}(y) \mu_\phi[a_0^{n - 1}] < e^{-\frac{nt}{2}}\right\}.
		\end{align*}
		Now by \thref{lem:cm-4-3}, if $v\mu_\phi[a_0^{n - 1}] \leq \frac{1}{2}$, then we have
		\[
			-\frac{\log{\mu_\phi\{\tau_{[a_0^{n - 1}]} > v\}}}{v\mu_\phi[a_0^{n - 1}]} \leq -\frac{\log{\mu_\phi\{\tau_{[a_0^{n - 1}]} > v\}}}{v} \leq \lambda_2.
		\]
		Rearranging this, we get
		\[
			-\log\left(1 - \mu_\phi\{\tau_{[a_0^{n - 1}]} < v\}\right) = -\log{\mu_\phi\{\tau_{[a_0^{n - 1}]} > v\}} \leq \lambda_2 v,
		\]
		which gives
		\[
			\mu_\phi\{\tau_{[a_0^{n - 1}]} < v\} \leq 1 - e^{-\lambda_2 v} \leq \lambda_2 v,
		\]
		because $1 - e^{-u} \leq u$ for all $u \in \reals$.
		
		Since $0 \leq \mu_\phi[a_0^{n - 1}] \leq 1$ for all $a_0^{n - 1}$, if we let $v = e^{-\frac{nt}{2}}$, then we have
		\begin{align*}
			T'_1 &\leq \sum_{a_0^{n - 1} \in A^n} \mu_\phi[a_0^{n - 1}]\mu_\phi\left\{y \midmid \tau_{[a_0^{n - 1}]}(y) < e^{-\frac{nt}{2}}\right\} \\
				&\leq \sum_{a_0^{n - 1} \in A^n} \mu_\phi[a_0^{n - 1}] \lambda_2 e^{-\frac{nt}{2}} \\
				&= \lambda_2 e^{-\frac{nt}{2}},
		\end{align*}
		provided that $e^{-\frac{nt}{2}}\mu_\phi[a_0^{n - 1}] \leq \frac{1}{2}$. We want this to be true for all $n$ and all words $a_0^{n - 1}$ of length $n$, so this restriction is equivalent to $e^{-\frac{t}{2}} \leq \frac{1}{2}$ which is the same as $t \geq 2\log{2}$. So we take $t_0 = \max\{2\log{C}, 2 \log{2}\}$.
		
		Finally, we combine $T'_1$ and $T'_2$ to get
		\[
			(\mu_\phi \otimes \mu_\phi)\left\{(x, y) \midmid \frac{1}{n}\log{W_n(x, y)} < h(\mu_\phi) - t\right\} \leq \lambda_2 e^{-\frac{nt}{2}} + e^{-\frac{Bnt^2}{4}} \leq C_1e^{-C_2 nt},
		\]
		for some constants $C_1, C_2 > 0$.
	\end{proof}
\end{theorem}


\appendix
\chapter{Auxiliary results}
This appendix contains results which are used throughout the dissertation. We separate the appendix into sections depending on which chapter the result is first used.

\section{Entropy}
We use the following result in \thref{prop:walters-cor-4-2-1} and also \thref{lem:pp-3-3}.

\begin{definition}
	Let $f : X \to \reals$ be a function of a convex set $X$. We say that $f$ is \key{strictly convex} if, for all $x, y \in X$ and for all $\alpha \in [0, 1]$, we have
	\[
		f(\alpha x + (1 - \alpha)y) \leq \alpha f(x) + (1 - \alpha)f(y),
	\]
	with equality if and only if $x = y$ or $\alpha = 0$ or $1$.
	
	In general, $f$ is strictly convex if, for any set $\{x_j \in X \mid j \in \{1, \dots, k\}\}$ and for any $\{\alpha_j \in [0, 1] \mid j \in \{1, \dots, k\},\ \sum_{j = 1}^k{\alpha_j} = 1\}$, we have
	\[
		f\left(\sum_{j = 1}^k{\alpha_j x_j}\right) \leq \sum_{j = 1}^k{\alpha_j} f(x_j),
	\]
	with equality if and only if $x_1 = x_2 = \dots = x_k$ whenever $\alpha_j \neq 0$ for all $j = 1, \dots, k$.
\end{definition}

\begin{theorem} \label{thm:walters-4-2-xlogx-convex}
	Let $f : [0, +\infty) \to \reals$ be the function defined by
	\[
		f(x) =
		\begin{cases}
			0, & \text{if } x = 0; \\
			x\log{x}, & \text{if } x \neq 0.
		\end{cases}
	\]
	Then $f$ is \emph{strictly convex}.
	\begin{proof}
		For $x > 0$ we have $f'(x) = 1 + \log{x}$ and $f''(x) = 1 / x$. Suppose that $y > x$ and let $\alpha \in (0, 1)$. In particular, this means that $x < \alpha x + (1 - \alpha)y < y$. Now $f$ is clearly continuous, and hence we may apply the Mean Value Theorem. So there exists $z \in (\alpha x + (1 - \alpha)y, y)$ such that
		\[
			f'(z) = \frac{f(y) - f(\alpha x + (1 - \alpha)y)}{y - \alpha x - (1 - \alpha)y},
		\]
		and so
		\[
			f(y) - f(\alpha x + (1 - \alpha)y) = f'(z)\alpha(y - x).
		\]
		
		Similarly, there exists $w \in (x, \alpha x + (1 - \alpha)y)$ such that
		\[
			f(\alpha x + (1 - \alpha)y) - f(x) = f'(w)(1 - \alpha)(y - x).
		\]
		
		Since $x > 0$, we have $f''(x) = 1 / x > 0$ and so $f'(z) > f'(w)$. Hence
		\begin{align*}
			&(1 - \alpha)(f(y) - f(\alpha x + (1 - \alpha)y)) = f'(z)\alpha(1 - \alpha)(y - x) \\
			& \quad > f'(w)\alpha(1 - \alpha)(y - x) = \alpha(f(\alpha x + (1 - \alpha)y) - f(x)).
		\end{align*}
		Cancelling terms, this gives
		\[
			f(\alpha x + (1 - \alpha)y) < \alpha f(x) + (1 - \alpha)f(y)
		\]
		for $y > x > 0$. We may apply the same argument for $x, y \geq 0$ and $x \neq y$.
	\end{proof}
\end{theorem}

\section{Gibbs measures}
The follow result is used in the proof of \thref{lem:pp-prop-3-4}. First, we need to a definition.

\begin{definition}
	Let $X$ be a convex set and suppose $f : X \to \reals$ (or $\complex$). We say that the function $f$ is \key{concave} on $X$ if for all $x, y \in X$ we have
	\[
		f(\alpha x + (1 - \alpha)y) \geq \alpha f(x) + (1 - \alpha)f(y),
	\]
	for all $\alpha \in [0, 1]$. The function $f$ is \key{strictly concave} if the inequality is strict.~\cite[p11]{cambini-martein:generalized}
\end{definition}

\begin{lemma} \label{lem:pp-3-3}
	Suppose that $(p_1, \dots, p_k), (q_1, \dots, q_k)$ are probability vectors with $p_j > 0$ for all $j = 1, \dots, k$. Then
	\[
		-\sum_{j = 1}^k{q_j \log{q_j}} + \sum_{j = 1}^k{q_j \log{p_j}} \leq 0,
	\]
	with equality if and only if $p_j = q_j$ for all $j = 1, \dots, k$.
	\begin{proof}
		We have
		\begin{align}
			-\sum_{j = 1}^k{q_j \log{q_j}} + \sum_{j = 1}^k{q_j \log{p_j}} &= \sum_{j = 1}^k{q_j \log\left(\frac{p_j}{q_j}\right)} \nonumber \\
				&= \sum_{j = 1}^k{-p_j \frac{q_j}{p_j} \log\left(\frac{q_j}{p_j}\right)} \nonumber \\
				&= \sum_{j = 1}^k{p_j \phi\left(\frac{q_j}{p_j}\right)}, \label{fml:pp-3-3-sums}
		\end{align}
		where $\phi(x) = -x \log{x}$, with the convention that $\phi(0) = 0$. By \thref{thm:walters-4-2-xlogx-convex}, we have that $x \log{x}$ is a strictly convex function, and so $\phi$ is strictly concave. Continuing from \eqref{fml:pp-3-3-sums}, we have
		\[
			-\sum_{j = 1}^k{q_j \log{q_j}} + \sum_{j = 1}^k{q_j \log{p_j}} \leq \phi\left(\sum_{j = 1}^k{p_j \frac{q_j}{p_j}}\right) = \phi(1) = 0,
		\]
		with equality if and only if all the $(q_j / p_j)$ terms are equal.
	\end{proof}
\end{lemma}

\section{Concentration bounds}

The following result is used in the proof of \thref{thm:cm-4-1}.

\begin{proposition}\label{prop:logs-thm-4-1}
	Let $\ell, k \in \naturals$ be such that $\ell + k \leq n$. Then
	\[
		\left|\left(\frac{\ell}{n}\right)\log\left(\frac{\ell}{n}\right) - \left(\frac{\ell + k}{n}\right)\log\left(\frac{\ell + k}{n}\right)\right| \leq \left(\frac{k}{n}\right)\log{n}.
	\]
	
	\begin{proof}
		We have
		\begin{align*}
			&\left(\frac{\ell}{n}\right)\log\left(\frac{\ell}{n}\right) - \left(\frac{\ell + k}{n}\right)\log\left(\frac{\ell + k}{n}\right) \\
				&\quad = \left(\frac{k}{n}\right)\log{n} - \left[\left(\frac{\ell + k}{n}\right)\log(\ell + k) - \left(\frac{\ell}{n}\right)\log{\ell}\right].
		\end{align*}
		Since $k \geq 1$, we have
		\begin{align*}
			0 &\leq \left(\frac{\ell + k}{n}\right)\log(\ell + k) - \left(\frac{\ell}{n}\right)\log{\ell} \\
				&\leq \left(\frac{\ell + k}{n}\right)\log(\ell + k) + \left(\frac{k - \ell}{n}\right)\log{\ell} \\
				&\leq \left(\frac{\ell + k}{n}\right)\log{n} + \left(\frac{k - \ell}{n}\right)\log{n} \\
				&\leq 2\left(\frac{k}{n}\right)\log{n}.
		\end{align*}
		Hence
		\[
			-\left(\frac{k}{n}\right)\log{n} \leq \left(\frac{\ell}{n}\right)\log\left(\frac{\ell}{n}\right) - \left(\frac{\ell + k}{n}\right)\log\left(\frac{\ell + k}{n}\right) \leq \left(\frac{k}{n}\right)\log{n},
		\]
		in other words,
		\[
			\left|\left(\frac{\ell}{n}\right)\log\left(\frac{\ell}{n}\right) - \left(\frac{\ell + k}{n}\right)\log\left(\frac{\ell + k}{n}\right)\right| \leq \left(\frac{k}{n}\right)\log{n}.
		\]
	\end{proof}
\end{proposition}

The following result allows us to write the $k$-block conditional empirical entropy $\hat{h}_{k(n)}(x_0^{n - 1})$ in a more useful form. It is used in the proof of \thref{thm:cm-4-2}.

\begin{lemma}\label{lem:cm-4-1}
	Let $\phi \in F_\theta$. We have
	\begin{equation}
		\hat{h}_{k(n)}(x_0^{n - 1}) = \frac{1}{n}\sum_{j = 0}^{n - 1}(-\phi \circ \sigma^j(x)) + \hat{\Delta}_{k(n)}(x_0^{n - 1}) + O(\theta^{k(n)}),
	\end{equation}
	where
	\[
		|E(\hat{\Delta}_{k(n)})| \leq \frac{M|A|^{k(n)}}{n},
	\]
	for some $M > 0$.
	\begin{proof}
		We follow the proof given in \cite[p10-11]{chazottes-maldonado:cbfee}.
		
		We have
		\begin{align}
			\hat{h}_k(x_0^{n - 1}) &= h_k(\E_k(\seedot; x_0^{n - 1})) \nonumber \\
				&= -\sum_{a_0^{k - 1} \in A^k}{\E_k(a_0^{k - 1}; x_0^{n - 1}) \log{\frac{\E_k(a_0^{k - 1}; x_0^{n - 1})}{\E_{k - 1}(a_0^{k - 2}; x_0^{n - 1})}}} \nonumber \\
				&= \hat{\Delta}_k(x_0^{n - 1}) -\sum_{a_0^{k - 1} \in A^k}\E_k(a_0^{k - 1}; x_0^{n - 1}) \log{\frac{\mu_\phi[a_0^{k - 1}]}{\mu_\phi[a_1^{k - 1}]}}, \label{fml:cm-11}
		\end{align}
		where
		\begin{align*}
			\hat{\Delta}_k(x_0^{n - 1}) &=-\sum_{a_0^{k - 1} \in A^k}{\E_k(a_0^{k - 1}; x_0^{n - 1}) \log{\frac{\E_k(a_0^{k - 1}; x_0^{n - 1})}{\E_{k - 1}(a_0^{k - 2}; x_0^{n - 1})}}} \\
				& \qquad + \sum_{a_0^{k - 1} \in A^k}\E_k(a_0^{k - 1}; x_0^{n - 1}) \log{\frac{\mu_\phi[a_0^{k - 1}]}{\mu_\phi[a_1^{k - 1}]}} \\
				&=-\sum_{a_0^{k - 1} \in A^k}{\E_k(a_0^{k - 1}; x_0^{n - 1}) \log{\frac{\E_k(a_0^{k - 1}; x_0^{n - 1})}{\mu_\phi[a_0^{k - 1}]}}} \\
				& \qquad + \sum_{a_0^{k - 1} \in A^k}\E_k(a_0^{k - 1}; x_0^{n - 1}) \log{\frac{\E_{k - 1}(a_0^{k - 2}; x_0^{n - 1})}{\mu_\phi[a_1^{k - 1}]}}.
		\end{align*}
		Since $\E_k(\seedot; x_0^{n - 1})$ is locally $\sigma$-invariant, we have
		\[
			\sum_{a_0 \in A}{\E_k(a_0^{k - 1}; x_0^{n - 1})} = \E_{k - 1}(a_1^{k - 1}; x_0^{n - 1}).
		\]
		It is also clear that
		\[
			\sum_{a_{k - 1} \in A}{\mathcal{E}_k(a_0^{k - 1}; x_0^{n - 1})} = \mathcal{E}_{k - 1}(a_0^{k - 2}; x_0^{n - 1}).
		\]
		This means that
		\begin{align*}
			\sum_{a_0^{k - 1} \in A^k}&\E_k(a_0^{k - 1}; x_0^{n - 1}) \log{\frac{\E_{k - 1}(a_0^{k - 2}; x_0^{n - 1})}{\mu_\phi[a_1^{k - 1}]}} \\
				&= \sum_{a_0^{k - 1} \in A^k}\E_k(a_0^{k - 1}; x_0^{n - 1}) \log{\frac{\sum_{a_{k} \in A}{\mathcal{E}_k(a_0^{k - 1}; x_0^{n - 1})}}{\mu_\phi[a_1^{k - 1}]}} \\
				&= \sum_{a_0^{k - 1} \in A^k}\E_k(a_0^{k - 1}; x_0^{n - 1}) \log{\frac{\mathcal{E}_k(a_0^{k - 1}; x_0^{n - 1})}{\mu_\phi[a_1^{k - 1}]}} \\
				&= \sum_{a_1^{k - 1} \in A^{k - 1}}\sum_{a_0 \in A}\E_k(a_0^{k - 1}; x_0^{n - 1}) \log{\frac{\mathcal{E}_k(a_0^{k - 1}; x_0^{n - 1})}{\mu_\phi[a_1^{k - 1}]}} \\
				&= \sum_{a_1^{k - 1} \in A^{k - 1}}\E_{k - 1}(a_1^{k - 1}; x_0^{n - 1}) \log{\frac{\mathcal{E}_{k - 1}(a_1^{k - 1}; x_0^{n - 1})}{\mu_\phi[a_1^{k - 1}]}} \\
				&= H_{k - 1}(\E_{k - 1}(\seedot; x_0^{n - 1})).
		\end{align*}
		Hence $\hat{\Delta}_k(x_0^{n - 1}) = -H_k(\E_k(\seedot; x_0^{n - 1})) + H_{k - 1}(\E_{k - 1}(\seedot; x_0^{n - 1}))$.
		
		There is a formula due to \cite[Formulae (4.15), (4.16)]{gabrielli-galves-guiol:fluctuations} which gives the bound
		\[
			|E_{\mu_\phi}(\hat{\Delta}_{k(n)})| \leq \frac{M|A|^k}{n}
		\]
		for all $n \geq 1$, where $M > 0$ is a strictly positive constant.
		
		We now deal with the summation in Formula \eqref{fml:cm-11}. For $y \in \Sigma$ we put
		\[
			\phi_k(y) = \log\frac{\mu_\phi[y_0^{k - 1}]}{\mu_\phi[y_1^{k - 1}]}.
		\]
		By \thref{prop:pp-3-2}, for all $y \in \Sigma$ we have
		\[
			\|\phi_k(y) - \phi(y)\|_\infty = \left\|\log\frac{\mu_\phi[y_0^{k - 1}]}{\mu_\phi[y_1^{k - 1}]} - \phi(y)\right\|_\infty \leq |\phi|_\theta \theta^k.
		\]
		So we may reasonably replace any instance of $\phi_k(y)$ with $\phi(y) + O(\theta^k)$. By the way $\E_k(\seedot; x_0^{n - 1})$ is defined, we have
		\begin{align*}
			-\sum_{a_0^{k - 1} \in A^k}\E_k(a_0^{k - 1}; x_0^{n - 1}) \log{\frac{\mu_\phi[a_0^{k - 1}]}{\mu_\phi[a_1^{k - 1}]}} &= -\sum_{a_0^{k - 1} \in A^k}\E_k(a_0^{k - 1}; x_0^{n - 1}) \phi_k(a) \\
				&= \frac{1}{n}\sum_{j = 0}^{n - 1}{(-\phi \circ \sigma^j(x))} + O(\theta^k).
		\end{align*}
		The result follows by substituting this back into Formula \eqref{fml:cm-11}.
	\end{proof}
\end{lemma}


\bibliographystyle{alpha}
\bibliography{references}

% If you need more than one appendix, then just use another \chapter command
%\chapter{Yet Another Appendix}

\end{document}
}

% Uncomment the line below to suppress the `List of Tables' page (optional)
\tablespagefalse

% Uncomment the line below to suppress the `List of Figures' page (optional)
\figurespagefalse

\beforeabstract

Given a measure-preserving transformation, we can use various techniques to calculate its entropy, which is an isomorphism invariant of measure-preserving transformations. If we have an unknown ergodic source which outputs a typical sample path $(x_0, x_1, \dots, x_{n - 1})$, then we cannot use these techniques, and instead we must use methods to estimate the entropy. Two such methods are plug-in estimators and the hitting time estimator.

This dissertation gives an accessible account of a recent paper by Chazottes and Maldonado, and so we find inequalities to describe the fluctuating behaviour of these entropy estimators on Gibbs measures, a class of measures on shifts of finite type.

\afterabstract

% The next part is optional; however it is a good place to thank your
% supervisor and the people responsible for providing computer support ;-)
\prefacesection{Acknowledgements}
I would like to express my gratitude to my supervisor Professor Richard Sharp for his guidance and support throughout the writing of this dissertation.

% The next line is NOT optional and MUST appear
\afterpreface

% Finally, you can start writing about all the new theorems you have proved
% and all the new results that you have discovered

\prefacesection{Conventions}
The following outlines the conventions used throughout this dissertation.

\section*{Notation}
\begin{trivlist}
	\item $\naturals = \{1, 2, 3, \ldots\}$, the set of natural numbers -- positive integers \emph{not} including $0$.
	\item $\naturals_0 = \{0, 1, 2, \ldots\}$, the set of nonnegative integers.
	\item $\reals^+ = \{a \in \reals \mid a \geq 0\}$, the set of nonnegative real numbers including $0$.
\end{trivlist}

\section*{Propositions, theorems, examples, etc.}
Throughout this dissertation a black square $\blacksquare$ marks the end of each definition, remark and results where the proof has been omitted. If we have a claim within a proof, then the proof of the claim is also marked with a black square.

A white square $\square$ is used to mark the end of all other proofs.

\chapter{Introduction}
\section{Overview}
The main aim of this dissertation is to provide a more accessible account of \cite{chazottes-maldonado:cbfee}, which we will do in Chapter \ref{chap:concentration-bounds}. To achieve this, the following chapters are devoted to describing the key concepts required to provide the reader with the relevant background knowledge.

A property of measure-preserving transformations called \key{entropy} makes up a crucial part of this dissertation. We will show that, if two measure-preserving transformations are `the same', then they have the same entropy. (In Chapter \ref{chap:entropy} we will formally define `the same'.)

The main ideas in \cite{chazottes-maldonado:cbfee} focus on methods for estimating entropy. For example, there is a theorem which says that, for almost all $(x, y)$ we have
\[
	\frac{1}{n} \log{W_n(x, y)} \to h(\nu),
\]
as $n \to +\infty$, where $W_n$ is a function we will define later and $h(\nu)$ is the entropy of a measure $\nu$. We will see later that this is called the hitting time entropy estimator.

Although it may seem obvious, it is worth emphasising that entropy estimators give \emph{estimates} for the entropy. For example, consider two sample pairs $(x_1, y_1)$, $(x_2, y_2)$ which achieve convergence for the above function. If we fix $n \geq 1$, it is possible that $\frac{1}{n} \log{W_n(x_1, y_1)}$ gives a value which is close to $h(\nu)$, whereas $\frac{1}{n} \log{W_n(x_2, y_2)}$ gives a value which is far away from $h(\nu)$. In the final part of this dissertation, we will be particularly interested in these fluctuation properties of entropy estimators.

We will work with \key{Gibbs measures}, which is a class of measures on shifts of finite type with a distinguishing property. Therefore Chapter \ref{chap:sft} will provide the relevant background for shifts of finite type, and Chapter \ref{chap:gibbs} will define Gibbs measures and its main properties.

\section{Preliminaries}
Before we begin with the main background material, this section briefly introduces some concepts and definitions which will be used throughout this dissertation.

\begin{definition}
	Let $(X, d_X)$ and $(Y, d_Y)$ be metric spaces. A function $f : X \to Y$ is a \key{Lipschitz function} if there exists a constant $K > 0$ such that
	\[
		d_Y(f(x), f(y)) \leq Kd_X(x, y)
	\]
	for all $x, y \in X$. If this is the case, we say that $f$ is a Lipschitz function with \key{Lipschitz constant} $K$.~\cite[p154]{searcoid:metric-spaces}
\end{definition}

\begin{definition}
	A transformation $T : (X_1, \B_1, \mu_1) \to (X_2, \B_2, \mu_2)$ is \key{measure-preserving} if:
	\begin{enumerate}
		\item $T$ is measurable, i.e. if $B_2 \in \B_2$, then $T^{-1}{B_2} \in \B_1$, and
		\item $\mu_1(T^{-1}{B_2}) = \mu_2(B_2)$ for all $B_2 \in \B_2$.
	\end{enumerate}
	This agrees with our usual definition when $X_1 = X_2$.
\end{definition}

\begin{definition}
	Let $X$ be a compact metric space with Borel $\sigma$-algebra $\B$. We let $M(X)$ denote the set of all probability measures on $(X, \B)$.
	
	Let $T : X \to X$ be a continuous mapping on $X$. We let $M(X, T)$ denote the set of $T$-invariant probability measures on $(X, B)$.
\end{definition}

\begin{definition}
	The \key{symmetric difference} of two sets $A, B$ is defined $(A \setminus B) \cup (B \setminus A)$. We will write this as
	\[
		A \symdiff B := (A \setminus B) \cup (B \setminus A).
	\]
\end{definition}
%\chapter{Background}
%\section{Preliminaries}
%This section briefly introduces some concepts which will be used throughout this dissertation.
%
%\begin{definition}
%	Let $(X, d_X)$ and $(Y, d_Y)$ be metric spaces. A function $f : X \to Y$ is a \key{Lipschitz function} if there exists a constant $K > 0$ such that
%	\[
%		d_Y(f(x), f(y)) \leq Kd_X(x, y)
%	\]
%	for all $x, y \in X$. We say that $f$ is a Lipschitz function with \key{Lipschitz constant} $K$.~\cite[p154]{searcoid:metric-spaces}
%%	
%%	More generally, we say that $f$ is \key{H\"older continuous} if there exists some constants $K > 0$ and $\alpha \in (0, 1]$ such that
%%	\[
%%	d_Y(f(x), f(y)) \leq K(d_X(x, y))^\alpha
%%	\]
%%	for all $x, y \in X$. In this case, we say that $f$ is H\"older continuous with \key{H\"older exponent} $\alpha$ and \key{H\"older constant} $K$.~\cite[p143]{brin-stuck:dynsys}
%\end{definition}
%
%\begin{definition}
%	A transformation $T : (X_1, \B_1, \mu_1) \to (X_2, \B_2, \mu_2)$ is \key{measure-preserving} if:
%	\begin{enumerate}
%		\item $T$ is measurable, i.e. if $B_2 \in \B_2$, then $T^{-1}{B_2} \in \B_1$, and
%		\item $\mu_1(T^{-1}{B_2}) = \mu_2(B_2)$ for all $B_2 \in \B_2$.
%	\end{enumerate}
%	This agrees with our usual definition when $X_1 = X_2$.
%\end{definition}
%
%\begin{definition}
%	Let $X$ be a compact metric space with Borel $\sigma$-algebra $\B$. We let $M(X)$ denote the set of all probability measures on $(X, \B)$.
%	
%	Let $T : X \to X$ be a continuous mapping on $X$. We let $M(X, T)$ denote the set of $T$-invariant probability measures on $(X, B)$.
%\end{definition}
%
%\begin{definition}
%	The \key{symmetric difference} of two sets $A, B$ is defined $(A \setminus B) \cup (B \setminus A)$. We will write this as
%	\begin{equation*}
%		A \symdiff B := (A \setminus B) \cup (B \setminus A).
%	\end{equation*}
%\end{definition}

\chapter{Shifts of finite type}
\section{The basics}
Let $A$ be a $k \times k$ matrix with entries in $\{0, 1\}$. A \key{(two-sided) shift of finite type} $\Sigma_A$ is defined by
\[
	\Sigma_A := \{(x_j)_{j = -\infty}^\infty \mid A_{x_j, x_{j + 1}} = 1,\ j \in \integers\}.
\]
Similarly, a \key{(one-sided) shift of finite type} $\Sigma_A^+$ is defined by
\[
	\Sigma_A^+ := \{(x_j)_{j = 0}^\infty \mid A_{x_j, x_{j + 1}} = 1,\ j \in \naturals_0\}.
\]

Let $x = (x_j)_{j = -\infty}^\infty \in \Sigma_A$. We define the \key{(two-sided, left) shift map} $\sigma : \Sigma_A \to \Sigma_A$ by
\[
	(\sigma(x))_j = x_{j + 1}.
\]
which shifts each coordinate of $x$ one position to the left.

Now let $x = (x_j)_{j = 0}^\infty \in \Sigma_A^+$. We similarly define the \key{(one-sided, left) shift map} $\sigma^+ : \Sigma_A^+ \to \Sigma_A^+$ by
\[
	(\sigma^+(x))_j = x_{j + 1}.
\]
As with the one-sided case, this shifts the coordinates of $x$ one position to the left but also deletes the first coordinate $x_0$. It is clear that $\sigma^+$ is not invertible, whereas $\sigma$ is invertible.

To avoid excessive use of subscripts and superscripts, we will often write $\sigma$ for both the one-sided and two-sided shift maps. It should be clear from the context which of these maps $\sigma$ denotes.

If $x = (x_j)_{j = -\infty}^\infty \in \Sigma_A$, then we call $(x_j)_{j = -\infty}^0$ the \key{past}, $x_0$ the \key{present}, and $(x_j)_{j = 0}^\infty$ the \key{future}.

\subsection{Irreducibility, aperiodicity and cylinders}
Let $A$ be a $k \times k$ matrix with entries in $\{0, 1\}$, and let $\Sigma_A$ (or $\Sigma_A^+$) be the associated shift of finite type. We may consider $A$ to be the adjacency matrix of a directed graph $G_A$ with $k$ vertices.

We say that $A$ is \key{irreducible} if, for each $i, j \in \{1, \dots, k\}$, there exists $n = n(i, j) > 0$ such that $(A^n)_{i, j} > 0$. Alternatively, $A$ is irreducible if there exists an edge-path between any two vertices in the corresponding graph $G_A$. In this case, we say that the shift of finite type $\Sigma_A$ (or $\Sigma_A^+$) is irreducible.

If there exists $n > 0$ such that $(A^n)_{i, j} > 0$ for all $i, j \in \{1, \dots, k\}$, then we say that $A$ is \key{aperiodic}. That is, $A$ is aperiodic if all edge-paths between any two vertices in $G_A$ can be chosen to be of the same length. As before, this means that $\Sigma_A$ (or $\Sigma_A^+$) is aperiodic.

A \key{cylinder} $C$ on $\Sigma_A$ is defined
\[
	C = [y_{-m}, \dots, y_{-1}, y_0, y_1, \dots, y_n]_{-m, n} := \{(x_j)_{j = -\infty}^\infty \in \Sigma \mid x_j = y_j \text{ for } -m \leq j \leq n\}.
\]
Similarly, a cylinder $C^+$ on $\Sigma_A^+$ is given by
\[
	C^+ = [y_0, y_1, \dots, y_{n - 1}, y_n]_{0, n} := \{(x_j)_{j = 0}^\infty \in \Sigma \mid x_j = y_j \text{ for } 0 \leq j \leq n\}.
\]
In other words, a cylinder is the set of all sequences which agree in the given positions.

\section{Function spaces for shifts of finite type}
\emph{This section predominantly follows \cite[Chapter 1]{parry-pollicott:zeta-fns-periodic-orbits}.}

Let $\Sigma_A$ and $\Sigma_A^+$ be two-sided and one-sided shifts of finite type, respectively, and let $\theta \in (0, 1)$ be fixed.

\subsection{Metrics for shifts of finite type}
Let $x = (x_j)_{j = -\infty}^\infty, y = (y_j)_{j = -\infty}^\infty \in \Sigma_A$. We define $n = n(x, y) \geq 0$ to be the largest integer such that $x_j = y_j$ for all $|j| < n$, but $x_n \neq y_n$ or $x_{-n} \neq y_{-n}$. If $x_j = y_j$ for all $j \in \integers$ then we define $n = +\infty$.

We define the map $d_\theta : \Sigma_A \times \Sigma_A \to \reals^+$ by
\[
	d_\theta(x, y) =
	\begin{cases}
		\theta^n, & \text{if } x \neq y; \\
		0, & \text{if } x = y.
	\end{cases}
\]
It can be shown that $d_\theta$ is a \key{metric} on $\Sigma_A$, so sequences in $\Sigma_A$ are `close' if they agree for a large number of leading coordinates.

Similarly, if $x = (x_j)_{j = 0}^\infty, y = (y_j)_{j = 0}^\infty \in \Sigma_A^+$, then $n = n(x, y) \geq 0$ is defined to be the largest integer such that $x_j = y_j$ for all $0 \leq j < n$ but $x_n \neq y_n$. We define the map $d_\theta : \Sigma_A^+ \times \Sigma_A^+ \to \reals^+$ in the same way as the two-sided case, and it can be shown that this also defines a metric on $\Sigma_A^+$.

\subsection{The space of Lipschitz functions}
\begin{definition}
	Let $f : \Sigma_A \to \complex$ be a continuous function and let $n \geq 0$. We define the \key{$n$-th variation of $f$} by
	\[
		\var_n(f) = \sup\{|f(x) - f(y)| \mid x, y \in \Sigma_A,\ x_j = y_j \text{ for } |j| < n\}.
	\]
	
	Similarly, if $g : \Sigma_A^+ \to \complex$ is a continuous function, then the \key{$n$-th variation of $g$} is given by
	\[
	\var_n(g) = \sup\{|g(x) - g(y)| \mid x, y \in \Sigma_A^+,\ x_j = y_j \text{ for } 0 \leq j < n\}.
	\]
	That is, $\var_n(g)$ indicates how much $g$ varies on cylinders of length $n$.~\cite[Lecture 8]{magic-ergodic}
\end{definition}

It is easy to see that $\var_n(f) \leq K\theta^n$ for all $n \geq 0$ if and only if $|f(x) - f(y)| \leq Kd_\theta(x, y)$, i.e. $f$ is a Lipschitz function. This allows the following definition to be given in terms of $n$-th variations of continuous functions.

\begin{definition}
	Define
	\[
		F_\theta = F_\theta(\Sigma_A) = \{f \in C(\Sigma_A, \complex) \mid \var_n(f) \leq K\theta^n \text{ for all } n \geq 0, \text{ for some } K > 0\}
	\]
	to be the space of Lipschitz functions with respect to the metric $d_\theta$.
	
	We define $F_\theta^+ = F_\theta^+(\Sigma_A^+)$ in the same way, replacing $\Sigma_A$ with $\Sigma_A^+$.
\end{definition}

\begin{definition}
	Let $f \in F_\theta$ (or $F_\theta^+)$. Define
	\[
		|f|_\theta = \sup_{n \geq 0}\left\{\frac{\var_n(f)}{\theta^n}\right\}
	\]
	to be the \key{least Lipschitz constant} of $f$. In other words, $|f|_\theta$ is the smallest $K > 0$ such that $\var_n(f) \leq K\theta^n$ for all $n \geq 0$.
	
	We can then define a \key{norm} on $F_\theta$ (or $F_\theta^+$) by
	\[
		\|f\|_\theta = |f|_\infty+ |f|_\theta,
	\]
	where $|f|_\infty = \sup_{x \in \Sigma}\{|f(x)|\}$.
\end{definition}

\begin{proposition}
	The spaces $(F_\theta, \|\cdot\|_\theta)$ and $(F_\theta^+, \|\cdot\|_\theta)$ are Banach spaces.
\end{proposition}

\begin{definition}
	Let $f, g \in F_\theta$ (or $F_\theta^+$). We say $f$ and $g$ are \key{cohomologous} if there exists a continuous function $h$ such that $f = g + h \circ \sigma - h$. In this case, we write $f \sim g$. (It is clear that $\sim$ is an equivalence relation.)
	
	If $f$ is cohomologous to $0$, then we say $f$ is a \key{coboundary}.
\end{definition}

\begin{proposition} \label{prop:pp-1-2}
	Suppose that $f \in F_\theta$. Then there exists $g, h \in F_{\theta^{1 / 2}}$ such that $f = g + h - h \circ \sigma$, where $g(x) = g(y)$ if $x_j = y_j$ for all $j \geq 0$. In other words, the value of $g(x)$ is determined only by the future coordinates of $x$.
	\begin{proof}
		For each $j = 1, \dots, k$ we choose a sequence $(a_n^{(j)})_{n = -\infty}^0$ from the past such that $a_0^{(j)} = j$. Now we define a function $\phi : \Sigma_A \to \Sigma_A : x \mapsto x'$, where
		\[
			(x')_n =
			\begin{cases}
				x_n, & \text{if } n \geq 0; \\
				a_n^{(x_0)}, & \text{if } n \leq 0.
			\end{cases}
		\]
		So $\phi$ fixes the future coordinates of $x$, but replaces the past with $(a_n^{(x_0)})_{n = -\infty}^0$.
		
		Let $h(x) := \sum_{n = 0}^\infty{(f(\sigma^n{x}) - f(\sigma^n \phi{x}))}$. Note that $h$ converges because for all $n \geq 0$,
		\[
			|f(\sigma^n{x}) - f(\sigma^n \phi{x})| \leq \var_n(f) \leq |f|_\theta \theta^n \to 0,
		\]
		as $n \to +\infty$. We have
		\begin{align*}
			h(x) - h(\sigma{x}) &= \sum_{n = 0}^\infty{\left(f(\sigma^n{x}) - f(\sigma^n \phi{x})\right)} - \sum_{n = 0}^\infty{\left(f(\sigma^{n + 1}{x}) - f(\sigma^n \phi \sigma{x})\right)} \\
				&= f(x) - \left(f(\phi{x}) + \sum_{n = 0}^\infty{(f(\sigma^{n + 1} \phi{x}) - f(\sigma^n \phi \sigma{x}))}\right) \\
				&= f(x) - g(x),
		\end{align*}
		where $g(x) := f(\phi{x}) + \sum_{n = 0}^\infty{\left(f(\sigma^{n + 1} \phi{x}) - f(\sigma^n \phi \sigma{x})\right)}$. Since all terms in $g$ contain $\phi$, we see that $g$ depends only on its future coordinates.
		
		It remains to show that $h \in F_{\theta^{1 / 2}}$ and we will immediately have $g \in F_{\theta^{1 / 2}}$. It is sufficient to show that $\var_{2N}(h) \leq K\theta^N$ for all $N \geq 0$, for some constant $K > 0$, because this gives that
		\[
			\var_{2N + 1}(h) \leq K\theta^N = \frac{K}{\theta^{1 / 2}}(\theta^{1 / 2})^{2N + 1}.
		\]
		
		Let $x, y \in \Sigma_A$ be such that $x_j = y_j$ for $|j| \leq 2N$. Then for all $n = 0, \dots, N$, we have
		\[
			|f(\sigma^n{x}) - f(\sigma^n{y})| \leq |f|_\theta \theta^{2N - n} \quad \text{and} \quad |f(\sigma^n \phi{x}) - f(\sigma^n \phi{y})| \leq |f|_\theta \theta^{2N - n}.
		\]
		By definition, for all $n \geq 0$ we have
		\[
			|f(\sigma^n{x}) - f(\sigma^n \phi{x})| \leq |f|_\theta \theta^n \quad \text{and} \quad |f(\sigma^n {y}) - f(\sigma^n \phi{y})| \leq |f|_\theta \theta^n.
		\]
		Hence
		\begin{align*}
			|h(x) - h(y)| &= \left|\sum_{n = 0}^\infty{\left(f(\sigma^n{x}) - f(\sigma^n \phi{x}) - f(\sigma^n{y}) + f(\sigma^n \phi{y})\right)}\right| \\
				&\leq \sum_{n = 0}^N{|f(\sigma^n{x}) - f(\sigma^n{y})| + | f(\sigma^n \phi{x}) -  f(\sigma^n \phi{y})|} \\
				&\quad + \sum_{n = N + 1}^\infty{|f(\sigma^n{x}) - f(\sigma^n \phi{x})| + | f(\sigma^n \phi{y}) -  f(\sigma^n {y})|} \\
				&\leq 2|f|_\theta \sum_{n = 0}^N{\theta^{2N - n}} + 2|f|_\theta \sum_{n = N + 1}^\infty{\theta^n} \\
				&= 2|f|_\theta \theta^{2N} \left(\frac{\theta^{-N - 1} - 1}{\theta^{-1} - 1}\right) + |f|_\theta \frac{\theta^{N + 1}}{1 - \theta} \\
				&\leq 4|f|_\theta \frac{\theta^N}{1 - \theta}.
		\end{align*}
		Therefore
		\[
			\var_{2N}(h) \leq \left(\frac{4|f|_\theta}{1 - \theta}\right)\theta^N
		\]
		and so $h \in F_{\theta^{1 / 2}}$.
	\end{proof}
\end{proposition}

The above result allows us to write $f \in F_\theta$ using $g \in F_{\theta^{1 / 2}}^+$. We can then apply results to $g$ which hold for $F_\theta^+$ but not necessarily for $F_\theta$. We will encounter such uses for this proposition in due course.

\chapter{Gibbs measures} \label{chap:gibbs}
\section{Overview}
This chapter introduces Gibbs measures, a class of probability measures on shifts of finite type. Gibbs measures have properties which we make use of when we discuss \cite{chazottes-maldonado:cbfee} in Chapter \ref{chap:concentration-bounds}.

Throughout this chapter we will let $\Sigma = \Sigma_A^+$, where $A$ is an irreducible $k \times k$ matrix.

\section{The Ruelle operator}
\emph{This section predominantly follows material from \cite[Chapter 2]{parry-pollicott:zeta-fns-periodic-orbits}.}

We briefly introduce a theorem due to Ruelle. Later, it will be apparent that this theorem is useful for constructing Gibbs measures.

\begin{definition}
	Let $f \in F_\theta^+$. The \key{Ruelle operator} (or \key{transfer operator}) $L_f : F_\theta^+ \to F_\theta^+$ (or, more generally, $L_f : C(\Sigma, \complex) \to C(\Sigma, \complex)$) is defined
	\[
	(L_f{w})(x) = \sum_{y \in \Sigma \midcolon \sigma{y} = x}{e^{f(y)} w(y)} = \sum_{j \midcolon A_{j, x_0} = 1}{e^{f(j, x_0, x_1, \dots)} w(j, x_0, x_1, \dots)},
	\]
	where $x = (x_j)_{j = 0}^\infty \in \Sigma$. This is a bounded linear operator.
	
	The $n$-th iterate of $L_f$ is given by
	\[
	(L_f^n{w})(x) = \sum_{y \in \Sigma \midcolon \sigma^n{y} = x}{e^{f^n(y)} w(y)}.
	\]
	
	If $f$ is also real-valued and $L_f{1} = 1$, then we say that $f$ or $L_f$ is \key{normalised}.
\end{definition}

%\begin{proposition}
%	Let $f \in F_\theta^+$ with $f = u + iv$, where $u, v \in F_\theta^+$ are real-valued functions. If $L_u$ is normalised, i.e. $L_u{1} = 1$, then for all $n \geq 0$,
%	\[
%	|L_f^n{w}|_\theta \leq K|w|_\infty + \theta^n |w|_\theta
%	\]
%	for all $w \in F_\theta^+$, where $K > 0$ is a constant depending only on $f$ and $\theta$.
%\end{proposition}

\begin{theorem}[Ruelle's Perron-Frobenius Theorem] \label{thm:rpf}
	Suppose that $\Sigma = \Sigma_A^+$ is an aperiodic shift of finite type and let $f \in F_\theta^+$ be a real-valued function. Then
	\begin{enumerate}
		\item There is a simple maximal eigenvalue $\lambda$ of $L_f : C(\Sigma, \reals) \to C(\Sigma, \reals)$ with a corresponding eigenfunction $h \in C(\Sigma_A^+, \reals)$, with $h > 0$. \label{rpf:1}
		\item The remainder of the spectrum of $L_f$ is contained in a disc of radius strictly less than $\lambda$. \label{rpf:2}
		\item There is a unique probability measure $\mu$ such that $L_f^*{\mu} = \lambda\mu$. That is,
		\[
		\int{L_f{w}\ d\mu} = \lambda \int{w\ d\mu},
		\]
		for all $w \in C(\Sigma, \reals)$. Additionally, if $h$ is the eigenfunction as in \ref{rpf:1} and $\int{h\ d\mu} = 1$, then the measure $\nu$ defined by $d\nu = h\ d\mu$ is a $\sigma$-invariant probability measure. \label{rpf:3}
		\item If $h$ is the eigenfunction as in \ref{rpf:1} and $\int{h\ d\mu} = 1$, then for all $w \in C(\Sigma, \reals)$,
		\[
		\frac{1}{\lambda^n}L_f^n{w} \to h \int{w\ d\mu}
		\]
		uniformly, as $n \to +\infty$. \label{rpf:4}
	\end{enumerate}
\end{theorem}

\section{Gibbs measures and basic results}
\emph{The remainder of this chapter predominantly follows \cite[Chapter 3]{parry-pollicott:zeta-fns-periodic-orbits}.}

\begin{definition}
	Let $\phi \in C(\Sigma, \reals)$ be a continuous function. Let $\mu$ be a probability measure such that there exists a constants $C = C(\phi) > 1$, $P = P(\phi) \in \reals$, such that
	\begin{equation}
		C^{-1} \leq \frac{\mu[x_0, \dots, x_n]}{\exp\left(-Pn + \sum_{j = 0}^{n - 1}{\phi(\sigma^j{x})} \right)} \leq C,
	\end{equation}
	for all $n \geq 0$ and for all $x \in \Sigma$. Then the measure $\mu$ is a \key{Gibbs measure for $\phi$} or a \key{Gibbs measure with potential $\phi$}. We will write $\mu_\phi = \mu$.
\end{definition}

\begin{remark}
	The Gibbs measure $\mu_\phi$ is not necessarily $\sigma$-invariant.
\end{remark}

Given any $\phi \in F_\theta$, it can be shown that there exists a Gibbs measure $\mu_\phi$ for $\phi$. We prove this in the following results.

\begin{proposition} \label{prop:pp-3-2}
	Suppose that $\phi \in F_\theta$ is a normalised, real-valued function. Then for all $x \in \Sigma$ we have
	\begin{equation}
		e^{-|\phi|_\theta \theta^n} \leq \frac{\mu[x_0, \dots, x_n]}{\mu[x_1, \dots, x_n]} e^{-\phi(x)} \leq e^{|\phi|_\theta \theta^n},
	\end{equation}
	where $L_\phi^*\mu = \mu$ is the unique measure in Ruelle's Perron-Frobenius Theorem \ref{rpf:3} (\thref{thm:rpf}) with $\lambda = 1$.
	\begin{proof}
		Let $w \in C(\Sigma, \reals)$. We have
		\begin{align*}
			\int{w \circ \sigma\ d\mu} &= \int L_\phi(w \circ \sigma)\ d\mu \\
				&= \int{\sum_{y \in \Sigma \midcolon \sigma{y} = x}{e^{\phi(y)} w \circ \sigma(y)}\ d\mu} \\
				&= \int{\sum_{y \in \Sigma \midcolon \sigma{y} = x}{e^{\phi(y)} w(x)}\ d\mu} \\
				&= \int{w L_\phi{1}\ d\mu} \\
				&= \int{w\ d\mu}.
		\end{align*}
		By a basic result~\cite[Lemma 11.3]{ergodic-lectures} in ergodic theory, it follows that $\mu$ is a $\sigma$-invariant measure. Hence
		\begin{align}
			\mu[x_1, \dots, x_n] &= \mu\left(\bigsqcup_{x_0 \midcolon A_{x_0, x_1} = 1}{[x_0, x_1, \dots, x_n]}\right) \nonumber \\
				&= \int{\sum_{x_0 \midcolon A_{x_0, x_1} = 1} \chi_{[x_0, x_1, \dots, x_n]}(z)\ d\mu} \nonumber \\
				&= \int{\sum_{y \midcolon \sigma{y} = z} e^{\phi(y)} \chi_{[x_0, \dots, x_n]}(y) e^{-\phi(y)}\ d\mu} & (\text{since } \chi_B(y) = 0 \text{ or } 1) \nonumber \\
				&= \int{L_\phi\chi_{[x_0, x_1, \dots, x_n]}(z) e^{-\phi(z)}\ d\mu} \nonumber \\
				&= \int{\left(\chi_{[x_0, \dots, x_n]} e^{-\phi}\right)(z)\ d\mu} \nonumber \\
				&= \int_{[x_0, \dots, x_n]}{e^{-\phi}\ d\mu}. \label{fml:prop-3-2-mu-e}
		\end{align}
		Now let $z, w \in [x_0, \dots, x_n]$. Since $\phi \in F_\theta \subset C(\Sigma, \reals)$, we have that $\var_n(\phi) \leq |\phi|_\theta \theta^n$ and hence
		\[
			e^{-|\phi|_\theta \theta^n} \leq e^{\phi(z) - \phi(w)} \leq e^{|\phi|_\theta \theta^n}.
		\]
		Then by \eqref{fml:prop-3-2-mu-e} we have
		\[
			\mu[x_0, \dots, x_n] e^{-|\phi|_\theta \theta^n} \leq \mu[x_1, \dots, x_n] e^{\phi(x)} \leq \mu[x_0, \dots, x_n] e^{|\phi|_\theta \theta^n},
		\]
		and so
		\[
			e^{-|\phi|_\theta \theta^n} \leq \frac{\mu[x_0, \dots, x_n]}{\mu[x_1, \dots, x_n]} e^{-\phi(x)} \leq e^{|\phi|_\theta \theta^n}.
		\]
	\end{proof}
\end{proposition}

\begin{corollary} \label{cor:pp-3-2-1}
	The measure $\mu$ in \thref{prop:pp-3-2} is a Gibbs measure for $\phi$ with $P = 0$.
	\begin{proof}
		We apply \thref{prop:pp-3-2} to $\phi$, $\phi \circ \sigma$, \dots, $\phi \circ \sigma^n$ to get the sequence of inequalities
		\[
			\left.
			\begin{matrix}
				e^{-|\phi|_\theta \theta^n} &\leq& \dfrac{\mu[x_0, \dots, x_n]}{\mu[x_1, \dots, x_n]} e^{-\phi(x)} &\leq& e^{|\phi|_\theta \theta^n} \\
				e^{-|\phi|_\theta \theta^{n - 1}} &\leq& \dfrac{\mu[x_1, \dots, x_n]}{\mu[x_2, \dots, x_n]} e^{-\phi(\sigma{x})} &\leq& e^{|\phi|_\theta \theta^{n - 1}} \\
				\vdots & & \vdots & & \vdots \\
				e^{-|\phi|_\theta} &\leq& \mu[x_n] e^{-\phi(\sigma^n{x})} &\leq& e^{|\phi|_\theta}
			\end{matrix}
			\right\}
		\]
		for all $x \in \Sigma$. Multiplying these together, we get
		\[
			\exp\left(-\sum_{j = 0}^n{|\phi|_\theta \theta^j}\right) \leq \frac{\mu[x_0, \dots, x_n]}{\exp\left(-\sum_{j = 0}^n{\phi(\sigma^j{x})}\right)} \leq \exp\left(\sum_{j = 0}^n{|\phi|_\theta \theta^j}\right),
		\]
		and so
		\[
			\exp\left(-\frac{|\phi|_\theta}{1 - \theta}\right) \leq \frac{\mu[x_0, \dots, x_n]}{\exp\left(-\sum_{j = 0}^n{\phi(\sigma^j{x})}\right)} \leq \exp\left(\frac{|\phi|_\theta}{1 - \theta}\right).
		\]
		Hence $\mu$ is a Gibbs measure for $\phi$ with $P = 0$.
	\end{proof}
\end{corollary}

We can generalise these results for Lipschitz functions $\phi \in F_\theta$ which are not normalised.

\begin{corollary}
	Suppose that $\phi \in F_\theta$ is a real-valued function. Let $\mu$ be the unique measure where $L_\phi^*\mu = \lambda\mu$, where $\lambda > 0$ is the maximal eigenvalue in Ruelle's Perron-Frobenius Theorem \ref{rpf:3}. Then $\mu$ is a Gibbs measure for $\phi$ with $P = \log{\lambda}$.
	\begin{proof}
		First note that $\psi := \phi - \log{(h \circ \sigma)} + \log{h} - \log{\lambda}$ is normalised, where $h > 0$ is the eigenfunction corresponding to $\lambda$ in \thref{thm:rpf}. We have
		\begin{align*}
			\exp\left(\sum_{j = 0}^{n - 1}{\psi(\sigma^j{x})}\right) &= \exp\left(\sum_{j = 0}^{n - 1}{\phi(\sigma^j{x}) - \log{h(\sigma^{j + 1}{x})} + \log{h(\sigma^j{x})} - \log{\lambda}}\right) \\
				&= \exp\left(-n\log{\lambda} + \sum_{j = 0}^{n - 1}{\phi(\sigma^j{x})}\right) \frac{h(x)}{h(\sigma^n{x})}.
		\end{align*}
		We apply \thref{cor:pp-3-2-1} to $\psi = \phi - \log{(h \circ \sigma)} + \log{h} - \log{\lambda}$ to get
		\[
			C_0^{-1} \leq \frac{\mu[x_0, \dots, x_n]}{\exp\left(-n\log{\lambda} + \sum_{j = 0}^{n - 1}{\phi(\sigma^j{x})}\right)} \leq C_0,
		\]
		for some constant $C_0 > 1$. Hence $\mu$ is a Gibbs measure for $\phi$ with $P = \log{\lambda}$.
	\end{proof}
\end{corollary}

In view of this, if $\phi \in F_\theta$ has Gibbs measure $\mu_\phi$ with $P(\phi) > 0$, then the Gibbs measure for $\phi - P(\phi)$ gives $P(\phi - P(\phi)) = 0$. In both cases we have the same Gibbs measure $\mu_\phi$.

\section{The variational principle and pressure}
There is a characteristic which sets Gibbs measures apart from other $\sigma$-invariant probability measures. To prove this property, we need some concepts and results.

Let $\mu$ be a $\sigma$-invariant probability measure on $\Sigma$. For $n \geq 0$ we have that $\mu[x_0, \dots, x_{n - 1}] > 0$ for $\mu$-almost every $x \in \Sigma$. For $j = 1, \dots, k$ we define
\begin{align*}
	\mu_n[j \mid \sigma^{-1}{x}] &:= \frac{\mu[j, x_0, \dots, x_{n - 1}]}{\mu[x_0, \dots, x_{n - 1}]} \\
		&= \frac{\mu([j] \cap \sigma^{-1}[x_0, \dots, x_{n - 1}])}{\mu(\sigma^{-1}[x_0, \dots, x_{n - 1}])} \\
		&= \mu\left([j] \midmid \bigjoin_{r = 0}^{n - 1}{\sigma^{-r}{\alpha}}\right)(x),
\end{align*}
where $\alpha = \{[j] \mid j \in \{1, \dots, k\}\}$ is the partition of $\Sigma$ by cylinders of length 1. It is clear that this is a probability distribution for $\mu$-almost every $x$.

\begin{proposition}\label{prop:mu-sq-bkt-pd}
	For $n \geq 0$, we have that
	\[
		\mu[j \mid \sigma^{-1}{x}] := \lim_{n \to +\infty}{\mu_n[j \mid \sigma^{-1}{x}]}
	\]
	is a well-defined probability distribution on $\{1, \dots, k\}$ for $\mu$-almost every $x \in \Sigma$.
	\begin{proof}
		First note that $\bigjoin_{r = 0}^{n - 1}{\sigma^{-r}{\alpha}} \to \B$, as $n \to +\infty$. By definition we also have $\mu([j] \mid \bigjoin_{r = 0}^{n - 1}{\sigma^{-r}{\alpha}}) = E_\mu(\chi_{[j]} \mid \bigjoin_{r = 0}^{n - 1}{\sigma^{-r}{\alpha}})$. Since $\chi_{[j]} \in L^1(\Sigma, \B, \mu)$, we may apply the Increasing Martingale Theorem so that
		\[
			\lim_{n \to +\infty}{\mu\left([j] \midmid \bigjoin_{r = 0}^{n - 1}{\sigma^{-r}{\alpha}}\right)} = \mu([j] \mid \B),
		\]
		for $\mu$-almost every $x$. Hence
		\[
			\mu[j \mid \sigma^{-1}{x}] = \mu([j] \mid \B)(x)
		\]
		for $\mu$-almost every $x$, i.e. $\mu[j \mid \sigma^{-1}{x}]$ is a well-defined probability distribution on $\{1, \dots, k\}$ for $\mu$-almost every $x$.
	\end{proof}
\end{proposition}

\begin{lemma} \label{lem:pp-prop-p36}
	We have
	\begin{equation}
		\sum_{j = 1}^k{\int{\psi(j, x_0, x_1, \dots) \mu[j \mid \sigma^{-1}{x}] \ d\mu}} = \int{\psi\ d\mu},
	\end{equation}
	for all $\psi \in C(\Sigma, \reals)$.
	\begin{proof}
		Let $\psi = \chi_{[r_0, \dots, r_t]}$, where $t \in \naturals_0$. We have
		\begin{align*}
			\sum_{j = 1}^k{\int{\psi(j, x_0, x_1, \dots) \mu[j \mid \sigma^{-1}{x}] \ d\mu}} &= \lim_{n \to +\infty}{\sum_{j = 1}^k{\int{\chi_{[r_0, \dots, r_t]} \mu_n[j \mid \sigma^{-1}{x}] \ d\mu}}} \\
				&= \mu[r_0, \dots, r_t] \sum_{j = 1}^k{\frac{\mu[j, x_0, \dots, x_{n - 1}]}{\mu[x_0, \dots, x_{n - 1}]}} \\
				&= \int{\psi\ d\mu}.
		\end{align*}
		So the result holds for characteristic functions. We then apply the definitions from measure theory to show that the result holds for $\psi \in C(\Sigma, \reals)$.
	\end{proof}
\end{lemma}

\begin{lemma} \label{lem:pp-prop-3-4}
	Suppose that $\phi \in F_\theta$ is a real-valued function and that $L_\phi$ is normalised. Let $\mu$ be a probability measure such that $L_\phi^*{\mu} = \mu$. Then for any $\sigma$-invariant probability measure $m$, we have
	\[
		h_m(\sigma) + \int{\phi\ dm} \leq 0,
	\]
	with equality if and only if $m = \mu$.
	\begin{proof}
		Let $m \in M(\Sigma, \sigma)$ be a $\sigma$-invariant probability measure. We define a probability distribution on $\{1, \dots, k\}$ by $\mu[j \mid \sigma^{-1}{x}]$ as in \thref{prop:mu-sq-bkt-pd}. If $m = \mu$, then $L_\phi^*{m} = m$ and so we have the probability distribution
		\[
			m[j \mid \sigma^{-1}{x}] = \exp(\phi(j, x_0, x_1, \dots))
		\]
		for all $x \in \Sigma$. We apply \thref{lem:pp-3-3} so that
		\[
			-\sum_{j = 1}^k{m[j \mid \sigma^{-1}{x}] \log{m[j \mid \sigma^{-1}{x}]}} + \sum_{j = 1}^k{m[j \mid \sigma^{-1}{x}] \phi(j, x_0, x_1, \dots)} \leq 0,
		\]
		for $m$-almost every $x$, with equality if and only if $m[j \mid \sigma^{-1}{x}] = \phi(j, x_0, x_1, \dots)$ for all $j = 1, \dots, k$.
		
		We integrate with respect to $m$ and apply \thref{lem:pp-prop-p36} to get
		\[
			h_m(\sigma) + \sum_{j = 1}^k{\int{m[j \mid \sigma^{-1}{x}] \phi(j, x_0, x_1, \dots)\ dm}} = h_m(\sigma) + \int{\phi\ dm} \leq 0,
		\]
		with equality if and only if $m[j \mid \sigma^{-1}{x}] = \phi(j, x_0, x_1, \dots)$ for $m$-almost every $x$. By \thref{lem:pp-prop-p36}, this equality condition is equivalent to
		\begin{align*}
			\int{\sum_{j = 1}^k{m[j \mid \sigma^{-1}{x}] \psi(j, x_0, x_1, \dots)}\ dm} &= \int{\sum_{j = 1}^k{\phi(j, x_0, x_1, \dots) \psi(j, x_0, x_1, \dots)}\ dm} \\
				&= \int{\psi\ dm}
		\end{align*}
		for all $\psi \in C(\Sigma, \reals)$. In other words, $\int{L_\phi{\psi}\ dm} = \int{g\ dm}$ for all $\psi$, and so we have $L_\phi^*{m} = m$. By Ruelle's Perron-Frobenius Theorem \ref{rpf:3}, $\mu$ is the unique $\sigma$-invariant probability measure such that $L_\phi^*{\mu} = \mu$, so we have
		\[
			h_m(\sigma) + \int{\phi\ dm} \leq 0,
		\]
		with equality if and only if $m = \mu$.
	\end{proof}
\end{lemma}

It can be shown that similar results hold for 2-sided shifts of finite type, and also for $\phi \in F_\theta$ where $L_\phi$ is not necessarily normalised.

The following result shows one of the main distinguishing characteristics of Gibbs measures.

\begin{theorem}[Variational Principle] \label{thm:variational-principle}
	Suppose that $\phi \in F_\theta$ (or $F_\theta^+$). The Gibbs measure $\mu_\phi$ is the unique $\sigma$-invariant probability measure such that
	\[
		h_m(\sigma) + \int{\phi\ dm} \leq h_{\mu_\phi}(\sigma) + \int{\phi\ d\mu_\phi}
	\]
	for all $m \in M(\Sigma, \sigma)$, with equality if and only if $m = \mu_\phi$.
	\begin{proof}
		Let $\phi \in F_\theta$. By \thref{prop:pp-1-2}, there exists $g \in F_{\theta^{1 / 2}}^+$, $u \in F_{\theta^{1 / 2}}$ such that $\phi = g + (u \circ \sigma) - u$. By Ruelle's Perron-Frobenius Theorem, we can write $g = \log(h \circ \sigma) - \log{h} + \log{\lambda} + k$, for some $k \in F_{\theta^{1/2}}^+$ such that $L_k^*{\mu_\phi} = \mu_\phi$ and $L_k$ is normalised.
		
		Let $m$ be a $\sigma$-invariant probability measure. By \thref{lem:pp-prop-3-4} we have
		\[
			h_m(\sigma) + \int{k\ dm} \leq h_{\mu_\phi}(\sigma) + \int{k\ d\mu_\phi} = 0.
		\]
		Substituting in $k = \phi - (u \circ \sigma) + u - \log(h \circ \sigma) + \log{h} - \log{\lambda}$ and cancelling terms, we get
		\[
			h_m(\sigma) + \int{\phi\ dm} \leq h_{\mu_\phi}(\sigma) + \int{\phi\ d\mu_\phi},
		\]
		with equality if and only if $m = \mu_\phi$.
	\end{proof}
\end{theorem}

From the proof of \thref{thm:variational-principle}, we see that
\[
	P(\phi) = \sup_{m \in M(\Sigma, \sigma)}\left\{h_m(\sigma) + \int{\phi\ dm}\right\} = h_{\mu_\phi}(\sigma) + \int{\phi\ d\mu_\phi}.
\]
So $P(\phi) = \log{\lambda}$, where $\lambda$ is the maximal positive eigenvalue for $L_{\phi'}$, where $\phi$ is cohomologous to $\phi' \in F_{\theta^{1 / 2}}^+$.

\begin{definition}
	We call
	\[
		P(\phi) := \sup_{m \in M(\Sigma, \sigma)}\left\{h_m(\sigma) + \int{\phi\ dm}\right\}
	\]
	the \key{pressure} of $\phi$.
	
	If a measure a $\sigma$-invariant probability measure $m$ attains this supremum, i.e. $P(\phi) = h_m(\sigma) + \int{\phi\ dm}$, then we say that $m$ is an \key{equilibrium state}.
\end{definition}

The Variational Principle gives that if we have $\phi \in F_\theta$, then the equilibrium state is unique and we can also define the pressure of $\phi$ by $P(\phi) = \log{\lambda}$.

\section{Gibbs measures are weak Bernoulli}
We now describe a particular property of Gibbs measures which we will use in Subsection \ref{ssec:hitting-times}. The following definitions and results follow \cite[Section 1.E]{bowen:equilibrium}.

\begin{definition}
	Let $\beta, \gamma$ be two finite partitions of a measure space $(X, \B, \mu)$ and let $\varepsilon > 0$ be given. We say that $\beta$ and $\gamma$ are \key{$\varepsilon$-independent} if
	\[
		\sum_{B \in \beta,\ C \in \gamma}{|\mu(B \cap C) - \mu(B)\mu(C)|} < \varepsilon.
	\]
\end{definition}

\begin{definition}
	Let $\xi = \left\{[j] \mid j \in \{1, \dots, k\}\right\}$ be the partition of $(\Sigma, \B, \mu)$ by cylinders of length 1. We say that $\xi$ is \key{weak Bernoulli} (for $\sigma$ and $\mu$) if for all $\varepsilon > 0$ there exists $N(\varepsilon) > 0$ such that for all $n \geq 1$, then the partitions
	\[
		\beta = \bigjoin_{j = 0}^n{\sigma^{-j}(\xi)} \quad \text{and} \quad \gamma = \bigjoin_{j = t}^{t + r}{\sigma^{-j}(\xi)}
	\]
	are $\varepsilon$-independent for all $r \geq 0$ and for all $t \geq n + N(\varepsilon)$.
\end{definition}

Before we state the main result, we need an auxiliary lemma.

\begin{lemma}\label{bowen:lem-1-12}
	Let $r \geq 0$, $f \in C(\Sigma, \reals)$ and $\var_r(f) = 0$. Let
	\begin{align*}
		F \in \{g \in C(\Sigma, \reals) \mid g & \geq 0,\ \nu(g) = 1,\ g(x) \leq B_m g(x') \text{ whenever } x_j = x'_j \text{ for all } j = 0, \dots, m\},
	\end{align*}
	where $B_m := \exp\left(\sum_{k = m + 1}^\infty{2b\alpha^k}\right)$, where $b > 0$, $\alpha \in (0, 1)$ are any pair of constants which satisfy $\var_k(\phi) \leq b\alpha^k$ for all $k > 0$.
	
	Then for any $n \geq 0$ we have
	\[
		\|\lambda^{-n - r}L_\phi^{n + r}(fF) - \nu(fF)h\| \leq M\nu(|fF|)\rho^n,
	\]
	where $\nu$, $\lambda$, $h$ are as in the Ruelle's Perron Frobenius Theorem, and $M > 0$, $\rho \in (0, 1)$ are constants.
\end{lemma}

\begin{theorem}\label{thm:gibbs-is-weak-bernoulli}
	Let $\xi = \left\{[j] \mid j \in \{1, \dots, k\}\right\}$ be the partition of $(\Sigma, \B, \mu)$ by cylinders of length 1. Then $\xi$ is weak Bernoulli for the Gibbs measure $\mu_\phi$.
	\begin{proof}
		Let $\varepsilon > 0$ be given. Suppose that $\phi \in C(\Sigma, \reals)$, $n \geq 1$ and $t \geq n + N(\varepsilon)$, for some $N(\varepsilon)$. Let $\beta, \gamma$ be partitions of $\Sigma$ defined by
		\[
		\beta = \bigjoin_{j = 0}^n{\sigma^{-j}(\xi)} \quad \text{and} \quad \gamma = \bigjoin_{j = t}^{t + r}{\sigma^{-j}(\xi)}.
		\]
		For all $B \in \beta$ we clearly have $\chi_B \in C(\Sigma, \reals)$ and $\var_r(\chi_B) = 0$ for all $r \geq 0$.
		
		Now consider $C \in \gamma$. Since $t \geq n$, we know that $B$ consists of cylinders of lengths strictly less than the lengths of cylinders in $C$, and therefore $\sigma^{-t}C$. It follows that the intersection $B \cap C$ depends only on $B$, and hence
		\[
			\mu_\phi(B \cap C) = \mu_\phi(B \cap \sigma^{-t}{C}).
		\]
		We then have, where $\nu$, $h$ and $\lambda$ are as in the Ruelle's Perron Frobenius Theorem,
		\begin{align*}
			\mu_\phi(B \cap C) &= \mu_\phi(B \cap \sigma^{-t}C) \\
				&= \mu_\phi(\chi_B \cdot \chi_{\sigma^{-t}}{C}) \\
				&= \mu_\phi(\chi_B \cdot (\chi_{C} \circ \sigma^t)) \\
				&= \nu(h \chi_B \cdot (\chi_C \circ \sigma^t)) \\
				&= \lambda^{-t}(L_\phi^*)^t\nu(h\chi_B \cdot (\chi_C \circ \sigma^t)) \\
				&= \nu(\lambda^{-t} L_\phi^t(h \chi_B \cdot (\chi_C \circ \sigma^t))) \\
				&= \nu(\lambda^{-t} L_\phi^t(h \chi_B) \cdot \chi_C).
		\end{align*}
		Consequently,
		\begin{align*}
			|\mu_\phi(B \cap C) - \mu_\phi(B)\mu_\phi(C)| &= |\nu(\lambda^{-t} L_\phi^t(h \chi_B) \cdot \chi_C) - \nu(h \chi_B)\nu(h \chi_C)| \\
				&= |\nu((\lambda^{-t} L_\phi^t(h \chi_B) - \nu(h \chi_B)h)\chi_C)| \\
				&\leq \|\lambda^{-t} L_\phi^t(h \chi_B) - \nu(h \chi_B)h\| \nu(C).
		\end{align*}
		Since $\chi_B \in C(\Sigma, \reals)$ and $\var_r(\chi_B) = 0$ for all $r \geq 0$, we may apply \thref{bowen:lem-1-12}. So if $t \geq s$, then
		\[
			\|\lambda^{-t}L_\phi^{t}(h \chi_B) - \nu(h \chi_B)h\| \leq M\nu(h \chi_B)\rho^{t - s},
		\]
		where $M > 0$, $\rho \in (0, 1)$ are constants. We therefore have
		\begin{align*}
			|\mu_\phi(B \cap C) - \mu_\phi(B)\mu_\phi(C)| &\leq M\nu(h \chi_B)\rho^{t - s} \nu(C) \\
				&= M\mu_\phi(B)\rho^{t - s} \nu(C) \\
				&= M'\mu_\phi(B)\mu_\phi(C)\rho^{t - s},
		\end{align*}
		where $M' = M(\inf{h})^{-1}$. Summing over all elements in the partitions $\beta, \gamma$, we get
		\[
			\sum_{B \in \beta,\ C \in \gamma}{|\mu_\phi(B \cap C) - \mu_\phi(B)\mu_\phi(C)|} \leq M'\rho^{t - s} < \varepsilon
		\]
		for sufficiently large $t - s$.
		
		Hence $\mu_\phi$ is weak Bernoulli.
	\end{proof}
\end{theorem}

\chapter{Entropy} \label{chap:entropy}
\section{Overview}
Entropy is an important property used to distinguish measure-preserving transformations from each other and is used extensively in ergodic theory. Chapter \ref{chap:concentration-bounds} looks at methods for estimating entropy and finding inequalities to describe how these `estimators' behave. This chapter focuses on defining entropy and explaining why it is a useful property.

Throughout this chapter $(X, \B, \mu)$ will denote a probability space.

\section{Isomorphisms of measure-preserving transformations}\label{sec:isos-of-mpts}
One of the main problems in ergodic theory is to classify measure-preserving transformations. To this end, we want to decide the conditions required for two measure-preserving transformations to be `the same' -- up to sets of measure zero.

\emph{This section predominantly follows material in \cite[Chapter 2]{walters:intro-to-ergodic-theory}.}

\subsection{Isomorphism and conjugacy of measure spaces}

We begin by defining when two probability spaces are isomorphic or conjugate.

\begin{definition}
	Two probability spaces $(X_1, \B_1, \mu_1), (X_2, \B_2, \mu_2)$ are \key{isomorphic} if there exists $M_1 \in \B_1$, $M_2 \in \B_2$ such that $\mu_1(M_1) = 1 = \mu_2(M_2)$ and if there exists an invertible measure-preserving transformation $\phi: M_1 \to M_2$.
\end{definition}

Let $A, C \subset \B$. We define an equivalence relation on $\B$: we have $A \sim C$ if and only if $\mu(A \symdiff C) = 0$. In other words, $A$ and $C$ belong to the same equivalence class if they are equal almost everywhere. It can be easily checked that $\sim$ is indeed an equivalence relation.

Let $\tilde{\B}$ denote the collection of all equivalence classes in $\B$. Since $\B$ is a $\sigma$-algebra, it is clear that $\tilde{\B}$ is also a $\sigma$-algebra. We can define a measure $\tilde{\mu} : \tilde{\B} \to \reals^+$ by $\tilde{\mu}(\tilde{B}) = \mu(B)$, where $B$ belongs to the equivalence class $\tilde{B}$.

\begin{definition}
	A \key{measure algebra} is a Boolean $\sigma$-algebra equipped with a measure.
\end{definition}

In view of this definition, we see that $(\tilde{\B}, \tilde{\mu})$ is a \key{measure algebra}.

\begin{definition}
	Let $(X_1, \B_1, \mu_1), (X_2, \B_2, \mu_2)$ be probability spaces with corresponding measure algebras $(\tilde{\B}_1, \tilde{\mu}_1), (\tilde{\B}_2, \tilde{\mu}_2)$, respectively.
	
	We say $(\tilde{\B}_1, \tilde{\mu}_1)$ and $(\tilde{\B}_2, \tilde{\mu}_2)$ are \key{isomorphic} if there exists a bijection $\phi : \tilde{\B}_2 \to \tilde{\B}_1$ which preserves complementation and countable unions and intersections such that $\tilde{\mu}_1(\phi \tilde{B}) = \tilde{\mu}_2(\tilde{B})$ for all $\tilde{B} \in \tilde{\B}_2$.
	
	The probability spaces $(X_1, \B_1, \mu_1)$ and $(X_2, \B_2, \mu_2)$ are \key{conjugate} if their corresponding measure algebras are isomorphic.
\end{definition}

\begin{proposition}
	If two probability spaces are isomorphic, then they are also conjugate.
	\begin{proof}
		Suppose $(X_1, \B_1, \mu_1), (X_2, \B_2, \mu_2)$ are isomorphic probability spaces with corresponding measure algebras $(\tilde{\B}_1, \tilde{\mu}_1), (\tilde{\B}_2, \tilde{\mu}_2)$. By definition, this means there exists $M_1 \in \B_1$, $M_2 \in \B_2$ such that $\mu_1(M_1) = 1 = \mu_2(M_2)$ and there exists an invertible measure-preserving transformation $\phi: M_1 \to M_2$.
		
		Now we can define the map
		\[
			\psi : \tilde{\B}_2 \to \tilde{\B}_1 : \tilde{B} \mapsto (\phi^{-1}(M_2 \cap B))^\sim.
		\]
		This is clearly a bijection and, since $\phi$ is measure-preserving and $M_2 = X_2$ almost everywhere, we have
		\[
			\tilde{\mu}_1(\psi\tilde{B}) = \tilde{\mu}_1(\phi^{-1}(M_2 \cap B))^\sim = \tilde{\mu}_2(M_2 \cap B)^\sim = \tilde{\mu}_2(\tilde{B}),
		\]
		for all $\tilde{B} \in \tilde{\B}_2$. Therefore the measure algebras are isomorphic and hence the corresponding measure spaces are conjugate.
	\end{proof}
\end{proposition}

The converse statement is not necessarily true. Indeed, suppose we have the probability space $(X_1, \B_1, \mu_1)$ consisting of exactly one point, and another probability space $(X_2, \B_2, \mu_2)$ consisting of exactly two points, with $\B_2 = \{\emptyset, X_2\}$. It is easy to see that the measure algebras are isomorphic and hence the measure spaces are conjugate.

We need to choose $M_1 \in \B_1$, $M_2 \in \B_2$ such that $\mu_1(M_1) = 1 = \mu_2(M_2)$; the only possibility is $M_1 = X_1$ and $M_2 = X_2$. However there does not exist bijection between these two sets, so the probability spaces are \emph{isomorphic}.

\subsection{A motivational example}
We describe a scenario when two measure-preserving transformations could be considered `the same'. We follow the example in \cite[p58]{walters:intro-to-ergodic-theory}.

We first introduce a new probability space.

\begin{comment}
Let $Y = \{0, 1\}$ and let $(p_0, p_1)$ be a probability vector with no zero entries. Then $(Y, 2^Y, \nu)$ is a measure space, with measure $\nu$ defined by $\nu(y) = p_y$ for $y \in Y$. Now let $X = \{(x_j)_{j = 0}^\infty \mid x_j \in Y\}$, the space of infinite sequences with entries in $Y = \{0, 1\}$.
\end{comment}
\subsubsection{Bernoulli shifts}
Let $Y = \{0, 1, \dots, k - 1\}$ be a set of $k - 1$ symbols and let $p = (p_0, p_1, \dots, p_{k - 1})$ be a probability vector with no zero entries. Let $X = \{(x_j)_{j = 0}^\infty \mid x_j \in Y \text{ for all } j \geq 0\}$ be the space of infinite sequences with entries in $Y$. We may define a measure $\nu$ on cylinders of length $n$ by
\[
	\nu[x_0, x_1, \dots, x_{n - 1}] = p_{x_0} p_{x_1} \dots p_{x_{n - 1}}.
\]
Such measures are known as \key{Bernoulli measures}. Let $\sigma : X \to X$ be the one-sided, left shift map on $X$.

\begin{proposition}
	The measure $\nu$ is $\sigma$-invariant.
	\begin{proof}
		We have
		\begin{align*}
			\nu(\sigma^{-1}[x_1, \dots, x_n]) &= \nu\left(\bigsqcup_{j = 0}^{k - 1}{[j, x_1, \dots, x_n]}\right) \\
				&= \sum_{j = 0}^{k - 1}{\nu[j, x_1, \dots, x_n]} \\
				&= \sum_{j = 0}^{k - 1}{p_j p_{x_1} \dots p_{x_n}} \\
				&= p_{x_1} \dots p_{x_n} \\
				&= \nu[x_1, \dots, x_n].
		\end{align*}
		(We have used the fact that $\sum_{j = 0}^{k - 1}{p_j} = 1$ on the penultimate line.)
	\end{proof}
\end{proposition}

The shift map $\sigma : (X, \nu) \to (X, \nu)$ is called the one-sided $(p_0, p_1, \dots, p_{k - 1})$-shift.

We are now ready to present two measure-preserving transformations which we argue are `the same'.

\subsubsection{The \texorpdfstring{$\mathbf{\left(\frac{1}{2}, \frac{1}{2}\right)}$}{(1/2, 1/2)}-shift and the doubling map}
Let $T : ([0, 1), \B, \mu) \to ([0, 1), \B, \mu) : x \mapsto 2x \bmod 1$ be the doubling map, where $\B$ is the Borel $\sigma$-algebra on $[0, 1)$ and $\mu$ is Lebesgue measure.

Let $\sigma : (X, \C, \nu) \to (X, \C, \nu)$ be the $\left(\frac{1}{2}, \frac{1}{2}\right)$-shift, where
\[
	X := \{(x_j)_{j = 0}^\infty \mid x_j \in \{0, 1\} \text{ for all } j \geq 0\},
\]
$\C$ is the $\sigma$-algebra generated by all cylinders in $X$, and $\nu$ is the Bernoulli measure as described above with $p = \left(\frac{1}{2}, \frac{1}{2}\right)$.

Define the map $\phi : X \to [0, 1)$ by
\[
	\phi(x_0, x_1, \dots) = \sum_{j = 0}^\infty{\frac{x_j}{2^{j + 1}}} = \frac{x_0}{2^1} + \frac{x_1}{2^2} + \frac{x_2}{2^3} + \dots.
\]
It is easy to see that $\phi$ maps the binary expansion of a number to the actual number itself.

Let $E := \{(x_j)_{j = 0}^\infty \in X \mid (x_j)_{j = N}^\infty \text{ is constant for some } N \geq 0\}$ be the set of sequences in $X$ whose coordinates are eventually constant. Now, if the binary expansion of a number is \emph{not} eventually constant, then this binary expansion is unique. Therefore $\phi$ is \emph{injective} on $X \setminus E$. It is also clear that $\phi$ is \emph{surjective}, since every number in $[0, 1)$ has at least one binary expansion. In addition, it is easy to that $\phi \circ \sigma = T \circ \phi$.

We now show that $\phi$ is measure-preserving. A dyadic interval is an interval of the form $\left[\frac{a}{2^s}, \frac{a + 1}{2^s}\right] \subset [0, 1)$, where $s \in \naturals$. We can write
\[
	\frac{a}{2^s} = \sum_{j = 0}^{s - 1}{\frac{a_j}{2^j}} \quad \text{and} \quad \frac{a + 1}{2^s} = \sum_{j = 0}^\infty{\frac{a_j}{2^j}},
\]
where $a_j \in \{0, 1\}$ for $j = 0, 1, \dots, s - 2$ and $a_k = 1$ for $k \geq s - 1$. In other words, the binary expansion of all numbers in the interval $\left[\frac{a}{2^s}, \frac{a + 1}{2^s}\right]$ agree in the first $s$ positions. Thus,
\begin{align*}
	\nu\left(\phi^{-1}\left[\frac{a}{2^s}, \frac{a + 1}{2^s}\right]\right) &= \nu[a_0, a_1, \dots, a_{s - 1}] \\
		&= \frac{1}{2^s} \\
		&= \mu\left[\frac{a}{2^s}, \frac{a + 1}{2^s}\right].
\end{align*}
Hence $\phi$ is measure-preserving on dyadic intervals, which generate the Borel $\sigma$-algebra $\B$ on $[0, 1)$. We may therefore apply the Kolmogorov Extension Theorem and it follows that $\phi$ is \emph{measure-preserving} on all Borel sets $B \in \B$.

Let $D := \left\{\frac{a}{2^s} \in [0, 1) \mid s \in \naturals,\ 0 \leq a < 2^s\right\}$ be the set of all dyadic rationals in $[0, 1)$. Clearly, $T^{-1}D = D$ and this means that $T^{-1}([0, 1) \setminus D) = [0, 1) \setminus D$. It is also clear that $\sigma^{-1}E = E$ and so $\sigma^{-1}(X \setminus E) = X \setminus E$. So by the above observations, we see that $\phi: X \setminus E \to [0, 1) \setminus D$ is a bijection. It is also clear that $\phi \circ \sigma(x) = T \circ \phi(x)$ for all $x \in X \setminus E$.

Finally, we have $D \subset \rationals$ which gives $\mu(D) = 0$, and we also note that there are countably many sequences in $E$, thus $\nu(E) = 0$. Therefore $\phi$ is an invertible measure-preserving transformation between $X$ and $[0, 1)$ (modulo sets of measure zero), that is, the measure-preserving transformations are \emph{isomorphic}. Therefore it makes sense to say that these measure-preserving transformations are `the same'.

\subsection{\texorpdfstring{\sloppy Isomorphism and conjugacy of measure-preserving transformations}{Isomorphism and conjugacy of measure-preserving transformations}}
We now formalise the ideas illustrated in the above example.

\begin{definition}
	\sloppy Let $(X_1, \B_1, \mu_1, T_1)$, $(X_2, \B_2, \mu_2, T_2)$ be measure-preserving transformations of probability spaces. We say that $T_1$ is \key{isomorphic} to $T_2$ if there exists $M_1 \in \B_1$, $M_2 \in \B_2$ such that $\mu_1(M_1) = 1 = \mu_2(M_2)$ with
	\begin{enumerate}
		\item $T_1{M_1} \subset M_1$ and $T_2{M_2} \subset M_2$, and \label{mpt-iso-i}
		\item there exists an invertible measure-preserving transformation $\phi : M_1 \to M_2$ such that $\phi \circ T_1(x) = T_2 \circ \phi(x)$ for all $x \in M_1$. \label{mpt-iso-ii}
	\end{enumerate}
	If this is the case, we write $T_1 \simeq T_2$.
\end{definition}

Now suppose that $T_1 \simeq T_2$ with $M_1$, $M_2$ and $\phi : M_1 \to M_2$ as in the above definition. Then for $n \geq 1$ we clearly have $T_1^n{M_1} \subset M_1$ and $T_2^n{M_2} \subset M_2$, satisfying condition \ref{mpt-iso-i}. This in turn gives that $\phi \circ T_1^n(x) = T_2^n \circ \phi(x)$ for all $x \in M_1$, satisfying condition \ref{mpt-iso-ii}. In other words, if $T_1 \simeq T_2$, then $T_1^n \simeq T_2^n$ for all $n \geq 1$.

We also have the notion of conjugacy of measure-preserving transformations.

\begin{definition}
	Let $(X_1, \B_1, \mu_1, T_1)$, $(X_2, \B_2, \mu_2, T_2)$ be measure-preserving transformations of probability spaces. We say that $T_1$ is \key{conjugate} to $T_2$ if there exists an isomorphism $\Phi : (\tilde{\B}_2, \tilde{\mu}_2) \to (\tilde{\B}_1, \tilde{\mu}_1)$ of measure algebras such that $\Phi \circ \tilde{T}_2^{-1} = \tilde{T}_1^{-1} \circ \Phi$.
\end{definition}

It can be easily checked that isomorphism and conjugacy are equivalence relations on the set of all measure-preserving transformations.

As with probability spaces, isomorphic measure-preserving transformations are also conjugate. We show this in the following result.

\begin{theorem}\label{thm:walters-2-5}
	Let $(X_1, \B_1, \mu_1, T_1)$, $(X_2, \B_2, \mu_2, T_2)$ be measure-preserving transformations of probability spaces and suppose that $T_1 \simeq T_2$. Then $T_1$ is conjugate to $T_2$.
	
	\begin{proof}
		Suppose that $T_1 \simeq T_2$, so there exists a measure-preserving transformation $\phi : M_1 \to M_2$ such that $\phi \circ T_1(x) = T_2 \circ \phi(x)$ for all $x \in M_1$, where $M_1, M_2$ are as in the definition.
		
		Define $\Phi : (\tilde{\B}_2, \tilde{\mu}_2) \to (\tilde{\B}_1, \tilde{\mu}_1)$ by $\Phi(\tilde{B}) \mapsto (\phi^{-1}(B \cap M_2))^\sim$ for $B \in B_2$. Recall that $\tilde{B}$ is an equivalence class, so it is easy to see that $\Phi$ is an isomorphism. We also have
		\[
			\tilde{T}_1^{-1} \circ \Phi(\tilde{B}) = \tilde{T}_1^{-1} \circ (\phi^{-1}(B \cap M_2))^\sim = \phi^{-1} \circ \tilde{T}_2^{-1} (B \cap M_2)^\sim = \Phi \circ \tilde{T}_2^{-1}(B)
		\]
		for all $B \in \B_2$. Hence $T_1$ is conjugate to $T_2$.
	\end{proof}
\end{theorem}

The converse of this theorem is not necessarily true. However, we will find it useful to know the conditions for which the converse holds. We need the following definition from \cite[Definition A.21]{einsiedler-ward:ergodic-nt}.

\begin{definition}
	Let $Y$ be a set of countably or finitely many points, where each $y \in Y$ has positive measure $p_y > 0$ such that $\sum_{y \in Y}{p_y} \leq 1$. Put $s := 1 - \sum_{y \in Y}{p_y}$ and let $\mathcal{L}[0, s]$ denote the $\sigma$-algebra of Lebesgue measurable sets on the closed interval $[0, s]$. Let $\lambda_{[0, s]}$ denote Lebesgue measure on $[0, s]$.
	
	If the probability space $(X, \B, \mu)$ is isomorphic to the probability space
	\[
		\left([0, s] \sqcup Y,\ \mathcal{L}[0, s],\ \lambda_{[0, s]} + \sum_{y \in Y}{p_y \delta_y} \right),
	\]
	where $\delta_y$ is the Dirac measure at $y$, then we say that $(X, \B, \mu)$ is a \key{Lebesgue space}.
\end{definition}

We will also use the following result, which is proved in \cite[Theorem 12]{royden:real-analysis}.

\begin{lemma} \label{lem:walters-thm-2-2}
	For $j = 1, 2$, let $(X_j, \B(X_j), \mu_j)$ be complete separable metric spaces endowed with Borel $\sigma$-algebra $\B(X_j)$ and probability measure $\mu_j$. Suppose that $\Phi: \tilde{\B}(X_2) \to \tilde{\B}(X_1)$ is an isomorphism of measure algebras. Then there exists $M_1 \in \B(X_1)$, $M_2 \in \B(X_2)$ such that $\mu_1(M_1) = 1 = \mu_2(M_2)$, and an invertible measure-preserving transformation $\phi: M_1 \to M_2$ such that $\Phi(\tilde{B}) = (\phi^{-1}(B \cap M_2))^\sim$ for all $B \in \B(X_2)$.
	
	If $\psi$ is any other isomorphism $(X_1, \B(X_1), \mu_1)$ to $(X_2, \B(X_2), \mu_2)$ which induces $\Phi$, then $\mu_1\{x \in X_1 \mid \phi(x) \neq \psi(x)\} = 0$.
\end{lemma}

The following result gives the conditions for which the converse of \thref{thm:walters-2-5} is true.

\begin{theorem} \label{thm:walters-2-6}
	Suppose that either $(X_1, \B_1, \mu_1)$, $(X_2, \B_2, \mu_2)$ are Lebesgue spaces, or that $X_1, X_2$ are each complete separable metric spaces with corresponding Borel $\sigma$-algebras $\B_1, \B_2$. Suppose that $T_1 : X_1 \to X_1$, $T_2 : X_2 \to X_2$ are measure-preserving transformations and that $T_1$ is conjugate to $T_2$. Then $T_1 \simeq T_2$.
	\begin{proof}
		Suppose that $\Phi : (\tilde{B}_2, \tilde{\mu}_2) \to (\tilde{B}_1, \tilde{\mu}_1)$ is an isomorphism of measure algebras such that $\Phi \circ \tilde{T}_2^{-1} = \tilde{T}_1^{-1} \circ \Phi$. By \thref{lem:walters-thm-2-2} there exists sets $X'_1 \in \B_1$, $X'_2 \in \B_2$ such that $\mu_1(X'_1) = 1 = \mu(X'_2)$, and there exists an invertible measure-preserving transformation $\phi : X'_1 \to X'_2$ such that $\Phi(\tilde{B}) = (\phi^{-1}(B \cap X'_2))^\sim$ for all $B \in \B_2$. Then we have $\tilde{\phi}^{-1} \circ \tilde{T}_2^{-1} = \tilde{T}_1^{-1} \circ \tilde{\phi}^{-1}$, i.e. $T_2 \circ \phi = \phi \circ T_1$ almost everywhere.
		
		Now put
		\[
			A_1 := \{x \in X_1 \mid T_2 \circ \phi(x) = \phi \circ T_1(x)\} \quad \text{and} \quad M_1 := \bigcap_{n = 0}^\infty{T_1^{-n}{A_1}}.
		\]
		Then $\mu_1(M_1) = 1$ and $T_1^{-1}{M_1} \supset M_1$ which means that $M_1 \supset T_1 M_1$. We then define $M_2 := \phi M_1$ so that $T_2 M_2 \subset M_2$. Hence $T_1 \simeq T_2$.
	\end{proof}
\end{theorem}

As we mentioned briefly at the beginning of Section \ref{sec:isos-of-mpts}, we want to be able to decide when two measure-preserving transformations are `the same'. In view of the above discussion, `the same' can be replaced with `conjugate' or `isomorphic'. \key{Entropy} is one of the main conjugacy and isomorphism invariants studied in ergodic theory, and the remainder of this chapter will describe how the entropy of a measure-preserving transformation is defined.

The rest of this chapter predominantly follows \cite[Chapter 4]{walters:intro-to-ergodic-theory} unless otherwise stated. In particular, any definitions relating to \emph{information} is derived from \cite[p33-34]{parry-pollicott:zeta-fns-periodic-orbits}

\section{Entropy of partitions and sub-\texorpdfstring{$\sigma$}{sigma}-algebras}
\subsection{Partitions and sub-\texorpdfstring{$\sigma$}{sigma}-algebras}

We begin with a finite partition $\alpha = \{A_1, \dots, A_m\}$ of $(X, \B, \mu)$, i.e. the $A_j$ are pairwise disjoint and $X = \bigsqcup_{j = 1}^m{A_j}$. For clarity, we will denote partitions by the Greek letters, usually $\alpha, \beta$ or $\gamma$. Consider the collection of all elements of $\B$ such that their unions are elements of $\alpha$. Such a collection is a sub-$\sigma$-algebra of $\B$ and we will denote it by $\A(\alpha)$.

On the other hand, consider a finite sub-$\sigma$-algebra $\C = \{C_1, \dots, C_n\}$ of $\B$. We will use script uppercase letters to denote sub-$\sigma$-algebras, usually $\A, \C$ or $\D$. We can form a partition of $X$ by $\{B_1, \dots, B_n\}$, where $B_j = C_j$ or $X \setminus C_j$. We denote this partition by $\alpha(\C)$.

Note that if $\C$ is a sub-$\sigma$-algebra of $\B$ and $\gamma$ is a partition of $X$, then $\A(\alpha(\C)) = \C$ and $\alpha(\A(\gamma)) = \gamma$. This means that there is a one-to-one correspondence between finite partitions of $X$ and finite sub-$\sigma$-algebras of $\B$. Hence, in a lot of cases, we may use ``partition'' and ``sub-$\sigma$-algebra'' interchangeably.

If $T: X \to X$ is a measure-preserving transformation and $n \geq 0$, then $T^{-n}{\alpha}$ denotes the partition $\{T^{-n}{A_1}, \dots, T^{-n}{A_k}\}$.

\begin{remark}
	Let $\alpha = \{A_1, \dots, A_m\}$ be a finite partition of $(X, \B, \mu)$. Throughout this chapter, we may assume without loss of generality that $\mu(A_j) > 0$ for all $j = 1, \dots, m$. Indeed, we may index $\alpha$ so that
	\[
		\mu(A_j)
		\begin{cases}
			> 0, & \text{if } 1 \leq j \leq p; \\
			= 0, & \text{if } p + 1 \leq j \leq m.
		\end{cases}
	\]
	Then we may form a new partition $\alpha'$ consisting of the sets $A_1, \dots, A_{p - 1}$ and $\bigsqcup_{j = p}^m{A_j}$. Clearly, the disjoint union has the same measure as $A_p$ and so all the sets in $\alpha'$ have strictly positive measure.
	
	This argument can be easily modified for countable partitions.
\end{remark}

\begin{definition}
	Suppose that $\alpha, \gamma$ are finite partitions of $(X, \B, \mu)$. If each element of $\alpha$ can be written as the union of elements of $\gamma$, then we write \key{$\alpha \leq \gamma$}. In particular, we have $\alpha \leq \gamma$ if and only if $\A(\alpha) \subset \A(\gamma)$, and $\A \subset \C$ if and only if $\alpha(\A) \leq \alpha(\C)$.
\end{definition}

\begin{definition}
	Let $\alpha = \{A_1, \dots, A_m\}$, $\gamma = \{C_1, \dots, C_n\}$ be two finite partitions of a measure space $(X, \B, \mu)$. We define their \key{join} $\alpha \join \gamma$ as the partition
	\[
		\alpha \join \gamma := \{A_j \cap C_k \mid 1 \leq j \leq m, 1 \leq k \leq n\}.
	\]
	If $\A, \C$ are finite sub-$\sigma$-algebras of $\B$, then we define the join $\A \join \C$ in the same way. If this is the case, then $\A \join \C$ is actually the smallest sub-$\sigma$-algebra of $\B$ containing both $\A$ and $\C$.
	
	It is clear that $\A \join \C$ is comprised of the unions of sets of the form $A \cap C$, where $A \in \A, C \in \C$.
	
	We also have the relations $\alpha(\A \join \C) = \alpha(\A) \join \alpha(\C)$ and $\A(\alpha \join \gamma) = \A(\alpha) \join \A(\gamma)$.
\end{definition}

\begin{remark}
	If $T : X \to X$ is a measure-preserving transformation and $n \geq 0$, then $T^{-n}$ preserves set theoretic operations and so we have
	\begin{enumerate}
		\item $\alpha(T^{-n}{\A}) = T^{-n}{\alpha(\A)}$,
		\item $\A(T^{-n}{\alpha}) = T^{-n}{\A(\alpha)}$,
		\item $T^{-n}(\A \join \C) = T^{-n}{\A} \join T^{-n}{\C}$,
		\item $T^{-n}(\alpha \join \gamma) = T^{-n}{\alpha} \join T^{-n}{\gamma}$,
		\item if $\alpha \leq \gamma$, then $T^{-n}{\alpha} \leq T^{-n}{\gamma}$,
		\item if $\A \subset \C$, then $T^{-n}{\A} \subset T^{-n}{\C}$.
	\end{enumerate}
\end{remark}

\begin{definition}
	Let $\alpha, \gamma$ be two partitions of $(X, \B, \mu)$. We say that $\alpha$ and $\gamma$ are \key{independent} if $\mu(A \cap C) = \mu(A)\mu(C)$ for all $A \in \alpha$, $C \in \gamma$.
\end{definition}

\subsection{Motivation for information and entropy}
The following motivation for information and entropy follows that of \cite[Lecture 23]{ergodic-lectures}.

Suppose that we want to locate a point $x \in X$. To do this, we can partition the state space $X$ by the finite partition $\alpha = \{A_1, \dots, A_k\}$. We will later show that we may also consider countable partitions. If we find that $x \in A_j$, then we have received some \key{information}, and we think of $\mu(A_j)$ to be the probability that this happens.

We would like to define a function $I_\mu(\alpha) : X \to \reals^+$ such that $I_\mu(\alpha)(x)$ is the information received upon observing that $x \in A_j$. We want $I_\mu(\alpha)$ to only depend on $\mu(A_j)$, and in particular, we should receive more information if $\mu(A_j)$ is small, and we should receive less information if $\mu(A_j)$ is large. So we want $I_\mu(\alpha)$ to be of the form
\[
	I_\mu(\alpha)(x) = \sum_{A \in \alpha}{\chi_A(x)\phi(\mu(A))},
\]
where $\phi : [0, 1] \to \reals^+$ is some nonnegative function.

For two independent partitions $\alpha, \gamma$, the information gained from observing that $x \in A \cap C$, where $A \in \alpha, C \in \gamma$, should be equal to the information we gain from observing $x \in A$ in addition to observing $x \in C$. In view of this, we would require that $I_\mu(\alpha \join \gamma) = I_\mu(\alpha) + I_\mu(\gamma)$.

Combining the above requirements, we get that $\phi(\mu(A \cap C)) = \phi(\mu(A)\mu(C)) = \phi(\mu(A)) + \phi(\mu(C))$. For $\phi$ to be a continuous function, we see that $\phi(t)$ must be a multiple of $-\log{t}$. This gives rise to the following definitions.

\subsection{Information and entropy of partitions}
\begin{definition}
	Let $\alpha$ be a partition of $(X, \B, \mu)$. We define the \key{information} $I_\mu(\alpha) : X \to \reals^+$ of the partition $\alpha$ (or of the sub-$\sigma$-algebra $\A(\alpha)$) by
	\[
		I_\mu(\A(\alpha))(x) = I_\mu(\alpha)(x) := -\sum_{A \in \alpha}{\chi_A(x) \log{\mu(A)}}.
	\]
	We define the \key{entropy} $H_\mu(\alpha)$ of the partition $\alpha$ (or of the sub-$\sigma$-algebra $\A(\alpha)$) to be the average of the information, i.e.
	\begin{align*}
		H_\mu(\A(\alpha)) = H_\mu(\alpha) &:= \int{I_\mu(\alpha)\ d\mu} \\
			&= \int{-\sum_{A \in \alpha}{\chi_A \log{\mu(A)}}\ d\mu} \\
			&= -\sum_{A \in \alpha}{\mu(A) \log{\mu(A)}}.
	\end{align*}
	Whenever we use this definition and those derived from it, we will use the convention that $x \log x = 0$ if $x = 0$.
\end{definition}

\begin{remark}
	If $\alpha = \{X, \emptyset\}$, then we don't gain any information from performing observations on this partition, so $H(\alpha) = 0$. This can also be easily verified from the definition above.
\end{remark}

It is useful to know that, given a partition of $(X, \B, \mu)$ into $k$ sets, we can find an upper bound for the entropy of the partition.

\begin{proposition} \label{prop:walters-cor-4-2-1}
	Let $\alpha = \{A_1, \dots, A_k\}$ be a partition of $(X, \B, \mu)$ into $k$ sets. Then $H_\mu(\alpha) \leq \log{k}$.
	
	In particular, we have $H_\mu(\alpha) = \log{k}$ if and only if $\mu(A_j) = 1 / k$ for all $j = 1, \dots k$.
	
	\begin{proof}
		By \thref{thm:walters-4-2-xlogx-convex}, $x \log{x}$ is strictly convex. This means that for any partition $\alpha = \{A_1, \dots, A_k\}$ of $(X, \B, \mu)$ and for any $\{\lambda_j \in [0, 1] \mid j \in \{1, \dots, k\},\ \sum_{j = 1}^k{\lambda_j} = 1\}$, we have
		\[
			\left(\sum_{j = 1}^k{\lambda_j \mu(A_j)}\right) \log{\left(\sum_{j = 1}^k{\lambda_j \mu(A_j)}\right)} \leq \sum_{j = 1}^k{\lambda_j \mu(A_j) \log{\mu(A_j)}},
		\]
		with equality if and only if $\mu(A_1) = \mu(A_2) = \dots = \mu(A_k)$ whenever $\lambda_j \neq 0$ for all $j = 1, \dots, k$.
		
		Substituting in $\lambda_j = 1 / k$ for all $j = 1, \dots, k$ and rearranging, we get
		\[
			H_\mu(\alpha) = -\sum_{j = 1}^k{\mu(A_j) \log{\mu(A_j)}} \leq -\log{\frac{1}{k}} = \log{k},
		\]
		with equality if and only if $\mu(A_j) = 1 / k$ for all $j = 1, \dots, k$.
	\end{proof}
\end{proposition}

\section{Conditional entropy}
\subsection{Conditional expectation}
The definitions and results in this subsection follow those in \cite[p8-9]{walters:intro-to-ergodic-theory}.
\begin{definition}
	Suppose that $\mu$, $\nu$ are probability measures on a measurable space $(X, \B)$. If all sets $B \in \B$ with $\mu$-measure zero are also sets of $\nu$-measure zero, then we say that $\nu$ is \key{absolutely continuous} with respect to $\mu$. If this is the case, we write $\nu \ll \mu$.
	
	Stated alternatively, we have $\nu \ll \mu$ if, for all $B \in \B$ such that $\mu(B) = 0$, then $\nu(B) = 0$.
	
	Note that there may be more sets of $\nu$-measure zero. In the case where $\nu \ll \mu$ and $\mu \ll \nu$, we say that $\mu$ and $\nu$ are \key{equivalent}.
\end{definition}

\begin{theorem}[Radon-Nikodym Theorem] \label{thm:radon-nikodym}
	Suppose that $\mu, \nu$ are probability measures on a measurable space $(X, \B)$. Then $\nu \ll \mu$ if and only if there exists a nonnegative $\mu$-integrable function $f \in L^1(X, \B, \mu)$ where $f \geq 0$, $\int{f\ d\mu} = 1$, such that $\nu(B) = \int_B{f\ d\mu}$ for all $B \in \B$.
	
	Moreover, the function $f$ is unique almost everywhere, i.e. if there exists another function $g$ which satisfies the above properties, then $f = g$ $\mu$-almost everywhere.
\end{theorem}

The Radon-Nikodym Theorem allows us to define the conditional expectation operator.

\begin{definition}
	Let $(X, \B, \mu)$ be a measure space and let $\C$ be a sub-$\sigma$-algebra of $\B$. The \key{conditional expectation} operator $E_\mu(\seedot \mid \C) : L^1(X, \B, \mu) \to L^1(X, \C, \mu)$ is defined as follows.
	
	If $f \in L^1(X, \B, \mu)$ is a nonnegative real-valued integrable function, then
	\[
		\nu_f(C) = a^{-1}\int_C{f\ d\mu},
	\]
	for $C \in \C$, where $a = \int_X{f\ d\mu}$, defines a probability measure $\nu_f$ on $(X, \C)$ with $\nu_f \ll \mu$. By \thref{thm:radon-nikodym}, there exists a nonnegative function $E_\mu(f \mid \C) \in L^1(X, \C, \mu)$ such that $\int_C{E_\mu(f \mid \C)\ d\mu} = \int_C{f\ d\mu}$ for all $C \in \C$. Furthermore, $E_\mu(f \mid \C)$ is unique almost everywhere.
	
	If $f$ is a real-valued function, we consider the positive and negative parts of $f = f^+ - f^-$, where $f^+, f^- \geq 0$, and define $E_\mu(f \mid \C) := E_\mu(f^+ \mid \C) - E_\mu(f^- \mid \C)$.
	
	If $f$ is complex-valued, we take the real and imaginary parts of $f$ and define $E_\mu(f \mid \C)$ linearly as above.
\end{definition}

The conditional expectation operator $E_\mu(f \mid \C)$ is uniquely determined by the requirement that $E_\mu(f \mid \C)$ is $\C$-measurable, and also that
\[
	\int_C{f\ d\mu} = \int_C{E_\mu(f \mid \C)\ d\mu},
\]
for all $C \in \C$. With this in mind, we can think of $E_\mu(f \mid \C)$ as the best approximation of $f$ in the smaller space $\C$ of measurable functions.~\cite[Lecture 21]{ergodic-lectures}

\subsubsection{Properties of \texorpdfstring{$E_\mu(\seedot \mid \C)$}{the conditional expectation operator}}
\begin{enumerate}
	\item Conditional expectation $E_\mu(\seedot \mid \C)$ is a linear operator. \label{cond-exp:1}
	\item If $f \geq 0$, then $E_\mu(f \mid \C)$. \label{cond-exp:2}
	\item If $f \in L^1(X, \B, \mu)$ and $g$ is a $\C$-measurable bounded function, then $E_\mu(fg \mid \C) = gE_\mu(f \mid \C)$. \label{cond-exp:3}
	\item For $f \in L^1(X, \B, \mu)$, we have $\left|E_\mu(f \mid \C)\right| \leq E_\mu(|f| \mid \C)$. \label{cond-exp:4}
	\item If $\C_2 \subset \C_1$, then for $f \in L^1(X, \B, \mu)$, we have $E_\mu(E_\mu(f \mid \C_1) \mid \C_2) = E_\mu(f \mid \C_2)$. \label{cond-exp:5}
\end{enumerate}

If $f$ is an integrable function, then we can find $E_\mu(f \mid \C)$ using the following formula.

\begin{proposition}
	Let $\C$ be a finite or countable sub-$\sigma$-algebra of $\B$. Then
	\[
		E_\mu(f \mid \C)(x) = \sum_{C \in \gamma}{\int_{C}{f\ d\mu}\frac{\chi_{C}(x)}{\mu(C)}}.
	\]
	
	\begin{proof}
		We follow the proof given in \cite[Example 10.1.2]{bogachev:measure}.
		
		The summation is clearly an integrable function, and the $\C$-measurable functions are exactly the characteristic functions of $C \in \C$. Therefore the result is equivalent to
		\[
			\int{\chi_B(x) \cdot E_\mu(f \mid \C)(x)\ d\mu} = \int{\left(\chi_B(x) \cdot \sum_{C \in \gamma}{\int_{C}{f\ d\mu}\frac{\chi_{C}(x)}{\mu(C)}}\right)\ d\mu},
		\]
		for any $B \in \C$. This is clearly true since by definition,
		\[
			\int{\chi_B(x) \cdot E_\mu(f \mid \C)(x)\ d\mu} = \int_B{f\ d\mu},
		\]
		and
		\[
			\int{\left(\chi_B(x) \cdot \sum_{C \in \gamma}{\int_{C}{f\ d\mu}\frac{\chi_{C}(x)}{\mu(C)}}\right)\ d\mu} = \int{\left({\int_{B}{f\ d\mu}\frac{\chi_{B}(x)}{\mu(B)}}\right)\ d\mu} = \int_{B}{f\ d\mu}.
		\]
	\end{proof}
\end{proposition}

\begin{definition}
	Let $\C \subset \B$ be a sub-$\sigma$-algebra of a $\sigma$-algebra $\B$. The \key{conditional probability} of $B \in \B$ given $\C$ is defined
	\[
		\mu(B \mid \C) := E_\mu(\chi_B \mid \C).
	\]
\end{definition}

\subsection{Conditional information and entropy}
We can define the conditional information and entropy of a partition $\alpha$, given that we know the information gained from the partition $\gamma$. Conditional entropy will prove to be useful later when look at the entropy of measure-preserving transformations.

\begin{definition}
	Let $(X, \B, \mu, T)$ be a measure-preserving transformation on a probability space. Let $\A, \C$ be finite sub-$\sigma$-algebras, where $\alpha(\A) = \{A_1, \dots, A_p\}$, $\alpha(\C) = \{C_1, \dots, C_q\}$. The \key{conditional entropy} of $\alpha$ given $\C$ is defined
	\begin{align*}
		H_\mu(\alpha(\A) \mid \alpha(\C)) = H_\mu(\A \mid \C) &:= -\sum_{k = 1}^q{\mu(C_k) \sum_{j = 1}^{p}{\frac{\mu(A_j \cap C_k)}{\mu(C_k)} \log{\frac{\mu(A_j \cap C_k)}{\mu(C_k)}}}} \\
			&= -\sum_{j, k}{\mu(A_j \cap C_k) \log{\frac{\mu(A_j \cap C_k)}{\mu(C_k)}}}.
	\end{align*}
\end{definition}

\begin{remark}
	If $\mathcal{N} = \{X, \emptyset\}$, then we have $H_\mu(\alpha \mid \mathcal{N}) = H_\mu(\alpha)$. Again, this is because we gain no information from $\mathcal{N}$.
\end{remark}

\begin{theorem} \label{thm:walters-4.3}
	Let $(X, \B, \mu)$ be a probability space and let $\A, \B, \D$ be finite sub-$\sigma$-algebras of $\B$. Suppose $T : X \to X$ is a measure-preserving transformation. Then
	\begin{enumerate}
		\item $H_\mu(\A \join \C \mid \D) = H_\mu(\A \mid \D) + H_\mu(\C \mid \A \join \D)$, \label{walters-thm-4.3:1}
		\item $H_\mu(\A \join \C) = H_\mu(\A) + H_\mu(\C \mid \A)$, \label{walters-thm-4.3:2}
		\item if $\A \subset \C$, then $H_\mu(\A \mid \D) \leq H_\mu(\C \mid \D)$, \label{walters-thm-4.3:3}
		\item if $\A \subset \C$, then $H_\mu(\A) \leq H_\mu(\C)$, \label{walters-thm-4.3:4}
		\item if $\C \subset \D$, then $H_\mu(\A \mid \C) \geq H_\mu(\A \mid \D)$, \label{walters-thm-4.3:5}
		\item $H_\mu(\A) \geq H_\mu(\A \mid \D)$, \label{walters-thm-4.3:6}
		\item $H_\mu(\A \join \C \mid \D) \leq H_\mu(\A \mid \D) + H_\mu(\C \mid \D)$, \label{walters-thm-4.3:7}
		\item $H_\mu(\A \join \C) \leq H_\mu(\A) + H_\mu(\C)$, \label{walters-thm-4.3:8}
		\item $H_\mu(T^{-1}\A \mid T^{-1}\C) = H_\mu(\A \mid \C)$, \label{walters-thm-4.3:9}
		\item $H_\mu(T^{-1}\A) = H_\mu(\A)$. \label{walters-thm-4.3:10}
	\end{enumerate}
	\begin{proof} \hfill
		\begin{enumerate}
			\item We have
				\[
					H_\mu(\A \join \C \mid \D) = -\sum_{j, k, m}{\mu(A_j \cap C_k \cap D_m) \log{\frac{\mu(A_j \cap C_k \cap D_m)}{\mu(D_m)}}}.
				\]
				If $\mu(A_j \cap D_m) \neq 0$, then
				\[
					\frac{\mu(A_j \cap C_k \cap D_m)}{\mu(D_m)} = \frac{\mu(A_j \cap C_k \cap D_m)}{\mu(A_j \cap D_m)} \frac{\mu(A_j \cap D_m)}{\mu(D_m)}.
				\]
				If $\mu(A_j \cap D_m) = 0$, then the above evaluates to zero anyway, so we ignore such terms. Therefore we have
				\begin{align*}
					H_\mu(\A \join \C \mid \D) &= -\sum_{j, k, m}{\mu(A_j \cap C_k \cap D_m) \log{\frac{\mu(A_j \cap D_m)}{\mu(D_m)}}} \\
						& \qquad - \sum_{j, k, m}{\mu(A_j \cap C_k \cap D_m) \log{\frac{\mu(A_j \cap C_k \cap D_m)}{\mu(A_j \cap D_m)}}} \\
						&= -\sum_{j, m}{\mu(A_j \cap D_m) \log{\frac{\mu(A_j \cap D_m)}{\mu(D_m)}}} + H_\mu(\C \mid \A \join \D) \\
						&= H_\mu(\A \mid \D) + H_\mu(\C \mid \A \join \D).
				\end{align*}
			\item We put $\D = \{X, \emptyset\}$ in \ref{walters-thm-4.3:1}. Then by the above remark the result follows immediately.
			\item Suppose that $\A \subset \C$. Then
				\begin{align*}
					H_\mu(\C \mid \D) &= H_\mu(\A \join \C \mid \D) & \text{(since } \A \subset \C) \\
						&= H_\mu(\A \mid \D) + H_\mu(\C \mid \A \join \D) & \text{(by \ref{walters-thm-4.3:1})} \\
						&\geq H_\mu(\A \mid \D).
				\end{align*}
			\item As with \ref{walters-thm-4.3:2}, we put $\D = \{X, \emptyset\}$ in \ref{walters-thm-4.3:3}.
			\item Suppose that $\C \subset \D$ and fix $j, k$. We have
				\[
					\sum_{m}{\frac{\mu(D_m \cap C_k)}{\mu(C_k)}} = 1
				\]
				and so we may apply Theorem \ref{thm:walters-4-2-xlogx-convex}. So with $f(x) = x\log{x}$, we have
				\begin{equation} \label{fml:walters-4-3-5-ineq}
					f\left(\sum_{m}{\frac{\mu(D_m \cap C_k)}{\mu(C_k)} \frac{\mu(A_j \cap D_m)}{\mu(D_m)}}\right) \leq \sum_{m}{\frac{\mu(D_m \cap C_k)}{\mu(C_k)} f\left(\frac{\mu(A_j \cap D_m)}{\mu(D_m)}\right)}.
				\end{equation}
				Since $\C \subset \D$, we have
				\begin{align*}
					f\left(\sum_{m}{\frac{\mu(D_m \cap C_k)}{\mu(C_k)} \frac{\mu(A_j \cap D_m)}{\mu(D_m)}}\right) &= f\left(\frac{\mu(A_j \cap C_k)}{\mu(C_k)}\right) \\
						&= \frac{\mu(A_j \cap C_k)}{\mu(C_k)} \log{\frac{\mu(A_j \cap C_k)}{\mu(C_k)}}.
				\end{align*}
				Then multiplying \eqref{fml:walters-4-3-5-ineq} by $\mu(C_k)$ and then summing over $j, k$, we get
				\begin{align*}
					-H_\mu(\A \mid \C) &= \sum_{j, k}{\mu(A_j \cap C_k) \log{\frac{\mu(A_j \cap C_k)}{\mu(C_k)}}} \\
						&\leq \sum_{j, k, m}{\mu(D_m \cap C_k) \frac{\mu(A_j \cap D_m)}{\mu(D_m)} \log{\frac{\mu(A_j \cap D_m)}{\mu(D_m)}}} \\
						&= \sum_{j, m}{\mu(A_j \cap D_m) \log{\frac{\mu(A_j \cap D_m)}{\mu(D_m)}}} \\
						&= -H_\mu(\A \mid \D).
				\end{align*}
				Hence $H_\mu(\A \mid \C) \geq H_\mu(\A \mid \D)$.
			\item We put $\C = \{X, \emptyset\}$ in \ref{walters-thm-4.3:5}.
			\item We have
				\begin{align*}
					H_\mu(\A \join \C \mid \D) &= H_\mu(\A \mid \D) + H_\mu(\C \mid \A \join \D) & \text{(by \ref{walters-thm-4.3:1})} \\
						&\leq H_\mu(\A \mid \D) + H_\mu(\C \mid \D) & \text{(by \ref{walters-thm-4.3:5})}.
				\end{align*}
			\item We put $\D = \{X, \emptyset\}$ in \ref{walters-thm-4.3:7}.
			\item This follows since $T$ is measure-preserving and by the definition conditional entropy.
			\item This is also immediate from the definitions.
		\end{enumerate}
	\end{proof}
\end{theorem}

Using conditional expectation, we can define conditional entropy for a finite sub-$\sigma$-algebra $\A$ of $\B$ given an arbitrary (not necessarily finite) sub-$\sigma$-algebra $\C$ of $\B$. We first suppose that $\C$ is finite so that $\alpha(\C) = \{C_1, \dots, C_q\}$, and also let $\alpha(\A) = \{A_1, \dots, A_p\}$. Note that
\[
		E_\mu(\chi_{A_j} \mid \C)(x) = \sum_{k = 1}^{q}{\int_{C_k}{\chi_{A_j}\ d\mu}\frac{\chi_{C_k}(x)}{\mu(C_k)}}.
\]

Then
\begin{align*}
	H_\mu(\alpha(\A) \mid \alpha(\C)) = H_\mu(\A \mid \C) &= -\sum_{j = 1}^{p}{\sum_{k = 1}^{q}{\mu(A_j \cap C_k) \log{\frac{\mu(A_j \cap C_k)}{\mu(C_k)}}}} \\
		&= -\sum_{j = 1}^{p}{\int{\chi_{A_j} \log{E_\mu(\chi_{A_j} \mid \C)}\ d\mu}} \\
		&= -\int{\sum_{j = 1}^{p}{E_\mu(\chi_{A_j} \mid \C) \log{E_\mu(\chi_{A_j} \mid \C)}}\ d\mu}.
\end{align*}

We can therefore make the following definition for countable sub-$\sigma$-algebras $\C$ of $\B$.

\begin{definition}
	Let $(X, \B, \mu)$ be a probability space. Suppose that $\A$ is a finite sub-$\sigma$-algebra of $\B$ and that $\C$ is an \emph{arbitrary} sub-$\sigma$-algebra of $\B$. Denote the partition $\alpha(\A) = \{A_1, \dots, A_p\}$. The \key{conditional entropy} of $\alpha$ given $\C$ is given by
	\[
		H_\mu(\alpha(\A) \mid \alpha(\C)) = H_\mu(\A \mid \C) := -\int{\sum_{j = 1}^{p}{\mu(A_j \mid \C) \log{\mu(A_j \mid \C)}}\ d\mu}.
	\]
\end{definition}

\begin{lemma} \label{lem:walters-4-6}
	Suppose that $\A_1 \subset \A_2 \subset \dots \subset \A_n \subset \dots$ is an increasing sequence of sub-$\sigma$-algebras of $\B$, and write $\A := \bigjoin_{n = 1}^\infty{\A_n}$. Then for all $f \in L^2(X, \B, \mu)$ we have $\|E_\mu(f \mid \A_n) - E_\mu(f \mid \A)\|_2 \to 0$, as $n \to \infty$.
	\begin{proof}
		By definition, the operator $E_\mu(\seedot \mid \A_n)$ maps functions from from $L^2(X, \B, \mu)$ to $L^2(X, \A_n, \mu)$. We let $A \in \A$ and choose a sequence $A_n \in \A_n$ such that $\mu(A_n \symdiff A) \to 0$, as $n \to +\infty$. (This is possible because $\A_n$ is an increasing sequence.)
		
		Since $E_\mu(\chi_A \mid \A_n)$ is a best approximation to $\chi_A$ in $L^2(X, \A_n, \mu)$, we have
		\[
			\|E_\mu(\chi_A \mid \A_n) - \chi_A\|_2^2 \leq \|\chi_{A_n} - \chi_A\|_2^2 = \mu(A_n \symdiff A) \to 0,
		\]
		as $n \to +\infty$.
		
		The set of all finite linear combinations of characteristic functions are dense in $L^2(X, \A, \mu)$ and so for all $g \in L^2(X, \A, \mu)$, we have
		\begin{equation} \label{fml:lem-4-6-star}
			\|E_\mu(g \mid \A_n) - g\|_2 \to 0,
		\end{equation}
		as $n \to +\infty$. Therefore if $f \in L^2(X, \B, \mu)$, then by property \ref{cond-exp:5} on page \pageref{cond-exp:5}, we have $E_\mu(E_\mu(f \mid \A) \mid \A_n) = E_\mu(f \mid \A_n)$ because $\A_n \subset \A$ for all $n \geq 1$. Hence by \eqref{fml:lem-4-6-star} we have
		\[
			\|E_\mu(f \mid \A_n) - E_\mu(f \mid \A)\|_2 = \|E_\mu(E_\mu(f \mid \A) \mid \A_n) - E_\mu(f \mid \A)\|_2 \to 0,
		\]
		as $n \to +\infty$, as required.
	\end{proof}
\end{lemma}

We also have the following theorem.

\begin{theorem}[Increasing Martingale Theorem] \label{thm:increasing-martingale}
	Suppose that $\A_1 \subset \A_2 \subset \dots \subset \A_n \subset \dots$ is an increasing sequence of sub-$\sigma$-algebras of $\B$ such that $\A_n \to \A$, as $n \to +\infty$. Then for all $f \in L^1(X, \B, \mu)$ we have
	\begin{enumerate}
		\item $E_\mu(f \mid \A_n) \to E_\mu(f \mid \A)$ $\mu$-almost everywhere, as $n \to +\infty$, and
		\item $E_\mu(f \mid \A_n) \to E_\mu(f \mid \A)$ in $L_1$, as $n \to +\infty$.
	\end{enumerate}
\end{theorem}

Note that \thref{thm:walters-4.3} was given in terms of \emph{finite} sub-$\sigma$-algebras and hence finite partitions. By the following theorem, we can in fact extend these results for \emph{countable} sub-$\sigma$-algebras and partitions.

\begin{theorem} \label{thm:walters-4-7}
	Suppose that $\A$ is a \emph{finite} sub-$\sigma$-algebra of $\B$. Furthermore, suppose that $\C_1 \subset \C_2 \subset \dots \subset \C_n \subset \dots$ is an increasing sequence of sub-$\sigma$-algebras of $\B$, and put $\C:= \bigvee_{n = 1}^\infty{\C_n}$. Then $H_\mu(\A \mid \C_n) \to H_\mu(\A \mid \C)$, as $n \to +\infty$.
	\begin{proof}
		Let $\alpha(\A) = \{A_1, \dots, \A_k\}$. By \thref{lem:walters-4-6}, $\|E_\mu(\chi_{A_j} \mid \C_n) - E_\mu(\chi_{A_j} \mid \C)\|_2 \to 0$, as $n \to +\infty$ for $j = 1, \dots, k$. So $E_\mu(\chi_{A_j} \mid \C_n)$ converges in measure to $E_\mu(\chi_{A_j} \mid \C)$, i.e. given $\varepsilon > 0$, we have that
		\[
			\lim_{n \to +\infty}{\mu\left\{x \in X \midmid \left|E_\mu(\chi_{A_j} \mid \C_n)(x) - E_\mu(\chi_{A_j} \mid \C)(x)\right| \geq \varepsilon\right\}} = 0.
		\]
		So it is clear that $-\sum_{j = 1}^k{E_\mu(\chi_{A_j} \mid \C_n) \log{E_\mu(\chi_{A_j} \mid \C_n)}}$ also converges in measure to $-\sum_{j = 1}^k{E_\mu(\chi_{A_j} \mid \C) \log{E_\mu(\chi_{A_j} \mid \C)}}$.
		
		Since $E_\mu(\seedot \mid \C)$ is a positive linear operator and since $\sum_{j = 1}^k{\chi_{A_j}} = 1$, we have $0 \leq E_\mu(\chi_{A_j} \mid \C)(x) \leq 1$ for $\mu$-almost every $x$. Hence
		\begin{align*}
			-\sum_{j = 1}^k{\mu(A_j \mid \C)(x) \log{\mu(A_j \mid \C)(x)}} &= -\sum_{j = 1}^k{E_\mu(\chi_{A_j} \mid \C)(x) \log{E_\mu(\chi_{A_j} \mid \C)(x)}} \\
				&\leq k \max_{t \in [0, 1]}(-t \log{t}) \\
				&= ke.
		\end{align*}
		So all functions of this form are bounded by $ke$ and hence converge in $L^1(\mu)$. Therefore, $H_\mu(\A \mid \C_n) \to H_\mu(\A \mid \C)$, as $n \to +\infty$.
	\end{proof}
\end{theorem}

As a result of this theorem, given a countable (not necessarily finite) sub-$\sigma$-algebra $\C$, we can find an increasing sequence $\C_1 \subset \C_2 \subset \dots \subset \C_n \subset \dots$ such that $\C_n \to \C$, as $n \to +\infty$. We then apply \thref{thm:walters-4-7} and we see that any result involving finite sub-$\sigma$-algebras can be extended for countable sub-$\sigma$-algebras.

\section{\texorpdfstring{\sloppy Entropy of measure-preserving transformations}{Entropy of measure-preserving transformations}}
So far, we have be focusing exclusively on the entropy of partitions and sub-$\sigma$-algebras, but we can now introduce a measure-preserving transformation $T : X \to X$. We can think of $T$ as the passing of a day in time, and so $H_\mu\left(\bigjoin_{j = 0}^{n - 1}{T^{-j}{\alpha}}\right)$ is the average information we gain after $n$ days. Given a partition $\alpha$, it is natural to define the entropy of $T$ by the average information we obtain \emph{per day}. First, we need to ensure that this is well-defined.

\begin{theorem} \label{thm:walters-4-9}
	Let $(a_n)_{n = 1}^\infty$ be a sequence of real numbers such that $a_{n + p} \leq a_n + a_p$ for all $n, p \geq 1$. Then $\lim_{n \to +\infty}(a_n / n)$ exists and equals $\inf_{n \geq 1}(a_n / n)$.
	
	This limit could be $-\infty$, but if $a_n$ is bounded below then, by the properties of the sequence, the limit is non-negative.
	\begin{proof}
		Fix $p \geq 1$. We can write $n = kp + j$ for some $0 \leq j < p$, and then
		\[
			\frac{a_n}{n} = \frac{a_{kp + j}}{kp + j} \leq \frac{a_j}{kp} + \frac{a_{kp}}{kp} \leq \frac{a_j}{kp} + \frac{ka_p}{kp} = \frac{a_j}{kp} + \frac{a_p}{p}.
		\]
		We have that $k \to +\infty$, as $n \to +\infty$, and so
		\[
			\frac{a_j}{kp} \to 0,
		\]
		as $n \to +\infty$. Putting the above results together, we have
		\[
			\limsup_{n \to +\infty}{\frac{a_n}{n}} \leq \frac{a_p}{p}.
		\]
		Since $p$ is fixed, we have
		\[
			\limsup_{n \to +\infty}{\frac{a_n}{n}} \leq \inf_{p \geq 1}{\frac{a_p}{p}}.
		\]
		On the other hand, it is clear that
		\[
			\inf_{p \geq 1}{\frac{a_p}{p}} \leq \liminf_{n \to +\infty}{\frac{a_n}{n}}.
		\]
		Therefore
		\[
			\lim_{n \to +\infty}{\frac{a_n}{n}}
		\]
		exists and is equal to the infimum.
	\end{proof}
\end{theorem}

\begin{corollary} \label{cor:walters-4-9-1}
	Let $T : X \to X$ be a measure-preserving transformation and suppose that $\A$ is a finite sub-$\sigma$-algebra of $\B$. Then
	\[
		\lim_{n \to +\infty}{\frac{1}{n} H_\mu\left(\bigjoin_{j = 0}^{n - 1}{T^{-j}{\A}}\right)}
	\]
	exists.
	\begin{proof}
		Let the sequence $(a_n)_{n = 1}^\infty$ be defined by $a_n = H_\mu\left(\bigjoin_{j = 0}^{n - 1}{T^{-j}{\A}}\right) \geq 0$. For any $n, p \geq 1$ we have
		\begin{align*}
			a_{n + p} &= H_\mu\left(\bigjoin_{j = 0}^{n + p - 1}{T^{-j}{\A}}\right) \\
				&\leq H_\mu\left(\bigjoin_{j = 0}^{n - 1}{T^{-j}{\A}}\right) + H_\mu\left(\bigjoin_{j = n}^{n + p - 1}{T^{-j}{\A}}\right) & \text{(by \thref{thm:walters-4.3} \ref{walters-thm-4.3:8})} \\
				&= a_n + H_\mu\left(\bigjoin_{j = 0}^{p - 1}{T^{-j}{\A}}\right) & \text{(by \thref{thm:walters-4.3} \ref{walters-thm-4.3:10})} \\
				&= a_n + a_p.
		\end{align*}
		By \thref{thm:walters-4-9}, the limit of $a_n$ exists and hence
		\[
			\lim_{n \to +\infty}{\frac{1}{n} H_\mu\left(\bigjoin_{j = 0}^{n - 1}{T^{-j}{\A}}\right)}
		\]
		exists.
	\end{proof}
\end{corollary}

This ensures that the following definition is well-defined.

\begin{definition}
	Let $(X, \B, \mu, T)$ be a measure-preserving transformation of a probability space and let $\alpha$ be a finite partition of $X$. The \key{entropy of $T$ with respect to $\alpha$} is defined
	\[
		h_\mu(T, \A(\alpha)) = h_\mu(T, \alpha) := \lim_{n \to +\infty}{\frac{1}{n} H_\mu\left(\bigjoin_{j = 0}^{n - 1}{}T^{-j}{\alpha}\right)}.
	\]
\end{definition}

\begin{theorem} \label{thm:walters-4.12}
	Let $\A$, $\C$ be finite sub-algebras of $\B$ and let $T$ be a measure-preserving transformation of a probability space $(X, \B, \mu)$ Then
	\begin{enumerate}
		\item $h_\mu(T, \A) \leq H_\mu(\A)$, \label{walters:thm-4-12:1}
		\item $h_\mu(T, \A \join \C) \leq h_\mu(T, \A) + h_\mu(T, \C)$, \label{walters:thm-4-12:2}
		\item if $\A \subset \C$ then $h_\mu(T, \A) \leq h_\mu(T, \C)$, \label{walters:thm-4-12:3}
		\item $h_\mu(T, \A) \leq h_\mu(T, \C) + H_\mu(\A \mid \C)$, \label{walters:thm-4-12:4}
		\item $h_\mu(T, T^{-1}{\A}) = h_\mu(T, \A)$, \label{walters:thm-4-12:5}
		\item if $m \geq 1$ then $h_\mu(T, \A) = h_\mu\left(T, \bigjoin\limits_{j = 0}^{m - 1}{T^{-j}{\A}}\right)$, \label{walters:thm-4-12:6}
		\item if $T$ is invertible and $m \geq 1$ then $h_\mu(T, \A) = h_\mu\left(T, \bigjoin\limits_{j = -m}^m{T^{-j}{\A}}\right)$. \label{walters:thm-4-12:7}
	\end{enumerate}
	\begin{proof} \hfill
		\begin{enumerate}
			\item For all $n \geq 1$,
				\begin{align*}
					\frac{1}{n} H_\mu\left(\bigjoin_{j = 0}^{n - 1}{T^{-j}{\A}}\right) &\leq \frac{1}{n} \sum_{j = 0}^{n - 1}{H_\mu({T^{-j}{\A}})} & \text{(by \thref{thm:walters-4.3} \ref{walters-thm-4.3:8})} \\
						&\leq \frac{1}{n} \sum_{j = 0}^{n - 1}{H_\mu(\A)} & \text{(by \thref{thm:walters-4.3} \ref{walters-thm-4.3:10})} \\
						&= H_\mu(\A).
				\end{align*}
			\item We have
				\begin{align*}
					H_\mu\left(\bigjoin_{j = 0}^{n - 1}{T^{-j}(\A \join \C)}\right) &= H_\mu\left(\bigjoin_{j = 0}^{n - 1}{T^{-j}{\A}} \join \bigjoin_{j = 0}^{n - 1}{T^{-j}{\C}}\right) \\
						&\leq H_\mu\left(\bigjoin_{j = 0}^{n - 1}{T^{-j}{\A}}\right) + H_\mu\left(\bigjoin_{j = 0}^{n - 1}{T^{-j}{\C}}\right)
				\end{align*}
				by \thref{thm:walters-4.3} \ref{walters-thm-4.3:8}.
			\item Since $T$ preserves set theoretic operations, if $\A \subset \C$, then for all $n \geq 1$ we have
				\[
					\bigjoin_{j = 0}^{n - 1}{T^{-j}{\A}} \subset \bigjoin_{j = 0}^{n - 1}{T^{-j}{\C}}.
				\]
				Applying \thref{thm:walters-4.3} \ref{walters-thm-4.3:4} we get
				\[
					H_\mu\left(\bigjoin_{j = 0}^{n - 1}{T^{-j}{\A}}\right) \leq H_\mu\left(\bigjoin_{j = 0}^{n - 1}{T^{-j}{\C}}\right).
				\]
			\item We have
				\begin{align*}
					H_\mu\left(\bigjoin_{j = 0}^{n - 1}{T^{-j}{\A}}\right) &\leq H_\mu\left(\bigjoin_{j = 0}^{n - 1}{T^{-j}{\A}} \join \bigjoin_{j = 0}^{n - 1}{T^{-j}{\C}}\right) \\ & \hspace{60mm} \text{(by \thref{thm:walters-4.3} \ref{walters-thm-4.3:4})} \\
						&= H_\mu\left(\bigjoin_{j = 0}^{n - 1}{T^{-j}{\C}}\right) + H_\mu\left(\bigjoin_{j = 0}^{n - 1}{T^{-j}{\A}} \midmid \bigjoin_{j = 0}^{n - 1}{T^{-j}{\C}}\right) \\ & \hspace{60mm} \text{(by \thref{thm:walters-4.3} \ref{walters-thm-4.3:2}).}
				\end{align*}
				By applying \thref{thm:walters-4.3} \ref{walters-thm-4.3:7} repeatedly, we have
				\begin{align*}
					H_\mu\left(\bigjoin_{j = 0}^{n - 1}{T^{-j}{\A}} \midmid \bigjoin_{j = 0}^{n - 1}{T^{-j}{\C}}\right) &\leq \sum_{j = 0}^{n - 1}{H_\mu\left(T^{-j}{\A} \midmid \bigjoin_{k = 0}^{n - 1}{T^{-k}{\C}}\right)} \\
						&\leq \sum_{j = 0}^{n - 1}{H_\mu\left(T^{-j}{\A} \midmid T^{-j}{\C}\right)} & \text{(by \thref{thm:walters-4.3} \ref{walters-thm-4.3:5})} \\
						&= nH_\mu(\A \mid \C) & \text{(by \thref{thm:walters-4.3} \ref{walters-thm-4.3:9}).}
				\end{align*}
				Combining this with the previous result, we have
				\begin{align*}
					h_\mu(T, \A) &\leq \lim_{n \to +\infty}{\left[\frac{1}{n} H_\mu\left(\bigjoin_{j = 0}^{n - 1}{T^{-j}{\C}}\right) + nH_\mu(\A \mid \C)\right]} \\
						&= h_\mu(T, \C) + H_\mu(\A \mid \C).
				\end{align*}
			\item By \thref{thm:walters-4.3} \ref{walters-thm-4.3:10} we have
				\begin{align*}
					h_\mu(T, T^{-1}{\A}) &= \lim_{n \to +\infty}{\frac{1}{n} H_\mu\left(\bigjoin_{j = 1}^{n - 1}{T^{-j}{\A}}\right)} \\
						&= \lim_{n \to +\infty}{\frac{1}{n} H_\mu\left(\bigjoin_{j = 0}^{n - 1}{T^{-j}{\A}}\right)} \\
						&= h_\mu(T, \A)
				\end{align*}
			\item Let $m \geq 1$ be fixed. Then
				\begin{align*}
					h_\mu\left(T, \bigjoin_{j = 0}^{m}{T^{-j}{\A}}\right) &= \lim_{n \to +\infty}{\frac{1}{n} H_\mu\left(\bigjoin_{j = 0}^{n - 1}{T^{-j}\left(\bigjoin_{k = 0}^{m}{T^{-k}{\A}}\right)}\right)} \\
						&= \lim_{n \to +\infty}{\frac{1}{n} H_\mu\left(\bigjoin_{j = 0}^{n + m - 1}{T^{-j}{\A}}\right)} \\
						&= \lim_{n \to +\infty}{\left(\frac{m + n}{n}\right) \frac{1}{m + n} H_\mu\left(\bigjoin_{j = 0}^{n + m - 1}{T^{-j}{\A}}\right)} \\
						&= h_\mu(T, \A)
				\end{align*}
			\item Suppose that $T$ is invertible and fix $m \geq 1$. We have
				\begin{align*}
					h_\mu\left(T, \bigjoin_{j = -m}^{m}{T^{-j}{\A}}\right) &= h_\mu\left(T, \bigjoin_{j = 0}^{2m}{T^{-j}{\A}}\right) & \text{(by \ref{walters:thm-4-12:5})} \\
						&= h_\mu(T, \A) & \text{(by \ref{walters:thm-4-12:6}).}
				\end{align*}
		\end{enumerate}
	\end{proof}
\end{theorem}

There is an alternative definition for $h_\mu(T, \alpha)$ which is useful if we want to utilise results relating to conditional entropy.

\begin{theorem}
	Suppose that $(X, \B, \mu, T)$ is a measure-preserving transformation of a probability space and that $\alpha$ be a finite partition of $X$. The entropy of $T$ with respect to $\alpha$ (or $\A(\alpha)$) may also be given by
	\[
		h_\mu(T, \A(\alpha)) = h_\mu(T, \alpha) = H_\mu\left(\alpha \midmid \bigjoin_{j = 1}^\infty {T^{-j}{\alpha}}\right).
	\]
	
	\begin{proof}
		We follow the proof given in \cite[Lecture 24]{ergodic-lectures}.
		
		Let $\A := \A(\alpha)$. We have
		\begin{align*}
			H_\mu\left(\bigjoin_{j = 0}^{n - 1}{T^{-j}{\A}}\right) &= H_\mu\left(\bigjoin_{j = 1}^{n - 1}{T^{-j}{\A}}\right) + H_\mu\left(\A \midmid \bigjoin_{j = 1}^{n - 1}{T^{-j}{\A}}\right) & \text{(by \thref{thm:walters-4.3} \ref{walters-thm-4.3:2})} \\
				&= H_\mu\left(\bigjoin_{j = 0}^{n - 2}{T^{-j}{\A}}\right) + H_\mu\left(\A \midmid \bigjoin_{j = 1}^{n - 1}{T^{-j}{\A}}\right) & \text{(by \thref{thm:walters-4.3} \ref{walters-thm-4.3:10})}.
		\end{align*}
		By induction, this means that
		\begin{align*}
			\frac{1}{n} H_\mu\left(\bigjoin_{j = 0}^{n - 1}{T^{-j}{\A}}\right) &= \frac{1}{n}\left[H_\mu\left(\A \midmid \bigjoin_{j = 1}^{n - 2}{T^{-j}{\A}}\right) + H_\mu\left(\A \midmid \bigjoin_{j = 1}^{n - 3}{T^{-j}{\A}}\right)\right. \\
				& \left. \vphantom{\bigjoin_{j = 1}^{n - 2}{T^{-j}{\A}}} \qquad + \dots + H_\mu(\A \mid T^{-1}{\A}) + H_\mu(\A)\right].
		\end{align*}
		By \thref{thm:walters-4.3} \ref{walters-thm-4.3:5} we have
		\[
			H_\mu\left(\A \midmid \bigjoin_{j = 1}^{n - 1}{T^{-j}{\A}}\right) \leq H_\mu\left(\A \midmid \bigjoin_{j = 1}^{n - 2}{T^{-j}{\A}}\right) \leq \dots \leq H_\mu(\A).
		\]
		We may now apply \thref{thm:increasing-martingale} to get that
		\[
			H_\mu\left(\A \midmid \bigjoin_{j = 1}^{n - 1}{T^{-j}{\A}}\right) \to H_\mu\left(\A \midmid \bigjoin_{j = 1}^\infty{T^{-j}{\A}}\right),
		\]
		as $n \to +\infty$, and therefore
		\begin{align*}
			h_\mu(T, \A) = \lim_{n \to +\infty}{\frac{1}{n} H_\mu\left(\bigjoin_{j = 0}^{n - 1}{}T^{-j}{\A}\right)} &= H_\mu\left(\A \midmid \bigjoin_{j = 1}^\infty{T^{-j}{\A}}\right).
		\end{align*}
	\end{proof}
\end{theorem}

Finally, we can think of the entropy of a measure-preserving transformation $T$, regardless of the choice of partition, to be maximal average information we can gain per day. We state this formally as follows.

\begin{definition}
	Let $(X, \B, \mu, T)$ be a measure-preserving transformation of a probability space. The \key{entropy of $T$} is defined
	\[
		h_\mu(T) := \sup_{\alpha}{h_\mu(T, \alpha)} = \sup_{\A}{h_\mu(T, \A)},
	\]
	where the supremum is taken over all finite measurable partitions $\alpha$ or finite sub-$\sigma$-algebras of $\B$, respectively.
\end{definition}

\begin{remark}
	If $T = \id_X$, then $h(T) = 0$. In general, if $h(T) = 0$, then $h(T, \alpha) = 0$ for all finite partitions $\alpha$. This means that we don't obtain `much' new information each day, i.e. $\bigvee_{j = 0}^{n - 1}{T^{-j}{\alpha}}$ doesn't change `much', as $n \to +\infty$. One such measure-preserving transformation which has this property is the identity transformation on $X$.
\end{remark}

\begin{theorem}
	Entropy is a conjugacy invariant and hence is also an isomorphism invariant.
	\begin{proof}
		Let $(X_1, \B_1, \mu_1, T_1), (X_2, \B_2, \mu_2, T_2)$ be measure-preserving transformations of probability spaces. Let $\Phi : (\tilde{B}_2, \tilde{\mu}_2) \to (\tilde{B}_1, \tilde{\mu}_1)$ be an isomorphism of measure algebras such that $\Phi \circ \tilde{T}_2^{-1} = \tilde{T}_1^{-1} \circ \Phi$. We aim to show that $h_{\mu_1}(T_1) = h_{\mu_2}(T_2)$.
		
		Let $\A_2$ be an arbitrary finite sub-$\sigma$-algebra of $\B_2$ and write $\alpha(\A_2) = \{A_1, \dots, A_r\}$. Since $\Phi$ is an isomorphism of measure algebras, we can choose $C_j \in \B_1$ such that $\tilde{C}_j = \Phi(\tilde{A}_j)$. Using this, we define $\gamma := \{C_1, \dots, C_r\}$, which we see is a partition of $(X_1, \B_1, \mu_1)$. We write $\A_1 := \A(\gamma)$.
		
		For any $(q_0, q_1, \dots, q_{n - 1})$, where $q_j \in \{1, \dots, r\}$ for each $j$, we have
		\begin{align*}
			\Phi\left(\bigcap_{j = 0}^{n - 1}{(T_2^{-j} A_{q_j})^\sim}\right) &= \Phi\left(\bigcap_{j = 0}^{n - 1}{\tilde{T}_2^{-j} \tilde{A}_{q_j}}\right) \\
				&= \bigcap_{j = 0}^{n - 1}{\tilde{T}_1^{-j} \Phi(\tilde{A}_{q_j})} \\
				&= \bigcap_{j = 0}^{n - 1}{\tilde{T}_1^{-j} \tilde{C}_{q_j}} \\
				&= \bigcap_{j = 0}^{n - 1}{(T_1^{-j} C_{q_j})^\sim}.
		\end{align*}
		Hence the sets $\bigcap_{j = 0}^{n - 1}{(T_2^{-j} A_{q_j})^\sim}$ and $\bigcap_{j = 0}^{n - 1}{(T_1^{-j} C_{q_j})^\sim}$ have the same measure. Recall that the entropy of a partition is completely determined by the measure of the elements in the partition. This means that
		\[
			H_{\mu_1}\left(\bigjoin_{j = 0}^{n - 1}{T_1^{-j}{\A_1}}\right) = H_{\mu_2}\left(\bigjoin_{j = 0}^{n - 1}{T_2^{-j}{\A_2}}\right),
		\]
		and hence $h_{\mu_1}(T_1, \A_1) = h_{\mu_2}(T_2, \A_2)$. Since $\A_2$ was chosen to be an arbitrary sub-$\sigma$-algebra of $\B_1$, this means that $h_{\mu_1}(T_1) \geq h_{\mu_2}(T_2)$.
		
		We repeat the proof, but choose an arbitrary finite sub-$\sigma$-algebra of $\B_1$ to get the reverse inequality $h_{\mu_1}(T_1) \leq h_{\mu_2}(T_2)$, and hence $h_{\mu_1}(T_1) = h_{\mu_2}(T_2)$.
	\end{proof}
\end{theorem}

\section{Calculating \texorpdfstring{$h_\mu(T)$}{h(T)}}
Recall that the entropy of a measure-preserving transformation $T$ is defined $h_\mu(T) := \sup_{\A}{h_\mu(T, \A)}$, where the supremum is taken over all finite sub-$\sigma$-algebras of $\B$ (or, equivalently, over all finite partitions of $(X, \B, \mu)$). Of course, it is difficult to consider all finite partitions, so we want to find criteria which guarantee that $h_\mu(T) = h_\mu(T, \A)$ instead.

One key result is the \key{Kolmogorov-Sinai Theorem}. We will prove this in \thref{thm:kolmogorov-sinai}, but to do this we need some preliminary results.

\begin{lemma} \label{lem:walters-4-15}
	Let $r \in \naturals$ be fixed and let $\varepsilon > 0$ be given. Then there exists $\delta > 0$ such that, if we have two partitions of $r$ sets $\alpha = \{A_1, \dots, A_r\}$, $\gamma = \{C_1, \dots, C_r\}$ of $(X, \B, \mu)$ such that
	\[
		\sum_{j = 1}^r{\mu(A_j \symdiff C_j)} < \delta,
	\]
	then we have $H_\mu(\alpha \mid \gamma) + H_\mu(\gamma \mid \alpha) < \varepsilon$.
	
	\begin{proof}
		Let $\varepsilon > 0$ be given. We choose $\delta > 0$ such that $\delta < 1 / 4$ and
		\[
			-r(r - 1) \delta \log{\delta} - (1 - \delta) \log(1 - \delta) < \frac{\varepsilon}{2}.
		\]
		
		Let $\beta$ be the partition of $(X, \B, \mu)$ consisting of the sets of the form $A_j \cap C_k$, where $j \neq k$, and the set $\bigcup_{j = 1}^r{A_j \cap C_j}$. It is then clear that $\alpha \join \gamma = \gamma \join \beta$. For $j \neq k$, we also have
		\[
			A_j \cap C_k \subset \bigcup_{n = 1}^r{A_n \symdiff C_n}.
		\]
		This gives, by the hypothesis, $\mu(A_j \cap C_k) < \delta$ for $j \neq k$. By the definition of symmetric difference, we also have
		\[
			\mu\left(\bigcup_{j = 1}^r{A_j \cap C_j}\right) > 1 - \delta.
		\]
		Hence
		\begin{align*}
			H_\mu(\beta) &= -\sum_{j \neq k}{\mu(A_j \cap C_k) \log{\mu(A_j \cap C_k)}} - \mu\left(\bigcup_{j = 1}^r{A_j \cap C_j}\right) \log{\mu\left(\bigcup_{j = 1}^r{A_j \cap C_j}\right)} \\
				&< -r(r - 1) \delta \log{\delta} - (1 - \delta) \log(1 - \delta) \\
				&< \frac{\varepsilon}{2}.
		\end{align*}
		We therefore have
		\begin{align*}
			H_\mu(\gamma) + H_\mu(\alpha \mid \gamma) &= H_\mu(\alpha \join \gamma) & \text{(by \thref{thm:walters-4.3} \ref{walters-thm-4.3:2})} \\
				&= H_\mu(\gamma \join \beta) \\
				&\leq H_\mu(\gamma) + H_\mu(\beta) & \text{(by \thref{thm:walters-4.3} \ref{walters-thm-4.3:8})} \\
				&< H_\mu(\gamma) + \frac{\varepsilon}{2},
		\end{align*}
		and hence $H_\mu(\alpha \mid \gamma) < \varepsilon / 2$.
		
		We repeat this argument using $\alpha \join \gamma = \alpha \join \beta$ to get $H_\mu(\gamma \mid \alpha) < \varepsilon / 2$. Combining these two results, we get $H_\mu(\alpha \mid \gamma) + H_\mu(\gamma \mid \alpha) < \varepsilon$.
	\end{proof}
\end{lemma}

\begin{theorem} \label{thm:walters-4-16}
	Suppose that $\C$ is a finite sub-$\sigma$-algebra of $\B$ and that $\B_0$ is an algebra such that $\B(\B_0) = \B$ $\mu$-almost everywhere. Then given any $\varepsilon > 0$, there exists a finite algebra $\D \subset \B_0$ such that
	\[
		H_\mu(\D \mid \C) + H_\mu(\C \mid \D) < \varepsilon.
	\]
	
	\begin{proof}
		Let $\varepsilon > 0$ be given and write $\alpha(\C) = \{C_1, \dots, C_r\}$. We choose $\delta > 0$ as in \thref{lem:walters-4-15}, where $r, \varepsilon$ here are as in the lemma. It suffices to show that, for each $\tau > 0$, there exists a partition $\D = \{D_1, \dots, D_r\}$, where $D_j \in \B_0$ and $\mu(C_j \symdiff D_j) < \tau$, for all $j = 1, \dots, r$. This is because we may then choose $\tau$ such that $r\tau \leq \delta$ and then apply \thref{lem:walters-4-15}.
		
		To begin, we choose $\lambda > 0$ such that $\lambda(r - 1)[1 + r(r - 1)] < \tau$. For each $j = 1, \dots, r$, choose $B_j \in \B_0$ such that $\mu(C_j \symdiff B_j) < \lambda$. Now if $j \neq k$, then $B_j \cap B_k \subset (B_j \symdiff C_j) \cup (\B_j \symdiff C_j)$. It follows that $\mu(B_j \cap B_k) < 2\lambda$. We let $N := \bigcup_{j \neq k}{(B_j \cap B_k)}$, so that $\mu(N) < r(r - 1)\lambda$.
		
		Now for $j = 1, \dots, r - 1$ we define $D_j = B_j \setminus N$, and $D_r = X \setminus \bigcup_{j = 1}^{r - 1}{D_j}$. This clearly defines a partition $\D := \{D_1, \dots, D_r\}$ of $X$, and each $D_j \in \B_0$, since $\B_0$ is an algebra (i.e. is closed under finite unions and complementation).
		
		If $j < r$, then $D_j \symdiff C_j \subset (B_j \symdiff C_j) \cup N$. Then by countable subadditivity,
		\begin{align*}
			\mu(D_j \symdiff C_j) &\leq \mu(B_j \symdiff C_j) + \mu(N) \\
				&< \lambda + r(r - 1)\lambda \\
				&= \lambda[1 + r(r - 1)] \\
				&< \tau.
		\end{align*}
		For the last part of $\D$, we use the fact that $D_r \symdiff C_r \subset \bigcup_{j = 1}^{r - 1}{(D_j \symdiff C_j)}$. Then $\mu(D_r \symdiff C_r) < (r - 1)\lambda[1 + r(r - 1)] < \tau$.
	\end{proof}
\end{theorem}

\begin{corollary} \label{cor:walters-4-16-1}
	Let $\A_1 \subset \A_2 \subset \dots \subset \A_n \subset \dots$ be an increasing sequence of finite sub-algebras of $\B$. Suppose that $\C$ is a finite sub-algebra of $\B$ such that $\C \subset \bigjoin_{n \geq 1}{\A_n}$ $\mu$-almost everywhere. Then $H_\mu(\C \mid \A_n) \to 0$, as $n \to +\infty$.
	
	Note: We say that $\A \subset \C$ $\mu$-almost everywhere if, for all $A \in \A$, there exists $C \in \C$ such that $\mu(A \symdiff C) = 0$.
	
	\begin{proof}
		Let $\varepsilon > 0$ be given. Write $\B_0 := \bigcup_{n = 1}^\infty{\A_n}$ so that $\B_0$ is an algebra. Since $\C \subset \bigjoin_{n \geq 1}{\A_n}$ $\mu$-almost everywhere, we have $\C \subset \B(\B_0)$ $\mu$-almost everywhere. By \thref{thm:walters-4-16}, there exists a finite sub-algebra $\D_\varepsilon$ of $\B_0$ such that $H_\mu(\C \mid \D_\varepsilon) < \varepsilon$.
		
		Since $\A_n$ is increasing and $\D_\varepsilon$ is finite, we have $\D_\varepsilon \subset \A_{n_0}$ for some $n_0 \in \naturals$. Then for all $n \geq n_0$, we have
		\[
			H_\mu(\C \mid \A_n) \leq H_\mu(\C \mid \A_{n_0}) \leq H_\mu(\C \mid \D_\varepsilon) < \varepsilon.
		\]
		Hence $H_\mu(\C \mid \A_n) \to 0$, as $n \to +\infty$.
	\end{proof}
\end{corollary}

We are ready to prove one of the main results of this chapter.

\begin{theorem}[Kolmogorov-Sinai Theorem] \label{thm:kolmogorov-sinai}
	Let $T : X \to X$ be an invertible measure-preserving transformation of a probability space $(X, \B, \mu)$. Suppose that $\A$ is a finite sub-$\sigma$-algebra of $\B$ such that
	\[
		\bigjoin_{n = -\infty}^\infty{T^{-n}{\A}} = \B
	\]
	$\mu$-almost everywhere. Then $h_\mu(T) = h_\mu(T, \A)$.
	
	\begin{proof}
		Let $\C$ be a finite sub-$\sigma$-algebra of $\B$. We want to show that $h_\mu(T, \C) \leq h_\mu(T, \A)$, i.e. $h_\mu(T, \A)$ achieves the supremum as in the definition of $h_\mu(T)$. We have
		\begin{align*}
			h_\mu(T, \C) &\leq h_\mu\left(T, \bigjoin_{j = -n}^n{T^{-j}{\A}}\right) + H_\mu\left(\C \midmid \bigjoin_{j = -n}^n{T^{-j}{\A}}\right) & \text{(by \thref{thm:walters-4.12} \ref{walters:thm-4-12:4})} \\
				&= h_\mu(T, \A) + H_\mu\left(\C \midmid \bigjoin_{j = -n}^n{T^{-j}{\A}} \right) & \text{(by \thref{thm:walters-4.12} \ref{walters:thm-4-12:7})}
		\end{align*}
		We let $\A_n := \bigjoin_{j = -n}^n{T^{-j}{\A}}$ so that $\A_n$ is an increasing sequence of finite sub-algebras of $\B$. Since $\bigjoin_{n = -\infty}^\infty{T^n{\A}} = \B$ $\mu$-almost everywhere, and $\C \subset \B$, we may apply \thref{cor:walters-4-16-1}. So $H_\mu(\C \mid \A_n) \to 0$, as $n \to +\infty$, and hence $h_\mu(T, \C) \leq h_\mu(T, \A)$.
	\end{proof}
\end{theorem}

There is a similar result which does not require that $T$ is invertible.

\begin{theorem} \label{thm:walters-4-18}
	Let $T : X \to X$ be a (not necessarily invertible) measure-preserving transformation of a probability space $(X, \B, \mu)$. Suppose that $\A$ is a finite sub-algebra of $\B$ such that
	\[
		\bigjoin_{n = 0}^\infty{T^{-n}{\A}} = \B
	\]
	$\mu$-almost everywhere. Then $h_\mu(T) = h_\mu(T, \A)$.
	
	\begin{proof}
		We repeat the same proof for \thref{thm:kolmogorov-sinai}, but replace $\bigjoin_{j = -n}^n{T^{-j}{\A}}$ with $\bigjoin_{j = 0}^n{T^{-j}{\A}}$ and apply \thref{thm:walters-4.12} \ref{walters:thm-4-12:6} instead of \ref{walters:thm-4-12:7}.
	\end{proof}
\end{theorem}

The following result gives a useful criterion for deciding if an invertible measure-preserving transformation has zero entropy.

\begin{corollary}
	Suppose that $T : X \to X$ is an invertible measure-preserving transformation of a probability space $(X, \B, \mu)$, and that
	\[
		\bigjoin_{n = 0}^\infty{T^{-n}{\A}} = \B
	\]
	$\mu$-almost everywhere for some finite sub-algebra $\A$ of $\B$. Then $h_\mu(T) = 0$.
	
	\begin{proof}
		By \thref{thm:walters-4-18}, we have
		\[
			h_\mu(T) = h_\mu(T, \A) = \lim_{n \to +\infty}{H_\mu\left(\A \midmid \bigjoin_{j = 1}^n{T^{-j}{\A}}\right)}.
		\]
		Since $T$ is a measure-preserving transformation, we have $\bigjoin_{j = 1}^\infty{T^{-j}{\A}} = T^{-1}{\B} = \B$ $\mu$-almost everywhere.
		
		We write $\A_n := \bigjoin_{j = 1}^\infty{T^{-j}{\A}}$, so that $\A_n$ is an increasing sequence of sub-algebras of $\B$, and $\bigjoin_{n = 1}^\infty{\A_n} = \B$ $\mu$-almost everywhere. In particular, for all $n \geq 1$ we have $\A \subset \A_n$ $\mu$-almost everywhere and so we may apply \thref{cor:walters-4-16-1}. So $H_\mu(\A \mid \A_n) \to 0$, as $n \to +\infty$. Hence $h_\mu(T) = 0$.
	\end{proof}
\end{corollary}

\section{Shifts of finite type} \label{sec:entropy:sft}
We now consider entropy for shifts of finite type, which we will be interested in later on. The following definitions appear in \cite{chazottes-maldonado:cbfee}.

Let $A$ be a finite alphabet and let $\Sigma := \{(x_j)_{j = 0}^\infty \mid x_j \in A\}$ denote the full (one-sided) shift. As usual, $\sigma : \Sigma \to \Sigma$ will be the (one-sided) shift map. For the sake of readability, if $(x_j)_{j = 0}^\infty \in \Sigma$, then we will write $x_m^n := (x_j)_{j = m}^n$.

In this setting, our measure-preserving transformation will always be $\sigma$ and so we are more interested in $\sigma$-invariant probability measures than the shift map itself. We will define the entropy of such measures in the remainder of this chapter and we will see that each definition is consistent with our previous discussion.

For the remainder of this section, we let $\alpha_k := \{[a_0^{k - 1}] \mid a_0^{k - 1} \in A^k\}$ denote the partition of $(\Sigma, \B, \nu)$ by cylinders of length $k$.

\begin{definition}
	Let $\nu$ be a $\sigma$-invariant probability measure on $\Sigma$. For shifts of finite type, for $k > 1$ we can define the \key{$k$-block entropy} of $\nu$ by
	\[
		H_k(\nu) := -\sum_{a_0^{k - 1} \in A^k}{\nu[a_0^{k - 1}] \log\nu[a_0^{k - 1}]},
	\]
	where the sum is taken over all sequences of length $k$.
\end{definition}

By definition, the entropy of $\alpha_k$ is
\[
	H_\nu(\alpha_k) = -\sum_{a_0^{k - 1} \in A^k}{\nu[a_0^{k - 1}] \log\nu[a_0^{k - 1}]} = H_k(\nu).
\]
In other words, $k$-block entropy is the entropy of the partition of $\Sigma$ by cylinders of length $k$.

\begin{definition}
	Let $\nu$ be a $\sigma$-invariant probability measure on $\Sigma$. The \key{entropy} of $\nu$ is given by
	\[
		h(\nu) := \lim_{k \to +\infty}{\frac{1}{k} H_k(\nu)}.
	\]
\end{definition}

Let $\A$ be the sub-algebra of $\B$ consisting of unions cylinders of length $1$. Then $\bigjoin_{k = 0}^\infty{\sigma^{-k}{\A}} = \B$ and hence \thref{thm:walters-4-18} applies. So
\[
	h_\nu(\sigma) = h_\nu(\sigma, \A) = \lim_{k \to +\infty}{\frac{1}{k} H_\nu\left(\bigjoin_{j = 0}^{k - 1}{\sigma^{-j}{\A}}\right)} = \lim_{k \to +\infty}{\frac{1}{k} H_k\left(\nu\right)} = h(\nu).
\]
Therefore this definition is consistent with our previous results.

\begin{definition}
	The \key{conditional $k$-block entropy}, where $k \geq 2$, is defined
	\[
		h_k(\nu) := -\sum_{a_0^{k - 1} \in A^k}{\nu[a_0^{k - 1}] \log{\frac{\nu[a_0^{k - 1}]}{\nu[a_0^{k - 2}]}}}.
	\]
\end{definition}

Note that $[a_0^{k - 1}] \subset [a_0^{k - 2}]$ and hence for $C \in \alpha_{k - 1}$,
\[
	[a_0^{k - 1}] \cap C =
	\begin{cases}
		\left[a_0^{k - 1}\right],	& \text{if } C = [a_0^{k - 2}]; \\
		\emptyset,	& \text{otherwise}.
	\end{cases}
\]
Then by the definition of conditional entropy we have
\begin{align*}
	H_\nu(\alpha_k \mid \alpha_{k - 1}) &= -\sum_{a_0^{k - 1} \in A^k}{\nu([a_0^{k - 2}] \cap [a_0^{k - 1}]) \log{\frac{[a_0^{k - 2}] \cap [a_0^{k - 1}]}{[a_0^{k - 2}]}}} \\
		&= -\sum_{a_0^{k - 1} \in A^k}{\nu[a_0^{k - 1}] \log{\frac{[a_0^{k - 1}]}{[a_0^{k - 2}]}}} \\
		&= h_k(\nu),
\end{align*}
and so the definitions are once again consistent. For $k \geq 2$, we also clearly have the relation
\[
	h_k(\nu) = H_k(\nu) - H_{k - 1}(\nu).
\]

Finally, we have \key{relative $k$-block entropy} which we will use later.

\begin{definition}
	For $k \geq 1$, the \key{$k$-block relative entropy} of a measure $\nu$ with respect to a measure $\mu$ is defined
	\[
		H_k(\nu \mid \mu) = \sum_{a_0^{k - 1} \in A^k}{\nu[a_0^{k - 1}] \log{\frac{\nu[a_0^{k - 1}]}{\mu[a_0^{k - 1}]}}}.
	\]
\end{definition}

\chapter[Concentration bounds for entropy estimation]{Concentration bounds for entropy estimation of Gibbs measures}\label{chap:concentration-bounds}
\section{Overview}
We now have the required background knowledge to discuss \cite{chazottes-maldonado:cbfee}, which is concerned with methods for estimating the entropy of Gibbs measures. In particular, we will show that these methods yield estimated entropy values which concentrate around the actual value of entropy.

\section{Restrictions and notation}
\subsection{Full shifts}
Let $A$ be a finite alphabet. Throughout this chapter, $\Sigma = \{(x_j)_{j = 0}^\infty \mid x_j \in A\}$ will denote the full (one-sided) shift with the (one-sided) shift map $\sigma : \Sigma \to \Sigma$. Once again, we will write $x_m^n := (x_j)_{j = m}^n$. We will use the definitions from Section \ref{sec:entropy:sft}.

\subsection{Gibbs measures}
Let $\phi \in F_\theta$. Recall that if there exist constants $C = C(\phi) > 1$, $P = P(\phi)$ such that
\[
	C^{-1} \leq \frac{\mu_\phi[x_0^{n - 1}]}{\exp\left(-Pn + \sum_{j = 0}^{n - 1}{\phi(\sigma^j x)}\right)} \leq C,
\]
then $\mu_\phi$ is a Gibbs measure for $\phi$.

Throughout this chapter, we will assume that the pressure $P$ of $\phi$ is zero. With this in mind, recall that
Gibbs measures satisfy the Variational Principle (\thref{thm:variational-principle}), so we have
\[
	0 = P = h(\mu_\phi) + \int{\phi\ d\mu_\phi}.
\]
This gives the identity
\begin{equation}\label{fml:vp-identity}
	h(\mu_\phi) = -\int{\phi\ d\mu_\phi},
\end{equation}
which will be useful later in this chapter.

\subsection{Probability theory}
We will use and prove some results related to a couple of basic ideas in probability theory.
\begin{definition}
	The \key{expectation} of a continuous function $f : X \to \reals$ with respect to a probability measure $\mu$ is given by
	\[
		E_\mu(f) := \int{f\ d\mu}.
	\]
	The expectation is a weighted average, or mean, of the values $f$ takes.~\cite[p127]{gray:probability}
\end{definition}

\begin{definition}
	The \key{variance} of a function $f : X \to \reals$ with respect to a probability measure $\mu$ is given by
	\[
		\Var_{\mu}(f) := \int{\left(K(x) - \int{K(y)\ d\mu(y)}\right)^2\ d\mu(x)}.
	\]
	The variance gives a measurement of the spread of the values $f$ takes.
\end{definition}

\section{An exponential inequality}
Throughout the remainder of this chapter, we will assume that $\phi \in F_\theta$ and that $\mu_\phi$ is the unique Gibbs measure for $\phi$, which follows since $P(\phi) = 0$.

The results in Section \ref{sec:estimator-bounds} utilise an exponential inequality proved in \cite{collet-martinez-schmitt:exp-ineq}. We first introduce some definitions which are used in the inequality.

\begin{definition}
	Let $K : \Sigma^n \to \reals$ be a function of $n$ variables in $\Sigma$. For $j = 0, \dots, n - 1$, we define
	\[
		\Lip_j(K) := \sup_{\substack{x^{(0)}, x^{(1)}, \dots, x^{(n - 1)} \\ y^{(j)} \neq x^{(j)}}}{\frac{\left|K(x^{(0)}, \dots, x^{(n - 1)}) - K(x^{(0)}, \dots, x^{(j - 1)}, y^{(j)}, x^{(j + 1)}, \dots, x^{(n - 1)})\right|}{d_\theta(x^{(j)}, y^{(j)})}},
	\]
	where each $x^{(k)}, y^{(k)}$ is a sequence in $\Sigma$ for $k = 0, \dots, n - 1$.
	
	In other words, $\Lip_j(K)$ is a measurement of how much $K$ varies when the $j$-th variable is changed. There is an additional weighting which depends on how close the old $j$-th variable is to the new $j$-th variable.
	
	If $\Lip_j(K) < +\infty$ for all $j = 0, \dots, n - 1$, then we say that $K$ is a \key{separately Lipschitz function} of $n$ variables.
\end{definition}

We now present the exponential inequality, which appears in its original form in \cite[Theorem I.1]{collet-martinez-schmitt:exp-ineq}.

\begin{theorem}\label{thm:cm-3-1}
	Let $\mu_\phi$ be a Gibbs measure. Then there exists some constant $D = D(\phi) > 0$ such that
	\begin{equation}\label{fml:cms-exp-ineq}
		\int{e^{K(x, \dots, \sigma^{n - 1}{x})}\ d\mu_\phi(x)} \leq e^{\int{K(y, \dots, \sigma^{n - 1}{y})\ d\mu_\phi(y)}} \; e^{D \sum_{j = 0}^{n - 1}{\Lip_j^2(K)}}.
	\end{equation}
	 for all $n \in \naturals$ and any separately Lipschitz function $K : \Sigma^n \to \reals$.
\end{theorem}

We may write \eqref{fml:cms-exp-ineq} more succinctly by defining a measure $\mu_\phi^{(n)}$ on $\Sigma^n$ by
\[
	d\mu_\phi^{(n)}(x^{(0)}, \dots, x^{(n - 1)}) = d\mu_\phi(x^{(0)}) \prod_{j = 1}^{n - 1}{\delta(x^{(j)} - \sigma^j{x^{(0)}})},
\]
where $\delta(x - a) = \delta_a(x)$ is the Dirac measure at $a$. Using this, \eqref{fml:cms-exp-ineq} becomes
\[
	\int{e^K\ d\mu_\phi^{(n)}} \leq e^{\int{K\ d\mu_\phi^{(n)}}} \; e^{D \sum_{j = 0}^{n - 1}{\Lip_j^2(K)}}.
\]

\subsection{Important results}
We will mostly use the results in this subsection which follow from \thref{thm:cm-3-1}. To begin, we need Markov's Inequality.

\begin{lemma}[Markov's Inequality]\label{lem:markov-ineq}
	Let $f \geq 0$ be a nonnegative measurable function on a measure space $(X, \B, \mu)$. Then for all $t > 0$ we have
	\[
		\mu\{x \in X \mid f(x) \geq t\} \leq \frac{1}{t}\int{f\ d\mu}.
	\]
	
	\begin{proof}
		We follow the proof given in \cite[Theorem 3.1.1]{athreya-lahiri:measure-theory}.
		
		We have $f \geq 0$ and so
		\begin{align*}
			\int{f\ d\mu} &\geq \int_{\{f \geq t\}}{f\ d\mu} \\ &= \mu\{f(x) \mid x \in X \text{ and } f(x) \geq t \} \\ &\geq t\mu\{x \in X \mid f(x) \geq t \}.
		\end{align*}
		The result follows by dividing by $t$.
	\end{proof}
\end{lemma}

\begin{corollary}\label{cor:cm-3-1}
	For all $t > 0$ we have
	\begin{align}\label{fml:cm-4}
		\mu_\phi&\left\{x \midmid K(x, \sigma{x}, \dots, \sigma^{n - 1}{x}) \geq \int{K(y, \sigma{y}, \dots, \sigma^{n - 1}{y})\ d\mu_\phi(y)} + t\right\} \nonumber \\
			&\leq \exp\left(-\frac{t^2}{4D\sum_{j = 0}^{n - 1}{\Lip_j^2(K)}}\right),
	\end{align}
	for all $n \in \naturals$ and all separately Lipschitz $K$.
	\begin{proof}
		Since $K$ is a separately Lipschitz function, it is clear that $\lambda K$ is also a separately Lipschitz function for all $\lambda > 0$.
		
		Now, for all $\lambda > 0$ and $t > 0$ we have
		\begin{align}
			\mu_\phi&\left\{x \midmid K(x, \sigma{x}, \dots, \sigma^{n - 1}{x}) \geq \int{K(y, \sigma{y}, \dots, \sigma^{n - 1}{y})\ d\mu_\phi(y)} + t\right\} \nonumber \\
				&= \mu_\phi\left\{x \midmid e^{\lambda\left[K(x, \sigma{x}, \dots, \sigma^{n - 1}{x}) - \int{K(y, \sigma{y}, \dots, \sigma^{n - 1}{y})\ d\mu_\phi(y)}\right]} \geq e^{\lambda t}\right\} \nonumber \\
				&\leq \frac{1}{e^{\lambda t}} \int{e^{\lambda\left[K(x, \sigma{x}, \dots, \sigma^{n - 1}{x}) - \int{K(y, \sigma{y}, \dots, \sigma^{n - 1}{y})\ d\mu_\phi(y)}\right]}\ d\mu_\phi(x)} \nonumber \\
				& \leq \exp(-\lambda t) \exp\left(\lambda^2 D \sum_{j = 0}^{n - 1}{\Lip_j^2(K)}\right). \label{fml:cor-3-1-exp}
		\end{align}
		(We have used \thref{lem:markov-ineq} on the penultimate line and \thref{thm:cm-3-1} on the last line.)
		
		We now optimise \eqref{fml:cor-3-1-exp} over $\lambda$, that is, we find the value of $\lambda$ for which \eqref{fml:cor-3-1-exp} has derivative zero with respect to $\lambda$. We have
		\begin{align*}
			0 &= \frac{d}{d\lambda} \exp\left(-\lambda t + \lambda^2 D \sum_{j = 0}^{n - 1}{\Lip_j^2(K)}\right) \\
				&= \left(-t + 2\lambda D \sum_{j = 0}^{n - 1}{\Lip_j^2(K)}\right) \exp\left(-\lambda t + \lambda^2 D \sum_{j = 0}^{n - 1}{\Lip_j^2(K)}\right).
		\end{align*}
		Since $\exp\left(-\lambda t + \lambda^2 D \sum_{j = 0}^{n - 1}{\Lip_j^2(K)}\right) > 0$ for all $\lambda > 0$, we get that
		\[
			\lambda = \frac{t}{2D \sum_{j = 0}^{n - 1}{\Lip_j^2(K)}}.
		\]
		Substituting this value of $\lambda$ into \eqref{fml:cor-3-1-exp} gives
		\begin{align*}
			\mu_\phi&\left\{x \midmid K(x, \sigma{x}, \dots, \sigma^{n - 1}{x}) \geq \int{K(y, \sigma{y}, \dots, \sigma^{n - 1}{y})\ d\mu_\phi(y)} + t\right\} \\
			 &\leq \exp\left(-\frac{t^2}{2D \sum_{j = 0}^{n - 1}{\Lip_j^2(K)}} + \frac{t^2 D \sum_{j = 0}^{n - 1}{\Lip_j^2(K)}}{4D^2 \left(\sum_{j = 0}^{n - 1}{\Lip_j^2(K)}\right)^2}\right) \\
			 &= \exp\left(-\frac{t^2}{4D\sum_{j = 0}^{n - 1}{\Lip_j^2(K)}}\right),
		\end{align*}
		and the result holds.
	\end{proof}
\end{corollary}

The following result is an immediate consequence of \thref{cor:cm-3-1}.

\begin{corollary}\label{cor:cm-3-1-5}
	For all $t > 0$ we have
	\begin{align}\label{fml:cm-4-5}
		\mu_\phi&\left\{x \midmid \left| K(x, \sigma{x}, \dots, \sigma^{n - 1}{x}) - \int{K(y, \sigma{y}, \dots, \sigma^{n - 1}{y})\ d\mu_\phi(y)} \right| \geq t\right\} \nonumber \\
		&\leq 2\exp\left(-\frac{t^2}{4D\sum_{j = 0}^{n - 1}{\Lip_j^2(K)}}\right),
	\end{align}
	for all $n \in \naturals$ and all separately Lipschitz $K$.
	\begin{proof}
		Apply \thref{cor:cm-3-1} for $K$ and $-K$ and then the result follows by countable subadditivity.
	\end{proof}
\end{corollary}

We can also find an upper bound for the variance of $K$ with respect to $\mu_\phi$.

\begin{corollary} \label{cor:cm-3-2}
	For all $n \in \naturals$ and all separately Lipschitz functions $K$, we have
	\[
		\Var_{\mu_\phi}(K) := \int{\left[K(x, \dots, \sigma^{n - 1}{x}) - \int{K(y, \dots, \sigma^{n - 1}{y})\ d\mu_\phi(y)}\right]^2\ d\mu_\phi(x)} \leq 2D\sum_{j = 0}^{n - 1}{\Lip_j^2(K)}.
	\]
	
	\begin{proof}
		Suppose that $\lambda \neq 0$. Applying \thref{thm:cm-3-1} to $\lambda K$, subtracting $1$ and then dividing by $\lambda^2$, we get
		\begin{equation} \label{fml:cor-3-2-variance}
			\frac{1}{\lambda^2}\left(\int{e^{\lambda\left[K(x, \dots, \sigma^{n - 1}{x}) - \int{K(y, \dots, \sigma^{n - 1}{y})\ d\mu_\phi(y)}\right]}\ d\mu_\phi(x)} - 1\right) \leq \frac{1}{\lambda^2}\left(e^{\lambda^2 D \sum_{j = 0}^{n - 1}{\Lip_j^2(K)}} - 1\right).
		\end{equation}
		Now put $L := K(x, \dots, \sigma^{n - 1}{x}) - \int{K(y, \dots, \sigma^{n - 1}{y})\ d\mu_\phi(y)}$ and note that $\int{L\ d\mu_\phi(x)} = 0$. The Taylor expansion of the left-hand side of \eqref{fml:cor-3-2-variance} is
		\begin{align*}
			\frac{1}{\lambda^2}&\left[\int{\left(1 + \lambda L + \frac{\lambda^2 L^2}{2!} + \frac{\lambda^3 L^3}{3!} + \dots\right)\ d\mu_\phi(x)} - 1\right] \\
				&= \frac{1}{\lambda^2}\left(\int{1 + \lambda L\ d\mu_\phi(x)} - 1\right) + \frac{L^2}{2} + \int{O(\lambda)\ d\mu_\phi(x)} \\
				&= \frac{L^2}{2} + \int{O(\lambda)\ d\mu_\phi(x)} \\
				&\to \frac{L^2}{2},
		\end{align*}
		as $\lambda \to 0$.
		
		The Taylor expansion of the right-hand side of \eqref{fml:cor-3-2-variance} is
		\begin{align*}
			\frac{1}{\lambda^2}&\left[\left(1 + \lambda^2 D \sum_{j = 0}^{n - 1}{\Lip_j^2(K)} + \lambda^4 D^2 \left(\sum_{j = 0}^{n - 1}{\Lip_j^2(K)} \right)^2 + \dots\right) - 1 \right] \\
				&= D \sum_{j = 0}^{n - 1}{\Lip_j^2(K)} + O(\lambda^2) \\
				&\to D \sum_{j = 0}^{n - 1}{\Lip_j^2(K)},
		\end{align*}
		as $\lambda \to 0$. The result then follows by combining the two sides.
	\end{proof}
\end{corollary}

We can also apply the above results to find results about ergodic averages.

\begin{corollary}\label{cor:cm-3-3}
	Let $f : \Sigma \to \reals$ be a Lipschitz function. Then for all $t > 0$ and all $n \geq 1$ we have
	\begin{equation}
		\mu_\phi\left\{x \midmid \frac{1}{n}\sum_{j = 0}^{n - 1}{f(\sigma^j{x})} - \int{f\ d\mu_\phi} \geq t\right\} \leq \exp(-Bnt^2),
	\end{equation}
	where $B := (4D|f|_\theta^2)^{-1}$.
	\begin{proof}
		Let $K_0(x^{(0)}, x^{(1)}, \dots, x^{(n - 1)}) := f(x^{(0)}) + f(x^{(1)}) + \dots + f(x^{(n - 1)})$. Since $f$ is Lipschitz, it follows that $K_0$ is separately Lipschitz. In particular, note that for all $j = 0, \dots, n - 1$, we have
		\begin{align*}
			\Lip_j(K_0) &= \sup_{y^{(j)} \neq x^{(j)}}{\frac{|f(x^{(j)}) - f(y^{(j)})|}{d_\theta(x^{(j)}, y^{(j)})}} \\
				&\leq \sup_{m \geq 0}\left\{\frac{\var_m(f)}{\theta^m}\right\} \\
				&= |f|_\theta.
		\end{align*}
		
		Now we apply \thref{cor:cm-3-1} to $\frac{1}{n}K_0$, so that
		\begin{align*}
			\mu_\phi&\left\{x \midmid \frac{1}{n}\sum_{j = 0}^{n - 1}{f(\sigma^j{x})} - \frac{1}{n}\int{\sum_{j = 0}^{n - 1}{f(\sigma^j{y})}\ d\mu_\phi(y)} \geq t\right\} \\
			&=\mu_\phi\left\{x \midmid \frac{1}{n}\sum_{j = 0}^{n - 1}{f(\sigma^j{x})} - \int{f\ d\mu_\phi} \geq t\right\} \\
			&\leq \exp\left(-\frac{t^2}{4D\frac{1}{n^2}\sum_{j = 0}^{n - 1}{\Lip_j^2(K_0)}}\right) \\
			&\leq \exp\left(-\frac{nt^2}{4D|f|_\theta^2}\right) \\
			&= \exp(-Bnt^2).
		\end{align*}
	\end{proof}
\end{corollary}

This result implies that, as $t$ increases, the ergodic average $\frac{1}{n}\sum_{j = 0}^{n - 1}{f(\sigma^j{x})}$ is exponentially less likely to deviate from the space average $\int{f\ d\mu_\phi}$. In other words, the ergodic average concentrates around the space average, hence the term \emph{concentration bound}.

\subsection{Functions of \texorpdfstring{$n$}{n} symbols}
We will consider entropy estimators that utilise functions of $n$ \emph{symbols}, as opposed to functions of $n$ variables in $\Sigma$. We will discuss the estimators in more detail in the next section.

It is clear that we can identify a function $\tilde{K} : A^n \to \reals$ of $n$ symbols with a function $K : \Sigma^n \to \reals$ of $n$ sequences. We can therefore apply \thref{thm:cm-3-1} and its corollaries to $\tilde{K}$. However, for each $j = 0, \dots, n - 1$, we must replace $\Lip_j(K)$ in the above results with
\[
	\delta_j(\tilde{K}) := \sup_{\substack{a_0^{n - 1} \in A^k \\ b_j \neq a_j}}{|\tilde{K}(a_0, \dots, a_{n - 1}) - \tilde{K}(a_0, \dots, a_{j - 1}, b_j, a_{j + 1}, \dots, a_{n - 1})|},
\]
the \key{oscillation at the $j$-th coordinate}.

\section{Entropy estimators}\label{sec:estimator-bounds}
\subsection{Motivation}
Suppose we have an ergodic source with a typical sample output $(x_0, x_1, \dots, x_{n - 1}) \in A^n$. If we do not know the measure-preserving transformation which yields this output, we have to use alternative methods to estimate its entropy. To do this, we will consider \key{plug-in estimators} and the \key{hitting-time estimator}.

There are theorems which show that these estimators tend to the entropy $h(\nu)$, as $n \to +\infty$, for almost every sample sequence $(x_0, \dots, x_{n - 1})$. However, as we would expect, the values given by these estimators have some margin of error when $n$ is relatively small, i.e. they \key{fluctuate} around $h(\nu)$. Our aim is to find concentration bounds for these fluctuations.

\subsection{Plug-in estimators}
Before we define any \key{plug-in estimators}, we first introduce some definitions. The main concept employed by plug-in estimators is the \key{empirical frequency} with which a word $a_0^{k - 1}$ of length $k$ occurs in a sample path $x_0^{n - 1}$ of length $n$.\footnote{The word ``empirical'' relates to information gained by observations. In our case, empirical frequency is the frequency observed by taking a sample path and matching it with a word of length $k$.}

\begin{definition}
	The \key{empirical frequency} with which the word $a_0^{k - 1}$ occurs in $x_0^{n - 1}$ is given by
	\[
		\E_k(a_0^{k - 1}; x_0^{n - 1}) := \frac{1}{n}\card\left\{j \in \{0, \dots, n - 1\} \midmid \tilde{x}_j^{j + k - 1} = a_0^{k - 1} \right\},
	\]
	where $\tilde{x} = x_0^{n - 1} x_0^{n - 1} x_0^{n - 1} \dots$ is the sequence of period $n$ obtained by concatenating $x_0^{n - 1}$ continually, i.e. $\tilde{x}_j = x_{j \bmod n}$.
\end{definition}

The definition of $\tilde{x}$ means that $\E_k(\seedot; x_0^{n - 1})$ is a locally $\sigma$-invariant probability measure on $A^k$. That is, for all words $a_1^k \in A^k$ there exists some $a_0 \in A$ such that
\[
	\E_k(a_0^{k - 1}; x_0^{n - 1}) = \E_k(a_1^k; x_0^{n - 1}).
\]

Given an ergodic measure $\nu$, Birkhoff's Ergodic Theorem gives that for $\nu$-almost every $x \in \Sigma$ and for all $k \geq 1$, we have
\[
	\E_k(a_0^{k - 1}; x_0^{n - 1}) = \frac{1}{n}\card\left\{j \in \{0, \dots, n - 1\} \midmid \tilde{x}_j^{j + k - 1} = a_0^{k - 1} \right\} \to \nu[a_0^{k - 1}],
\]
as $n \to +\infty$.

We are now in the position to define the following examples of plug-in entropy estimators, which can be found in \cite[Definition 2.1]{chazottes-gabrielle:large-deviations}.

\begin{definition}
	Suppose $x_0^{n - 1} \in A^n$ is a word of length $n$.
	
	For $k \geq 1$, the \key{$k$-block empirical entropy} is defined
	\[
		\hat{H}_k(x_0^{n - 1}) := H_k(\E_k(\seedot; x_0^{n - 1})).
	\]
	
	For $k \geq 2$, the \key{$k$-block conditional empirical entropy} is defined
	\[
		\hat{h}_k(x_0^{n - 1}) := h_k(\E_k(\seedot; x_0^{n - 1})).
	\]
\end{definition}

We have that
\[
	\frac{1}{k} \hat{H}_k(x_0^{n - 1}) = \frac{1}{k} H_k(\E_k(\seedot; x_0^{n - 1})) \to \frac{1}{k} H_k(\nu),
\]
as $n \to +\infty$, for $\nu$-almost every $x \in \Sigma$. Recalling that $\frac{1}{k} H_k(\nu) \to h(\nu)$, as $k \to +\infty$, we have
\[
	\lim_{k \to +\infty} \lim_{n \to +\infty}{\frac{1}{k} \hat{H}_k(x_0^{n - 1})} = h(\nu).
\]

We can actually remove one of these limits: By \thref{prop:walters-cor-4-2-1}, we easily see that $0 \leq h(\nu) \leq \log{|A|}$ and so we have
\[
	\frac{\log{n}}{\log{|A|}} \leq \frac{\log{n}}{h(\nu)}.
\]
Hence we can define a monotonically increasing function $k : \naturals \to \naturals$ such that $k(n) \leq \frac{\log{n}}{h(\nu)}$. Then a result by Ornstein and Weiss~\cite{shields:ergodic} shows that
\[
	\lim_{n \to +\infty}{\frac{1}{k(n)}\hat{H}_{k(n)}(x_0^{n -1})} = h(\nu),
\]
for $\nu$-almost every $x \in \Sigma$.

For conditional empirical entropy, we have the following result proved in \cite[Theorem II.3.5]{shields:ergodic}.

\begin{theorem}[Entropy-Estimation Theorem] \label{thm:entropy-estimation}
	 Let $\nu$ be ergodic measure and let $\alpha \in (0, 1)$. Let $k : \naturals \to \naturals$ be a function such that $k(n) \leq \frac{(1 - \alpha)}{\log{|A|}}\log{n}$. Then
	 \[
	 	\lim_{n \to +\infty}{\hat{h}_{k(n)}(x_0^{n - 1})} = h(\nu),
	 \]
	 for $\nu$-almost every $x \in \Sigma$.
\end{theorem}

Therefore both kinds of empirical entropy can be used for estimating the actual entropy $h(\nu)$.

We now present a concentration bound for the $k$-block empirical entropy about its expectation value.

\begin{theorem}\label{thm:cm-4-1}
	For all $\alpha \in (0, 1)$, $t > 0$ and $n \geq 2$, if
	\[
		k(n) \leq \frac{\alpha}{2 \log{|A|}}\log{n},
	\]
	then
	\[
		\mu_\phi\left\{\left|\frac{\hat{H}_{k(n)}}{k(n)} - E_{\mu_\phi}\left(\frac{\hat{H}_{k(n)}}{k(n)}\right)\right| \geq t\right\} \leq \exp\left(-\frac{n^{1 - \alpha} t^2}{16D(\log{n})^2}\right),
	\]
	where $D > 0$ is as in \thref{thm:cm-3-1}.
	
	Furthermore, for all $n \geq 2$ we have
	\[
		\Var_{\mu_\phi}\left(\frac{\hat{H}_{k(n)}}{k(n)}\right) \leq 8D\frac{(\log{n})^2}{n^{1 - \alpha}}.
	\]
	\begin{proof}
		Let $k \in \naturals$ and define a function $\tilde{K} : A^n \to \reals$ by $\tilde{K}(s_0, \dots, s_{n - 1}) = \hat{H}_k(s_0^{n - 1})$. Recall that
		\[
			\delta_j(\tilde{K}) := \sup_{\substack{s_0^{n - 1} \in A^k \\ r_j \neq s_j}}{|\tilde{K}(s_0, \dots, s_{n - 1}) - \tilde{K}(s_0, \dots, s_{j - 1}, r_j, s_{j + 1}, \dots, s_{n - 1})|}
		\]
		replaces $\Lip_j$ in \thref{thm:cm-3-1} and its corollaries. To utilise these results, we must estimate $\delta_j$.
		\begin{claim}
			We claim that
			\begin{equation}
				\delta_j(\tilde{K}) \leq 2k|A|^k\frac{\log{n}}{n}, \label{fml:oscil-est}
			\end{equation}
			for $j = 0, \dots, n - 1$.
			
			Indeed, first suppose that $a_0^{k - 1} \in A^k$ is given. By the nature of how $\delta_j(\tilde{K})$ is defined, we want to consider the effect on the value of $\tilde{K}$ when one coordinate in $s_0^{n - 1}$ is changed. Since $\tilde{K}$ is defined in terms of the empirical frequency, we consider the largest effect on $\E_k(a_0^{k - 1}; s_0^{n - 1})$.
			
			If we change exactly one symbol $s_j$ in $s_0^{n - 1}$, then the value of $\E_k(a_0^{k - 1}; s_0^{n - 1})$ can decrease by \emph{at most} $k / n$. This is because, in the worst case scenario, $s_j$ matches all $k$ characters in $a_0^{k - 1}$ (and hence $\tilde{a}$). Likewise, changing one symbol in $s_0^{n - 1}$ can increase the value of $\E_k(a_0^{k - 1}; s_0^{n - 1})$ by at most $k / n$.
			
			By \thref{prop:logs-thm-4-1}, for $\ell, k \in \naturals$ with $\ell+ k \leq n$, we have that
			\[
				\left|\left(\frac{\ell}{n}\right)\log\left(\frac{\ell}{n}\right) - \left(\frac{\ell + k}{n}\right)\log\left(\frac{\ell + k}{n}\right)\right| \leq \frac{k}{n}\log{n}.
			\]
			Since $\tilde{K}(s_0^{n - 1}) = H_k(\E_k(\seedot; s_0^{n - 1}))$, by all the above results we have
			\begin{align*}
				\delta_j(\tilde{K}) &\leq 2\sum_{a_0^{k - 1} \in A^k}{\left|\left(\frac{\ell}{n}\right)\log\left(\frac{\ell}{n}\right) - \left(\frac{\ell + k}{n}\right)\log\left(\frac{\ell + k}{n}\right)\right|} \\
				 &\leq 2k|A|^k \frac{\log{n}}{n},
			\end{align*}
			for all $j = 0, \dots, n - 1$. Hence the claim holds.
		\end{claim}
		
		Now take $k(n) \leq \frac{\alpha}{2\log{|A|}}\log{n}$, where $\alpha \in (0, 1)$. By rearranging this, we have the relation $|A|^{2k(n)} \leq n^\alpha$. We now apply \thref{cor:cm-3-1-5} so that for all $t > 0$,
		\begin{align*}
			\mu_\phi\left\{\left| \frac{\hat{H}_{k(n)}}{k(n)} - E_{\mu_\phi}\left(\frac{\hat{H}_{k(n)}}{k(n)}\right) \right| \geq t\right\} &= \mu_\phi\left\{\left| \frac{\tilde{K}}{k(n)} - \int{\frac{\tilde{K}}{k(n)}\ d\mu_\phi} \right| \geq t\right\} \\
				&\leq 2\exp\left(-\frac{t^2 (k(n))^2}{4D\sum_{j = 0}^{n - 1}{\delta_j^2(\tilde{K})}}\right) \\
				&\leq 2\exp\left(-\frac{nt^2}{16D|A|^{2k(n)} (\log{n})^2}\right) \\
				&\leq 2\exp\left(-\frac{n^{1 - \alpha} t^2}{16D (\log{n})^2}\right).
		\end{align*}
		This completes the proof of the concentration bound.
		
		For the variance, we apply \thref{cor:cm-3-2} to get
		\begin{align*}
			\Var_{\mu_\phi}\left(\frac{\hat{H}_{k(n)}}{k(n)}\right) &= \Var_{\mu_\phi}\left(\frac{\tilde{K}}{k(n)}\right) \\
				&= \frac{1}{(k(n))^2}\int{\left[\tilde{K} - E_{\mu_\phi}(\tilde{K})\right]^2\ d\mu_\phi} \\
				&\leq \frac{2D}{(k(n))^2}\sum_{j = 0}^{n - 1}{\delta_j^2(\tilde{K})} \\
				&\leq \frac{8D (\log{n})^2}{n^{1 - \alpha}}.
		\end{align*}
	\end{proof}
\end{theorem}

This result gives a concentration bound about the expectation value. However, to quantify the accuracy of an entropy estimator, it makes more sense to find a concentration bound about the entropy $h(\mu_\phi)$. For this, we use $k$-block conditional empirical entropy.

\begin{theorem} \label{thm:cm-4-2}
	Suppose that $\theta < |A|^{-1}$. If $k(n) \leq \frac{\log{n}}{2\log{|A|}}$, then there exist strictly positive constants $c, \gamma, \Gamma, \xi > 0$ such that for all $t > 0$ we have
	\begin{equation}
		\mu_\phi\left\{\left|\hat{h}_{k(n)} - h(\mu_\phi)\right| \geq t + \frac{c}{n^\gamma}\right\} \leq 2\exp\left(-\frac{\Gamma n^\xi t^2}{(\log{n})^4}\right),
	\end{equation}
	for sufficiently large $n$.
	
	%Here, this upper bound decays superexponentially as $t \to +\infty$.
	If we think of $t + (c / n^\gamma)$ as the error of the $k$-block conditional empirical entropy, we see that the number of sequences which yield a `bad' estimate decays superexponentially as $t \to +\infty$. On the other hand, if we let $n \to +\infty$, then the upper bound will tend to zero for all $t > 0$. Hence $\mu_\phi$-almost every sequence gives an estimate which converges to $h(\mu_\phi)$ and this agrees with \thref{thm:entropy-estimation}
	
	\begin{proof}
		Recall that $\hat{h}_k = \hat{H}_k - \hat{H}_{k - 1}$. We put
		\[
			\tilde{K}'(s_0, \dots, s_{n - 1}) := \hat{h}_k(s_0^{n - 1}) = \hat{H}_k(s_0^{n - 1}) - \hat{H}_{k - 1}(s_0^{n - 1}).
		\]
		We estimate $\delta_j(\tilde{K}')$ by $2\delta_j(\tilde{K})$, where $\tilde{K}(s_0, \dots, s_{n - 1}) = \hat{H}_k(s_0^{n - 1})$ and using the estimate for $\delta_j(\tilde{K})$ in Formula \eqref{fml:oscil-est}.
		
		By \thref{lem:cm-4-1}, we have
		\begin{align}
			E_{\mu_\phi}&\left(\hat{h}_{k(n)}(x_0^{n - 1}) - h(\mu_\phi)\right) \nonumber \\
				&= E_{\mu_\phi}\left(\frac{1}{n}\sum_{j = 0}^{n - 1}(-\phi \circ \sigma^j) + \hat{\Delta}_{k(n)} + O(\theta^{k(n)}) - h(\mu_\phi)\right), \label{fml:cm-3-9-lem-ref}
		\end{align}
		where $\hat{\Delta}_{k(n)}(x_0^{n - 1}) = -H_{k(n)}(\E_{k(n)}(\seedot; x_0^{n - 1})) + H_{k(n) - 1}(\E_{k(n) - 1}(\seedot; x_0^{n - 1}))$. 
		Recalling that $h(\mu_\phi) = -\int{\phi\ d\mu_\phi}$, \eqref{fml:cm-3-9-lem-ref} becomes
		\begin{align*}
			E_{\mu_\phi}\left(\hat{h}_{k(n)}\right) - h(\mu_\phi) &= -\int{\phi\ d\mu_\phi} + E_{\mu_\phi}\left(\hat{\Delta}_{k(n)}\right) + O(\theta^{k(n)}) - h(\mu_\phi) \\
				&= h(\mu_\phi) + E_{\mu_\phi}\left(\hat{\Delta}_{k(n)}\right) + O(\theta^{k(n)}) - h(\mu_\phi) \\
				&= E_{\mu_\phi}\left(\hat{\Delta}_{k(n)}\right) + O(\theta^{k(n)}).
		\end{align*}
		
		We now put $k(n) = \frac{q\log{n}}{\log{|A|}}$, where $0 < q < 1$. If we choose $q$ to be
		\[
			q = \frac{1}{1 + \frac{\log{\theta^{-1}}}{\log{|A|}}},
		\]
		then using
		\[
			|E_{\mu_\phi}(\hat{h}_{k(n)}) - h(\mu_\phi)| \leq \frac{M|A|^{k(n)}}{n}
		\]
		from \thref{lem:cm-4-1}, a straightforward but tedious calculation shows that
		\begin{equation}\label{fml:cm-8}
			|E_{\mu_\phi}(\hat{h}_{k(n)}) - h(\mu_\phi)| \leq \frac{c}{n^\gamma},
		\end{equation}
		where $c > 0$ is a constant and $\gamma = \left(1 + \frac{\log{|A|}}{\log{\theta^{-1}}}\right)^{-1} > 0$. We then have
		\begin{align*}
			|\hat{h}_{k(n)} - h(\mu_\phi)| &= |\hat{h}_{k(n)} - E_{\mu_\phi}(\hat{k}_{k(n)}) - h(\mu_\phi) + E_{\mu_\phi}(\hat{k}_{k(n)})| \\
				&\leq |\hat{h}_{k(n)} - E_{\mu_\phi}(\hat{k}_{k(n)})| + |h(\mu_\phi) - E_{\mu_\phi}(\hat{k}_{k(n)})| \\
				&\leq |\hat{h}_{k(n)} - E_{\mu_\phi}(\hat{k}_{k(n)})| + \frac{c}{n^\gamma}.
		\end{align*}
		
		We now apply \thref{cor:cm-3-1-5} to get
		\begin{align*}
			\mu_\phi\left\{\left|\hat{h}_{k(n)} - h(\mu_\phi)\right| \geq t + \frac{c}{n^\gamma}\right\} &\leq \mu_\phi\left\{\left|\hat{h}_{k(n)} - E_{\mu_\phi}(\hat{k}_{k(n)})\right| \geq t\right\} \\
				&\leq 2\exp\left(-\frac{t^2}{4D\sum_{j = 0}^{n - 1}{\delta_j^2(\tilde{K}')}}\right) \\
				&\leq 2\exp\left(-\frac{nt^2}{64D(k(n))^2|A|^{2k(n)}(\log{n})^2}\right) \\
				&\leq 2\exp\left(-\frac{(\log{|A|})^2}{16D} \frac{nt^2}{|A|^{2k(n)}(\log{n})^4}\right) \\
				&\leq 2\exp\left(-\frac{(\log{|A|})^2}{16D} \frac{\theta^{2k(n)} nt^2}{(\log{n})^4}\right) \\
				&\leq 2\exp\left(-\frac{\Gamma n^\xi t^2}{(\log{n})^4}\right),
		\end{align*}
		where
		\[
			\Gamma = \frac{(\log{|A|})^2}{16D} \quad \text{and} \quad \xi = 1 - \frac{\log{\theta^{-1}}}{\log{|A|}}.
		\]
		The value of $\xi$ is found using the assumption that $k(n) \leq \frac{\log{n}}{2\log{|A|}}$. Furthermore, to guarantee that $\xi > 0$ we must have
		\[
			\xi = 1 - \frac{\log{\theta^{-1}}}{\log{|A|}} > 0 \iff \log{|A|} > \log{\theta^{-1}} \iff |A| > \theta^{-1},
		\]
		which is the first hypothesis in the theorem. This completes the proof.
	\end{proof}
\end{theorem}

\subsection{Hitting time estimators}\label{ssec:hitting-times}
We now focus on \key{hitting time entropy estimators}. Instead of using empirical frequencies, hitting time estimators use the first occurrence of the first $n$ symbols of a sequence $x \in \Sigma$ in another sequence $y \in \Sigma$. The following formal definition can be found in \cite[Definition 2.1]{chazottes-ugalde:hitting-times}.

\begin{definition}
	Let $x, y \in \Sigma$, let $a_0^{n - 1} \in A^n$. The (first) \key{hitting time of $x$ to a cylinder $[a_0^{n - 1}]$} is defined
	\[
		\tau_{[a_0^{n - 1}]}(x) := \inf\{j \geq 1 \mid x_j^{j + n - 1} = a_0^{n - 1}\}.
	\]
	
	The \key{following hitting time} or \key{waiting-time}~\cite[Section III.5]{shields:ergodic} is defined
	\[
		W_n(x, y) := \tau_{[x_0^{n - 1}]}(y) = \inf\{j \geq 1 \mid y_j^{j + n - 1} = x_0^{n - 1}\}.
	\]
\end{definition}

Note that, while the above definitions are largely the same, $\tau_{[a_0^{n - 1}]}$ is defined using a word $a_0^{n - 1}$ of length $n$, whereas $W_n$ is defined using sequences in $\Sigma$. In order to analyse the limiting behaviour as $n \to +\infty$, it makes more sense to work with $W_n$.

The following result relates the waiting-time function to the entropy $h(\nu)$ of certain measures $\nu$. This result is proved in \cite[Theorem III.5.1]{shields:ergodic}.

\begin{theorem}[Exact-Match Theorem] \label{thm:exact-match}
	If $\nu$ is weak Bernoulli, then
	\[
		\lim_{n \to +\infty}{\frac{1}{n}\log{W_n(x, y)}} = h(\nu),
	\]
	for $\nu \otimes \nu$-almost every $(x, y) \in \Sigma \otimes \Sigma$.
\end{theorem}

That is, if $\nu$ is weak Bernoulli, then the time it takes for one sequence to hit another tends to the entropy $h(\nu)$, for almost all pairs of sequences. This is the \key{hitting time entropy estimator}.

By \thref{thm:gibbs-is-weak-bernoulli}, Gibbs measures are weak Bernoulli and hence the Exact-Match Theorem applies. As with plug-in estimators, we also have concentration bounds for the hitting time estimator. To prove the main theorem, we will need two lemmata which follow from \cite[Theorem 1]{abadi:sharp}.

\begin{lemma}\label{lem:cm-4-2}
	There exists constants $C, c, \lambda_1, \lambda_2 > 0$ with $\lambda_1 < \lambda_2$ such that, for all $n \geq 1$ and for all $a_0^{n - 1} \in A^n$, there exists $\Lambda(a_0^{n - 1}) \in [\lambda_1, \lambda_2]$ such that for all $u > 0$ we have
	\[
		\left| \mu_\phi\left\{x \midmid \tau_{[a_0^{n - 1}]}(x) > \frac{u}{\Lambda(a_0^{n - 1}) \mu_\phi[a_0^{n - 1}]}\right\} - e^{-u} \right| \leq Ce^{-cu}.
	\]
\end{lemma}

\begin{lemma}\label{lem:cm-4-3}
	For all $v > 0$ and any word $a_0^{n - 1} \in A^n$ such that $v\mu_\phi[a_0^{n - 1}] \leq \frac{1}{2}$ we have
	\[
		\lambda_1 \leq -\frac{\log{\mu_\phi\{\tau_{[a_0^{n - 1}]} > v\}}}{v\mu_\phi[a_0^{n - 1}]} \leq \lambda_2,
	\]
	where $\lambda_1, \lambda_2 > 0$ are the same constants as in \thref{lem:cm-4-2}.
\end{lemma}

We are now ready to state and prove concentration bounds for the hitting time estimator.

\begin{theorem} \label{thm:cm-4-3}
	There exist strictly positive constants $C_1, C_2 > 0$, $t_0 > 0$ such that, for all $n \geq 1$ and for all $t > t_0$, we have
	\begin{equation}
		(\mu_\phi \otimes \mu_\phi)\left\{(x, y) \midmid \frac{1}{n}\log{W_n(x, y)} > h(\mu_\phi) + t\right\} \leq C_1 e^{-C_2 nt^2} \label{fml:cm-9}
	\end{equation}
	and
	\begin{equation}
		(\mu_\phi \otimes \mu_\phi)\left\{(x, y) \midmid \frac{1}{n}\log{W_n(x, y)} < h(\mu_\phi) - t\right\} \leq C_1 e^{-C_2 nt}. \label{fml:cm-10}
	\end{equation}	
	Essentially, \eqref{fml:cm-9} tells us how likely the hitting time estimator is to \emph{overestimate} the entropy, while \eqref{fml:cm-10} gives how likely it is to \emph{underestimate} the entropy. Note that the upper bound in \eqref{fml:cm-10} decays exponentially as $t \to +\infty$, whereas in \eqref{fml:cm-9} the bound decays superexponentially as $t \to +\infty$. In either case, this means that fewer pairs $(x, y)$ yield estimates which deviate very far from $h(\mu_\phi)$.
	
	If we let $n \to +\infty$, both \eqref{fml:cm-9} and \eqref{fml:cm-10} are bounded above by $0$ for any $t > t_0$. This means that $\mu_\phi \otimes \mu_\phi$-almost every $(x, y)$ gives an estimate which deviates by at most $t_0$ from $h(\mu_\phi)$, and this largely agrees with \thref{thm:exact-match}.
	
	Unfortunately, to prove this result we cannot use \thref{thm:cm-3-1} and its corollaries directly. Our approach to the proof of this theorem involves approximating the value of $W_n(x, y)$, and to do this we need to use the fact that $\mu_\phi$ is a Gibbs measure.
	
	\begin{proof}
		We concentrate on \eqref{fml:cm-9} first. We have
		\begin{align*}
			&(\mu_\phi \otimes \mu_\phi)\left\{(x, y) \midmid \frac{1}{n}\log{W_n(x, y)} > h(\mu_\phi) + t\right\} \\
				&= (\mu_\phi \otimes \mu_\phi)\left\{(x, y) \midmid \frac{1}{n}\log{W_n(x, y)} + \frac{1}{n}\log{\mu_\phi[x_0^{n - 1}]} - \frac{1}{n}\log{\mu_\phi[x_0^{n - 1}]} - h(\mu_\phi) > t\right\} \\
				&\leq T_1 + T_2, % (\mu_\phi \otimes \mu_\phi)\left\{(x, y) \midmid \frac{1}{n}\log{W_n(x, y)} + \frac{1}{n}\log{\mu_\phi[x_0^{n - 1}]} > \frac{t}{2}\right\}, \\
				%& \quad + \mu_\phi\left\{x \midmid - \frac{1}{n}\log{\mu_\phi[x_0^{n - 1}]} - h(\mu_\phi) > \frac{t}{2}\right\} \\
		\end{align*}
		where
		\begin{align*}
			T_1 &:= (\mu_\phi \otimes \mu_\phi)\left\{(x, y) \midmid \frac{1}{n}\log{W_n(x, y)} + \frac{1}{n}\log{\mu_\phi[x_0^{n - 1}]} > \frac{t}{2}\right\} \\
				&= (\mu_\phi \otimes \mu_\phi)\left\{(x, y) \midmid \frac{1}{n}\log\left(W_n(x, y) \mu_\phi[x_0^{n - 1}]\right) > \frac{t}{2}\right\}
		\end{align*}
		and
		\[
			T_2 := \mu_\phi\left\{x \midmid - \frac{1}{n}\log{\mu_\phi[x_0^{n - 1}]} - h(\mu_\phi) > \frac{t}{2}\right\}.
		\]
		
		We find an upper bound for $T_2$ first. By the definition of Gibbs measures, we have
		\begin{equation}\label{fml:gibbs-property}
			-\log{C} + \sum_{j = 0}^{n - 1}{\phi(\sigma^j{x})} \leq \log{\mu_\phi[x_0^{n - 1}]} \leq \log{C} + \sum_{j = 0}^{n - 1}{\phi(\sigma^j{x})}.
		\end{equation}
		Once again, recalling that $h(\mu_\phi) = -\int{\phi\ d\mu_\phi}$, we apply \thref{cor:cm-3-3} with $f = -\phi$ to get
		\begin{align*}
			T_2 &= \mu_\phi\left\{x \midmid - \frac{1}{n}\log{\mu_\phi[x_0^{n - 1}]} - h(\mu_\phi) > \frac{t}{2}\right\} \\
				&\leq \mu_\phi\left\{x \midmid -\frac{1}{n}\sum_{j = 0}^{n - 1}{\phi(\sigma^j{x})} -h(\mu_\phi) > \frac{t}{2} -\frac{1}{n}\log{C}\right\} \\
				&= \mu_\phi\left\{x \midmid -\frac{1}{n}\sum_{j = 0}^{n - 1}{\phi(\sigma^j{x})} + \int{\phi\ d\mu_\phi} > \frac{t}{2} -\frac{1}{n}\log{C}\right\} \\
				& \leq e^{-Bnt^2},
		\end{align*}
		for all $t > 2\log{C}$, where $B = (4D|f|_\theta^2)^{-1}$.
		
		We now find an upper bound for $T_1$. We have
		\begin{align*}
			T_1 &= (\mu_\phi \otimes \mu_\phi)\left\{(x, y) \midmid \frac{1}{n}\log\left(W_n(x, y) \mu_\phi[x_0^{n - 1}]\right) > \frac{t}{2}\right\} \\
				&= \sum_{a_0^{n - 1} \in A^n}{(\mu_\phi \otimes \mu_\phi)\left\{(x, y) \midmid x \in [a_0^{n - 1}],\ \frac{1}{n}\log\left(W_n(x, y) \mu_\phi[x_0^{n - 1}]\right) > \frac{t}{2}\right\}} \\
				&= \sum_{a_0^{n - 1} \in A^n}{\mu_\phi[a_0^{n - 1}] \mu_\phi\left\{y \midmid \tau_{[a_0^{n - 1}]}(y) \mu_\phi[a_0^{n - 1}] > e^\frac{nt}{2}\right\}}
		\end{align*}
		By \thref{lem:cm-4-2} with $u = e^\frac{nt}{2}$ we have
		\begin{align*}
			\left|T_1 - e^{-e^\frac{nt}{2}}\right| &= \left|\sum_{a_0^{n - 1} \in A^n}\mu_\phi[a_0^{n - 1}]\left(\mu_\phi\left\{y \midmid \tau_{[a_0^{n - 1}]}(y) \mu_\phi[a_0^{n - 1}] > e^\frac{nt}{2}\right\} - e^{-e^\frac{nt}{2}}\right)\right| \\
				& \leq \sum_{a_0^{n - 1} \in A^n}\mu_\phi[a_0^{n - 1}] \hat{C}e^{-\hat{c}e^\frac{nt}{2}} \\
				& = \hat{C}e^{-\hat{c}e^\frac{nt}{2}},
		\end{align*}
		for some constants $\hat{C}, \hat{c} > 0$. So we have $T_1 \leq C'e^{-c'e^\frac{nt}{2}}$, for some constants $C', c' > 0$.
		
		Finally, we combine $T_1$ and $T_2$ to get
		\begin{align*}
			(\mu_\phi \otimes \mu_\phi)\left\{(x, y) \midmid \frac{1}{n}\log{W_n(x, y)} > h(\mu_\phi) + t\right\} &\leq C'e^{-c'e^\frac{nt}{2}} + e^{-Bnt^2} \\
				& \leq C_1 e^{Bnt^2},
		\end{align*}
		for some constant $C_1 > 0$. This proves \eqref{fml:cm-9}.
		
		We now turn our attention to \eqref{fml:cm-10}. We have
		\begin{align*}
			&(\mu_\phi \otimes \mu_\phi)\left\{(x, y) \midmid \frac{1}{n}\log{W_n(x, y)} < h(\mu_\phi) - t\right\} \\
				&= (\mu_\phi \otimes \mu_\phi)\left\{(x, y) \midmid -\frac{1}{n}\log{W_n(x, y)} - \frac{1}{n}\log{\mu_\phi[x_0^{n - 1}]} + \frac{1}{n}\log{\mu_\phi[x_0^{n - 1}]} + h(\mu_\phi) > t\right\} \\
				&\leq T'_1 + T'_2,
		\end{align*}
		where
		\begin{align*}
			T'_1 &:= (\mu_\phi \otimes \mu_\phi)\left\{(x, y) \midmid -\frac{1}{n}\log{W_n(x, y)} - \frac{1}{n}\log{\mu_\phi[x_0^{n - 1}]} > \frac{t}{2}\right\} \\
				&= (\mu_\phi \otimes \mu_\phi)\left\{(x, y) \midmid -\frac{1}{n}\log\left(W_n(x, y) \mu_\phi[x_0^{n - 1}]\right) > \frac{t}{2}\right\}
		\end{align*}
		and
		\[
			T'_2 := \mu_\phi \left\{x \midmid \frac{1}{n}\log{\mu_\phi[x_0^{n - 1}]} + h(\mu_\phi) > \frac{t}{2}\right\}.
		\]
		
		To find an upper bound for $T'_2$, we once again use the Gibbs property in \eqref{fml:gibbs-property} and apply \thref{cor:cm-3-3} with $f = \phi$ to get
		\begin{align*}
			T'_2 &\leq \mu_\phi\left\{x \midmid \frac{1}{n}\sum_{j = 0}^{n - 1}{\phi(\sigma^j{x})} + h(\mu_\phi) > \frac{t}{2} - \frac{1}{n}\log{C}\right\} \\
				&= \mu_\phi\left\{x \midmid \frac{1}{n}\sum_{j = 0}^{n - 1}{\phi(\sigma^j{x})} - \int{\phi\ d\mu_\phi} > \frac{t}{2} - \frac{1}{n}\log{C}\right\} \\
				&\leq e^{-\frac{Bnt^2}{4}},
		\end{align*}
		for all $t > 2\log{C}$. (We have applied Corollary 4.6 for $t / 2$ instead of $t$.)
		
		As for $T_1$ above, for $T'_1$ we have
		\begin{align*}
			T'_1 &= (\mu_\phi \otimes \mu_\phi)\left\{(x, y) \midmid -\frac{1}{n}\log\left(W_n(x, y) \mu_\phi[x_0^{n - 1}]\right) > \frac{t}{2}\right\} \\
				&= \sum_{a_0^{n - 1} \in A^n}{(\mu_\phi \otimes \mu_\phi)\left\{(x, y) \midmid x \in [a_0^{n - 1}],\ -\frac{1}{n}\log\left(W_n(x, y) \mu_\phi[x_0^{n - 1}]\right) > \frac{t}{2}\right\}} \\
				&= \sum_{a_0^{n - 1} \in A^n} \mu_\phi[a_0^{n - 1}]\mu_\phi\left\{y \midmid \tau_{[a_0^{n - 1}]}(y) \mu_\phi[a_0^{n - 1}] < e^{-\frac{nt}{2}}\right\}.
		\end{align*}
		Now by \thref{lem:cm-4-3}, if $v\mu_\phi[a_0^{n - 1}] \leq \frac{1}{2}$, then we have
		\[
			-\frac{\log{\mu_\phi\{\tau_{[a_0^{n - 1}]} > v\}}}{v\mu_\phi[a_0^{n - 1}]} \leq -\frac{\log{\mu_\phi\{\tau_{[a_0^{n - 1}]} > v\}}}{v} \leq \lambda_2.
		\]
		Rearranging this, we get
		\[
			-\log\left(1 - \mu_\phi\{\tau_{[a_0^{n - 1}]} < v\}\right) = -\log{\mu_\phi\{\tau_{[a_0^{n - 1}]} > v\}} \leq \lambda_2 v,
		\]
		which gives
		\[
			\mu_\phi\{\tau_{[a_0^{n - 1}]} < v\} \leq 1 - e^{-\lambda_2 v} \leq \lambda_2 v,
		\]
		because $1 - e^{-u} \leq u$ for all $u \in \reals$.
		
		Since $0 \leq \mu_\phi[a_0^{n - 1}] \leq 1$ for all $a_0^{n - 1}$, if we let $v = e^{-\frac{nt}{2}}$, then we have
		\begin{align*}
			T'_1 &\leq \sum_{a_0^{n - 1} \in A^n} \mu_\phi[a_0^{n - 1}]\mu_\phi\left\{y \midmid \tau_{[a_0^{n - 1}]}(y) < e^{-\frac{nt}{2}}\right\} \\
				&\leq \sum_{a_0^{n - 1} \in A^n} \mu_\phi[a_0^{n - 1}] \lambda_2 e^{-\frac{nt}{2}} \\
				&= \lambda_2 e^{-\frac{nt}{2}},
		\end{align*}
		provided that $e^{-\frac{nt}{2}}\mu_\phi[a_0^{n - 1}] \leq \frac{1}{2}$. We want this to be true for all $n$ and all words $a_0^{n - 1}$ of length $n$, so this restriction is equivalent to $e^{-\frac{t}{2}} \leq \frac{1}{2}$ which is the same as $t \geq 2\log{2}$. So we take $t_0 = \max\{2\log{C}, 2 \log{2}\}$.
		
		Finally, we combine $T'_1$ and $T'_2$ to get
		\[
			(\mu_\phi \otimes \mu_\phi)\left\{(x, y) \midmid \frac{1}{n}\log{W_n(x, y)} < h(\mu_\phi) - t\right\} \leq \lambda_2 e^{-\frac{nt}{2}} + e^{-\frac{Bnt^2}{4}} \leq C_1e^{-C_2 nt},
		\]
		for some constants $C_1, C_2 > 0$.
	\end{proof}
\end{theorem}


\appendix
\chapter{Auxiliary results}
This appendix contains results which are used throughout the dissertation. We separate the appendix into sections depending on which chapter the result is first used.

\section{Entropy}
We use the following result in \thref{prop:walters-cor-4-2-1} and also \thref{lem:pp-3-3}.

\begin{definition}
	Let $f : X \to \reals$ be a function of a convex set $X$. We say that $f$ is \key{strictly convex} if, for all $x, y \in X$ and for all $\alpha \in [0, 1]$, we have
	\[
		f(\alpha x + (1 - \alpha)y) \leq \alpha f(x) + (1 - \alpha)f(y),
	\]
	with equality if and only if $x = y$ or $\alpha = 0$ or $1$.
	
	In general, $f$ is strictly convex if, for any set $\{x_j \in X \mid j \in \{1, \dots, k\}\}$ and for any $\{\alpha_j \in [0, 1] \mid j \in \{1, \dots, k\},\ \sum_{j = 1}^k{\alpha_j} = 1\}$, we have
	\[
		f\left(\sum_{j = 1}^k{\alpha_j x_j}\right) \leq \sum_{j = 1}^k{\alpha_j} f(x_j),
	\]
	with equality if and only if $x_1 = x_2 = \dots = x_k$ whenever $\alpha_j \neq 0$ for all $j = 1, \dots, k$.
\end{definition}

\begin{theorem} \label{thm:walters-4-2-xlogx-convex}
	Let $f : [0, +\infty) \to \reals$ be the function defined by
	\[
		f(x) =
		\begin{cases}
			0, & \text{if } x = 0; \\
			x\log{x}, & \text{if } x \neq 0.
		\end{cases}
	\]
	Then $f$ is \emph{strictly convex}.
	\begin{proof}
		For $x > 0$ we have $f'(x) = 1 + \log{x}$ and $f''(x) = 1 / x$. Suppose that $y > x$ and let $\alpha \in (0, 1)$. In particular, this means that $x < \alpha x + (1 - \alpha)y < y$. Now $f$ is clearly continuous, and hence we may apply the Mean Value Theorem. So there exists $z \in (\alpha x + (1 - \alpha)y, y)$ such that
		\[
			f'(z) = \frac{f(y) - f(\alpha x + (1 - \alpha)y)}{y - \alpha x - (1 - \alpha)y},
		\]
		and so
		\[
			f(y) - f(\alpha x + (1 - \alpha)y) = f'(z)\alpha(y - x).
		\]
		
		Similarly, there exists $w \in (x, \alpha x + (1 - \alpha)y)$ such that
		\[
			f(\alpha x + (1 - \alpha)y) - f(x) = f'(w)(1 - \alpha)(y - x).
		\]
		
		Since $x > 0$, we have $f''(x) = 1 / x > 0$ and so $f'(z) > f'(w)$. Hence
		\begin{align*}
			&(1 - \alpha)(f(y) - f(\alpha x + (1 - \alpha)y)) = f'(z)\alpha(1 - \alpha)(y - x) \\
			& \quad > f'(w)\alpha(1 - \alpha)(y - x) = \alpha(f(\alpha x + (1 - \alpha)y) - f(x)).
		\end{align*}
		Cancelling terms, this gives
		\[
			f(\alpha x + (1 - \alpha)y) < \alpha f(x) + (1 - \alpha)f(y)
		\]
		for $y > x > 0$. We may apply the same argument for $x, y \geq 0$ and $x \neq y$.
	\end{proof}
\end{theorem}

\section{Gibbs measures}
The follow result is used in the proof of \thref{lem:pp-prop-3-4}. First, we need to a definition.

\begin{definition}
	Let $X$ be a convex set and suppose $f : X \to \reals$ (or $\complex$). We say that the function $f$ is \key{concave} on $X$ if for all $x, y \in X$ we have
	\[
		f(\alpha x + (1 - \alpha)y) \geq \alpha f(x) + (1 - \alpha)f(y),
	\]
	for all $\alpha \in [0, 1]$. The function $f$ is \key{strictly concave} if the inequality is strict.~\cite[p11]{cambini-martein:generalized}
\end{definition}

\begin{lemma} \label{lem:pp-3-3}
	Suppose that $(p_1, \dots, p_k), (q_1, \dots, q_k)$ are probability vectors with $p_j > 0$ for all $j = 1, \dots, k$. Then
	\[
		-\sum_{j = 1}^k{q_j \log{q_j}} + \sum_{j = 1}^k{q_j \log{p_j}} \leq 0,
	\]
	with equality if and only if $p_j = q_j$ for all $j = 1, \dots, k$.
	\begin{proof}
		We have
		\begin{align}
			-\sum_{j = 1}^k{q_j \log{q_j}} + \sum_{j = 1}^k{q_j \log{p_j}} &= \sum_{j = 1}^k{q_j \log\left(\frac{p_j}{q_j}\right)} \nonumber \\
				&= \sum_{j = 1}^k{-p_j \frac{q_j}{p_j} \log\left(\frac{q_j}{p_j}\right)} \nonumber \\
				&= \sum_{j = 1}^k{p_j \phi\left(\frac{q_j}{p_j}\right)}, \label{fml:pp-3-3-sums}
		\end{align}
		where $\phi(x) = -x \log{x}$, with the convention that $\phi(0) = 0$. By \thref{thm:walters-4-2-xlogx-convex}, we have that $x \log{x}$ is a strictly convex function, and so $\phi$ is strictly concave. Continuing from \eqref{fml:pp-3-3-sums}, we have
		\[
			-\sum_{j = 1}^k{q_j \log{q_j}} + \sum_{j = 1}^k{q_j \log{p_j}} \leq \phi\left(\sum_{j = 1}^k{p_j \frac{q_j}{p_j}}\right) = \phi(1) = 0,
		\]
		with equality if and only if all the $(q_j / p_j)$ terms are equal.
	\end{proof}
\end{lemma}

\section{Concentration bounds}

The following result is used in the proof of \thref{thm:cm-4-1}.

\begin{proposition}\label{prop:logs-thm-4-1}
	Let $\ell, k \in \naturals$ be such that $\ell + k \leq n$. Then
	\[
		\left|\left(\frac{\ell}{n}\right)\log\left(\frac{\ell}{n}\right) - \left(\frac{\ell + k}{n}\right)\log\left(\frac{\ell + k}{n}\right)\right| \leq \left(\frac{k}{n}\right)\log{n}.
	\]
	
	\begin{proof}
		We have
		\begin{align*}
			&\left(\frac{\ell}{n}\right)\log\left(\frac{\ell}{n}\right) - \left(\frac{\ell + k}{n}\right)\log\left(\frac{\ell + k}{n}\right) \\
				&\quad = \left(\frac{k}{n}\right)\log{n} - \left[\left(\frac{\ell + k}{n}\right)\log(\ell + k) - \left(\frac{\ell}{n}\right)\log{\ell}\right].
		\end{align*}
		Since $k \geq 1$, we have
		\begin{align*}
			0 &\leq \left(\frac{\ell + k}{n}\right)\log(\ell + k) - \left(\frac{\ell}{n}\right)\log{\ell} \\
				&\leq \left(\frac{\ell + k}{n}\right)\log(\ell + k) + \left(\frac{k - \ell}{n}\right)\log{\ell} \\
				&\leq \left(\frac{\ell + k}{n}\right)\log{n} + \left(\frac{k - \ell}{n}\right)\log{n} \\
				&\leq 2\left(\frac{k}{n}\right)\log{n}.
		\end{align*}
		Hence
		\[
			-\left(\frac{k}{n}\right)\log{n} \leq \left(\frac{\ell}{n}\right)\log\left(\frac{\ell}{n}\right) - \left(\frac{\ell + k}{n}\right)\log\left(\frac{\ell + k}{n}\right) \leq \left(\frac{k}{n}\right)\log{n},
		\]
		in other words,
		\[
			\left|\left(\frac{\ell}{n}\right)\log\left(\frac{\ell}{n}\right) - \left(\frac{\ell + k}{n}\right)\log\left(\frac{\ell + k}{n}\right)\right| \leq \left(\frac{k}{n}\right)\log{n}.
		\]
	\end{proof}
\end{proposition}

The following result allows us to write the $k$-block conditional empirical entropy $\hat{h}_{k(n)}(x_0^{n - 1})$ in a more useful form. It is used in the proof of \thref{thm:cm-4-2}.

\begin{lemma}\label{lem:cm-4-1}
	Let $\phi \in F_\theta$. We have
	\begin{equation}
		\hat{h}_{k(n)}(x_0^{n - 1}) = \frac{1}{n}\sum_{j = 0}^{n - 1}(-\phi \circ \sigma^j(x)) + \hat{\Delta}_{k(n)}(x_0^{n - 1}) + O(\theta^{k(n)}),
	\end{equation}
	where
	\[
		|E(\hat{\Delta}_{k(n)})| \leq \frac{M|A|^{k(n)}}{n},
	\]
	for some $M > 0$.
	\begin{proof}
		We follow the proof given in \cite[p10-11]{chazottes-maldonado:cbfee}.
		
		We have
		\begin{align}
			\hat{h}_k(x_0^{n - 1}) &= h_k(\E_k(\seedot; x_0^{n - 1})) \nonumber \\
				&= -\sum_{a_0^{k - 1} \in A^k}{\E_k(a_0^{k - 1}; x_0^{n - 1}) \log{\frac{\E_k(a_0^{k - 1}; x_0^{n - 1})}{\E_{k - 1}(a_0^{k - 2}; x_0^{n - 1})}}} \nonumber \\
				&= \hat{\Delta}_k(x_0^{n - 1}) -\sum_{a_0^{k - 1} \in A^k}\E_k(a_0^{k - 1}; x_0^{n - 1}) \log{\frac{\mu_\phi[a_0^{k - 1}]}{\mu_\phi[a_1^{k - 1}]}}, \label{fml:cm-11}
		\end{align}
		where
		\begin{align*}
			\hat{\Delta}_k(x_0^{n - 1}) &=-\sum_{a_0^{k - 1} \in A^k}{\E_k(a_0^{k - 1}; x_0^{n - 1}) \log{\frac{\E_k(a_0^{k - 1}; x_0^{n - 1})}{\E_{k - 1}(a_0^{k - 2}; x_0^{n - 1})}}} \\
				& \qquad + \sum_{a_0^{k - 1} \in A^k}\E_k(a_0^{k - 1}; x_0^{n - 1}) \log{\frac{\mu_\phi[a_0^{k - 1}]}{\mu_\phi[a_1^{k - 1}]}} \\
				&=-\sum_{a_0^{k - 1} \in A^k}{\E_k(a_0^{k - 1}; x_0^{n - 1}) \log{\frac{\E_k(a_0^{k - 1}; x_0^{n - 1})}{\mu_\phi[a_0^{k - 1}]}}} \\
				& \qquad + \sum_{a_0^{k - 1} \in A^k}\E_k(a_0^{k - 1}; x_0^{n - 1}) \log{\frac{\E_{k - 1}(a_0^{k - 2}; x_0^{n - 1})}{\mu_\phi[a_1^{k - 1}]}}.
		\end{align*}
		Since $\E_k(\seedot; x_0^{n - 1})$ is locally $\sigma$-invariant, we have
		\[
			\sum_{a_0 \in A}{\E_k(a_0^{k - 1}; x_0^{n - 1})} = \E_{k - 1}(a_1^{k - 1}; x_0^{n - 1}).
		\]
		It is also clear that
		\[
			\sum_{a_{k - 1} \in A}{\mathcal{E}_k(a_0^{k - 1}; x_0^{n - 1})} = \mathcal{E}_{k - 1}(a_0^{k - 2}; x_0^{n - 1}).
		\]
		This means that
		\begin{align*}
			\sum_{a_0^{k - 1} \in A^k}&\E_k(a_0^{k - 1}; x_0^{n - 1}) \log{\frac{\E_{k - 1}(a_0^{k - 2}; x_0^{n - 1})}{\mu_\phi[a_1^{k - 1}]}} \\
				&= \sum_{a_0^{k - 1} \in A^k}\E_k(a_0^{k - 1}; x_0^{n - 1}) \log{\frac{\sum_{a_{k} \in A}{\mathcal{E}_k(a_0^{k - 1}; x_0^{n - 1})}}{\mu_\phi[a_1^{k - 1}]}} \\
				&= \sum_{a_0^{k - 1} \in A^k}\E_k(a_0^{k - 1}; x_0^{n - 1}) \log{\frac{\mathcal{E}_k(a_0^{k - 1}; x_0^{n - 1})}{\mu_\phi[a_1^{k - 1}]}} \\
				&= \sum_{a_1^{k - 1} \in A^{k - 1}}\sum_{a_0 \in A}\E_k(a_0^{k - 1}; x_0^{n - 1}) \log{\frac{\mathcal{E}_k(a_0^{k - 1}; x_0^{n - 1})}{\mu_\phi[a_1^{k - 1}]}} \\
				&= \sum_{a_1^{k - 1} \in A^{k - 1}}\E_{k - 1}(a_1^{k - 1}; x_0^{n - 1}) \log{\frac{\mathcal{E}_{k - 1}(a_1^{k - 1}; x_0^{n - 1})}{\mu_\phi[a_1^{k - 1}]}} \\
				&= H_{k - 1}(\E_{k - 1}(\seedot; x_0^{n - 1})).
		\end{align*}
		Hence $\hat{\Delta}_k(x_0^{n - 1}) = -H_k(\E_k(\seedot; x_0^{n - 1})) + H_{k - 1}(\E_{k - 1}(\seedot; x_0^{n - 1}))$.
		
		There is a formula due to \cite[Formulae (4.15), (4.16)]{gabrielli-galves-guiol:fluctuations} which gives the bound
		\[
			|E_{\mu_\phi}(\hat{\Delta}_{k(n)})| \leq \frac{M|A|^k}{n}
		\]
		for all $n \geq 1$, where $M > 0$ is a strictly positive constant.
		
		We now deal with the summation in Formula \eqref{fml:cm-11}. For $y \in \Sigma$ we put
		\[
			\phi_k(y) = \log\frac{\mu_\phi[y_0^{k - 1}]}{\mu_\phi[y_1^{k - 1}]}.
		\]
		By \thref{prop:pp-3-2}, for all $y \in \Sigma$ we have
		\[
			\|\phi_k(y) - \phi(y)\|_\infty = \left\|\log\frac{\mu_\phi[y_0^{k - 1}]}{\mu_\phi[y_1^{k - 1}]} - \phi(y)\right\|_\infty \leq |\phi|_\theta \theta^k.
		\]
		So we may reasonably replace any instance of $\phi_k(y)$ with $\phi(y) + O(\theta^k)$. By the way $\E_k(\seedot; x_0^{n - 1})$ is defined, we have
		\begin{align*}
			-\sum_{a_0^{k - 1} \in A^k}\E_k(a_0^{k - 1}; x_0^{n - 1}) \log{\frac{\mu_\phi[a_0^{k - 1}]}{\mu_\phi[a_1^{k - 1}]}} &= -\sum_{a_0^{k - 1} \in A^k}\E_k(a_0^{k - 1}; x_0^{n - 1}) \phi_k(a) \\
				&= \frac{1}{n}\sum_{j = 0}^{n - 1}{(-\phi \circ \sigma^j(x))} + O(\theta^k).
		\end{align*}
		The result follows by substituting this back into Formula \eqref{fml:cm-11}.
	\end{proof}
\end{lemma}


\bibliographystyle{alpha}
\bibliography{references}

% If you need more than one appendix, then just use another \chapter command
%\chapter{Yet Another Appendix}

\end{document}
