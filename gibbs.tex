\chapter{Gibbs measures} \label{chap:gibbs}
\section{Overview}
This chapter introduces Gibbs measures, a class of probability measures on shifts of finite type. Gibbs measures have properties which we make use of when we discuss \cite{chazottes-maldonado:cbfee} in Chapter \ref{chap:concentration-bounds}.

Throughout this chapter we will let $\Sigma = \Sigma_A^+$, where $A$ is an irreducible $k \times k$ matrix.

\section{The Ruelle operator}
\emph{This section predominantly follows material from \cite[Chapter 2]{parry-pollicott:zeta-fns-periodic-orbits}.}

We briefly introduce a theorem due to Ruelle. Later, it will be apparent that this theorem is useful for constructing Gibbs measures.

\begin{definition}
	Let $f \in F_\theta^+$. The \key{Ruelle operator} (or \key{transfer operator}) $L_f : F_\theta^+ \to F_\theta^+$ (or, more generally, $L_f : C(\Sigma, \complex) \to C(\Sigma, \complex)$) is defined
	\[
	(L_f{w})(x) = \sum_{y \in \Sigma \midcolon \sigma{y} = x}{e^{f(y)} w(y)} = \sum_{j \midcolon A_{j, x_0} = 1}{e^{f(j, x_0, x_1, \dots)} w(j, x_0, x_1, \dots)},
	\]
	where $x = (x_j)_{j = 0}^\infty \in \Sigma$. This is a bounded linear operator.
	
	The $n$-th iterate of $L_f$ is given by
	\[
	(L_f^n{w})(x) = \sum_{y \in \Sigma \midcolon \sigma^n{y} = x}{e^{f^n(y)} w(y)}.
	\]
	
	If $f$ is also real-valued and $L_f{1} = 1$, then we say that $f$ or $L_f$ is \key{normalised}.
\end{definition}

%\begin{proposition}
%	Let $f \in F_\theta^+$ with $f = u + iv$, where $u, v \in F_\theta^+$ are real-valued functions. If $L_u$ is normalised, i.e. $L_u{1} = 1$, then for all $n \geq 0$,
%	\[
%	|L_f^n{w}|_\theta \leq K|w|_\infty + \theta^n |w|_\theta
%	\]
%	for all $w \in F_\theta^+$, where $K > 0$ is a constant depending only on $f$ and $\theta$.
%\end{proposition}

\begin{theorem}[Ruelle's Perron-Frobenius Theorem] \label{thm:rpf}
	Suppose that $\Sigma = \Sigma_A^+$ is an aperiodic shift of finite type and let $f \in F_\theta^+$ be a real-valued function. Then
	\begin{enumerate}
		\item There is a simple maximal eigenvalue $\lambda$ of $L_f : C(\Sigma, \reals) \to C(\Sigma, \reals)$ with a corresponding eigenfunction $h \in C(\Sigma_A^+, \reals)$, with $h > 0$. \label{rpf:1}
		\item The remainder of the spectrum of $L_f$ is contained in a disc of radius strictly less than $\lambda$. \label{rpf:2}
		\item There is a unique probability measure $\mu$ such that $L_f^*{\mu} = \lambda\mu$. That is,
		\[
		\int{L_f{w}\ d\mu} = \lambda \int{w\ d\mu},
		\]
		for all $w \in C(\Sigma, \reals)$. Additionally, if $h$ is the eigenfunction as in \ref{rpf:1} and $\int{h\ d\mu} = 1$, then the measure $\nu$ defined by $d\nu = h\ d\mu$ is a $\sigma$-invariant probability measure. \label{rpf:3}
		\item If $h$ is the eigenfunction as in \ref{rpf:1} and $\int{h\ d\mu} = 1$, then for all $w \in C(\Sigma, \reals)$,
		\[
		\frac{1}{\lambda^n}L_f^n{w} \to h \int{w\ d\mu}
		\]
		uniformly, as $n \to +\infty$. \label{rpf:4}
	\end{enumerate}
\end{theorem}

\section{Gibbs measures and basic results}
\emph{The remainder of this chapter predominantly follows \cite[Chapter 3]{parry-pollicott:zeta-fns-periodic-orbits}.}

\begin{definition}
	Let $\phi \in C(\Sigma, \reals)$ be a continuous function. Let $\mu$ be a probability measure such that there exist constants $C = C(\phi) > 1$, $P = P(\phi) \in \reals$, such that
	\begin{equation}
		C^{-1} \leq \frac{\mu[x_0, \dots, x_{n - 1}]}{\exp\left(-Pn + \sum_{j = 0}^{n - 1}{\phi(\sigma^j{x})} \right)} \leq C,
	\end{equation}
	for all $n \geq 0$ and for all $x \in \Sigma$. Then the measure $\mu$ is a \key{Gibbs measure for $\phi$} or a \key{Gibbs measure with potential $\phi$}. We will write $\mu_\phi = \mu$.
\end{definition}

\begin{remark}
	The Gibbs measure $\mu_\phi$ is not necessarily $\sigma$-invariant.
\end{remark}

Given any $\phi \in F_\theta$, it can be shown that there exists a Gibbs measure $\mu_\phi$ for $\phi$. We prove this in the following results.

\begin{proposition} \label{prop:pp-3-2}
	Suppose that $\phi \in F_\theta$ is a normalised, real-valued function. Then for all $x \in \Sigma$ we have
	\begin{equation}
		e^{-|\phi|_\theta \theta^n} \leq \frac{\mu[x_0, \dots, x_n]}{\mu[x_1, \dots, x_n]} e^{-\phi(x)} \leq e^{|\phi|_\theta \theta^n},
	\end{equation}
	where $L_\phi^*\mu = \mu$ is the unique measure in Ruelle's Perron-Frobenius Theorem \ref{rpf:3} (\thref{thm:rpf}) with $\lambda = 1$.
	\begin{proof}
		Let $w \in C(\Sigma, \reals)$. We have
		\begin{align*}
			\int{w \circ \sigma\ d\mu} &= \int L_\phi(w \circ \sigma)\ d\mu \\
				&= \int{\sum_{y \in \Sigma \midcolon \sigma{y} = x}{e^{\phi(y)} w \circ \sigma(y)}\ d\mu} \\
				&= \int{\sum_{y \in \Sigma \midcolon \sigma{y} = x}{e^{\phi(y)} w(x)}\ d\mu} \\
				&= \int{w L_\phi{1}\ d\mu} \\
				&= \int{w\ d\mu}.
		\end{align*}
		By a basic result~\cite[Lemma 11.3]{ergodic-lectures} in ergodic theory, it follows that $\mu$ is a $\sigma$-invariant measure. Hence
		\begin{align}
			\mu[x_1, \dots, x_n] &= \mu\left(\bigsqcup_{x_0 \midcolon A_{x_0, x_1} = 1}{[x_0, x_1, \dots, x_n]}\right) \nonumber \\
				&= \int{\sum_{x_0 \midcolon A_{x_0, x_1} = 1} \chi_{[x_0, x_1, \dots, x_n]}(z)\ d\mu} \nonumber \\
				&= \int{\sum_{y \midcolon \sigma{y} = z} e^{\phi(y)} \chi_{[x_0, \dots, x_n]}(y) e^{-\phi(y)}\ d\mu} & (\text{since } \chi_B(y) = 0 \text{ or } 1) \nonumber \\
				&= \int{L_\phi\chi_{[x_0, x_1, \dots, x_n]}(z) e^{-\phi(z)}\ d\mu} \nonumber \\
				&= \int{\left(\chi_{[x_0, \dots, x_n]} e^{-\phi}\right)(z)\ d\mu} \nonumber \\
				&= \int_{[x_0, \dots, x_n]}{e^{-\phi}\ d\mu}. \label{fml:prop-3-2-mu-e}
		\end{align}
		Now let $z, w \in [x_0, \dots, x_n]$. Since $\phi \in F_\theta \subset C(\Sigma, \reals)$, we have that $\var_n(\phi) \leq |\phi|_\theta \theta^n$ and hence
		\[
			e^{-|\phi|_\theta \theta^n} \leq e^{\phi(z) - \phi(w)} \leq e^{|\phi|_\theta \theta^n}.
		\]
		Then by \eqref{fml:prop-3-2-mu-e} we have
		\[
			\mu[x_0, \dots, x_n] e^{-|\phi|_\theta \theta^n} \leq \mu[x_1, \dots, x_n] e^{\phi(x)} \leq \mu[x_0, \dots, x_n] e^{|\phi|_\theta \theta^n},
		\]
		and so
		\[
			e^{-|\phi|_\theta \theta^n} \leq \frac{\mu[x_0, \dots, x_n]}{\mu[x_1, \dots, x_n]} e^{-\phi(x)} \leq e^{|\phi|_\theta \theta^n}.
		\]
	\end{proof}
\end{proposition}

\begin{corollary} \label{cor:pp-3-2-1}
	The measure $\mu$ in \thref{prop:pp-3-2} is a Gibbs measure for $\phi$ with $P = 0$.
	\begin{proof}
		We apply \thref{prop:pp-3-2} to $\phi$, $\phi \circ \sigma$, \dots, $\phi \circ \sigma^n$ to get the sequence of inequalities
		\[
			\left.
			\begin{matrix}
				e^{-|\phi|_\theta \theta^n} &\leq& \dfrac{\mu[x_0, \dots, x_n]}{\mu[x_1, \dots, x_n]} e^{-\phi(x)} &\leq& e^{|\phi|_\theta \theta^n} \\
				e^{-|\phi|_\theta \theta^{n - 1}} &\leq& \dfrac{\mu[x_1, \dots, x_n]}{\mu[x_2, \dots, x_n]} e^{-\phi(\sigma{x})} &\leq& e^{|\phi|_\theta \theta^{n - 1}} \\
				\vdots & & \vdots & & \vdots \\
				e^{-|\phi|_\theta} &\leq& \mu[x_n] e^{-\phi(\sigma^n{x})} &\leq& e^{|\phi|_\theta}
			\end{matrix}
			\right\}
		\]
		for all $x \in \Sigma$. Multiplying these together, we get
		\[
			\exp\left(-\sum_{j = 0}^n{|\phi|_\theta \theta^j}\right) \leq \frac{\mu[x_0, \dots, x_n]}{\exp\left(-\sum_{j = 0}^n{\phi(\sigma^j{x})}\right)} \leq \exp\left(\sum_{j = 0}^n{|\phi|_\theta \theta^j}\right),
		\]
		and so
		\[
			\exp\left(-\frac{|\phi|_\theta}{1 - \theta}\right) \leq \frac{\mu[x_0, \dots, x_n]}{\exp\left(-\sum_{j = 0}^n{\phi(\sigma^j{x})}\right)} \leq \exp\left(\frac{|\phi|_\theta}{1 - \theta}\right).
		\]
		Hence $\mu$ is a Gibbs measure for $\phi$ with $P = 0$.
	\end{proof}
\end{corollary}

We can generalise these results for Lipschitz functions $\phi \in F_\theta$ which are not normalised.

\begin{corollary}
	Suppose that $\phi \in F_\theta$ is a real-valued function. Let $\mu$ be the unique measure where $L_\phi^*\mu = \lambda\mu$, where $\lambda > 0$ is the maximal eigenvalue in Ruelle's Perron-Frobenius Theorem \ref{rpf:3}. Then $\mu$ is a Gibbs measure for $\phi$ with $P = \log{\lambda}$.
	\begin{proof}
		First note that $\psi := \phi - \log{(h \circ \sigma)} + \log{h} - \log{\lambda}$ is normalised, where $h > 0$ is the eigenfunction corresponding to $\lambda$ in \thref{thm:rpf}. We have
		\begin{align*}
			\exp\left(\sum_{j = 0}^{n - 1}{\psi(\sigma^j{x})}\right) &= \exp\left(\sum_{j = 0}^{n - 1}{\phi(\sigma^j{x}) - \log{h(\sigma^{j + 1}{x})} + \log{h(\sigma^j{x})} - \log{\lambda}}\right) \\
				&= \exp\left(-n\log{\lambda} + \sum_{j = 0}^{n - 1}{\phi(\sigma^j{x})}\right) \frac{h(x)}{h(\sigma^n{x})}.
		\end{align*}
		We apply \thref{cor:pp-3-2-1} to $\psi = \phi - \log{(h \circ \sigma)} + \log{h} - \log{\lambda}$ to get
		\[
			C_0^{-1} \leq \frac{\mu[x_0, \dots, x_n]}{\exp\left(-n\log{\lambda} + \sum_{j = 0}^{n - 1}{\phi(\sigma^j{x})}\right)} \leq C_0,
		\]
		for some constant $C_0 > 1$. Hence $\mu$ is a Gibbs measure for $\phi$ with $P = \log{\lambda}$.
	\end{proof}
\end{corollary}

In view of this, if $\phi \in F_\theta$ has Gibbs measure $\mu_\phi$ with $P(\phi) > 0$, then the Gibbs measure for $\phi - P(\phi)$ gives $P(\phi - P(\phi)) = 0$. In both cases we have the same Gibbs measure $\mu_\phi$.

\section{The variational principle and pressure}
There is a characteristic which sets Gibbs measures apart from other $\sigma$-invariant probability measures. To prove this property, we need some concepts and results.

Let $\mu$ be a $\sigma$-invariant probability measure on $\Sigma$. For $n \geq 0$ we have that $\mu[x_0, \dots, x_{n - 1}] > 0$ for $\mu$-almost every $x \in \Sigma$. For $j = 1, \dots, k$ we define
\begin{align*}
	\mu_n[j \mid \sigma^{-1}{x}] &:= \frac{\mu[j, x_0, \dots, x_{n - 1}]}{\mu[x_0, \dots, x_{n - 1}]} \\
		&= \frac{\mu([j] \cap \sigma^{-1}[x_0, \dots, x_{n - 1}])}{\mu(\sigma^{-1}[x_0, \dots, x_{n - 1}])} \\
		&= \mu\left([j] \midmid \bigjoin_{r = 0}^{n - 1}{\sigma^{-r}{\alpha}}\right)(x),
\end{align*}
where $\alpha = \{[j] \mid j \in \{1, \dots, k\}\}$ is the partition of $\Sigma$ by cylinders of length 1. It is clear that this is a probability distribution for $\mu$-almost every $x$.

\begin{proposition}\label{prop:mu-sq-bkt-pd}
	For $n \geq 0$, we have that
	\[
		\mu[j \mid \sigma^{-1}{x}] := \lim_{n \to +\infty}{\mu_n[j \mid \sigma^{-1}{x}]}
	\]
	is a well-defined probability distribution on $\{1, \dots, k\}$ for $\mu$-almost every $x \in \Sigma$.
	\begin{proof}
		First note that $\bigjoin_{r = 0}^{n - 1}{\sigma^{-r}{\alpha}} \to \B$, as $n \to +\infty$. By definition we also have $\mu([j] \mid \bigjoin_{r = 0}^{n - 1}{\sigma^{-r}{\alpha}}) = E_\mu(\chi_{[j]} \mid \bigjoin_{r = 0}^{n - 1}{\sigma^{-r}{\alpha}})$. Since $\chi_{[j]} \in L^1(\Sigma, \B, \mu)$, we may apply the Increasing Martingale Theorem so that
		\[
			\lim_{n \to +\infty}{\mu\left([j] \midmid \bigjoin_{r = 0}^{n - 1}{\sigma^{-r}{\alpha}}\right)} = \mu([j] \mid \B),
		\]
		for $\mu$-almost every $x$. Hence
		\[
			\mu[j \mid \sigma^{-1}{x}] = \mu([j] \mid \B)(x)
		\]
		for $\mu$-almost every $x$, i.e. $\mu[j \mid \sigma^{-1}{x}]$ is a well-defined probability distribution on $\{1, \dots, k\}$ for $\mu$-almost every $x$.
	\end{proof}
\end{proposition}

\begin{lemma} \label{lem:pp-prop-p36}
	We have
	\begin{equation}
		\sum_{j = 1}^k{\int{\psi(j, x_0, x_1, \dots) \mu[j \mid \sigma^{-1}{x}] \ d\mu}} = \int{\psi\ d\mu},
	\end{equation}
	for all $\psi \in C(\Sigma, \reals)$.
	\begin{proof}
		Let $\psi = \chi_{[r_0, \dots, r_t]}$, where $t \in \naturals_0$. We have
		\begin{align*}
			\sum_{j = 1}^k{\int{\psi(j, x_0, x_1, \dots) \mu[j \mid \sigma^{-1}{x}] \ d\mu}} &= \lim_{n \to +\infty}{\sum_{j = 1}^k{\int{\chi_{[r_0, \dots, r_t]} \mu_n[j \mid \sigma^{-1}{x}] \ d\mu}}} \\
				&= \mu[r_0, \dots, r_t] \sum_{j = 1}^k{\frac{\mu[j, x_0, \dots, x_{n - 1}]}{\mu[x_0, \dots, x_{n - 1}]}} \\
				&= \int{\psi\ d\mu}.
		\end{align*}
		So the result holds for characteristic functions. We then apply the definitions from measure theory to show that the result holds for $\psi \in C(\Sigma, \reals)$.
	\end{proof}
\end{lemma}

\begin{lemma} \label{lem:pp-prop-3-4}
	Suppose that $\phi \in F_\theta$ is a real-valued function and that $L_\phi$ is normalised. Let $\mu$ be a probability measure such that $L_\phi^*{\mu} = \mu$. Then for any $\sigma$-invariant probability measure $m$, we have
	\[
		h_m(\sigma) + \int{\phi\ dm} \leq 0,
	\]
	with equality if and only if $m = \mu$.
	\begin{proof}
		Let $m \in M(\Sigma, \sigma)$ be a $\sigma$-invariant probability measure. We define a probability distribution on $\{1, \dots, k\}$ by $\mu[j \mid \sigma^{-1}{x}]$ as in \thref{prop:mu-sq-bkt-pd}. If $m = \mu$, then $L_\phi^*{m} = m$ and so we have the probability distribution
		\[
			m[j \mid \sigma^{-1}{x}] = \exp(\phi(j, x_0, x_1, \dots))
		\]
		for all $x \in \Sigma$. We apply \thref{lem:pp-3-3} so that
		\[
			-\sum_{j = 1}^k{m[j \mid \sigma^{-1}{x}] \log{m[j \mid \sigma^{-1}{x}]}} + \sum_{j = 1}^k{m[j \mid \sigma^{-1}{x}] \phi(j, x_0, x_1, \dots)} \leq 0,
		\]
		for $m$-almost every $x$, with equality if and only if $m[j \mid \sigma^{-1}{x}] = \phi(j, x_0, x_1, \dots)$ for all $j = 1, \dots, k$.
		
		We integrate with respect to $m$ and apply \thref{lem:pp-prop-p36} to get
		\[
			h_m(\sigma) + \sum_{j = 1}^k{\int{m[j \mid \sigma^{-1}{x}] \phi(j, x_0, x_1, \dots)\ dm}} = h_m(\sigma) + \int{\phi\ dm} \leq 0,
		\]
		with equality if and only if $m[j \mid \sigma^{-1}{x}] = \phi(j, x_0, x_1, \dots)$ for $m$-almost every $x$. By \thref{lem:pp-prop-p36}, this equality condition is equivalent to
		\begin{align*}
			\int{\sum_{j = 1}^k{m[j \mid \sigma^{-1}{x}] \psi(j, x_0, x_1, \dots)}\ dm} &= \int{\sum_{j = 1}^k{\phi(j, x_0, x_1, \dots) \psi(j, x_0, x_1, \dots)}\ dm} \\
				&= \int{\psi\ dm}
		\end{align*}
		for all $\psi \in C(\Sigma, \reals)$. In other words, $\int{L_\phi{\psi}\ dm} = \int{g\ dm}$ for all $\psi$, and so we have $L_\phi^*{m} = m$. By Ruelle's Perron-Frobenius Theorem \ref{rpf:3}, $\mu$ is the unique $\sigma$-invariant probability measure such that $L_\phi^*{\mu} = \mu$, so we have
		\[
			h_m(\sigma) + \int{\phi\ dm} \leq 0,
		\]
		with equality if and only if $m = \mu$.
	\end{proof}
\end{lemma}

It can be shown that similar results hold for 2-sided shifts of finite type, and also for $\phi \in F_\theta$ where $L_\phi$ is not necessarily normalised.

The following result shows one of the main distinguishing characteristics of Gibbs measures.

\begin{theorem}[Variational Principle] \label{thm:variational-principle}
	Suppose that $\phi \in F_\theta$ (or $F_\theta^+$). The Gibbs measure $\mu_\phi$ is the unique $\sigma$-invariant probability measure such that
	\[
		h_m(\sigma) + \int{\phi\ dm} \leq h_{\mu_\phi}(\sigma) + \int{\phi\ d\mu_\phi}
	\]
	for all $m \in M(\Sigma, \sigma)$, with equality if and only if $m = \mu_\phi$.
	\begin{proof}
		Let $\phi \in F_\theta$. By \thref{prop:pp-1-2}, there exists $g \in F_{\theta^{1 / 2}}^+$, $u \in F_{\theta^{1 / 2}}$ such that $\phi = g + (u \circ \sigma) - u$. By Ruelle's Perron-Frobenius Theorem, we can write $g = \log(h \circ \sigma) - \log{h} + \log{\lambda} + k$, for some $k \in F_{\theta^{1/2}}^+$ such that $L_k^*{\mu_\phi} = \mu_\phi$ and $L_k$ is normalised.
		
		Let $m$ be a $\sigma$-invariant probability measure. By \thref{lem:pp-prop-3-4} we have
		\[
			h_m(\sigma) + \int{k\ dm} \leq h_{\mu_\phi}(\sigma) + \int{k\ d\mu_\phi} = 0.
		\]
		Substituting in $k = \phi - (u \circ \sigma) + u - \log(h \circ \sigma) + \log{h} - \log{\lambda}$ and cancelling terms, we get
		\[
			h_m(\sigma) + \int{\phi\ dm} \leq h_{\mu_\phi}(\sigma) + \int{\phi\ d\mu_\phi},
		\]
		with equality if and only if $m = \mu_\phi$.
	\end{proof}
\end{theorem}

From the proof of \thref{thm:variational-principle}, we see that
\[
	P(\phi) = \sup_{m \in M(\Sigma, \sigma)}\left\{h_m(\sigma) + \int{\phi\ dm}\right\} = h_{\mu_\phi}(\sigma) + \int{\phi\ d\mu_\phi}.
\]
So $P(\phi) = \log{\lambda}$, where $\lambda$ is the maximal positive eigenvalue for $L_{\phi'}$, where $\phi$ is cohomologous to $\phi' \in F_{\theta^{1 / 2}}^+$.

\begin{definition}
	We call
	\[
		P(\phi) := \sup_{m \in M(\Sigma, \sigma)}\left\{h_m(\sigma) + \int{\phi\ dm}\right\}
	\]
	the \key{pressure} of $\phi$.
	
	If a measure a $\sigma$-invariant probability measure $m$ attains this supremum, i.e. $P(\phi) = h_m(\sigma) + \int{\phi\ dm}$, then we say that $m$ is an \key{equilibrium state}.
\end{definition}

The Variational Principle gives that if we have $\phi \in F_\theta$, then the equilibrium state is unique and we can also define the pressure of $\phi$ by $P(\phi) = \log{\lambda}$.

\section{Gibbs measures are weak Bernoulli}
We now describe a particular property of Gibbs measures which we will use in Subsection \ref{ssec:hitting-times}. The following definitions and results follow \cite[Section 1.E]{bowen:equilibrium}.

\begin{definition}
	Let $\beta, \gamma$ be two finite partitions of a measure space $(X, \B, \mu)$ and let $\varepsilon > 0$ be given. We say that $\beta$ and $\gamma$ are \key{$\varepsilon$-independent} if
	\[
		\sum_{B \in \beta,\ C \in \gamma}{|\mu(B \cap C) - \mu(B)\mu(C)|} < \varepsilon.
	\]
\end{definition}

\begin{definition}
	Let $\xi = \left\{[j] \mid j \in \{1, \dots, k\}\right\}$ be the partition of $(\Sigma, \B, \mu)$ by cylinders of length 1. We say that $\xi$ is \key{weak Bernoulli} (for $\sigma$ and $\mu$) if for all $\varepsilon > 0$ there exists $N(\varepsilon) > 0$ such that for all $n \geq 1$, then the partitions
	\[
		\beta = \bigjoin_{j = 0}^n{\sigma^{-j}(\xi)} \quad \text{and} \quad \gamma = \bigjoin_{j = t}^{t + r}{\sigma^{-j}(\xi)}
	\]
	are $\varepsilon$-independent for all $r \geq 0$ and for all $t \geq n + N(\varepsilon)$.
\end{definition}

Before we state the main result, we need an auxiliary lemma.

\begin{lemma}\label{bowen:lem-1-12}
	Let $r \geq 0$, $f \in C(\Sigma, \reals)$ and $\var_r(f) = 0$. Let
	\begin{align*}
		F \in \{g \in C(\Sigma, \reals) \mid g & \geq 0,\ \nu(g) = 1,\ g(x) \leq B_m g(x') \text{ whenever } x_j = x'_j \text{ for all } j = 0, \dots, m\},
	\end{align*}
	where $B_m := \exp\left(\sum_{k = m + 1}^\infty{2b\alpha^k}\right)$, where $b > 0$, $\alpha \in (0, 1)$ are any pair of constants which satisfy $\var_k(\phi) \leq b\alpha^k$ for all $k > 0$.
	
	Then for any $n \geq 0$ we have
	\[
		\|\lambda^{-n - r}L_\phi^{n + r}(fF) - \nu(fF)h\| \leq M\nu(|fF|)\rho^n,
	\]
	where $\nu$, $\lambda$, $h$ are as in the Ruelle's Perron Frobenius Theorem, and $M > 0$, $\rho \in (0, 1)$ are constants.
\end{lemma}

\begin{theorem}\label{thm:gibbs-is-weak-bernoulli}
	Let $\xi = \left\{[j] \mid j \in \{1, \dots, k\}\right\}$ be the partition of $(\Sigma, \B, \mu)$ by cylinders of length 1. Then $\xi$ is weak Bernoulli for the Gibbs measure $\mu_\phi$.
	\begin{proof}
		Let $\varepsilon > 0$ be given. Suppose that $\phi \in C(\Sigma, \reals)$, $n \geq 1$ and $t \geq n + N(\varepsilon)$, for some $N(\varepsilon)$. Let $\beta, \gamma$ be partitions of $\Sigma$ defined by
		\[
		\beta = \bigjoin_{j = 0}^n{\sigma^{-j}(\xi)} \quad \text{and} \quad \gamma = \bigjoin_{j = t}^{t + r}{\sigma^{-j}(\xi)}.
		\]
		For all $B \in \beta$ we clearly have $\chi_B \in C(\Sigma, \reals)$ and $\var_r(\chi_B) = 0$ for all $r \geq 0$.
		
		Now consider $C \in \gamma$. Since $t \geq n$, we know that $B$ consists of cylinders of lengths strictly less than the lengths of cylinders in $C$, and therefore $\sigma^{-t}C$. It follows that the intersection $B \cap C$ depends only on $B$, and hence
		\[
			\mu_\phi(B \cap C) = \mu_\phi(B \cap \sigma^{-t}{C}).
		\]
		We then have, where $\nu$, $h$ and $\lambda$ are as in the Ruelle's Perron Frobenius Theorem,
		\begin{align*}
			\mu_\phi(B \cap C) &= \mu_\phi(B \cap \sigma^{-t}C) \\
				&= \mu_\phi(\chi_B \cdot \chi_{\sigma^{-t}}{C}) \\
				&= \mu_\phi(\chi_B \cdot (\chi_{C} \circ \sigma^t)) \\
				&= \nu(h \chi_B \cdot (\chi_C \circ \sigma^t)) \\
				&= \lambda^{-t}(L_\phi^*)^t\nu(h\chi_B \cdot (\chi_C \circ \sigma^t)) \\
				&= \nu(\lambda^{-t} L_\phi^t(h \chi_B \cdot (\chi_C \circ \sigma^t))) \\
				&= \nu(\lambda^{-t} L_\phi^t(h \chi_B) \cdot \chi_C).
		\end{align*}
		Consequently,
		\begin{align*}
			|\mu_\phi(B \cap C) - \mu_\phi(B)\mu_\phi(C)| &= |\nu(\lambda^{-t} L_\phi^t(h \chi_B) \cdot \chi_C) - \nu(h \chi_B)\nu(h \chi_C)| \\
				&= |\nu((\lambda^{-t} L_\phi^t(h \chi_B) - \nu(h \chi_B)h)\chi_C)| \\
				&\leq \|\lambda^{-t} L_\phi^t(h \chi_B) - \nu(h \chi_B)h\| \nu(C).
		\end{align*}
		Since $\chi_B \in C(\Sigma, \reals)$ and $\var_r(\chi_B) = 0$ for all $r \geq 0$, we may apply \thref{bowen:lem-1-12}. So if $t \geq s$, then
		\[
			\|\lambda^{-t}L_\phi^{t}(h \chi_B) - \nu(h \chi_B)h\| \leq M\nu(h \chi_B)\rho^{t - s},
		\]
		where $M > 0$, $\rho \in (0, 1)$ are constants. We therefore have
		\begin{align*}
			|\mu_\phi(B \cap C) - \mu_\phi(B)\mu_\phi(C)| &\leq M\nu(h \chi_B)\rho^{t - s} \nu(C) \\
				&= M\mu_\phi(B)\rho^{t - s} \nu(C) \\
				&= M'\mu_\phi(B)\mu_\phi(C)\rho^{t - s},
		\end{align*}
		where $M' = M(\inf{h})^{-1}$. Summing over all elements in the partitions $\beta, \gamma$, we get
		\[
			\sum_{B \in \beta,\ C \in \gamma}{|\mu_\phi(B \cap C) - \mu_\phi(B)\mu_\phi(C)|} \leq M'\rho^{t - s} < \varepsilon
		\]
		for sufficiently large $t - s$.
		
		Hence $\mu_\phi$ is weak Bernoulli.
	\end{proof}
\end{theorem}
