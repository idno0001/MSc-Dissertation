\chapter{Background}
\section{Preliminaries}
This section briefly introduces some concepts which will be used throughout this dissertation.

\begin{definition}
	Let $(X, d_X)$ and $(Y, d_Y)$ be metric spaces. A function $f : X \to Y$ is a \key{Lipschitz function} if there exists a constant $C \in \reals^+$ such that
	\[
	d_Y(f(x), f(y)) \leq Cd_X(x, y)
	\]
	for all $x, y \in X$. We say that $f$ is a Lipschitz function with \key{Lipschitz constant} $C$.~\cite[p154]{searcoid:metric-spaces}
	
	More generally, $f$ is \key{H\"older continuous} if there exists $C \in \reals^+$ and $\alpha \in (0, 1]$ such that
	\[
	d_Y(f(x), f(y)) \leq C(d_X(x, y))^\alpha
	\]
	for all $x, y \in X$. In this case, we say that $f$ is H\"older continuous with \key{H\"older exponent} $\alpha$ and \key{H\"older constant} $C$.~\cite[p143]{brin-stuck:dynsys}
\end{definition}

\section{Shifts of finite type}
\emph{A large portion of this section follows \cite[Chapter 1]{parry-pollicott:zeta-fns-periodic-orbits}.}
\subsection{The basics}
Let $A$ be a $k \times k$ matrix with entries in $\{0, 1\}$. A \key{(two-sided) shift of finite type} $\Sigma_A$ is defined by
\[
	\Sigma_A = \{(x_j)_{j = -\infty}^\infty \mid A_{x_j, x_{j + 1}} = 1,\ j \in \integers\}.
\]
Similarly, a \key{(one-sided) shift of finite type} $\Sigma_A^+$ is defined by
\[
	\Sigma_A^+ = \{(x_j)_{j = 0}^\infty \mid A_{x_j, x_{j + 1}} = 1,\ j \in \naturals\}.
\]

Let $x = (x_j)_{j = -\infty}^\infty \in \Sigma_A$. We define the \key{(two-sided, left) shift map} $\sigma : \Sigma_A \to \Sigma_A$ by
\[
	(\sigma(x))_j = x_{j + 1}.
\]
which shifts each coordinate of $x$ one position to the left.

Now let $x = (x_j)_{j = 0}^\infty \in \Sigma_A^+$. We similarly define the \key{(one-sided, left) shift map} $\sigma^+ : \Sigma_A^+ \to \Sigma_A^+$ by
\[
	(\sigma^+(x))_j = x_{j + 1}.
\]
As with the one-sided case, this shifts the coordinates of $x$ one position to the left but also deletes the first coordinate $x_0$. It is clear that $\sigma^+$ is not invertible, whereas $\sigma$ is invertible.

Let $x = (x_j)_{j = -\infty}^\infty$. We call $(x_j)_{j = -\infty}^0$ the \key{past}, $x_0$ the \key{present}, and $(x_j)_{j = 0}^\infty$ the \key{future}.

Let $A$ be a $k \times k$ matrix with entries in $\{0, 1\}$. We say that $A$ is \key{irreducible} if, for each $i, j \in \{1, \dots, k\}$, there exists $n = n(i, j) > 0$ such that $(A^n)_{i, j} > 0$. If there exists $n > 0$ such that $(A^n)_{i, j} > 0$ for all $i, j \in \{1, \dots, k\}$, then we say that $A$ is \key{aperiodic}.

If we consider $A$ as an adjacency matrix of a graph $G_A$, then $A$ is irreducible if there exists an edge-path between any two vertices in $G_A$. If all edge-paths can be chosen to be of the same length, then $A$ is aperiodic.

A shift of finite type $\Sigma_A$ is irreducible or aperiodic if the associated matrix $A$ is irreducible or aperiodic, respectively.

A \key{cylinder} $C$ on $\Sigma_A$ is defined
\[
	C = [i_{-m}, \dots, i_{-1}, i_0, i_1, \dots, i_n]_{-m, n} = \{(x_j)_{j = -\infty}^\infty \in \Sigma \mid x_j = i_j \text{ for } -m \leq j \leq n\}.
\]
Similarly, a cylinder $C^+$ on $\Sigma_A^+$ is given by
\[
	C^+ = [i_0, i_1, \dots, i_{n - 1}, i_n]_{0, n} = \{(x_j)_{j = 0}^\infty \in \Sigma \mid x_j = i_j \text{ for } 0 \leq j \leq n\}.
\]
That is, a cylinder is the set of all sequences which agree in the given positions.

\subsection{Function spaces for shifts of finite type}
\subsubsection{Two-sided shifts of finite type}
Let $\Sigma = \Sigma_A$ be a shift of finite type and let $\theta \in (0, 1)$ be given.

Let $x = (x_j)_{j = -\infty}^\infty, y = (y_j)_{j = -\infty}^\infty \in \Sigma$. We define $n = n(x, y) \geq 0$ to be the largest integer such that $x_j = y_j$ for all $|j| < n$, but $x_n \neq y_n$ or $x_{-n} \neq y_{-n}$. If $x_j = y_j$ for all $j \in \integers$ then we define $n = +\infty$.

We define the map $d_\theta : \Sigma \times \Sigma \to \reals^+$ by
\[
	d_\theta(x, y) =
	\begin{cases}
		\theta^n, & \text{if } x \neq y; \\
		0 & \text{if } x = y.
	\end{cases}
\]
It can be shown that $d_\theta$ is a \key{metric} on $\Sigma$, so sequences in $\Sigma$ are `close' if they agree for a large number of leading coordinates.

Let $f : \Sigma \to \complex$ be a continuous function and let $n \geq 0$. We define the \key{$n$-th variation of $f$} by
\[
	\var_n(f) = \sup\{|f(x) - f(y)| \mid x, y \in \Sigma,\ x_j = y_j \text{ for } |j| < n\}.
\]
That is, $\var_n(f)$ is a measure of how much $f$ varies on cylinders of length $n$.~\cite[Lecture 8]{magic-ergodic}

It is easy to see that $|f(x) - f(y)| \leq Cd_\theta(x, y)$ if and only if $\var_n(f) \leq C\theta^n$ for $n = 0, 1, 2, \dots$, where $C > 0$ is some constant.

We define
\begin{align*}
	F_\theta = F_\theta(\Sigma) = \{f : \Sigma \to \complex &\mid f \text{ is continuous}, \ \var_n(f) \leq C\theta^n \text{ for all } n \geq 0, \\
			& \quad \text{for some } C > 0\}
\end{align*}
to be the space of Lipschitz functions with respect to the metric $d_\theta$.

Let $f \in F_\theta(\Sigma)$. Define
\[
	|f|_\theta = \sup_{n \geq 0}\left\{\frac{\var_n(f)}{\theta^n}\right\}
\]
to be the least Lipschitz constant of $f$. In other words, $|f|_\theta$ is the smallest $C > 0$ such that $\var_n(f) \leq C\theta^n$ for all $n \geq 0$.

We can then define a \key{norm} on $F_\theta$ by
\[
	\|f\|_\theta = |f|_\infty+ |f|_\theta,
\]
where $|f|_\infty = \sup_{x \in \Sigma}\{|f(x)|\}$.

\subsubsection{One-sided shifts of finite type}
Similar definitions hold for one-sided shifts of finite type $\Sigma^+ = \Sigma_A^+$.

If $x = (x_j)_{j = 0}^\infty, y = (y_j)_{j = 0}^\infty \in \Sigma^+$ then we define $n = n(x, y) \geq 0$ to be the largest integer such that $x_j = y_j$ for all $0 \leq j < n$ but $x_n \neq y_n$. We then define the metric $d_\theta : \Sigma^+ \times \Sigma^+ \to \reals^+$ in the same way as before.

The \key{$n$-th variation of $f$} is given by
\[
	\var_n(f) = \sup\{|f(x) - f(y)| \mid x, y \in \Sigma^+,\ x_j = y_j \text{ for } 0 \leq j < n\}.
\]
The space of Lipschitz functions with respect to $d_\theta$ is similarly given by
\begin{align*}
	F_\theta^+ = F_\theta^+(\Sigma^+) = \{f : \Sigma^+ \to \complex &\mid f \text{ is continuous}, \ \var_n(f) \leq C\theta^n \text{ for all } n \geq 0, \\
			& \quad \text{for some } C > 0\}.
\end{align*}
If $f \in F_\theta^+(\Sigma^+)$ then the definitions of $|f|_\theta$, $|f|_\infty$ and $\|f\|_\theta$ are identical to the two-sided analogues.

\begin{proposition}
	The spaces $(F_\theta, \|\cdot\|_\theta)$ and $(F_\theta^+, \|\cdot\|_\theta)$ are Banach spaces.
	
	Moreover, if $f, g \in F_\theta$ (or $F_\theta^+$) then $\|fg\|_\theta \leq \|f\|_\theta |g|_\infty + |f|_\infty \|g\|_\theta$. If $f$ is nowhere zero then $\|1 / f\|_\theta \leq |1 / f^2|_\infty \|f\|_\theta$.
\end{proposition}

\begin{proposition}
	Suppose $f \in F_\theta$. Then there exists $g, h \in F_{\theta^{\frac{1}{2}}}$ such that $f = g + h - h \circ \sigma$, where $g(x) = g(y)$ if $x_j = y_j$ for all $j \geq 0$, i.e. the value of $g(x)$ is determined only by the future coordinates of $x$.
\end{proposition}

If a function $f : \Sigma \to \complex$ is $\alpha$-H\"older continuous, $\alpha \in (0, 1]$, then it is a Lipschitz function with respect to $d_{\theta^\alpha}$.

Suppose we have $0 < \theta < \theta' < 1$. Then
\[
	F_{\theta'}(\Sigma) \supset F_\theta(\Sigma) \quad \text{and} \quad F_{\theta'}^+(\Sigma^+) \supset F_\theta^+(\Sigma^+).
\]
We can therefore define
\[
	F = \bigcup_{0 < \theta < 1}{F_\theta(\Sigma)} \quad \text{and} \quad F^+ = \bigcup_{0 < \theta < 1}{F_\theta^+(\Sigma^+)},
\]
the spaces of all H\"older continuous functions.

We now consider a class of functions lying in $F_\theta^+$ for all $0 < \theta < 1$. For all $m \geq 1$ we define
\[
	F_m^+ = \{f : \Sigma^+ \to \complex \mid f(x) = f(y) \text{ if } x_j = y_j, \text{ for all } 0 \leq j < m\},
\]
the set of all locally constant functions which depend on the first $m$ terms of $x \in \Sigma^+$. It is clear that
\[
	F_1^+ \subset F_2^+ \subset F_3^+ \subset \dots.
\]
and, for $f \in F_m^+$, $\var_m(f) = 0$. Hence
\[
	\bigcup_{m = 1}^\infty{F_m^+} \subset \bigcap_{0 < \theta < 1}{F_\theta^+}.
\]

\begin{proposition}
	Suppose $0 < \theta < \theta' < 1$. Then for all $m \geq 0$ we have
	\begin{equation*}
		|f - f_m|_{\theta'} \leq |f|_\theta \left(\frac{\theta}{\theta'}\right)^m.
	\end{equation*}
\end{proposition}

\section{The Ruelle operator}
\begin{theorem}[Ruelle's Perron-Frobenius Theorem]
	Coming soon!
\end{theorem}
