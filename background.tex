\chapter{Background}
\section{Preliminaries}
This section briefly introduces some concepts which will be used throughout this dissertation.

\begin{definition}
	Let $(X, d_X)$ and $(Y, d_Y)$ be metric spaces. A function $f : X \to Y$ is a \key{Lipschitz function} if there exists a constant $K \in \reals^+$ such that
	\[
		d_Y(f(x), f(y)) \leq Kd_X(x, y)
	\]
	for all $x, y \in X$. We say that $f$ is a Lipschitz function with \key{Lipschitz constant} $K$.~\cite[p154]{searcoid:metric-spaces}
	
	More generally, we say that $f$ is \key{H\"older continuous} if there exists some constants $K \in \reals^+$ and $\alpha \in (0, 1]$ such that
	\[
	d_Y(f(x), f(y)) \leq K(d_X(x, y))^\alpha
	\]
	for all $x, y \in X$. In this case, we say that $f$ is H\"older continuous with \key{H\"older exponent} $\alpha$ and \key{H\"older constant} $K$.~\cite[p143]{brin-stuck:dynsys}
\end{definition}

\begin{definition}
	A transformation $T : (X_1, \B_1, \mu_1) \to (X_2, \B_2, \mu_2)$ is \key{measure-preserving} if:
	\begin{enumerate}
		\item $T$ is measurable, i.e. if $B_2 \in \B_2$, then $T^{-1}{B_2} \in \B_1$, and
		\item $\mu_1(T^{-1}{B_2}) = \mu_2(B_2)$ for all $B_2 \in \B_2$.
	\end{enumerate}
	This agrees with our usual definition when $X_1 = X_2$.
\end{definition}

\begin{definition}
	Let $X$ be a compact metric space with Borel $\sigma$-algebra $\B$. We let $M(X)$ denote the set of all probability measures on $(X, \B)$.
	
	Let $T : X \to X$ be a continuous mapping on $X$. We let $M(X, T)$ denote the set of $T$-invariant probability measures on $(X, B)$.
\end{definition}

\begin{definition}
	The \key{symmetric difference} of two sets $A, B$ is defined $(A \setminus B) \cup (B \setminus A)$. We will write this as
	\begin{equation*}
		A \symdiff B = (A \setminus B) \cup (B \setminus A).
	\end{equation*}
\end{definition}

\section{Shifts of finite type}
\emph{A large portion of this section follows \cite[Chapter 1]{parry-pollicott:zeta-fns-periodic-orbits}.}
\subsection{The basics}
Let $A$ be a $k \times k$ matrix with entries in $\{0, 1\}$. A \key{(two-sided) shift of finite type} $\Sigma_A$ is defined by
\[
	\Sigma_A = \{(x_j)_{j = -\infty}^\infty \mid A_{x_j, x_{j + 1}} = 1,\ j \in \integers\}.
\]
Similarly, a \key{(one-sided) shift of finite type} $\Sigma_A^+$ is defined by
\[
	\Sigma_A^+ = \{(x_j)_{j = 0}^\infty \mid A_{x_j, x_{j + 1}} = 1,\ j \in \naturals_0\}.
\]

Let $x = (x_j)_{j = -\infty}^\infty \in \Sigma_A$. We define the \key{(two-sided, left) shift map} $\sigma : \Sigma_A \to \Sigma_A$ by
\[
	(\sigma(x))_j = x_{j + 1}.
\]
which shifts each coordinate of $x$ one position to the left.

Now let $x = (x_j)_{j = 0}^\infty \in \Sigma_A^+$. We similarly define the \key{(one-sided, left) shift map} $\sigma^+ : \Sigma_A^+ \to \Sigma_A^+$ by
\[
	(\sigma^+(x))_j = x_{j + 1}.
\]
As with the one-sided case, this shifts the coordinates of $x$ one position to the left but also deletes the first coordinate $x_0$. It is clear that $\sigma^+$ is not invertible, whereas $\sigma$ is invertible.

To avoid excessive use of subscripts and superscripts, we will often write $\sigma$ for both the one-sided and two-sided shift maps. It should be clear from the context which of these maps $\sigma$ denotes.

If $x = (x_j)_{j = -\infty}^\infty \in \Sigma_A$, then we call $(x_j)_{j = -\infty}^0$ the \key{past}, $x_0$ the \key{present}, and $(x_j)_{j = 0}^\infty$ the \key{future}.

\subsubsection{Irreducibility, aperiodicity and cylinders}
Let $A$ is a $k \times k$ matrix with entries in $\{0, 1\}$, and let $\Sigma_A$ (or $\Sigma_A^+$) be the associated shift of finite type. We may consider $A$ to be the adjacency matrix of a directed graph $G_A$ with $k$ vertices.

We say that $A$ is \key{irreducible} if, for each $i, j \in \{1, \dots, k\}$, there exists $n = n(i, j) > 0$ such that $(A^n)_{i, j} > 0$. Alternatively, $A$ is irreducible if there exists an edge-path between any two vertices in the corresponding graph $G_A$. In this case, we say that the shift of finite type $\Sigma_A$ (or $\Sigma_A^+$) is irreducible.

If there exists $n > 0$ such that $(A^n)_{i, j} > 0$ for all $i, j \in \{1, \dots, k\}$, then we say that $A$ is \key{aperiodic}. That is, $A$ is aperiodic if all edge-paths between any two vertices in $G_A$ can be chosen to be of the same length. As before, this means that $\Sigma_A$ (or $\Sigma_A^+$) is aperiodic.

A \key{cylinder} $C$ on $\Sigma_A$ is defined
\[
	C = [i_{-m}, \dots, i_{-1}, i_0, i_1, \dots, i_n]_{-m, n} = \{(x_j)_{j = -\infty}^\infty \in \Sigma \mid x_j = i_j \text{ for } -m \leq j \leq n\}.
\]
Similarly, a cylinder $C^+$ on $\Sigma_A^+$ is given by
\[
	C^+ = [i_0, i_1, \dots, i_{n - 1}, i_n]_{0, n} = \{(x_j)_{j = 0}^\infty \in \Sigma \mid x_j = i_j \text{ for } 0 \leq j \leq n\}.
\]
In other words, a cylinder is the set of all sequences which agree in the given positions.

\subsection{Function spaces for shifts of finite type}
Let $\Sigma_A$ and $\Sigma_A^+$ be two-sided and one-sided shifts of finite type, respectively, and let $\theta \in (0, 1)$ be fixed.

\subsubsection{Metrics for shifts of finite type}
Let $x = (x_j)_{j = -\infty}^\infty, y = (y_j)_{j = -\infty}^\infty \in \Sigma_A$. We define $n = n(x, y) \geq 0$ to be the largest integer such that $x_j = y_j$ for all $|j| < n$, but $x_n \neq y_n$ or $x_{-n} \neq y_{-n}$. If $x_j = y_j$ for all $j \in \integers$ then we define $n = +\infty$.

We define the map $d_\theta : \Sigma_A \times \Sigma_A \to \reals^+$ by
\[
	d_\theta(x, y) =
	\begin{cases}
		\theta^n, & \text{if } x \neq y; \\
		0 & \text{if } x = y.
	\end{cases}
\]
It can be shown that $d_\theta$ is a \key{metric} on $\Sigma_A$, so sequences in $\Sigma_A$ are `close' if they agree for a large number of leading coordinates.

Similarly, if $x = (x_j)_{j = 0}^\infty, y = (y_j)_{j = 0}^\infty \in \Sigma_A^+$, then $n = n(x, y) \geq 0$ is defined to be the largest integer such that $x_j = y_j$ for all $0 \leq j < n$ but $x_n \neq y_n$. We define the map $d_\theta : \Sigma_A^+ \times \Sigma_A^+ \to \reals^+$ in the same way as the two-sided case, and it can be shown that $d_\theta$ is also a metric.

\subsubsection{The space of Lipschitz functions}
\begin{definition}
	Let $f : \Sigma_A \to \complex$ be a continuous function and let $n \geq 0$. We define the \key{$n$-th variation of $f$} by
	\[
		\var_n(f) = \sup\{|f(x) - f(y)| \mid x, y \in \Sigma_A,\ x_j = y_j \text{ for } |j| < n\}.
	\]
	
	Similarly, if $g : \Sigma_A^+ \to \complex$ is a continuous function, then the \key{$n$-th variation of $g$} is given by
	\[
	\var_n(g) = \sup\{|g(x) - g(y)| \mid x, y \in \Sigma_A^+,\ x_j = y_j \text{ for } 0 \leq j < n\}.
	\]
	That is, $\var_n(g)$ indicates how much $g$ varies on cylinders of length $n$.~\cite[Lecture 8]{magic-ergodic}
\end{definition}

It is easy to see that $\var_n(f) \leq K\theta^n$ for all $n \geq 0$ if and only if $|f(x) - f(y)| \leq Kd_\theta(x, y)$, i.e. $f$ is a Lipschitz function. This allows the following definition to be given in terms of $n$-th variations of continuous functions.

\begin{definition}
	Define
	\[
		F_\theta = F_\theta(\Sigma_A) = \{f \in C(\Sigma_A, \complex) \mid \var_n(f) \leq K\theta^n \text{ for all } n \geq 0, \text{ for some } K > 0\}
	\]
	to be the space of Lipschitz functions with respect to the metric $d_\theta$.
	
	We define $F_\theta^+ = F_\theta^+(\Sigma_A^+)$ in the same way, replacing $\Sigma_A$ with $\Sigma_A^+$.
\end{definition}

\begin{definition}
	Let $f \in F_\theta$ (or $F_\theta^+)$. Define
	\[
		|f|_\theta = \sup_{n \geq 0}\left\{\frac{\var_n(f)}{\theta^n}\right\}
	\]
	to be the least Lipschitz constant of $f$. In other words, $|f|_\theta$ is the smallest $K > 0$ such that $\var_n(f) \leq K\theta^n$ for all $n \geq 0$.
	
	We can then define a \key{norm} on $F_\theta$ (or $F_\theta^+$) by
	\[
		\|f\|_\theta = |f|_\infty+ |f|_\theta,
	\]
	where $|f|_\infty = \sup_{x \in \Sigma}\{|f(x)|\}$.
\end{definition}

\begin{proposition}
	The spaces $(F_\theta, \|\cdot\|_\theta)$ and $(F_\theta^+, \|\cdot\|_\theta)$ are Banach spaces.
	%
	%Moreover, if $f, g \in F_\theta$ (or $F_\theta^+$) then $\|fg\|_\theta \leq \|f\|_\theta |g|_\infty + |f|_\infty \|g\|_\theta$. If $f$ is nowhere zero then $\|1 / f\|_\theta \leq |1 / f^2|_\infty \|f\|_\theta$.
\end{proposition}

\begin{definition}
	Let $f, g \in F_\theta$ (or $F_\theta^+$). We say $f$ and $g$ are \key{cohomologous} if there exists a continuous function $h$ such that $f = g + h \circ \sigma - h$. In this case, we write $f \sim g$.
	
	If $f$ is cohomologous to $0$, then we say $f$ is a \key{coboundary}.
\end{definition}

\begin{remark}
	It is clear that $\sim$ is an equivalence relation.
\end{remark}

\begin{proposition} \label{prop:pp-1-2}
	Suppose $f \in F_\theta$. Then there exists $g, h \in F_{\theta^{1 / 2}}$ such that $f = g + h - h \circ \sigma$, where $g(x) = g(y)$ if $x_j = y_j$ for all $j \geq 0$. In other words, the value of $g(x)$ is determined only by the future coordinates of $x$.
	\begin{proof}
		\hl{TODO: Prove this.}
	\end{proof}
\end{proposition}

\begin{comment}
If a function $f : \Sigma \to \complex$ is $\alpha$-H\"older continuous, $\alpha \in (0, 1]$, then it is a Lipschitz function with respect to $d_{\theta^\alpha}$.

Suppose we have $0 < \theta < \theta' < 1$. Then
\[
	F_{\theta'}(\Sigma) \supset F_\theta(\Sigma) \quad \text{and} \quad F_{\theta'}^+(\Sigma^+) \supset F_\theta^+(\Sigma^+).
\]
We can therefore define
\[
	F = \bigcup_{0 < \theta < 1}{F_\theta(\Sigma)} \quad \text{and} \quad F^+ = \bigcup_{0 < \theta < 1}{F_\theta^+(\Sigma^+)},
\]
the spaces of all H\"older continuous functions.


We now consider a class of functions lying in $F_\theta^+$ for all $0 < \theta < 1$. For all $m \geq 1$ we define
\[
	F_m^+ = \{f : \Sigma^+ \to \complex \mid f(x) = f(y) \text{ if } x_j = y_j, \text{ for all } 0 \leq j < m\},
\]
the set of all locally constant functions which depend on the first $m$ terms of $x \in \Sigma^+$. It is clear that
\[
	F_1^+ \subset F_2^+ \subset F_3^+ \subset \dots.
\]
and, for $f \in F_m^+$, $\var_m(f) = 0$. Hence
\[
	\bigcup_{m = 1}^\infty{F_m^+} \subset \bigcap_{0 < \theta < 1}{F_\theta^+}.
\]

\begin{proposition}
	Suppose $0 < \theta < \theta' < 1$. Then for all $m \geq 0$ we have
	\begin{equation*}
		|f - f_m|_{\theta'} \leq |f|_\theta \left(\frac{\theta}{\theta'}\right)^m.
	\end{equation*}
\end{proposition}
\end{comment}

\section{The Ruelle operator}
Throughout this section, let $\Sigma = \Sigma_A^+$ be a (one-sided) shift of finite type.

\begin{definition}
	Let $f \in F_\theta^+$. The \key{Ruelle operator} (or \key{transfer operator}) $L_f : F_\theta^+ \to F_\theta^+$ (or, more generally, $L_f : C(\Sigma, \complex) \to C(\Sigma, \complex)$) is defined
	\[
		(L_f{w})(x) = \sum_{y \in \Sigma \midcolon \sigma{y} = x}{e^{f(y)} w(y)} = \sum_{j \midcolon A_{j, x_0} = 1}{e^{f(j, x_0, x_1, \dots)} w(j, x_0, x_1, \dots)},
	\]
	where $x = (x_j)_{j = 0}^\infty \in \Sigma$. This is a bounded linear operator.
	
	The $n$-th iterate of $L_f$ is given by
	\[
		(L_f^n{w})(x) = \sum_{y \in \Sigma \midcolon \sigma^n{y} = x}{e^{f^n(y)} w(y)}.
	\]
	
	If $f$ is also real-valued and $L_f{1} = 1$, then we say that $f$ or $L_f$ is \key{normalised}.
\end{definition}

\begin{proposition}
	Let $f \in F_\theta^+$ with $f = u + iv$, where $u, v \in F_\theta^+$ are real-valued functions. If $L_u$ is normalised, i.e. $L_u{1} = 1$, then for all $n \geq 0$,
	\[
		|L_f^n{w}|_\theta \leq K|w|_\infty + \theta^n |w|_\theta
	\]
	for all $w \in F_\theta^+$, where $K > 0$ is a constant depending only on $f$ and $\theta$.
\end{proposition}

\begin{theorem}[Ruelle's Perron-Frobenius Theorem] \label{thm:rpf}
	Suppose $\Sigma = \Sigma_A^+$ is an aperiodic shift of finite type and let $f \in F_\theta^+$ be a real-valued function. Then
	\begin{enumerate}
		\item There is a simple maximal eigenvalue $\lambda$ of $L_f : C(\Sigma, \reals) \to C(\Sigma, \reals)$ with a corresponding eigenfunction $h \in C(\Sigma_A^+, \reals)$, with $h > 0$. \label{rpf:1}
		\item The remainder of the spectrum of $L_f$ is contained in a disc of radius strictly less than $\lambda$. \label{rpf:2}
		\item There is a unique probability measure $\mu$ such that $L_f^*{\mu} = \lambda\mu$. That is,
		\[
			\int{L_f{v}\ d\mu} = \lambda \int{v\ d\mu},
		\]
		for all $v \in C(\Sigma, \reals)$. Additionally, if $h$ is the eigenfunction as in \ref{rpf:1} and $\int{h\ d\mu} = 1$, then the measure $\nu$ defined by $d\nu = h\ d\mu$ is a $\sigma$-invariant probability measure. \label{rpf:3}
		\item If $h$ is the eigenfunction as in \ref{rpf:1} and $\int{h\ d\mu} = 1$, then for all $v \in C(\Sigma, \reals)$,
		\[
			\frac{1}{\lambda^n}L_f^n{v} \to h \int{v\ d\mu}
		\]
		uniformly. \label{rpf:4}
	\end{enumerate}
\end{theorem}
